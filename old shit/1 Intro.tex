\chapter{Introduction}


% In the 21st century, the landscape of public discourse has undergone a significant transformation with the rise of short-form video content platforms such as TikTok, Instagram Reels, and YouTube Shorts. Among these, TikTok has seen a giant increase in popularity, becoming one of the most popular social media services, with more than $1.6$ billion active monthly users worldwide \citep{tiktok_popular}. The shift from more traditional text-dominated social media such as Facebook and X (formerly known as Twitter) to this newer short-form video content format presents new challenges in computationally analyzing public discourse, and requires new methods of analysis. 

% TikTok contains a wide variety of content, ranging from comedic sketches to serious political debates, and everything in between. This multi-faceted nature of TikTok's content makes it an interesting platform for studying. Furthermore, the app has various 



In the 21st century, the landscape of public discourse has undergone a significant transformation with the rise of short-form video content platforms such as TikTok, Instagram Reels, and YouTube Shorts. Among these, TikTok has seen a giant increase in popularity, becoming one of the most popular social media services, with more than $1.6$ billion active monthly users worldwide \citep{tiktok_popular}. The shift from more traditional text-dominated social media such as Facebook and X (formerly known as Twitter) to this newer short-form video content format presents new challenges in computationally analyzing public discourse, and requires new methods of analysis.

TikTok's unique features, such as the ability to "stitch" videos, have created a network of interconnected content that spans a wide variety of topics and formats, ranging from comedic sketches to serious political debates. This network-like structure, where videos can respond to other videos, offers a new dimension to social media interactions and public discussions. However, it also raises significant questions about data privacy, platform security, and the broader societal impact of such platforms. [...]\\
Traditionally, natural language processing methods have been sufficient for analyzing older social media platforms like Facebook and X. However, the multi-media nature of short-form video content introduces new analytical challenges. Understanding this new method of communication requires a multifaceted approach, combining visual, auditory, and textual analysis of individual videos with an understanding of the broader network structure created by these interactions.

The recent introduction of the TikTok Research API \citep{tiktokresearchapi} has opened up new possibilities for studying this platform and its impact. Of particular interest are the platform's reactionary video types such as "stitches" and "duets," which allow users to directly build upon or respond to other videos. These features create a graph-like structure of content, where videos have direct relations to other videos. The purpose and tone of these reactions vary widely depending on the context – from political debates to comedic expansions on a theme.\\\\
This project aims to explore and improve our understanding of how people communicate using stitches on TikTok. By analyzing both the topological structure of the TikTok stitch network and the individual contents of each video, we seek to uncover patterns and insights into this new form of public discourse. Our approach will combine image processing, natural language processing methods, and network analysis to tackle the complex, multi-modal nature of this content.

As we dive into this analysis, we must also consider the broader implications of TikTok's growing influence. The platform's rise hasn't come without concerns, both security-related and societal.  [CITATIONS NEEDED] \citep{tiktok_alcohol_good}, \citep{covidFakeNewsTikTok}, which have been the subject of numerous discussions and articles \citep{security_tiktok, security_tiktok2, security_tiktok3, security_tiktok4}. By studying the structure and content of TikTok's network, we may gain valuable insights not only into communication patterns but also into the potential impacts of this new medium on public opinion, information spread, and social dynamics.

Through this research, we aim to contribute to a deeper understanding of how short-form video content, particularly on TikTok, is shaping modern communication and public discourse. This knowledge could have far-reaching implications for fields ranging from media studies and social psychology to political science and information security.



% NEW CHATGPT INTRO 

% In the 21st century, the landscape of public discourse has been reshaped by the rapid emergence and growth of short-form video platforms like TikTok, Instagram Reels, YouTube Shorts, and Facebook Shorties. Of these, TikTok stands out as the largest and most influential platform, known for pioneering short-form content and fostering diverse forms of user interaction. With more than 1.6 billion active monthly users (Howarth, 2024), TikTok has become a central hub for global social interactions and cultural exchange.

% TikTok’s unique features, particularly the "stitch" function, facilitate the creation of complex, interconnected networks of content where users respond directly to others’ videos. This distinctive form of engagement allows a range of topics and formats to proliferate, spanning lighthearted comedic skits to serious political discussions. The structure of these interactions is inherently graph-like, where videos are nodes and "stitch" responses form directed edges between them.

% Understanding communication on TikTok poses new challenges compared to traditional platforms like Facebook and X (formerly Twitter). The content on TikTok is inherently multi-modal, encompassing auditory, visual, and textual elements that require integrated analytical approaches. While natural language processing (NLP) has been instrumental in examining text-centric social media, analyzing TikTok’s content demands a combination of NLP, image processing, and network analysis to capture the full scope of its communication dynamics.

% The introduction of the TikTok Research API (TikTok, 2023) has expanded the scope for quantitative studies of the platform. However, despite significant interest in TikTok, much existing research remains qualitative or lacks network-centric analysis. This project fills that gap by assembling the first comprehensive, network-centric TikTok dataset composed of stitches across 30 different hashtags, spanning categories from political and educational to entertainment content.

% Our objective is to characterize what makes communication on TikTok unique. We aim to identify communicative patterns and explore how the structure of stitch networks varies across content types. By leveraging methods such as frequent subgraph mining, graph embeddings, and sentiment analysis, we seek to uncover the distinguishing features of interactions on the platform. This exploration leads to key research questions: What are the characteristic communicative patterns on TikTok? How do stitch network structures differ between various content genres?

% Ultimately, this research seeks to contribute valuable insights into modern public discourse and the broader implications of short-form video communication. The findings have the potential to impact fields such as media studies, social psychology, political science, and information security by providing a deeper understanding of how TikTok shapes conversations and influences public opinion in the digital age.


