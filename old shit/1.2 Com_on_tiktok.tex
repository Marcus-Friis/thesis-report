\section{Communication on TikTok}
TikTok offers a range of interactive features that shape how users communicate on the platform. One of the key features is hashtags, which serve as markers for different topics and communities, helping users find and engage with specific content. However, beyond hashtags, TikTok introduced the stitch feature in 2020, adding a new layer to communication. A stitch allows users to respond to another video by incorporating a clip from the original video, but only up to five seconds. Importantly, users cannot stitch a video that is already a stitch of another video, which creates a limitation in the communication structure, often leading to dyads or star-like topologies in the user interaction graph. TikTok also gives users control over this feature, allowing them to disable stitching either for specific videos or across all their content. When a video is stitched, the description will show “stitch with @username” and, most of the time\footnote{We have made a few observations that some accounts have removed the hyperlink from the title, rendering us unable to be redirected to the stitched video}, provide a direct link to the original video. \\

In addition to stitching, TikTok allows users to reply to videos in comments. Interestingly, users can also respond to comments with videos, further blurring the line between text and video-based communication. The duet feature, another popular form of interaction, allows users to create a side-by-side or picture-in-picture collaboration, making it perfect for real-time reactions or joint performances. Another key communication feature is the use of original sound or music\_id, where users borrow audio from other videos\citep{tiktok_features}. This creates a sound-based network, linking different videos and users through shared audio, fostering connections and interactions across the platform.

TikTok’s features like stitches, duets, comments, and original sounds all help create an active and creative environment where users can interact in different ways. However, how they use these features is influenced by the platform’s design and algorithm, which affect what content gets seen and shared.

For this study, we focus only on stitches to maintain a clear direction and avoid becoming overwhelmed by the vast range of interaction types on TikTok. While duets and other forms of video replies offer valuable insights, the volume of data could make the analysis too complex. By narrowing our scope to stitches, we ensure that we can thoroughly explore the dynamics within this specific feature without losing focus. Additionally, there is more than enough data from stitches alone to provide a comprehensive view of communication patterns on TikTok. 

