\chapter{Results}
We introduce the TikTok StitchGraph dataset. From this, applying the presented methodology, we present a look into the foundational subgraphs, and how the collected hashtags relate in their topology. All results are derived from analyzing the largest weakly connected component of the networks.

Importantly, the sentiment results are skewed by human error. \textit{\#challenge}, \textit{\#football}, \textit{\#makeup}, and \textit{\#minecraft} are all missing some sentiment labels. \textit{\#makeup} and \textit{\#minecraft} are only missing a small portion, \textit{\#football} is missing all edges not classified as speech by the applied audio classifier, and \textit{\#challenge} is missing essentially all sentiment labels. This directly impacts the mined sentiment subgraphs and their support, and the sentiment Bag-Of-Subgraphs graph representations. Results based on these should be interpreted accordingly. 

\section{TikTok Subgraphs and Motifs}
In this section, we present the findings of the subgraph analysis. The results are illustrated through figures that show various subgraphs. Each subgraph is accompanied by two numbers indicating the support for TikTok and Twitter, respectively ($TikTok \hspace{4pt} | \hspace{4pt} \textcolor{TwitterBlue}{Twitter}$). Importantly, the reported support values represent transactional support and should be interpreted accordingly.

This section is organized as follows: First, we highlight observations regarding cyclic subgraphs and their absence in the mined results. Next, we examine the hierarchical relationships among subgraphs and the relative prevalence of stars and chains. This is followed by an exploration of adding sentiment and directionality as dimensions in the analysis, comparing TikTok’s patterns to those observed on Twitter. Lastly, we discuss the absence of any significant motifs under the used null model. 

\subsection{Subgraphs}

The first notable observation is the lack of cyclic subgraphs. As seen in Figure \ref{fig:cycle_subgraphs}, the most common cycle is a square, with a support of $16$. This support is low enough such that subgraph mining will not find any subgraph that is a supergraph of a cycle. Interestingly, the support of cycles with an even number of vertices is consistently higher than that of cycles with an odd number of vertices. An odd-numbered cycle means that a user has to be both a stitcher and a stitchee for these cycles to occur. This is indicative of it being rare for a user to both stitch and be stitched, which is also reflected by the mean $38$ magnitude difference in the number of user followers between stitchers and stitchees. In contrast, this tendency is not observed in the Twitter data. Instead, all reported cyclic subgraphs occur almost equally, with most of them having Twitter support $4$. 

\begin{figure}[!htbp]
    \centering
    \begin{comment}
from math import pi, sin, cos

def plot(num_points: int, scaling: int = 1) -> None:
    angle = 2 * pi / num_points
    for i in range(num_points):
        x = sin(i * angle) * scaling
        y = cos(i * angle) * scaling
        print(f'\Vertex[x={x:.2f}, y={y:.2f}]{{{i}}}')
    
    print()
    for i in range(num_points):
        j = (i + 1) % num_points
        print(f'\Edge({i})({j})')
    
if __name__ == '__main__':
    import sys
    scaling = 0.7
    plot(int(sys.argv[1]), scaling)
\end{comment}

\SetVertexStyle[Shape=circle, InnerSep=1, MinSize=4, FillColor=black, LineColor=black]
\SetEdgeStyle[Color=gray, Arrow=-stealth]

\newcommand{\gtriangle}{
% triangle
\begin{tikzpicture}
\Vertex[x=0.00, y=0.70]{0}
\Vertex[x=0.61, y=-0.35]{1}
\Vertex[x=-0.61, y=-0.35]{2}

\Edge(0)(1)
\Edge(1)(2)
\Edge(2)(0)
\end{tikzpicture}
}

\newcommand{\gsquare}{
% square
\begin{tikzpicture}
\Vertex[x=0.00, y=0.70]{0}
\Vertex[x=0.70, y=0.00]{1}
\Vertex[x=0.00, y=-0.70]{2}
\Vertex[x=-0.70, y=-0.00]{3}

\Edge(0)(1)
\Edge(1)(2)
\Edge(2)(3)
\Edge(3)(0)
\end{tikzpicture}
}
\newcommand{\gpentagon}{
% pentagon
\begin{tikzpicture}
\Vertex[x=0.00, y=0.70]{0}
\Vertex[x=0.67, y=0.22]{1}
\Vertex[x=0.41, y=-0.57]{2}
\Vertex[x=-0.41, y=-0.57]{3}
\Vertex[x=-0.67, y=0.22]{4}

\Edge(0)(1)
\Edge(1)(2)
\Edge(2)(3)
\Edge(3)(4)
\Edge(4)(0)
\end{tikzpicture}
}

\newcommand{\ghexagon}{
% hexagon
\begin{tikzpicture}
\Vertex[x=0.00, y=0.70]{0}
\Vertex[x=0.61, y=0.35]{1}
\Vertex[x=0.61, y=-0.35]{2}
\Vertex[x=0.00, y=-0.70]{3}
\Vertex[x=-0.61, y=-0.35]{4}
\Vertex[x=-0.61, y=0.35]{5}

\Edge(0)(1)
\Edge(1)(2)
\Edge(2)(3)
\Edge(3)(4)
\Edge(4)(5)
\Edge(5)(0)
\end{tikzpicture}
}

\newcommand{\gseptagon}{
% septagon
\begin{tikzpicture}
\Vertex[x=0.00, y=0.70]{0}
\Vertex[x=0.55, y=0.44]{1}
\Vertex[x=0.68, y=-0.16]{2}
\Vertex[x=0.30, y=-0.63]{3}
\Vertex[x=-0.30, y=-0.63]{4}
\Vertex[x=-0.68, y=-0.16]{5}
\Vertex[x=-0.55, y=0.44]{6}

\Edge(0)(1)
\Edge(1)(2)
\Edge(2)(3)
\Edge(3)(4)
\Edge(4)(5)
\Edge(5)(6)
\Edge(6)(0)
\end{tikzpicture}
}

\newcommand{\goctagon}{
% octagon
\begin{tikzpicture}
\Vertex[x=0.00, y=0.70]{0}
\Vertex[x=0.49, y=0.49]{1}
\Vertex[x=0.70, y=0.00]{2}
\Vertex[x=0.49, y=-0.49]{3}
\Vertex[x=0.00, y=-0.70]{4}
\Vertex[x=-0.49, y=-0.49]{5}
\Vertex[x=-0.70, y=-0.00]{6}
\Vertex[x=-0.49, y=0.49]{7}

\Edge(0)(1)
\Edge(1)(2)
\Edge(2)(3)
\Edge(3)(4)
\Edge(4)(5)
\Edge(5)(6)
\Edge(6)(7)
\Edge(7)(0)

\end{tikzpicture}
}


% graphs
% 8 17 7 12 7 10
% lcc
% 6 16 7 11 7 10

\newcommand{\cyclesubgraphs}{
\begin{tabular}{c c c c c c}
    \gtriangle & \gsquare & \gpentagon & \ghexagon & \gseptagon & \goctagon \\
     $8$ & $17$ & $7$ & $12$ & $7$ & $10$ 
\end{tabular}
}


\newcommand{\cyclesubgraphslcc}{
\begin{tabular}{c c c c c c}
    \gtriangle & \gsquare & \gpentagon & \ghexagon & \gseptagon & \goctagon \\
     $6 \hspace{4pt} | \hspace{4pt} \textcolor{TwitterBlue}{3}$ & $16 \hspace{4pt} | \hspace{4pt} \textcolor{TwitterBlue}{4}$ & $7 \hspace{4pt} | \hspace{4pt} \textcolor{TwitterBlue}{4}$ & $11 \hspace{4pt} | \hspace{4pt} \textcolor{TwitterBlue}{4}$ & $7 \hspace{4pt} | \hspace{4pt} \textcolor{TwitterBlue}{4}$ & $10 \hspace{4pt} | \hspace{4pt} \textcolor{TwitterBlue}{4}$ 
\end{tabular}
}


\cyclesubgraphslcc

    \caption{The figure shows the cyclic subgraphs identified within the largest connected components of user graphs, ranging from a triangle to an octagon. The numbers beneath denote the support of the subgraph in TikTok and Twitter in that order.}
    \label{fig:cycle_subgraphs}
\end{figure}


Given that cycles are not present in any mined subgraph, the complete subgraph hierarchy of up to six vertices can be mapped out, as presented in Figure \ref{fig:subgraph_hierarchy}. From it, the hierarchy of subgraphs is apparent, illustrating how the support of any child subgraph is at most equal to its parent. We see that stars and star-like patterns generally have slightly higher support than chains and chain-like patterns. Furthermore, we note that, without cycles, any mined subgraph will always be an interpolation between stars and chains.

\begin{figure}[!htbp]
    \centering
    \SetVertexStyle[Shape=circle, InnerSep=1, MinSize=4, FillColor=black, LineColor=black]
\SetEdgeStyle[Color=gray, Arrow=-stealth]

\newcommand{\dyad}{
\begin{tikzpicture}
	\Vertex[x=0, y=0]{0}
    \Vertex[x=1, y=0]{1}
    \Edge[](0)(1)
\end{tikzpicture}
}

\newcommand{\triad}{
\begin{tikzpicture}
	\Vertex[x=0, y=0]{0}
    \Vertex[x=0.5, y=0]{1}
    \Vertex[x=1, y=0]{2}
    \Edge[](0)(1)
    \Edge[](1)(2)
\end{tikzpicture}
}

\newcommand{\threechain}{
\begin{tikzpicture}
	\Vertex[x=0, y=0]{0}
    \Vertex[x=0.33, y=0]{1}
    \Vertex[x=0.66, y=0]{2}
    \Vertex[x=1, y=0]{3}
    \Edge[](0)(1)
    \Edge[](1)(2)
    \Edge[](2)(3)
\end{tikzpicture}
}

\newcommand{\threestar}{
\begin{tikzpicture}
	\Vertex[x=0, y=0]{0}
    \Vertex[x=0, y=0.5]{1}
    \Vertex[x=-0.433, y=-0.25]{2}
    \Vertex[x=0.433, y=-0.25]{3}
    \Edge[](1)(0)
    \Edge[](2)(0)
    \Edge[](3)(0)
\end{tikzpicture}
}

\newcommand{\fourchain}{
\begin{tikzpicture}
	\Vertex[x=0, y=0]{0}
    \Vertex[x=0.33, y=0]{1}
    \Vertex[x=0.66, y=0]{2}
    \Vertex[x=1, y=0]{3}
    \Vertex[x=1.33, y=0]{4}
    \Edge[](0)(1)
    \Edge[](1)(2)
    \Edge[](2)(3)
    \Edge[](3)(4)
\end{tikzpicture}
}

\newcommand{\fourstar}{
\begin{tikzpicture}
	\Vertex[x=0, y=0]{0}
    \Vertex[x=0.33, y=0.33]{1}
    \Vertex[x=0.33, y=-0.33]{2}
    \Vertex[x=-0.33, y=0.33]{3}
    \Vertex[x=-0.33, y=-0.33]{4}
    \Edge[](1)(0)
    \Edge[](2)(0)
    \Edge[](3)(0)
    \Edge[](4)(0)
\end{tikzpicture}
}

\newcommand{\starchain}{
\begin{tikzpicture}
	\Vertex[x=0, y=0]{0}
    \Vertex[x=0, y=0.5]{1}
    \Vertex[x=-0.433, y=-0.25]{2}
    \Vertex[x=0.433, y=-0.25]{3}
    \Vertex[x=0.866, y=-0.5]{4}
    \Edge[](1)(0)
    \Edge[](2)(0)
    \Edge[](3)(0)
    \Edge[](4)(3)
\end{tikzpicture}
}

\newcommand{\hierarchygraph}{
\begin{tikzpicture}[node distance=2.5cm]
    % Nodes
    \node (dyad) {\makecell{\dyad\\ $36$}};
    \node (triad) [below=0.7cm of dyad] {\makecell{\triad\\ $34$}};
    \node (threechain) [below left of=triad] {\makecell{\threechain\\ $32$}};
    \node (threestar) [below right of=triad] {\makecell{\threestar\\ $33$}};
    \node (fourchain) [below left of=threechain] {\makecell{\fourchain\\ $29$}};
    \node (fourstar) [below right of=threestar] {\makecell{\fourstar\\ $32$}};
    \node (starchain) [below right of=threechain] {\makecell{\starchain\\$30$}};

    \draw (dyad) -- (triad);
    \draw (triad) -- (threechain);
    \draw (triad) -- (threestar);
    \draw (threechain) -- (fourchain);
    \draw (threechain) -- (starchain);
    \draw (threestar) -- (fourstar);
    \draw (threestar) -- (starchain);
\end{tikzpicture}
}



\newcommand{\fivechain}{
\begin{tikzpicture}
	\Vertex[x=0, y=0]{0}
	\Vertex[x=0.33, y=0]{1}
	\Vertex[x=0.66, y=0]{2}
	\Vertex[x=1, y=0]{3}
	\Vertex[x=1.33, y=0]{4}
	\Vertex[x=1.66, y=0]{5}
    \Edge[](0)(1)
    \Edge[](1)(2)
    \Edge[](2)(3)
    \Edge[](3)(4)
    \Edge[](4)(5)
\end{tikzpicture}
}

\newcommand{\threestarchain}{
\begin{tikzpicture}
	\Vertex[x=-0.165, y=0.285]{0}
	\Vertex[x=-0.165, y=-0.285]{1}
	\Vertex[x=0, y=0]{2}
	\Vertex[x=0.33, y=0]{3}
	\Vertex[x=0.66, y=0]{4}
	\Vertex[x=1, y=0]{5}
    \Edge[](0)(2)
    \Edge[](1)(2)
    \Edge[](2)(3)
    \Edge[](3)(4)
    \Edge[](4)(5)
\end{tikzpicture}
}

\newcommand{\twothreestarchain}{
\begin{tikzpicture}
	\Vertex[x=-0.25, y=0.433]{0}
	\Vertex[x=-0.25, y=-0.433]{1}
	\Vertex[x=0, y=0]{2}
	\Vertex[x=0.5, y=0]{3}
	\Vertex[x=0.75, y=-0.433]{4}
	\Vertex[x=0.75, y=0.433]{5}
    \Edge[](0)(2)
    \Edge[](1)(2)
    \Edge[](2)(3)
    \Edge[](3)(4)
    \Edge[](3)(5)
\end{tikzpicture}
}

\newcommand{\threestartwochainz}{
\begin{tikzpicture}
	\Vertex[x=0, y=0]{0}
    \Vertex[x=0, y=0.5]{1}
    \Vertex[x=-0.433, y=-0.25]{2}
    \Vertex[x=0.433, y=-0.25]{3}
    \Vertex[x=0.866, y=-0.5]{4}
    \Vertex[x=-0.866, y=-0.5]{5}
    \Edge[](1)(0)
    \Edge[](2)(0)
    \Edge[](3)(0)
    \Edge[](4)(3)
    \Edge[](5)(2)
\end{tikzpicture}
}

\newcommand{\fourstarchain}{
\begin{tikzpicture}
	\Vertex[x=0, y=0]{0}
    \Vertex[x=0.33, y=0.33]{1}
    \Vertex[x=0.33, y=-0.33]{2}
    \Vertex[x=-0.33, y=0.33]{3}
    \Vertex[x=-0.33, y=-0.33]{4}
    \Vertex[x=0.66, y=-0.66]{5}
    \Edge[](1)(0)
    \Edge[](2)(0)
    \Edge[](3)(0)
    \Edge[](4)(0)
    \Edge[](5)(2)
\end{tikzpicture}
}

\newcommand{\fivestar}{
\begin{tikzpicture}
	\Vertex[x=0, y=0]{0}
    \Vertex[x=0, y=0.5]{1}
    \Vertex[x=-0.47552825814757677, y=0.15450849718747373]{2}
    \Vertex[x=-0.2938926261462366, y=-0.40450849718747367]{3}
    \Vertex[x=0.2938926261462365, y=-0.4045084971874738]{4}
    \Vertex[x=0.4755282581475768, y=0.15450849718747361]{5}
    \Edge[](1)(0)
    \Edge[](2)(0)
    \Edge[](3)(0)
    \Edge[](4)(0)
    \Edge[](5)(0)
\end{tikzpicture}
}

\newcommand{\hierarchygraphbig}{
\begin{tikzpicture}[node distance=2.5cm]
    % Nodes
    \node (dyad) {\makecell{\dyad\\ $36$}};
    \node (triad) [below=0.7cm of dyad] {\makecell{\triad\\ $34$}};
    \node (threechain) [below left of=triad] {\makecell{\threechain\\ $32$}};
    \node (threestar) [below right of=triad] {\makecell{\threestar\\ $33$}};
    \node (fourchain) [below left of=threechain] {\makecell{\fourchain\\ $29$}};
    \node (fourstar) [below right of=threestar] {\makecell{\fourstar\\ $32$}};
    \node (starchain) [below right of=threechain] {\makecell{\starchain\\$30$}};
    \node (fivechain) [below left of=fourchain] {\makecell{\fivechain\\ $27$}};
    \node (threestarchain) [below of=fourchain] {\makecell{\threestarchain\\ $28$}};
    \node (twothreestarchain) [below left=2cm and 0cm of starchain] {\makecell{\twothreestarchain\\ $25$}};
    \node (threestartwochainz) [below right=2cm and 0cm of starchain] {\makecell{\threestartwochainz\\ $28$}};
    \node (fourstarchain) [below of=fourstar] {\makecell{\fourstarchain\\ $28$}};
    \node (fivestar) [below right of=fourstar] {\makecell{\fivestar\\ $32$}};

    \draw (dyad) -- (triad);
    \draw (triad) -- (threechain);
    \draw (triad) -- (threestar);
    \draw (threechain) -- (fourchain);
    \draw (threechain) -- (starchain);
    \draw (threestar) -- (fourstar);
    \draw (threestar) -- (starchain);
    \draw (fourchain) -- (fivechain);
    \draw (fourchain) -- (threestarchain);
    \draw (starchain) -- (threestarchain);
    \draw (starchain) -- (twothreestarchain);
    \draw (starchain) -- (threestartwochainz);
    \draw (starchain) -- (fourstarchain);
    \draw (fourstar) -- (fourstarchain);
    \draw (fourstar) -- (fivestar);
    
\end{tikzpicture}
}

\newcommand{\hierarchygraphlccbig}{
\begin{tikzpicture}[node distance=2.5cm]
    % Nodes
    \node (dyad) {\makecell{\dyad\\ $36 \hspace{4pt} | \hspace{4pt} \textcolor{TwitterBlue}{6}$}};
    \node (triad) [below=0.7cm of dyad] {\makecell{\triad\\ $34 \hspace{4pt} | \hspace{4pt} \textcolor{TwitterBlue}{6}$}};
    \node (threechain) [below left of=triad] {\makecell{\threechain\\ $29 \hspace{4pt} | \hspace{4pt} \textcolor{TwitterBlue}{5}$}};
    \node (threestar) [below right of=triad] {\makecell{\threestar\\ $33 \hspace{4pt} | \hspace{4pt} \textcolor{TwitterBlue}{6}$}};
    \node (fourchain) [below left of=threechain] {\makecell{\fourchain\\ $26 \hspace{4pt} | \hspace{4pt} \textcolor{TwitterBlue}{5}$}};
    \node (fourstar) [below right of=threestar] {\makecell{\fourstar\\ $32 \hspace{4pt} | \hspace{4pt} \textcolor{TwitterBlue}{5}$}};
    \node (starchain) [below right of=threechain] {\makecell{\starchain\\$28 \hspace{4pt} | \hspace{4pt} \textcolor{TwitterBlue}{5}$}};
    \node (fivechain) [below left of=fourchain] {\makecell{\fivechain\\ $24 \hspace{4pt} | \hspace{4pt} \textcolor{TwitterBlue}{5}$}};
    \node (threestarchain) [below of=fourchain] {\makecell{\threestarchain\\ $26 \hspace{4pt} | \hspace{4pt} \textcolor{TwitterBlue}{5}$}};
    \node (threestartwochainz) [below left=2cm and 0cm of starchain] {\makecell{\threestartwochainz\\ $24 \hspace{4pt} | \hspace{4pt} \textcolor{TwitterBlue}{5}$}};
    \node (twothreestarchain) [below right=2cm and 0cm of starchain] {\makecell{\twothreestarchain\\ $22 \hspace{4pt} | \hspace{4pt} \textcolor{TwitterBlue}{5}$}};
    \node (fourstarchain) [below of=fourstar] {\makecell{\fourstarchain\\ $27 \hspace{4pt} | \hspace{4pt} \textcolor{TwitterBlue}{4}$}};
    \node (fivestar) [below right of=fourstar] {\makecell{\fivestar\\ $32 \hspace{4pt} | \hspace{4pt} \textcolor{TwitterBlue}{5}$}};

    \draw (dyad) -- (triad);
    \draw (triad) -- (threechain);
    \draw (triad) -- (threestar);
    \draw (threechain) -- (fourchain);
    \draw (threechain) -- (starchain);
    \draw (threestar) -- (fourstar);
    \draw (threestar) -- (starchain);
    \draw (fourchain) -- (fivechain);
    \draw (fourchain) -- (threestarchain);
    \draw (starchain) -- (threestarchain);
    \draw (starchain) -- (twothreestarchain);
    \draw (starchain) -- (threestartwochainz);
    \draw (starchain) -- (fourstarchain);
    \draw (fourstar) -- (fourstarchain);
    \draw (fourstar) -- (fivestar);
    
\end{tikzpicture}
}


\hierarchygraphlccbig
    \caption{The figure shows the frequent undirected subgraphs identified in TikTok user graphs using gSpan on the largest connected component. The lines between subgraphs illustrate their hierarchical relationships, where each subgraph extends from the one above. The numbers are the supports of the given subgraphs in TikTok and Twitter respectively. Simpler structures, such as dyads $S_1$ ($support=36$), form the foundation, while more complex patterns like chains and star-like structures emerge as extensions with lower support. This highlights how TikTok stitch patterns evolve, often around central hubs or sequential interactions.}
    \label{fig:subgraph_hierarchy}
\end{figure}

These findings are also true going beyond subgraphs with six vertices, as seen in Figure \ref{fig:large_subgraphs}. The most common substructure with more than six vertices is a six-star $S_6$, which appears in $29$ user graphs' largest weak component. This aligns with the notion of central users often acting as hubs, either for generating reactions or being highly active in reacting to others. This is a general trend for most of the subgraphs, with pure stars having higher support than pure chains. In fact, for subgraphs where $|V| \geq 8$, the pure chain does not appear within the mined set of graphs with a minimum support threshold of $21$. The rest of the subgraphs are somewhere in between stars and chains, often occurring in the form of stars extended with a chain, or a chain connecting two stars. The support for these interpolations appears to correlate with the degree to which they resemble a star or a chain. By comparison, the chains' support is generally equal to the stars' support in Twitter networks, indicating a preference for chains when compared to TikTok.

\begin{figure}[!htbp]
    \centering
    \begin{adjustwidth}{-\textwidth}{-\textwidth}
        \centering    
        \setlength{\tabcolsep}{9pt}
\begin{tabular}{cccccccccc}
$|V| = 7$&\makecell{\begin{tikzpicture}
	\Vertex[x=-0.50, y=-0.14]{0}
	\Vertex[x=-0.21, y=-0.13]{1}
	\Vertex[x=-0.34, y=-0.39]{2}
	\Vertex[x=-0.37, y=0.11]{3}
	\Vertex[x=-0.05, y=-0.37]{4}
	\Vertex[x=-0.08, y=0.13]{5}
	\Vertex[x=0.08, y=-0.11]{6}
	\Edge[color=gray](0)(1)
	\Edge[color=gray](1)(2)
	\Edge[color=gray](1)(3)
	\Edge[color=gray](1)(4)
	\Edge[color=gray](1)(5)
	\Edge[color=gray](1)(6)
\end{tikzpicture}
\\$29 \hspace{4pt} | \hspace{4pt} \textcolor{TwitterBlue}{5}$
}
&\makecell{\begin{tikzpicture}
	\Vertex[x=0.50, y=0.39]{0}
	\Vertex[x=0.24, y=0.20]{1}
	\Vertex[x=-0.04, y=-0.01]{2}
	\Vertex[x=-0.20, y=0.30]{3}
	\Vertex[x=-0.38, y=-0.02]{4}
	\Vertex[x=-0.15, y=-0.33]{5}
	\Vertex[x=0.21, y=-0.25]{6}
	\Edge[color=gray](0)(1)
	\Edge[color=gray](1)(2)
	\Edge[color=gray](2)(3)
	\Edge[color=gray](2)(4)
	\Edge[color=gray](2)(5)
	\Edge[color=gray](2)(6)
\end{tikzpicture}
\\$27 \hspace{4pt} | \hspace{4pt} \textcolor{TwitterBlue}{4}$
}
&\makecell{\begin{tikzpicture}
	\Vertex[x=0.35, y=0.38]{0}
	\Vertex[x=0.24, y=0.13]{1}
	\Vertex[x=-0.03, y=0.07]{2}
	\Vertex[x=-0.29, y=0.00]{3}
	\Vertex[x=-0.50, y=0.18]{4}
	\Vertex[x=-0.40, y=-0.25]{5}
	\Vertex[x=0.45, y=-0.04]{6}
	\Edge[color=gray](0)(1)
	\Edge[color=gray](1)(2)
	\Edge[color=gray](2)(3)
	\Edge[color=gray](3)(4)
	\Edge[color=gray](3)(5)
	\Edge[color=gray](1)(6)
\end{tikzpicture}
\\$25 \hspace{4pt} | \hspace{4pt} \textcolor{TwitterBlue}{5}$
}
&\makecell{\begin{tikzpicture}
	\Vertex[x=0.31, y=0.50]{0}
	\Vertex[x=0.20, y=0.30]{1}
	\Vertex[x=0.09, y=0.10]{2}
	\Vertex[x=-0.02, y=-0.12]{3}
	\Vertex[x=-0.26, y=-0.04]{4}
	\Vertex[x=-0.13, y=-0.33]{5}
	\Vertex[x=0.17, y=-0.28]{6}
	\Edge[color=gray](0)(1)
	\Edge[color=gray](1)(2)
	\Edge[color=gray](2)(3)
	\Edge[color=gray](3)(4)
	\Edge[color=gray](3)(5)
	\Edge[color=gray](3)(6)
\end{tikzpicture}
\\$25 \hspace{4pt} | \hspace{4pt} \textcolor{TwitterBlue}{4}$
}
&\makecell{\begin{tikzpicture}
	\Vertex[x=0.21, y=0.50]{0}
	\Vertex[x=0.13, y=0.34]{1}
	\Vertex[x=0.05, y=0.19]{2}
	\Vertex[x=-0.03, y=0.03]{3}
	\Vertex[x=-0.12, y=-0.14]{4}
	\Vertex[x=-0.30, y=-0.16]{5}
	\Vertex[x=-0.03, y=-0.30]{6}
	\Edge[color=gray](0)(1)
	\Edge[color=gray](1)(2)
	\Edge[color=gray](2)(3)
	\Edge[color=gray](3)(4)
	\Edge[color=gray](4)(5)
	\Edge[color=gray](4)(6)
\end{tikzpicture}
\\$24 \hspace{4pt} | \hspace{4pt} \textcolor{TwitterBlue}{5}$
}
&\makecell{\begin{tikzpicture}
	\Vertex[x=0.29, y=0.50]{0}
	\Vertex[x=0.17, y=0.32]{1}
	\Vertex[x=0.06, y=0.14]{2}
	\Vertex[x=-0.04, y=-0.05]{3}
	\Vertex[x=-0.26, y=-0.08]{4}
	\Vertex[x=-0.47, y=-0.13]{5}
	\Vertex[x=0.08, y=-0.24]{6}
	\Edge[color=gray](0)(1)
	\Edge[color=gray](1)(2)
	\Edge[color=gray](2)(3)
	\Edge[color=gray](3)(4)
	\Edge[color=gray](4)(5)
	\Edge[color=gray](3)(6)
\end{tikzpicture}
\\$24 \hspace{4pt} | \hspace{4pt} \textcolor{TwitterBlue}{5}$
}
&\makecell{\begin{tikzpicture}
	\Vertex[x=0.18, y=0.50]{0}
	\Vertex[x=0.11, y=0.34]{1}
	\Vertex[x=0.04, y=0.19]{2}
	\Vertex[x=-0.03, y=0.03]{3}
	\Vertex[x=-0.10, y=-0.12]{4}
	\Vertex[x=-0.17, y=-0.28]{5}
	\Vertex[x=-0.25, y=-0.43]{6}
	\Edge[color=gray](0)(1)
	\Edge[color=gray](1)(2)
	\Edge[color=gray](2)(3)
	\Edge[color=gray](3)(4)
	\Edge[color=gray](4)(5)
	\Edge[color=gray](5)(6)
\end{tikzpicture}
\\$23 \hspace{4pt} | \hspace{4pt} \textcolor{TwitterBlue}{5}$
}
&\makecell{\begin{tikzpicture}
	\Vertex[x=0.39, y=0.33]{0}
	\Vertex[x=0.20, y=0.13]{1}
	\Vertex[x=0.02, y=-0.09]{2}
	\Vertex[x=-0.24, y=0.03]{3}
	\Vertex[x=-0.50, y=0.11]{4}
	\Vertex[x=-0.09, y=-0.36]{5}
	\Vertex[x=0.24, y=-0.28]{6}
	\Edge[color=gray](0)(1)
	\Edge[color=gray](1)(2)
	\Edge[color=gray](2)(3)
	\Edge[color=gray](3)(4)
	\Edge[color=gray](2)(5)
	\Edge[color=gray](2)(6)
\end{tikzpicture}
\\$23 \hspace{4pt} | \hspace{4pt} \textcolor{TwitterBlue}{4}$
}
\\[0.9cm]
$|V| = 8$&\makecell{\begin{tikzpicture}
	\Vertex[x=-0.30, y=-0.50]{0}
	\Vertex[x=-0.16, y=-0.19]{1}
	\Vertex[x=0.10, y=0.02]{2}
	\Vertex[x=-0.01, y=-0.49]{3}
	\Vertex[x=-0.43, y=0.01]{4}
	\Vertex[x=-0.49, y=-0.28]{5}
	\Vertex[x=0.17, y=-0.26]{6}
	\Vertex[x=-0.17, y=0.14]{7}
	\Edge[color=gray](0)(1)
	\Edge[color=gray](1)(2)
	\Edge[color=gray](1)(3)
	\Edge[color=gray](1)(4)
	\Edge[color=gray](1)(5)
	\Edge[color=gray](1)(6)
	\Edge[color=gray](1)(7)
\end{tikzpicture}
\\$26 \hspace{4pt} | \hspace{4pt} \textcolor{TwitterBlue}{5}$
}
&\makecell{\begin{tikzpicture}
	\Vertex[x=0.50, y=0.34]{0}
	\Vertex[x=0.25, y=0.17]{1}
	\Vertex[x=-0.03, y=-0.01]{2}
	\Vertex[x=-0.10, y=0.31]{3}
	\Vertex[x=-0.33, y=0.12]{4}
	\Vertex[x=-0.30, y=-0.18]{5}
	\Vertex[x=-0.03, y=-0.33]{6}
	\Vertex[x=0.23, y=-0.20]{7}
	\Edge[color=gray](0)(1)
	\Edge[color=gray](1)(2)
	\Edge[color=gray](2)(3)
	\Edge[color=gray](2)(4)
	\Edge[color=gray](2)(5)
	\Edge[color=gray](2)(6)
	\Edge[color=gray](2)(7)
\end{tikzpicture}
\\$26 \hspace{4pt} | \hspace{4pt} \textcolor{TwitterBlue}{4}$
}
&\makecell{\begin{tikzpicture}
	\Vertex[x=0.30, y=0.50]{0}
	\Vertex[x=0.19, y=0.31]{1}
	\Vertex[x=0.08, y=0.12]{2}
	\Vertex[x=-0.05, y=-0.09]{3}
	\Vertex[x=-0.25, y=0.05]{4}
	\Vertex[x=-0.26, y=-0.19]{5}
	\Vertex[x=-0.04, y=-0.33]{6}
	\Vertex[x=0.17, y=-0.20]{7}
	\Edge[color=gray](0)(1)
	\Edge[color=gray](1)(2)
	\Edge[color=gray](2)(3)
	\Edge[color=gray](3)(4)
	\Edge[color=gray](3)(5)
	\Edge[color=gray](3)(6)
	\Edge[color=gray](3)(7)
\end{tikzpicture}
\\$25 \hspace{4pt} | \hspace{4pt} \textcolor{TwitterBlue}{4}$
}
&\makecell{\begin{tikzpicture}
	\Vertex[x=0.30, y=0.41]{0}
	\Vertex[x=0.27, y=0.17]{1}
	\Vertex[x=0.07, y=0.04]{2}
	\Vertex[x=-0.13, y=-0.09]{3}
	\Vertex[x=-0.32, y=0.08]{4}
	\Vertex[x=-0.33, y=-0.22]{5}
	\Vertex[x=-0.06, y=-0.33]{6}
	\Vertex[x=0.50, y=0.10]{7}
	\Edge[color=gray](0)(1)
	\Edge[color=gray](1)(2)
	\Edge[color=gray](2)(3)
	\Edge[color=gray](3)(4)
	\Edge[color=gray](3)(5)
	\Edge[color=gray](3)(6)
	\Edge[color=gray](1)(7)
\end{tikzpicture}
\\$25 \hspace{4pt} | \hspace{4pt} \textcolor{TwitterBlue}{4}$
}
&\makecell{\begin{tikzpicture}
	\Vertex[x=0.30, y=0.50]{0}
	\Vertex[x=0.21, y=0.30]{1}
	\Vertex[x=0.15, y=0.07]{2}
	\Vertex[x=-0.07, y=-0.02]{3}
	\Vertex[x=-0.27, y=-0.12]{4}
	\Vertex[x=-0.49, y=-0.02]{5}
	\Vertex[x=-0.32, y=-0.35]{6}
	\Vertex[x=0.34, y=-0.07]{7}
	\Edge[color=gray](0)(1)
	\Edge[color=gray](1)(2)
	\Edge[color=gray](2)(3)
	\Edge[color=gray](3)(4)
	\Edge[color=gray](4)(5)
	\Edge[color=gray](4)(6)
	\Edge[color=gray](2)(7)
\end{tikzpicture}
\\$23 \hspace{4pt} | \hspace{4pt} \textcolor{TwitterBlue}{5}$
}
&\makecell{\begin{tikzpicture}
	\Vertex[x=0.44, y=0.50]{0}
	\Vertex[x=0.30, y=0.30]{1}
	\Vertex[x=0.17, y=0.09]{2}
	\Vertex[x=0.04, y=-0.15]{3}
	\Vertex[x=-0.22, y=-0.10]{4}
	\Vertex[x=-0.47, y=-0.09]{5}
	\Vertex[x=-0.01, y=-0.41]{6}
	\Vertex[x=0.27, y=-0.30]{7}
	\Edge[color=gray](0)(1)
	\Edge[color=gray](1)(2)
	\Edge[color=gray](2)(3)
	\Edge[color=gray](3)(4)
	\Edge[color=gray](4)(5)
	\Edge[color=gray](3)(6)
	\Edge[color=gray](3)(7)
\end{tikzpicture}
\\$23 \hspace{4pt} | \hspace{4pt} \textcolor{TwitterBlue}{4}$
}
&\makecell{\begin{tikzpicture}
	\Vertex[x=0.46, y=0.35]{0}
	\Vertex[x=0.22, y=0.16]{1}
	\Vertex[x=-0.01, y=-0.08]{2}
	\Vertex[x=-0.25, y=0.15]{3}
	\Vertex[x=-0.50, y=0.33]{4}
	\Vertex[x=-0.29, y=-0.26]{5}
	\Vertex[x=-0.00, y=-0.41]{6}
	\Vertex[x=0.28, y=-0.24]{7}
	\Edge[color=gray](0)(1)
	\Edge[color=gray](1)(2)
	\Edge[color=gray](2)(3)
	\Edge[color=gray](3)(4)
	\Edge[color=gray](2)(5)
	\Edge[color=gray](2)(6)
	\Edge[color=gray](2)(7)
\end{tikzpicture}
\\$23 \hspace{4pt} | \hspace{4pt} \textcolor{TwitterBlue}{4}$
}
&\makecell{\begin{tikzpicture}
	\Vertex[x=0.22, y=0.50]{0}
	\Vertex[x=0.13, y=0.36]{1}
	\Vertex[x=0.04, y=0.21]{2}
	\Vertex[x=-0.04, y=0.07]{3}
	\Vertex[x=-0.13, y=-0.08]{4}
	\Vertex[x=-0.23, y=-0.23]{5}
	\Vertex[x=-0.41, y=-0.24]{6}
	\Vertex[x=-0.16, y=-0.40]{7}
	\Edge[color=gray](0)(1)
	\Edge[color=gray](1)(2)
	\Edge[color=gray](2)(3)
	\Edge[color=gray](3)(4)
	\Edge[color=gray](4)(5)
	\Edge[color=gray](5)(6)
	\Edge[color=gray](5)(7)
\end{tikzpicture}
\\$22 \hspace{4pt} | \hspace{4pt} \textcolor{TwitterBlue}{5}$
}
&\makecell{\begin{tikzpicture}
	\Vertex[x=0.26, y=0.50]{0}
	\Vertex[x=0.18, y=0.35]{1}
	\Vertex[x=0.11, y=0.20]{2}
	\Vertex[x=0.04, y=0.05]{3}
	\Vertex[x=-0.03, y=-0.12]{4}
	\Vertex[x=-0.21, y=-0.16]{5}
	\Vertex[x=-0.37, y=-0.22]{6}
	\Vertex[x=0.08, y=-0.26]{7}
	\Edge[color=gray](0)(1)
	\Edge[color=gray](1)(2)
	\Edge[color=gray](2)(3)
	\Edge[color=gray](3)(4)
	\Edge[color=gray](4)(5)
	\Edge[color=gray](5)(6)
	\Edge[color=gray](4)(7)
\end{tikzpicture}
\\$22 \hspace{4pt} | \hspace{4pt} \textcolor{TwitterBlue}{5}$
}
\\[0.9cm]
$|V| = 9$&\makecell{\begin{tikzpicture}
	\Vertex[x=-0.28, y=-0.15]{0}
	\Vertex[x=0.11, y=0.00]{1}
	\Vertex[x=0.28, y=0.39]{2}
	\Vertex[x=0.50, y=0.16]{3}
	\Vertex[x=-0.27, y=0.17]{4}
	\Vertex[x=0.49, y=-0.16]{5}
	\Vertex[x=-0.04, y=0.39]{6}
	\Vertex[x=-0.06, y=-0.38]{7}
	\Vertex[x=0.26, y=-0.39]{8}
	\Edge[color=gray](0)(1)
	\Edge[color=gray](1)(2)
	\Edge[color=gray](1)(3)
	\Edge[color=gray](1)(4)
	\Edge[color=gray](1)(5)
	\Edge[color=gray](1)(6)
	\Edge[color=gray](1)(7)
	\Edge[color=gray](1)(8)
\end{tikzpicture}
\\$25 \hspace{4pt} | \hspace{4pt} \textcolor{TwitterBlue}{5}$
}
&\makecell{\begin{tikzpicture}
	\Vertex[x=0.50, y=0.08]{0}
	\Vertex[x=0.28, y=0.16]{1}
	\Vertex[x=0.09, y=0.04]{2}
	\Vertex[x=-0.12, y=-0.09]{3}
	\Vertex[x=-0.24, y=0.12]{4}
	\Vertex[x=-0.35, y=-0.08]{5}
	\Vertex[x=-0.23, y=-0.29]{6}
	\Vertex[x=0.00, y=-0.29]{7}
	\Vertex[x=0.32, y=0.38]{8}
	\Edge[color=gray](0)(1)
	\Edge[color=gray](1)(2)
	\Edge[color=gray](2)(3)
	\Edge[color=gray](3)(4)
	\Edge[color=gray](3)(5)
	\Edge[color=gray](3)(6)
	\Edge[color=gray](3)(7)
	\Edge[color=gray](1)(8)
\end{tikzpicture}
\\$25 \hspace{4pt} | \hspace{4pt} \textcolor{TwitterBlue}{4}$
}
&\makecell{\begin{tikzpicture}
	\Vertex[x=0.32, y=0.50]{0}
	\Vertex[x=0.20, y=0.32]{1}
	\Vertex[x=0.08, y=0.13]{2}
	\Vertex[x=-0.05, y=-0.07]{3}
	\Vertex[x=-0.21, y=0.11]{4}
	\Vertex[x=-0.29, y=-0.08]{5}
	\Vertex[x=-0.18, y=-0.27]{6}
	\Vertex[x=0.04, y=-0.30]{7}
	\Vertex[x=0.18, y=-0.14]{8}
	\Edge[color=gray](0)(1)
	\Edge[color=gray](1)(2)
	\Edge[color=gray](2)(3)
	\Edge[color=gray](3)(4)
	\Edge[color=gray](3)(5)
	\Edge[color=gray](3)(6)
	\Edge[color=gray](3)(7)
	\Edge[color=gray](3)(8)
\end{tikzpicture}
\\$24 \hspace{4pt} | \hspace{4pt} \textcolor{TwitterBlue}{4}$
}
&\makecell{\begin{tikzpicture}
	\Vertex[x=0.19, y=0.39]{0}
	\Vertex[x=0.15, y=0.17]{1}
	\Vertex[x=0.15, y=-0.08]{2}
	\Vertex[x=-0.07, y=-0.17]{3}
	\Vertex[x=-0.28, y=-0.27]{4}
	\Vertex[x=-0.34, y=-0.50]{5}
	\Vertex[x=-0.50, y=-0.18]{6}
	\Vertex[x=0.28, y=-0.28]{7}
	\Vertex[x=0.39, y=-0.06]{8}
	\Edge[color=gray](0)(1)
	\Edge[color=gray](1)(2)
	\Edge[color=gray](2)(3)
	\Edge[color=gray](3)(4)
	\Edge[color=gray](4)(5)
	\Edge[color=gray](4)(6)
	\Edge[color=gray](2)(7)
	\Edge[color=gray](2)(8)
\end{tikzpicture}
\\$23 \hspace{4pt} | \hspace{4pt} \textcolor{TwitterBlue}{4}$
}
&\makecell{\begin{tikzpicture}
	\Vertex[x=0.50, y=0.45]{0}
	\Vertex[x=0.34, y=0.27]{1}
	\Vertex[x=0.17, y=0.07]{2}
	\Vertex[x=0.02, y=-0.15]{3}
	\Vertex[x=-0.22, y=-0.02]{4}
	\Vertex[x=-0.46, y=0.06]{5}
	\Vertex[x=-0.17, y=-0.34]{6}
	\Vertex[x=0.07, y=-0.42]{7}
	\Vertex[x=0.27, y=-0.26]{8}
	\Edge[color=gray](0)(1)
	\Edge[color=gray](1)(2)
	\Edge[color=gray](2)(3)
	\Edge[color=gray](3)(4)
	\Edge[color=gray](4)(5)
	\Edge[color=gray](3)(6)
	\Edge[color=gray](3)(7)
	\Edge[color=gray](3)(8)
\end{tikzpicture}
\\$23 \hspace{4pt} | \hspace{4pt} \textcolor{TwitterBlue}{4}$
}
&\makecell{\begin{tikzpicture}
	\Vertex[x=0.39, y=-0.25]{0}
	\Vertex[x=0.17, y=-0.23]{1}
	\Vertex[x=-0.05, y=-0.21]{2}
	\Vertex[x=-0.28, y=-0.19]{3}
	\Vertex[x=-0.33, y=0.03]{4}
	\Vertex[x=-0.50, y=-0.18]{5}
	\Vertex[x=-0.37, y=-0.41]{6}
	\Vertex[x=0.23, y=-0.46]{7}
	\Vertex[x=0.26, y=-0.02]{8}
	\Edge[color=gray](0)(1)
	\Edge[color=gray](1)(2)
	\Edge[color=gray](2)(3)
	\Edge[color=gray](3)(4)
	\Edge[color=gray](3)(5)
	\Edge[color=gray](3)(6)
	\Edge[color=gray](1)(7)
	\Edge[color=gray](1)(8)
\end{tikzpicture}
\\$23 \hspace{4pt} | \hspace{4pt} \textcolor{TwitterBlue}{4}$
}
&\makecell{\begin{tikzpicture}
	\Vertex[x=-0.50, y=0.14]{0}
	\Vertex[x=-0.31, y=0.04]{1}
	\Vertex[x=-0.11, y=-0.08]{2}
	\Vertex[x=0.06, y=0.08]{3}
	\Vertex[x=-0.11, y=0.15]{4}
	\Vertex[x=-0.30, y=-0.20]{5}
	\Vertex[x=-0.15, y=-0.31]{6}
	\Vertex[x=0.03, y=-0.26]{7}
	\Vertex[x=0.12, y=-0.10]{8}
	\Edge[color=gray](0)(1)
	\Edge[color=gray](1)(2)
	\Edge[color=gray](2)(3)
	\Edge[color=gray](2)(4)
	\Edge[color=gray](2)(5)
	\Edge[color=gray](2)(6)
	\Edge[color=gray](2)(7)
	\Edge[color=gray](2)(8)
\end{tikzpicture}
\\$23 \hspace{4pt} | \hspace{4pt} \textcolor{TwitterBlue}{4}$
}
\\[0.9cm]
$|V| = 10$&\makecell{\begin{tikzpicture}
	\Vertex[x=-0.17, y=-0.25]{0}
	\Vertex[x=0.13, y=-0.02]{1}
	\Vertex[x=0.34, y=0.31]{2}
	\Vertex[x=0.50, y=0.10]{3}
	\Vertex[x=0.08, y=0.37]{4}
	\Vertex[x=-0.15, y=0.24]{5}
	\Vertex[x=-0.25, y=-0.00]{6}
	\Vertex[x=0.31, y=-0.36]{7}
	\Vertex[x=0.49, y=-0.16]{8}
	\Vertex[x=0.05, y=-0.39]{9}
	\Edge[color=gray](0)(1)
	\Edge[color=gray](1)(2)
	\Edge[color=gray](1)(3)
	\Edge[color=gray](1)(4)
	\Edge[color=gray](1)(5)
	\Edge[color=gray](1)(6)
	\Edge[color=gray](1)(7)
	\Edge[color=gray](1)(8)
	\Edge[color=gray](1)(9)
\end{tikzpicture}
\\$24 \hspace{4pt} | \hspace{4pt} \textcolor{TwitterBlue}{5}$
}
&\makecell{\begin{tikzpicture}
	\Vertex[x=0.35, y=0.36]{0}
	\Vertex[x=0.30, y=0.15]{1}
	\Vertex[x=0.10, y=0.04]{2}
	\Vertex[x=-0.10, y=-0.06]{3}
	\Vertex[x=-0.17, y=0.15]{4}
	\Vertex[x=-0.31, y=0.03]{5}
	\Vertex[x=-0.30, y=-0.17]{6}
	\Vertex[x=-0.14, y=-0.29]{7}
	\Vertex[x=0.04, y=-0.24]{8}
	\Vertex[x=0.50, y=0.07]{9}
	\Edge[color=gray](0)(1)
	\Edge[color=gray](1)(2)
	\Edge[color=gray](2)(3)
	\Edge[color=gray](3)(4)
	\Edge[color=gray](3)(5)
	\Edge[color=gray](3)(6)
	\Edge[color=gray](3)(7)
	\Edge[color=gray](3)(8)
	\Edge[color=gray](1)(9)
\end{tikzpicture}
\\$24 \hspace{4pt} | \hspace{4pt} \textcolor{TwitterBlue}{4}$
}
&\makecell{\begin{tikzpicture}
	\Vertex[x=-0.04, y=0.17]{0}
	\Vertex[x=0.04, y=0.05]{1}
	\Vertex[x=0.15, y=-0.06]{2}
	\Vertex[x=0.07, y=-0.22]{3}
	\Vertex[x=0.01, y=-0.37]{4}
	\Vertex[x=-0.13, y=-0.42]{5}
	\Vertex[x=0.08, y=-0.50]{6}
	\Vertex[x=0.11, y=-0.14]{7}
	\Vertex[x=0.30, y=-0.11]{8}
	\Vertex[x=0.25, y=0.04]{9}
	\Edge[color=gray](0)(1)
	\Edge[color=gray](1)(2)
	\Edge[color=gray](2)(3)
	\Edge[color=gray](3)(4)
	\Edge[color=gray](4)(5)
	\Edge[color=gray](4)(6)
	\Edge[color=gray](2)(7)
	\Edge[color=gray](2)(8)
	\Edge[color=gray](2)(9)
\end{tikzpicture}
\\$23 \hspace{4pt} | \hspace{4pt} \textcolor{TwitterBlue}{4}$
}
&\makecell{\begin{tikzpicture}
	\Vertex[x=-0.14, y=-0.50]{0}
	\Vertex[x=-0.10, y=-0.31]{1}
	\Vertex[x=-0.04, y=-0.10]{2}
	\Vertex[x=0.14, y=0.02]{3}
	\Vertex[x=-0.14, y=0.09]{4}
	\Vertex[x=-0.24, y=-0.03]{5}
	\Vertex[x=-0.23, y=-0.17]{6}
	\Vertex[x=0.09, y=-0.25]{7}
	\Vertex[x=0.17, y=-0.13]{8}
	\Vertex[x=0.01, y=0.11]{9}
	\Edge[color=gray](0)(1)
	\Edge[color=gray](1)(2)
	\Edge[color=gray](2)(3)
	\Edge[color=gray](2)(4)
	\Edge[color=gray](2)(5)
	\Edge[color=gray](2)(6)
	\Edge[color=gray](2)(7)
	\Edge[color=gray](2)(8)
	\Edge[color=gray](2)(9)
\end{tikzpicture}
\\$23 \hspace{4pt} | \hspace{4pt} \textcolor{TwitterBlue}{4}$
}
&\makecell{\begin{tikzpicture}
	\Vertex[x=0.50, y=0.34]{0}
	\Vertex[x=0.33, y=0.21]{1}
	\Vertex[x=0.15, y=0.06]{2}
	\Vertex[x=-0.01, y=-0.13]{3}
	\Vertex[x=-0.18, y=0.05]{4}
	\Vertex[x=-0.37, y=0.19]{5}
	\Vertex[x=-0.24, y=-0.19]{6}
	\Vertex[x=-0.12, y=-0.35]{7}
	\Vertex[x=0.08, y=-0.36]{8}
	\Vertex[x=0.21, y=-0.21]{9}
	\Edge[color=gray](0)(1)
	\Edge[color=gray](1)(2)
	\Edge[color=gray](2)(3)
	\Edge[color=gray](3)(4)
	\Edge[color=gray](4)(5)
	\Edge[color=gray](3)(6)
	\Edge[color=gray](3)(7)
	\Edge[color=gray](3)(8)
	\Edge[color=gray](3)(9)
\end{tikzpicture}
\\$22 \hspace{4pt} | \hspace{4pt} \textcolor{TwitterBlue}{4}$
}
\end{tabular}

    \end{adjustwidth}
    \caption{The figure displays frequent subgraphs with seven or more nodes in the user graphs, grouped by their number their number of vertices $|V|$, down to support threshold $22$ (see the full figure in Appendix \ref{lab:fsm_appendix} Table \ref{fig:large_subgraphs_full}). The numbers are the support in TikTok and Twitter respectively. Generally, stars and star-like patterns are the most common for each size, with chains and chain-like patterns having systematically lower supports.}
    \label{fig:large_subgraphs}
\end{figure}

Nuancing the subgraphs with direction provides further insight into the tendencies in stitch behavior. As reflected by the lack of odd-numbered cycles, a user being both stitched and stitching another user is rare. This is further reinforced by the directed subgraphs in Figure \ref{fig:directed_subgraphs}, showing an out-two-star has support $34$, an in-two-star has support $33$, but a mixed-direction two-star has support $22$, meaning it is comparatively uncommon for a user to both stitch and be stitched. The same trend is reflected in all larger structures, with three-stars $S_3$ having out-star support $32$, in-star support $31$ and mixed-star support $17$ or $14$ based on the specific directionality. Conversely, Twitter exhibits relatively uniform levels of support across in-, out- and mixed-direction two-stars $S_2$. Moreover, three-stars $S_3$ display interesting supports. Although the in- and out-three-star $S_3$ Twitter support is comparable, the two mixed-direction three-stars $S_3$ vary greatly compared to all previous Twitter support patterns. 

From the figure, the square from above is also specified with direction. As hypothesized, the square consists of two users collectively stitching two other users, forming a weak cycle. This pattern has equivalent support with the undirected square, and no other directed squares are mined, leading to this structure largely explaining the undirected squares support. 

\begin{figure}[h]
    \centering
    \begin{adjustwidth}{-\textwidth}{-\textwidth}
        \centering
        \setlength{\tabcolsep}{9pt}
\begin{tabular}{ccccccccc}
$|V| = 3$&\makecell{\begin{tikzpicture}
	\Vertex[x=0.06, y=0.50]{0}
	\Vertex[x=-0.06, y=0.14]{1}
	\Vertex[x=-0.18, y=-0.23]{2}
	\Edge[color=gray,Direct](0)(1)
	\Edge[color=gray,Direct](2)(1)
\end{tikzpicture}
\\$32 \hspace{4pt} | \hspace{4pt} \textcolor{TwitterBlue}{5}$
}
&\makecell{\begin{tikzpicture}
	\Vertex[x=0.18, y=0.02]{0}
	\Vertex[x=0.50, y=0.38]{1}
	\Vertex[x=-0.14, y=-0.33]{2}
	\Edge[color=gray,Direct](0)(1)
	\Edge[color=gray,Direct](0)(2)
\end{tikzpicture}
\\$31 \hspace{4pt} | \hspace{4pt} \textcolor{TwitterBlue}{6}$
}
&\makecell{\begin{tikzpicture}
	\Vertex[x=0.18, y=0.02]{0}
	\Vertex[x=0.50, y=0.38]{1}
	\Vertex[x=-0.14, y=-0.33]{2}
	\Edge[color=gray,Direct](0)(1)
	\Edge[color=gray,Direct](2)(0)
\end{tikzpicture}
\\$16 \hspace{4pt} | \hspace{4pt} \textcolor{TwitterBlue}{5}$
}
\\[0.9cm]
$|V| = 4$&\makecell{\begin{tikzpicture}
	\Vertex[x=0.04, y=0.05]{0}
	\Vertex[x=0.10, y=-0.23]{1}
	\Vertex[x=-0.01, y=0.32]{2}
	\Vertex[x=0.15, y=-0.50]{3}
	\Edge[color=gray,Direct](0)(1)
	\Edge[color=gray,Direct](0)(2)
	\Edge[color=gray,Direct](3)(1)
\end{tikzpicture}
\\$29 \hspace{4pt} | \hspace{4pt} \textcolor{TwitterBlue}{5}$
}
&\makecell{\begin{tikzpicture}
	\Vertex[x=0.17, y=0.49]{0}
	\Vertex[x=-0.10, y=0.19]{1}
	\Vertex[x=-0.50, y=0.28]{2}
	\Vertex[x=0.02, y=-0.20]{3}
	\Edge[color=gray,Direct](0)(1)
	\Edge[color=gray,Direct](2)(1)
	\Edge[color=gray,Direct](3)(1)
\end{tikzpicture}
\\$28 \hspace{4pt} | \hspace{4pt} \textcolor{TwitterBlue}{5}$
}
&\makecell{\begin{tikzpicture}
	\Vertex[x=0.19, y=-0.10]{0}
	\Vertex[x=0.49, y=0.17]{1}
	\Vertex[x=-0.20, y=0.02]{2}
	\Vertex[x=0.28, y=-0.50]{3}
	\Edge[color=gray,Direct](0)(1)
	\Edge[color=gray,Direct](0)(2)
	\Edge[color=gray,Direct](0)(3)
\end{tikzpicture}
\\$27 \hspace{4pt} | \hspace{4pt} \textcolor{TwitterBlue}{6}$
}
&\makecell{\begin{tikzpicture}
	\Vertex[x=0.40, y=0.11]{0}
	\Vertex[x=-0.00, y=0.50]{1}
	\Vertex[x=-0.50, y=-0.00]{2}
	\Vertex[x=-0.10, y=-0.40]{3}
	\Edge[color=gray,Direct](0)(1)
	\Edge[color=gray,Direct](0)(3)
	\Edge[color=gray,Direct](2)(3)
	\Edge[color=gray,Direct](2)(1)
\end{tikzpicture}
\\$16 \hspace{4pt} | \hspace{4pt} \textcolor{TwitterBlue}{4}$
}
&\makecell{\begin{tikzpicture}
	\Vertex[x=0.04, y=0.05]{0}
	\Vertex[x=0.10, y=-0.23]{1}
	\Vertex[x=-0.01, y=0.32]{2}
	\Vertex[x=0.15, y=-0.50]{3}
	\Edge[color=gray,Direct](0)(1)
	\Edge[color=gray,Direct](0)(2)
	\Edge[color=gray,Direct](1)(3)
\end{tikzpicture}
\\$14 \hspace{4pt} | \hspace{4pt} \textcolor{TwitterBlue}{5}$
}
&\makecell{\begin{tikzpicture}
	\Vertex[x=0.04, y=0.05]{0}
	\Vertex[x=0.10, y=-0.23]{1}
	\Vertex[x=-0.01, y=0.32]{2}
	\Vertex[x=0.15, y=-0.50]{3}
	\Edge[color=gray,Direct](0)(1)
	\Edge[color=gray,Direct](2)(0)
	\Edge[color=gray,Direct](3)(1)
\end{tikzpicture}
\\$14 \hspace{4pt} | \hspace{4pt} \textcolor{TwitterBlue}{5}$
}
&\makecell{\begin{tikzpicture}
	\Vertex[x=0.19, y=-0.10]{0}
	\Vertex[x=0.49, y=0.17]{1}
	\Vertex[x=-0.20, y=0.02]{2}
	\Vertex[x=0.28, y=-0.50]{3}
	\Edge[color=gray,Direct](0)(1)
	\Edge[color=gray,Direct](0)(2)
	\Edge[color=gray,Direct](3)(0)
\end{tikzpicture}
\\$13 \hspace{4pt} | \hspace{4pt} \textcolor{TwitterBlue}{4}$
}
&\makecell{\begin{tikzpicture}
	\Vertex[x=0.19, y=-0.10]{0}
	\Vertex[x=0.49, y=0.17]{1}
	\Vertex[x=-0.20, y=0.02]{2}
	\Vertex[x=0.28, y=-0.50]{3}
	\Edge[color=gray,Direct](0)(1)
	\Edge[color=gray,Direct](2)(0)
	\Edge[color=gray,Direct](3)(0)
\end{tikzpicture}
\\$13 \hspace{4pt} | \hspace{4pt} \textcolor{TwitterBlue}{1}$
}
\end{tabular}

    \end{adjustwidth}
    \caption{The figure shows the frequent directed subgraphs identified in the user graphs. Due to computational constraints, the analysis was limited to subgraphs with four or fewer nodes.
    We observe that chains and star structures with mixed direction occurs a lot less frequently than those with consistent directionality.} 
    \label{fig:directed_subgraphs}
\end{figure}

% Something something motifs

To determine whether any mined subgraph is a motif, subgraphs are compared to relevant null models. Doing this did not reveal significant motifs for undirected and directed subgraph mining. This result is likely due to the choice of null model, the simplicity of the mined subgraphs, and the specific support definition. Configuration models preserve the degree distribution, and many of the simple structures are explained by the degree distribution. For instance, stars can be entirely explained by the degree distribution: high-degree central nodes and low-degree peripheral nodes (each with an out-degree of $1$) will occur with equal prevalence in a configuration model. Other mined subgraphs are often star-like patterns, and in combination with the large size of many user graphs, the subgraphs are likely to emerge by chance based on the given degree distribution. This is further perpetuated by the support definition. Each subgraph only has to appear once for it to count towards the support, meaning the support definition does not account for differences in frequency within the user graphs and configuration models.   

% However, our analysis does not reveal significant motifs within the TikTok stitch networks. This result is likely due to the dominant star-like structures characteristic of these graphs. Configuration models preserve the degree distribution, and in the case of star structures, the high-degree central nodes (representing hubs) and low-degree peripheral nodes (degree of 1) are maintained. This results in similar subgraph occurrences between the observed graphs and the configuration models, making it challenging to identify motifs that exceed the baseline expectations.


\subsection{Sentiment Subgraphs}

Adding sentiment as a dimension yields the subgraphs illustrated in Figure \ref{fig:sentiment_subgraphs}. From these we see that positive and negative edges have a support of $30$, missing sentiment edges $27$, and neutral edges $22$. An explanation for the maximum support of $30$ lies in the size of the largest components, as ten user graphs have a largest component size of $|V_{lcc}| \leq 11$. Outside of these surface-level observations, no clear trends emerge with regard to sentiment patterns. 

A notable observation is the lack of anything between pure stars and chains, unlike in the previous subgraph mining. In fact, subgraphs with $|V| \geq 5$ consist of purely stars, further indicating that TikTok stitch networks are dominated by this pattern. Although no mechanism prevents other subgraphs from appearing, the likely reason for their absence is the increased number of possible subgraphs, especially due to the limited largest component sizes. As four new edge classes are introduced, the possible subgraphs increase exponentially.  

\begin{figure}[h]
    \centering
    \begin{adjustwidth}{-\textwidth}{-\textwidth}
        \centering
        \setlength{\tabcolsep}{9pt}
\begin{tabular}{ccccccccc}
$|V| = 2$&\makecell{\begin{tikzpicture}
	\Vertex[x=0.50, y=-0.00]{0}
	\Vertex[x=-0.28, y=0.00]{1}
	\Edge[color=SentimentPositive](0)(1)
\end{tikzpicture}
\\$30 \hspace{4pt} | \hspace{4pt} \textcolor{TwitterBlue}{6}$
}
&\makecell{\begin{tikzpicture}
	\Vertex[x=0.50, y=-0.00]{0}
	\Vertex[x=-0.28, y=0.00]{1}
	\Edge[color=SentimentNegative](0)(1)
\end{tikzpicture}
\\$30 \hspace{4pt} | \hspace{4pt} \textcolor{TwitterBlue}{5}$
}
&\makecell{\begin{tikzpicture}
	\Vertex[x=0.50, y=-0.00]{0}
	\Vertex[x=-0.28, y=0.00]{1}
	\Edge[color=SentimentMissing](0)(1)
\end{tikzpicture}
\\$27 \hspace{4pt} | \hspace{4pt} \textcolor{TwitterBlue}{-}$
}
&\makecell{\begin{tikzpicture}
	\Vertex[x=0.50, y=-0.00]{0}
	\Vertex[x=-0.28, y=0.00]{1}
	\Edge[color=SentimentNeutral](0)(1)
\end{tikzpicture}
\\$22 \hspace{4pt} | \hspace{4pt} \textcolor{TwitterBlue}{5}$
}
\\[0.9cm]
$|V| = 3$&\makecell{\begin{tikzpicture}
	\Vertex[x=0.06, y=0.50]{0}
	\Vertex[x=-0.06, y=0.14]{1}
	\Vertex[x=-0.18, y=-0.23]{2}
	\Edge[color=SentimentPositive](0)(1)
	\Edge[color=SentimentNegative](1)(2)
\end{tikzpicture}
\\$28 \hspace{4pt} | \hspace{4pt} \textcolor{TwitterBlue}{5}$
}
&\makecell{\begin{tikzpicture}
	\Vertex[x=0.06, y=0.50]{0}
	\Vertex[x=-0.06, y=0.14]{1}
	\Vertex[x=-0.18, y=-0.23]{2}
	\Edge[color=SentimentNegative](0)(1)
	\Edge[color=SentimentNegative](1)(2)
\end{tikzpicture}
\\$28 \hspace{4pt} | \hspace{4pt} \textcolor{TwitterBlue}{5}$
}
&\makecell{\begin{tikzpicture}
	\Vertex[x=0.06, y=0.50]{0}
	\Vertex[x=-0.06, y=0.14]{1}
	\Vertex[x=-0.18, y=-0.23]{2}
	\Edge[color=SentimentPositive](0)(1)
	\Edge[color=SentimentPositive](1)(2)
\end{tikzpicture}
\\$28 \hspace{4pt} | \hspace{4pt} \textcolor{TwitterBlue}{4}$
}
&\makecell{\begin{tikzpicture}
	\Vertex[x=0.06, y=0.50]{0}
	\Vertex[x=-0.06, y=0.14]{1}
	\Vertex[x=-0.18, y=-0.23]{2}
	\Edge[color=SentimentMissing](0)(1)
	\Edge[color=SentimentMissing](1)(2)
\end{tikzpicture}
\\$25 \hspace{4pt} | \hspace{4pt} \textcolor{TwitterBlue}{-}$
}
&\makecell{\begin{tikzpicture}
	\Vertex[x=0.06, y=0.50]{0}
	\Vertex[x=-0.06, y=0.14]{1}
	\Vertex[x=-0.18, y=-0.23]{2}
	\Edge[color=SentimentNeutral](0)(1)
	\Edge[color=SentimentPositive](1)(2)
\end{tikzpicture}
\\$22 \hspace{4pt} | \hspace{4pt} \textcolor{TwitterBlue}{5}$
}
&\makecell{\begin{tikzpicture}
	\Vertex[x=0.06, y=0.50]{0}
	\Vertex[x=-0.06, y=0.14]{1}
	\Vertex[x=-0.18, y=-0.23]{2}
	\Edge[color=SentimentNeutral](0)(1)
	\Edge[color=SentimentNegative](1)(2)
\end{tikzpicture}
\\$22 \hspace{4pt} | \hspace{4pt} \textcolor{TwitterBlue}{4}$
}
&\makecell{\begin{tikzpicture}
	\Vertex[x=0.06, y=0.50]{0}
	\Vertex[x=-0.06, y=0.14]{1}
	\Vertex[x=-0.18, y=-0.23]{2}
	\Edge[color=SentimentPositive](0)(1)
	\Edge[color=SentimentMissing](1)(2)
\end{tikzpicture}
\\$22 \hspace{4pt} | \hspace{4pt} \textcolor{TwitterBlue}{-}$
}
&\makecell{\begin{tikzpicture}
	\Vertex[x=0.06, y=0.50]{0}
	\Vertex[x=-0.06, y=0.14]{1}
	\Vertex[x=-0.18, y=-0.23]{2}
	\Edge[color=SentimentNegative](0)(1)
	\Edge[color=SentimentMissing](1)(2)
\end{tikzpicture}
\\$22 \hspace{4pt} | \hspace{4pt} \textcolor{TwitterBlue}{-}$
}
\\[0.9cm]
$|V| = 4$&\makecell{\begin{tikzpicture}
	\Vertex[x=0.17, y=0.49]{0}
	\Vertex[x=-0.10, y=0.19]{1}
	\Vertex[x=-0.50, y=0.28]{2}
	\Vertex[x=0.02, y=-0.20]{3}
	\Edge[color=SentimentPositive](0)(1)
	\Edge[color=SentimentPositive](1)(2)
	\Edge[color=SentimentNegative](1)(3)
\end{tikzpicture}
\\$27 \hspace{4pt} | \hspace{4pt} \textcolor{TwitterBlue}{4}$
}
&\makecell{\begin{tikzpicture}
	\Vertex[x=0.17, y=0.49]{0}
	\Vertex[x=-0.10, y=0.19]{1}
	\Vertex[x=-0.50, y=0.28]{2}
	\Vertex[x=0.02, y=-0.20]{3}
	\Edge[color=SentimentPositive](0)(1)
	\Edge[color=SentimentNegative](1)(2)
	\Edge[color=SentimentNegative](1)(3)
\end{tikzpicture}
\\$26 \hspace{4pt} | \hspace{4pt} \textcolor{TwitterBlue}{5}$
}
&\makecell{\begin{tikzpicture}
	\Vertex[x=0.17, y=0.49]{0}
	\Vertex[x=-0.10, y=0.19]{1}
	\Vertex[x=-0.50, y=0.28]{2}
	\Vertex[x=0.02, y=-0.20]{3}
	\Edge[color=SentimentPositive](0)(1)
	\Edge[color=SentimentPositive](1)(2)
	\Edge[color=SentimentPositive](1)(3)
\end{tikzpicture}
\\$25 \hspace{4pt} | \hspace{4pt} \textcolor{TwitterBlue}{4}$
}
&\makecell{\begin{tikzpicture}
	\Vertex[x=0.17, y=0.49]{0}
	\Vertex[x=-0.10, y=0.19]{1}
	\Vertex[x=-0.50, y=0.28]{2}
	\Vertex[x=0.02, y=-0.20]{3}
	\Edge[color=SentimentNegative](0)(1)
	\Edge[color=SentimentNegative](1)(2)
	\Edge[color=SentimentNegative](1)(3)
\end{tikzpicture}
\\$22 \hspace{4pt} | \hspace{4pt} \textcolor{TwitterBlue}{5}$
}
&\makecell{\begin{tikzpicture}
	\Vertex[x=0.17, y=0.49]{0}
	\Vertex[x=-0.10, y=0.19]{1}
	\Vertex[x=-0.50, y=0.28]{2}
	\Vertex[x=0.02, y=-0.20]{3}
	\Edge[color=SentimentNeutral](0)(1)
	\Edge[color=SentimentPositive](1)(2)
	\Edge[color=SentimentPositive](1)(3)
\end{tikzpicture}
\\$22 \hspace{4pt} | \hspace{4pt} \textcolor{TwitterBlue}{4}$
}
&\makecell{\begin{tikzpicture}
	\Vertex[x=0.17, y=0.49]{0}
	\Vertex[x=-0.10, y=0.19]{1}
	\Vertex[x=-0.50, y=0.28]{2}
	\Vertex[x=0.02, y=-0.20]{3}
	\Edge[color=SentimentNeutral](0)(1)
	\Edge[color=SentimentPositive](1)(2)
	\Edge[color=SentimentNegative](1)(3)
\end{tikzpicture}
\\$22 \hspace{4pt} | \hspace{4pt} \textcolor{TwitterBlue}{4}$
}
&\makecell{\begin{tikzpicture}
	\Vertex[x=0.17, y=0.49]{0}
	\Vertex[x=-0.10, y=0.19]{1}
	\Vertex[x=-0.50, y=0.28]{2}
	\Vertex[x=0.02, y=-0.20]{3}
	\Edge[color=SentimentPositive](0)(1)
	\Edge[color=SentimentPositive](1)(2)
	\Edge[color=SentimentMissing](1)(3)
\end{tikzpicture}
\\$22 \hspace{4pt} | \hspace{4pt} \textcolor{TwitterBlue}{-}$
}
&\makecell{\begin{tikzpicture}
	\Vertex[x=0.17, y=0.49]{0}
	\Vertex[x=-0.10, y=0.19]{1}
	\Vertex[x=-0.50, y=0.28]{2}
	\Vertex[x=0.02, y=-0.20]{3}
	\Edge[color=SentimentMissing](0)(1)
	\Edge[color=SentimentMissing](1)(2)
	\Edge[color=SentimentMissing](1)(3)
\end{tikzpicture}
\\$22 \hspace{4pt} | \hspace{4pt} \textcolor{TwitterBlue}{-}$
}
&&\makecell{\begin{tikzpicture}
	\Vertex[x=0.35, y=0.50]{0}
	\Vertex[x=0.09, y=0.18]{1}
	\Vertex[x=-0.17, y=-0.13]{2}
	\Vertex[x=-0.43, y=-0.45]{3}
	\Edge[color=SentimentPositive](0)(1)
	\Edge[color=SentimentNegative](1)(2)
	\Edge[color=SentimentNegative](2)(3)
\end{tikzpicture}
\\$21 \hspace{4pt} | \hspace{4pt} \textcolor{TwitterBlue}{5}$
}
&\makecell{\begin{tikzpicture}
	\Vertex[x=0.35, y=0.50]{0}
	\Vertex[x=0.09, y=0.18]{1}
	\Vertex[x=-0.17, y=-0.13]{2}
	\Vertex[x=-0.43, y=-0.45]{3}
	\Edge[color=SentimentPositive](0)(1)
	\Edge[color=SentimentPositive](1)(2)
	\Edge[color=SentimentPositive](2)(3)
\end{tikzpicture}
\\$21 \hspace{4pt} | \hspace{4pt} \textcolor{TwitterBlue}{4}$
}
&\makecell{\begin{tikzpicture}
	\Vertex[x=0.17, y=0.49]{0}
	\Vertex[x=-0.10, y=0.19]{1}
	\Vertex[x=-0.50, y=0.28]{2}
	\Vertex[x=0.02, y=-0.20]{3}
	\Edge[color=SentimentPositive](0)(1)
	\Edge[color=SentimentNegative](1)(2)
	\Edge[color=SentimentMissing](1)(3)
\end{tikzpicture}
\\$21 \hspace{4pt} | \hspace{4pt} \textcolor{TwitterBlue}{-}$
}
\\[0.9cm]
$|V| = 5$&\makecell{\begin{tikzpicture}
	\Vertex[x=0.31, y=0.26]{0}
	\Vertex[x=-0.04, y=0.14]{1}
	\Vertex[x=-0.16, y=0.50]{2}
	\Vertex[x=-0.40, y=0.03]{3}
	\Vertex[x=0.08, y=-0.21]{4}
	\Edge[color=SentimentPositive](0)(1)
	\Edge[color=SentimentPositive](1)(2)
	\Edge[color=SentimentNegative](1)(3)
	\Edge[color=SentimentNegative](1)(4)
\end{tikzpicture}
\\$26 \hspace{4pt} | \hspace{4pt} \textcolor{TwitterBlue}{4}$
}
&\makecell{\begin{tikzpicture}
	\Vertex[x=0.31, y=0.26]{0}
	\Vertex[x=-0.04, y=0.14]{1}
	\Vertex[x=-0.16, y=0.50]{2}
	\Vertex[x=-0.40, y=0.03]{3}
	\Vertex[x=0.08, y=-0.21]{4}
	\Edge[color=SentimentPositive](0)(1)
	\Edge[color=SentimentPositive](1)(2)
	\Edge[color=SentimentPositive](1)(3)
	\Edge[color=SentimentNegative](1)(4)
\end{tikzpicture}
\\$25 \hspace{4pt} | \hspace{4pt} \textcolor{TwitterBlue}{4}$
}
&\makecell{\begin{tikzpicture}
	\Vertex[x=0.31, y=0.26]{0}
	\Vertex[x=-0.04, y=0.14]{1}
	\Vertex[x=-0.16, y=0.50]{2}
	\Vertex[x=-0.40, y=0.03]{3}
	\Vertex[x=0.08, y=-0.21]{4}
	\Edge[color=SentimentPositive](0)(1)
	\Edge[color=SentimentPositive](1)(2)
	\Edge[color=SentimentPositive](1)(3)
	\Edge[color=SentimentPositive](1)(4)
\end{tikzpicture}
\\$24 \hspace{4pt} | \hspace{4pt} \textcolor{TwitterBlue}{4}$
}
&\makecell{\begin{tikzpicture}
	\Vertex[x=0.31, y=0.26]{0}
	\Vertex[x=-0.04, y=0.14]{1}
	\Vertex[x=-0.16, y=0.50]{2}
	\Vertex[x=-0.40, y=0.03]{3}
	\Vertex[x=0.08, y=-0.21]{4}
	\Edge[color=SentimentPositive](0)(1)
	\Edge[color=SentimentNegative](1)(2)
	\Edge[color=SentimentNegative](1)(3)
	\Edge[color=SentimentNegative](1)(4)
\end{tikzpicture}
\\$22 \hspace{4pt} | \hspace{4pt} \textcolor{TwitterBlue}{4}$
}
&\makecell{\begin{tikzpicture}
	\Vertex[x=0.31, y=0.26]{0}
	\Vertex[x=-0.04, y=0.14]{1}
	\Vertex[x=-0.16, y=0.50]{2}
	\Vertex[x=-0.40, y=0.03]{3}
	\Vertex[x=0.08, y=-0.21]{4}
	\Edge[color=SentimentNeutral](0)(1)
	\Edge[color=SentimentPositive](1)(2)
	\Edge[color=SentimentPositive](1)(3)
	\Edge[color=SentimentNegative](1)(4)
\end{tikzpicture}
\\$21 \hspace{4pt} | \hspace{4pt} \textcolor{TwitterBlue}{4}$
}
&\makecell{\begin{tikzpicture}
	\Vertex[x=0.31, y=0.26]{0}
	\Vertex[x=-0.04, y=0.14]{1}
	\Vertex[x=-0.16, y=0.50]{2}
	\Vertex[x=-0.40, y=0.03]{3}
	\Vertex[x=0.08, y=-0.21]{4}
	\Edge[color=SentimentNegative](0)(1)
	\Edge[color=SentimentNegative](1)(2)
	\Edge[color=SentimentNegative](1)(3)
	\Edge[color=SentimentNegative](1)(4)
\end{tikzpicture}
\\$21 \hspace{4pt} | \hspace{4pt} \textcolor{TwitterBlue}{4}$
}
\\[0.9cm]
$|V| = 6$&\makecell{\begin{tikzpicture}
	\Vertex[x=0.49, y=0.32]{0}
	\Vertex[x=0.16, y=0.07]{1}
	\Vertex[x=0.02, y=0.46]{2}
	\Vertex[x=-0.25, y=0.06]{3}
	\Vertex[x=0.04, y=-0.33]{4}
	\Vertex[x=0.50, y=-0.16]{5}
	\Edge[color=SentimentPositive](0)(1)
	\Edge[color=SentimentPositive](1)(2)
	\Edge[color=SentimentPositive](1)(3)
	\Edge[color=SentimentPositive](1)(4)
	\Edge[color=SentimentNegative](1)(5)
\end{tikzpicture}
\\$24 \hspace{4pt} | \hspace{4pt} \textcolor{TwitterBlue}{4}$
}
&\makecell{\begin{tikzpicture}
	\Vertex[x=0.49, y=0.32]{0}
	\Vertex[x=0.16, y=0.07]{1}
	\Vertex[x=0.02, y=0.46]{2}
	\Vertex[x=-0.25, y=0.06]{3}
	\Vertex[x=0.04, y=-0.33]{4}
	\Vertex[x=0.50, y=-0.16]{5}
	\Edge[color=SentimentPositive](0)(1)
	\Edge[color=SentimentPositive](1)(2)
	\Edge[color=SentimentPositive](1)(3)
	\Edge[color=SentimentNegative](1)(4)
	\Edge[color=SentimentNegative](1)(5)
\end{tikzpicture}
\\$23 \hspace{4pt} | \hspace{4pt} \textcolor{TwitterBlue}{4}$
}
&\makecell{\begin{tikzpicture}
	\Vertex[x=0.49, y=0.32]{0}
	\Vertex[x=0.16, y=0.07]{1}
	\Vertex[x=0.02, y=0.46]{2}
	\Vertex[x=-0.25, y=0.06]{3}
	\Vertex[x=0.04, y=-0.33]{4}
	\Vertex[x=0.50, y=-0.16]{5}
	\Edge[color=SentimentPositive](0)(1)
	\Edge[color=SentimentPositive](1)(2)
	\Edge[color=SentimentNegative](1)(3)
	\Edge[color=SentimentNegative](1)(4)
	\Edge[color=SentimentNegative](1)(5)
\end{tikzpicture}
\\$22 \hspace{4pt} | \hspace{4pt} \textcolor{TwitterBlue}{4}$
}
&\makecell{\begin{tikzpicture}
	\Vertex[x=0.49, y=0.32]{0}
	\Vertex[x=0.16, y=0.07]{1}
	\Vertex[x=0.02, y=0.46]{2}
	\Vertex[x=-0.25, y=0.06]{3}
	\Vertex[x=0.04, y=-0.33]{4}
	\Vertex[x=0.50, y=-0.16]{5}
	\Edge[color=SentimentPositive](0)(1)
	\Edge[color=SentimentPositive](1)(2)
	\Edge[color=SentimentPositive](1)(3)
	\Edge[color=SentimentPositive](1)(4)
	\Edge[color=SentimentPositive](1)(5)
\end{tikzpicture}
\\$21 \hspace{4pt} | \hspace{4pt} \textcolor{TwitterBlue}{4}$
}
&\makecell{\begin{tikzpicture}
	\Vertex[x=0.49, y=0.32]{0}
	\Vertex[x=0.16, y=0.07]{1}
	\Vertex[x=0.02, y=0.46]{2}
	\Vertex[x=-0.25, y=0.06]{3}
	\Vertex[x=0.04, y=-0.33]{4}
	\Vertex[x=0.50, y=-0.16]{5}
	\Edge[color=SentimentPositive](0)(1)
	\Edge[color=SentimentNegative](1)(2)
	\Edge[color=SentimentNegative](1)(3)
	\Edge[color=SentimentNegative](1)(4)
	\Edge[color=SentimentNegative](1)(5)
\end{tikzpicture}
\\$21 \hspace{4pt} | \hspace{4pt} \textcolor{TwitterBlue}{4}$
}
\\[0.9cm]
$|V| = 7$&\makecell{\begin{tikzpicture}
	\Vertex[x=-0.50, y=-0.14]{0}
	\Vertex[x=-0.21, y=-0.13]{1}
	\Vertex[x=-0.34, y=-0.39]{2}
	\Vertex[x=-0.37, y=0.11]{3}
	\Vertex[x=-0.05, y=-0.37]{4}
	\Vertex[x=-0.08, y=0.13]{5}
	\Vertex[x=0.08, y=-0.11]{6}
	\Edge[color=SentimentPositive](0)(1)
	\Edge[color=SentimentPositive](1)(2)
	\Edge[color=SentimentPositive](1)(3)
	\Edge[color=SentimentPositive](1)(4)
	\Edge[color=SentimentPositive](1)(5)
	\Edge[color=SentimentNegative](1)(6)
\end{tikzpicture}
\\$21 \hspace{4pt} | \hspace{4pt} \textcolor{TwitterBlue}{4}$
}
&\makecell{\begin{tikzpicture}
	\Vertex[x=-0.50, y=-0.14]{0}
	\Vertex[x=-0.21, y=-0.13]{1}
	\Vertex[x=-0.34, y=-0.39]{2}
	\Vertex[x=-0.37, y=0.11]{3}
	\Vertex[x=-0.05, y=-0.37]{4}
	\Vertex[x=-0.08, y=0.13]{5}
	\Vertex[x=0.08, y=-0.11]{6}
	\Edge[color=SentimentPositive](0)(1)
	\Edge[color=SentimentPositive](1)(2)
	\Edge[color=SentimentNegative](1)(3)
	\Edge[color=SentimentNegative](1)(4)
	\Edge[color=SentimentNegative](1)(5)
	\Edge[color=SentimentNegative](1)(6)
\end{tikzpicture}
\\$21 \hspace{4pt} | \hspace{4pt} \textcolor{TwitterBlue}{4}$
}
\end{tabular}

    \end{adjustwidth}
    \caption{The subgraphs identified in Figure \ref{fig:subgraph_hierarchy}, enriched with sentiment labels. Green edges represent \textcolor{SentimentPositive}{positive} sentiments, red for \textcolor{SentimentNegative}{negative}, gray for \textcolor{SentimentNeutral}{neutral}, and blue for \textcolor{SentimentMissing}{missing} sentiments. The subgraphs highlight how sentiment influences the structure of user interactions. Star-like structural patterns dominate, reflecting either central user receiving reactions, or a central user reacting. If a subgraph contains a \textcolor{SentimentMissing}{missing} sentiment edge, the Twitter support is undefined.}
    \label{fig:sentiment_subgraphs}
\end{figure}

Building on the undirected sentiment subgraphs, Figure \ref{fig:sentiment_directed} introduces directionality as an additional dimension. However, no new unexpected patterns emerge with regard to the combination of sentiment and direction. 

\begin{figure}[h]
    \centering
    \begin{adjustwidth}{-\textwidth}{-\textwidth}
        \centering
        \setlength{\tabcolsep}{9pt}
\begin{tabular}{cccccccccc}
$|V| = 2$&\makecell{\begin{tikzpicture}
	\Vertex[x=0.50, y=-0.00]{0}
	\Vertex[x=-0.28, y=0.00]{1}
	\Edge[color=SentimentPositive,Direct](0)(1)
\end{tikzpicture}
\\$30 \hspace{4pt} | \hspace{4pt} \textcolor{TwitterBlue}{6}$
}
&\makecell{\begin{tikzpicture}
	\Vertex[x=0.50, y=-0.00]{0}
	\Vertex[x=-0.28, y=0.00]{1}
	\Edge[color=SentimentNegative,Direct](0)(1)
\end{tikzpicture}
\\$30 \hspace{4pt} | \hspace{4pt} \textcolor{TwitterBlue}{5}$
}
&\makecell{\begin{tikzpicture}
	\Vertex[x=0.50, y=-0.00]{0}
	\Vertex[x=-0.28, y=0.00]{1}
	\Edge[color=SentimentMissing,Direct](0)(1)
\end{tikzpicture}
\\$27 \hspace{4pt} | \hspace{4pt} \textcolor{TwitterBlue}{-}$
}
&\makecell{\begin{tikzpicture}
	\Vertex[x=0.50, y=-0.00]{0}
	\Vertex[x=-0.28, y=0.00]{1}
	\Edge[color=SentimentNeutral,Direct](0)(1)
\end{tikzpicture}
\\$22 \hspace{4pt} | \hspace{4pt} \textcolor{TwitterBlue}{5}$
}
\\[0.9cm]
$|V| = 3$&\makecell{\begin{tikzpicture}
	\Vertex[x=0.06, y=0.50]{0}
	\Vertex[x=-0.06, y=0.14]{1}
	\Vertex[x=-0.18, y=-0.23]{2}
	\Edge[color=SentimentNegative,Direct](0)(1)
	\Edge[color=SentimentPositive,Direct](2)(1)
\end{tikzpicture}
\\$26 \hspace{4pt} | \hspace{4pt} \textcolor{TwitterBlue}{4}$
}
&\makecell{\begin{tikzpicture}
	\Vertex[x=0.06, y=0.50]{0}
	\Vertex[x=-0.06, y=0.14]{1}
	\Vertex[x=-0.18, y=-0.23]{2}
	\Edge[color=SentimentPositive,Direct](0)(1)
	\Edge[color=SentimentPositive,Direct](2)(1)
\end{tikzpicture}
\\$26 \hspace{4pt} | \hspace{4pt} \textcolor{TwitterBlue}{4}$
}
&\makecell{\begin{tikzpicture}
	\Vertex[x=0.18, y=0.02]{0}
	\Vertex[x=0.50, y=0.38]{1}
	\Vertex[x=-0.14, y=-0.33]{2}
	\Edge[color=SentimentPositive,Direct](0)(1)
	\Edge[color=SentimentPositive,Direct](0)(2)
\end{tikzpicture}
\\$24 \hspace{4pt} | \hspace{4pt} \textcolor{TwitterBlue}{4}$
}
&\makecell{\begin{tikzpicture}
	\Vertex[x=0.06, y=0.50]{0}
	\Vertex[x=-0.06, y=0.14]{1}
	\Vertex[x=-0.18, y=-0.23]{2}
	\Edge[color=SentimentMissing,Direct](0)(1)
	\Edge[color=SentimentMissing,Direct](2)(1)
\end{tikzpicture}
\\$24 \hspace{4pt} | \hspace{4pt} \textcolor{TwitterBlue}{-}$
}
&\makecell{\begin{tikzpicture}
	\Vertex[x=0.18, y=0.02]{0}
	\Vertex[x=0.50, y=0.38]{1}
	\Vertex[x=-0.14, y=-0.33]{2}
	\Edge[color=SentimentNegative,Direct](0)(1)
	\Edge[color=SentimentNegative,Direct](0)(2)
\end{tikzpicture}
\\$23 \hspace{4pt} | \hspace{4pt} \textcolor{TwitterBlue}{5}$
}
&\makecell{\begin{tikzpicture}
	\Vertex[x=0.18, y=0.02]{0}
	\Vertex[x=0.50, y=0.38]{1}
	\Vertex[x=-0.14, y=-0.33]{2}
	\Edge[color=SentimentNegative,Direct](0)(1)
	\Edge[color=SentimentPositive,Direct](0)(2)
\end{tikzpicture}
\\$23 \hspace{4pt} | \hspace{4pt} \textcolor{TwitterBlue}{5}$
}
&\makecell{\begin{tikzpicture}
	\Vertex[x=0.06, y=0.50]{0}
	\Vertex[x=-0.06, y=0.14]{1}
	\Vertex[x=-0.18, y=-0.23]{2}
	\Edge[color=SentimentNegative,Direct](0)(1)
	\Edge[color=SentimentNegative,Direct](2)(1)
\end{tikzpicture}
\\$23 \hspace{4pt} | \hspace{4pt} \textcolor{TwitterBlue}{5}$
}
&\makecell{\begin{tikzpicture}
	\Vertex[x=0.06, y=0.50]{0}
	\Vertex[x=-0.06, y=0.14]{1}
	\Vertex[x=-0.18, y=-0.23]{2}
	\Edge[color=SentimentNegative,Direct](0)(1)
	\Edge[color=SentimentMissing,Direct](2)(1)
\end{tikzpicture}
\\$22 \hspace{4pt} | \hspace{4pt} \textcolor{TwitterBlue}{-}$
}
&\makecell{\begin{tikzpicture}
	\Vertex[x=0.06, y=0.50]{0}
	\Vertex[x=-0.06, y=0.14]{1}
	\Vertex[x=-0.18, y=-0.23]{2}
	\Edge[color=SentimentPositive,Direct](0)(1)
	\Edge[color=SentimentMissing,Direct](2)(1)
\end{tikzpicture}
\\$22 \hspace{4pt} | \hspace{4pt} \textcolor{TwitterBlue}{-}$
}
&&\makecell{\begin{tikzpicture}
	\Vertex[x=0.06, y=0.50]{0}
	\Vertex[x=-0.06, y=0.14]{1}
	\Vertex[x=-0.18, y=-0.23]{2}
	\Edge[color=SentimentNegative,Direct](0)(1)
	\Edge[color=SentimentNeutral,Direct](2)(1)
\end{tikzpicture}
\\$21 \hspace{4pt} | \hspace{4pt} \textcolor{TwitterBlue}{4}$
}
&\makecell{\begin{tikzpicture}
	\Vertex[x=0.06, y=0.50]{0}
	\Vertex[x=-0.06, y=0.14]{1}
	\Vertex[x=-0.18, y=-0.23]{2}
	\Edge[color=SentimentPositive,Direct](0)(1)
	\Edge[color=SentimentNeutral,Direct](2)(1)
\end{tikzpicture}
\\$20 \hspace{4pt} | \hspace{4pt} \textcolor{TwitterBlue}{4}$
}
&\makecell{\begin{tikzpicture}
	\Vertex[x=0.18, y=0.02]{0}
	\Vertex[x=0.50, y=0.38]{1}
	\Vertex[x=-0.14, y=-0.33]{2}
	\Edge[color=SentimentMissing,Direct](0)(1)
	\Edge[color=SentimentMissing,Direct](0)(2)
\end{tikzpicture}
\\$20 \hspace{4pt} | \hspace{4pt} \textcolor{TwitterBlue}{-}$
}
&\makecell{\begin{tikzpicture}
	\Vertex[x=0.06, y=0.50]{0}
	\Vertex[x=-0.06, y=0.14]{1}
	\Vertex[x=-0.18, y=-0.23]{2}
	\Edge[color=SentimentMissing,Direct](0)(1)
	\Edge[color=SentimentNeutral,Direct](2)(1)
\end{tikzpicture}
\\$20 \hspace{4pt} | \hspace{4pt} \textcolor{TwitterBlue}{-}$
}
&\makecell{\begin{tikzpicture}
	\Vertex[x=0.18, y=0.02]{0}
	\Vertex[x=0.50, y=0.38]{1}
	\Vertex[x=-0.14, y=-0.33]{2}
	\Edge[color=SentimentPositive,Direct](0)(1)
	\Edge[color=SentimentNeutral,Direct](0)(2)
\end{tikzpicture}
\\$19 \hspace{4pt} | \hspace{4pt} \textcolor{TwitterBlue}{5}$
}
\\[0.9cm]
$|V| = 4$&\makecell{\begin{tikzpicture}
	\Vertex[x=0.17, y=0.49]{0}
	\Vertex[x=-0.10, y=0.19]{1}
	\Vertex[x=-0.50, y=0.28]{2}
	\Vertex[x=0.02, y=-0.20]{3}
	\Edge[color=SentimentNegative,Direct](0)(1)
	\Edge[color=SentimentPositive,Direct](2)(1)
	\Edge[color=SentimentPositive,Direct](3)(1)
\end{tikzpicture}
\\$23 \hspace{4pt} | \hspace{4pt} \textcolor{TwitterBlue}{4}$
}
&\makecell{\begin{tikzpicture}
	\Vertex[x=0.19, y=-0.10]{0}
	\Vertex[x=0.49, y=0.17]{1}
	\Vertex[x=-0.20, y=0.02]{2}
	\Vertex[x=0.28, y=-0.50]{3}
	\Edge[color=SentimentNegative,Direct](0)(1)
	\Edge[color=SentimentPositive,Direct](0)(2)
	\Edge[color=SentimentPositive,Direct](0)(3)
\end{tikzpicture}
\\$22 \hspace{4pt} | \hspace{4pt} \textcolor{TwitterBlue}{4}$
}
&\makecell{\begin{tikzpicture}
	\Vertex[x=0.19, y=-0.10]{0}
	\Vertex[x=0.49, y=0.17]{1}
	\Vertex[x=-0.20, y=0.02]{2}
	\Vertex[x=0.28, y=-0.50]{3}
	\Edge[color=SentimentNegative,Direct](0)(1)
	\Edge[color=SentimentNegative,Direct](0)(2)
	\Edge[color=SentimentPositive,Direct](0)(3)
\end{tikzpicture}
\\$21 \hspace{4pt} | \hspace{4pt} \textcolor{TwitterBlue}{5}$
}
&\makecell{\begin{tikzpicture}
	\Vertex[x=0.17, y=0.49]{0}
	\Vertex[x=-0.10, y=0.19]{1}
	\Vertex[x=-0.50, y=0.28]{2}
	\Vertex[x=0.02, y=-0.20]{3}
	\Edge[color=SentimentNegative,Direct](0)(1)
	\Edge[color=SentimentNegative,Direct](2)(1)
	\Edge[color=SentimentPositive,Direct](3)(1)
\end{tikzpicture}
\\$21 \hspace{4pt} | \hspace{4pt} \textcolor{TwitterBlue}{4}$
}
&\makecell{\begin{tikzpicture}
	\Vertex[x=0.19, y=-0.10]{0}
	\Vertex[x=0.49, y=0.17]{1}
	\Vertex[x=-0.20, y=0.02]{2}
	\Vertex[x=0.28, y=-0.50]{3}
	\Edge[color=SentimentPositive,Direct](0)(1)
	\Edge[color=SentimentPositive,Direct](0)(2)
	\Edge[color=SentimentPositive,Direct](0)(3)
\end{tikzpicture}
\\$21 \hspace{4pt} | \hspace{4pt} \textcolor{TwitterBlue}{4}$
}
&\makecell{\begin{tikzpicture}
	\Vertex[x=0.04, y=0.05]{0}
	\Vertex[x=0.10, y=-0.23]{1}
	\Vertex[x=-0.01, y=0.32]{2}
	\Vertex[x=0.15, y=-0.50]{3}
	\Edge[color=SentimentPositive,Direct](0)(1)
	\Edge[color=SentimentPositive,Direct](0)(2)
	\Edge[color=SentimentPositive,Direct](3)(1)
\end{tikzpicture}
\\$21 \hspace{4pt} | \hspace{4pt} \textcolor{TwitterBlue}{4}$
}
&\makecell{\begin{tikzpicture}
	\Vertex[x=0.17, y=0.49]{0}
	\Vertex[x=-0.10, y=0.19]{1}
	\Vertex[x=-0.50, y=0.28]{2}
	\Vertex[x=0.02, y=-0.20]{3}
	\Edge[color=SentimentNegative,Direct](0)(1)
	\Edge[color=SentimentPositive,Direct](2)(1)
	\Edge[color=SentimentMissing,Direct](3)(1)
\end{tikzpicture}
\\$21 \hspace{4pt} | \hspace{4pt} \textcolor{TwitterBlue}{-}$
}
&\makecell{\begin{tikzpicture}
	\Vertex[x=0.17, y=0.49]{0}
	\Vertex[x=-0.10, y=0.19]{1}
	\Vertex[x=-0.50, y=0.28]{2}
	\Vertex[x=0.02, y=-0.20]{3}
	\Edge[color=SentimentPositive,Direct](0)(1)
	\Edge[color=SentimentPositive,Direct](2)(1)
	\Edge[color=SentimentMissing,Direct](3)(1)
\end{tikzpicture}
\\$21 \hspace{4pt} | \hspace{4pt} \textcolor{TwitterBlue}{-}$
}
&\makecell{\begin{tikzpicture}
	\Vertex[x=0.04, y=0.05]{0}
	\Vertex[x=0.10, y=-0.23]{1}
	\Vertex[x=-0.01, y=0.32]{2}
	\Vertex[x=0.15, y=-0.50]{3}
	\Edge[color=SentimentNegative,Direct](0)(1)
	\Edge[color=SentimentPositive,Direct](0)(2)
	\Edge[color=SentimentPositive,Direct](3)(1)
\end{tikzpicture}
\\$20 \hspace{4pt} | \hspace{4pt} \textcolor{TwitterBlue}{4}$
}
&&\makecell{\begin{tikzpicture}
	\Vertex[x=0.17, y=0.49]{0}
	\Vertex[x=-0.10, y=0.19]{1}
	\Vertex[x=-0.50, y=0.28]{2}
	\Vertex[x=0.02, y=-0.20]{3}
	\Edge[color=SentimentPositive,Direct](0)(1)
	\Edge[color=SentimentPositive,Direct](2)(1)
	\Edge[color=SentimentPositive,Direct](3)(1)
\end{tikzpicture}
\\$20 \hspace{4pt} | \hspace{4pt} \textcolor{TwitterBlue}{4}$
}
&\makecell{\begin{tikzpicture}
	\Vertex[x=0.17, y=0.49]{0}
	\Vertex[x=-0.10, y=0.19]{1}
	\Vertex[x=-0.50, y=0.28]{2}
	\Vertex[x=0.02, y=-0.20]{3}
	\Edge[color=SentimentPositive,Direct](0)(1)
	\Edge[color=SentimentMissing,Direct](2)(1)
	\Edge[color=SentimentMissing,Direct](3)(1)
\end{tikzpicture}
\\$20 \hspace{4pt} | \hspace{4pt} \textcolor{TwitterBlue}{-}$
}
&\makecell{\begin{tikzpicture}
	\Vertex[x=0.17, y=0.49]{0}
	\Vertex[x=-0.10, y=0.19]{1}
	\Vertex[x=-0.50, y=0.28]{2}
	\Vertex[x=0.02, y=-0.20]{3}
	\Edge[color=SentimentMissing,Direct](0)(1)
	\Edge[color=SentimentMissing,Direct](2)(1)
	\Edge[color=SentimentMissing,Direct](3)(1)
\end{tikzpicture}
\\$20 \hspace{4pt} | \hspace{4pt} \textcolor{TwitterBlue}{-}$
}
&\makecell{\begin{tikzpicture}
	\Vertex[x=0.04, y=0.05]{0}
	\Vertex[x=0.10, y=-0.23]{1}
	\Vertex[x=-0.01, y=0.32]{2}
	\Vertex[x=0.15, y=-0.50]{3}
	\Edge[color=SentimentNegative,Direct](0)(1)
	\Edge[color=SentimentNegative,Direct](0)(2)
	\Edge[color=SentimentPositive,Direct](3)(1)
\end{tikzpicture}
\\$19 \hspace{4pt} | \hspace{4pt} \textcolor{TwitterBlue}{4}$
}
&\makecell{\begin{tikzpicture}
	\Vertex[x=-0.23, y=0.20]{0}
	\Vertex[x=-0.26, y=0.50]{1}
	\Vertex[x=-0.20, y=-0.10]{2}
	\Vertex[x=-0.17, y=-0.41]{3}
	\Edge[color=SentimentNegative,Direct](0)(1)
	\Edge[color=SentimentPositive,Direct](0)(2)
	\Edge[color=SentimentPositive,Direct](3)(2)
\end{tikzpicture}
\\$19 \hspace{4pt} | \hspace{4pt} \textcolor{TwitterBlue}{4}$
}
&\makecell{\begin{tikzpicture}
	\Vertex[x=0.17, y=0.49]{0}
	\Vertex[x=-0.10, y=0.19]{1}
	\Vertex[x=-0.50, y=0.28]{2}
	\Vertex[x=0.02, y=-0.20]{3}
	\Edge[color=SentimentNegative,Direct](0)(1)
	\Edge[color=SentimentPositive,Direct](2)(1)
	\Edge[color=SentimentNeutral,Direct](3)(1)
\end{tikzpicture}
\\$19 \hspace{4pt} | \hspace{4pt} \textcolor{TwitterBlue}{4}$
}
&\makecell{\begin{tikzpicture}
	\Vertex[x=0.35, y=0.50]{0}
	\Vertex[x=0.09, y=0.18]{1}
	\Vertex[x=-0.17, y=-0.13]{2}
	\Vertex[x=-0.43, y=-0.45]{3}
	\Edge[color=SentimentNegative,Direct](0)(1)
	\Edge[color=SentimentPositive,Direct](2)(1)
	\Edge[color=SentimentPositive,Direct](2)(3)
\end{tikzpicture}
\\$19 \hspace{4pt} | \hspace{4pt} \textcolor{TwitterBlue}{4}$
}
&\makecell{\begin{tikzpicture}
	\Vertex[x=0.17, y=0.49]{0}
	\Vertex[x=-0.10, y=0.19]{1}
	\Vertex[x=-0.50, y=0.28]{2}
	\Vertex[x=0.02, y=-0.20]{3}
	\Edge[color=SentimentNegative,Direct](0)(1)
	\Edge[color=SentimentNegative,Direct](2)(1)
	\Edge[color=SentimentMissing,Direct](3)(1)
\end{tikzpicture}
\\$19 \hspace{4pt} | \hspace{4pt} \textcolor{TwitterBlue}{-}$
}
&\makecell{\begin{tikzpicture}
	\Vertex[x=0.17, y=0.49]{0}
	\Vertex[x=-0.10, y=0.19]{1}
	\Vertex[x=-0.50, y=0.28]{2}
	\Vertex[x=0.02, y=-0.20]{3}
	\Edge[color=SentimentNegative,Direct](0)(1)
	\Edge[color=SentimentMissing,Direct](2)(1)
	\Edge[color=SentimentMissing,Direct](3)(1)
\end{tikzpicture}
\\$19 \hspace{4pt} | \hspace{4pt} \textcolor{TwitterBlue}{-}$
}
\\[0.9cm]
\end{tabular}

    \end{adjustwidth}
    \caption{Mined sentiment subgraphs, extending Figure \ref{fig:sentiment_subgraphs} with direction, showing all subgraphs with $support \geq 19$. The full figure can be found in Appendix \ref{lab:fsm_appendix} Figure \ref{fig:sentiment_directed_full}.}
    \label{fig:sentiment_directed}
\end{figure}



\clearpage
\section{Graph embeddings}

To compare the collected graphs with clustering, we employ various graph embedding techniques to find a representation that facilitates vectorized comparisons. To this end, we employ two approaches: we apply the graph representation learning algorithm Graph2Vec, and we use the identified subgraphs in a Bag-Of-Subgraphs approach, leading to the results seen in Figure \ref{fig:emb_scatter}. 

\begin{figure}[!htbp]
    \centering
    \begin{adjustwidth}{-\textwidth}{-\textwidth}
        \centering
        %% Creator: Matplotlib, PGF backend
%%
%% To include the figure in your LaTeX document, write
%%   \input{<filename>.pgf}
%%
%% Make sure the required packages are loaded in your preamble
%%   \usepackage{pgf}
%%
%% Also ensure that all the required font packages are loaded; for instance,
%% the lmodern package is sometimes necessary when using math font.
%%   \usepackage{lmodern}
%%
%% Figures using additional raster images can only be included by \input if
%% they are in the same directory as the main LaTeX file. For loading figures
%% from other directories you can use the `import` package
%%   \usepackage{import}
%%
%% and then include the figures with
%%   \import{<path to file>}{<filename>.pgf}
%%
%% Matplotlib used the following preamble
%%   \def\mathdefault#1{#1}
%%   \everymath=\expandafter{\the\everymath\displaystyle}
%%   
%%   \ifdefined\pdftexversion\else  % non-pdftex case.
%%     \usepackage{fontspec}
%%     \setmainfont{DejaVuSerif.ttf}[Path=\detokenize{/home/mahf/.local/lib/python3.10/site-packages/matplotlib/mpl-data/fonts/ttf/}]
%%     \setsansfont{DejaVuSans.ttf}[Path=\detokenize{/home/mahf/.local/lib/python3.10/site-packages/matplotlib/mpl-data/fonts/ttf/}]
%%     \setmonofont{DejaVuSansMono.ttf}[Path=\detokenize{/home/mahf/.local/lib/python3.10/site-packages/matplotlib/mpl-data/fonts/ttf/}]
%%   \fi
%%   \makeatletter\@ifpackageloaded{underscore}{}{\usepackage[strings]{underscore}}\makeatother
%%
\begingroup%
\makeatletter%
\begin{pgfpicture}%
\pgfpathrectangle{\pgfpointorigin}{\pgfqpoint{6.981240in}{2.985276in}}%
\pgfusepath{use as bounding box, clip}%
\begin{pgfscope}%
\pgfsetbuttcap%
\pgfsetmiterjoin%
\definecolor{currentfill}{rgb}{1.000000,1.000000,1.000000}%
\pgfsetfillcolor{currentfill}%
\pgfsetlinewidth{0.000000pt}%
\definecolor{currentstroke}{rgb}{1.000000,1.000000,1.000000}%
\pgfsetstrokecolor{currentstroke}%
\pgfsetdash{}{0pt}%
\pgfpathmoveto{\pgfqpoint{0.000000in}{0.000000in}}%
\pgfpathlineto{\pgfqpoint{6.981240in}{0.000000in}}%
\pgfpathlineto{\pgfqpoint{6.981240in}{2.985276in}}%
\pgfpathlineto{\pgfqpoint{0.000000in}{2.985276in}}%
\pgfpathlineto{\pgfqpoint{0.000000in}{0.000000in}}%
\pgfpathclose%
\pgfusepath{fill}%
\end{pgfscope}%
\begin{pgfscope}%
\pgfsetbuttcap%
\pgfsetmiterjoin%
\definecolor{currentfill}{rgb}{1.000000,1.000000,1.000000}%
\pgfsetfillcolor{currentfill}%
\pgfsetlinewidth{0.000000pt}%
\definecolor{currentstroke}{rgb}{0.000000,0.000000,0.000000}%
\pgfsetstrokecolor{currentstroke}%
\pgfsetstrokeopacity{0.000000}%
\pgfsetdash{}{0pt}%
\pgfpathmoveto{\pgfqpoint{0.669309in}{0.529443in}}%
\pgfpathlineto{\pgfqpoint{2.258429in}{0.529443in}}%
\pgfpathlineto{\pgfqpoint{2.258429in}{2.275433in}}%
\pgfpathlineto{\pgfqpoint{0.669309in}{2.275433in}}%
\pgfpathlineto{\pgfqpoint{0.669309in}{0.529443in}}%
\pgfpathclose%
\pgfusepath{fill}%
\end{pgfscope}%
\begin{pgfscope}%
\pgfpathrectangle{\pgfqpoint{0.669309in}{0.529443in}}{\pgfqpoint{1.589120in}{1.745990in}}%
\pgfusepath{clip}%
\pgfsetroundcap%
\pgfsetroundjoin%
\pgfsetlinewidth{1.003750pt}%
\definecolor{currentstroke}{rgb}{0.800000,0.800000,0.800000}%
\pgfsetstrokecolor{currentstroke}%
\pgfsetdash{}{0pt}%
\pgfpathmoveto{\pgfqpoint{0.879252in}{0.529443in}}%
\pgfpathlineto{\pgfqpoint{0.879252in}{2.275433in}}%
\pgfusepath{stroke}%
\end{pgfscope}%
\begin{pgfscope}%
\definecolor{textcolor}{rgb}{0.150000,0.150000,0.150000}%
\pgfsetstrokecolor{textcolor}%
\pgfsetfillcolor{textcolor}%
\pgftext[x=0.879252in,y=0.397499in,,top]{\color{textcolor}{\rmfamily\fontsize{8.000000}{9.600000}\selectfont\catcode`\^=\active\def^{\ifmmode\sp\else\^{}\fi}\catcode`\%=\active\def%{\%}5.0}}%
\end{pgfscope}%
\begin{pgfscope}%
\pgfpathrectangle{\pgfqpoint{0.669309in}{0.529443in}}{\pgfqpoint{1.589120in}{1.745990in}}%
\pgfusepath{clip}%
\pgfsetroundcap%
\pgfsetroundjoin%
\pgfsetlinewidth{1.003750pt}%
\definecolor{currentstroke}{rgb}{0.800000,0.800000,0.800000}%
\pgfsetstrokecolor{currentstroke}%
\pgfsetdash{}{0pt}%
\pgfpathmoveto{\pgfqpoint{1.285874in}{0.529443in}}%
\pgfpathlineto{\pgfqpoint{1.285874in}{2.275433in}}%
\pgfusepath{stroke}%
\end{pgfscope}%
\begin{pgfscope}%
\definecolor{textcolor}{rgb}{0.150000,0.150000,0.150000}%
\pgfsetstrokecolor{textcolor}%
\pgfsetfillcolor{textcolor}%
\pgftext[x=1.285874in,y=0.397499in,,top]{\color{textcolor}{\rmfamily\fontsize{8.000000}{9.600000}\selectfont\catcode`\^=\active\def^{\ifmmode\sp\else\^{}\fi}\catcode`\%=\active\def%{\%}7.5}}%
\end{pgfscope}%
\begin{pgfscope}%
\pgfpathrectangle{\pgfqpoint{0.669309in}{0.529443in}}{\pgfqpoint{1.589120in}{1.745990in}}%
\pgfusepath{clip}%
\pgfsetroundcap%
\pgfsetroundjoin%
\pgfsetlinewidth{1.003750pt}%
\definecolor{currentstroke}{rgb}{0.800000,0.800000,0.800000}%
\pgfsetstrokecolor{currentstroke}%
\pgfsetdash{}{0pt}%
\pgfpathmoveto{\pgfqpoint{1.692496in}{0.529443in}}%
\pgfpathlineto{\pgfqpoint{1.692496in}{2.275433in}}%
\pgfusepath{stroke}%
\end{pgfscope}%
\begin{pgfscope}%
\definecolor{textcolor}{rgb}{0.150000,0.150000,0.150000}%
\pgfsetstrokecolor{textcolor}%
\pgfsetfillcolor{textcolor}%
\pgftext[x=1.692496in,y=0.397499in,,top]{\color{textcolor}{\rmfamily\fontsize{8.000000}{9.600000}\selectfont\catcode`\^=\active\def^{\ifmmode\sp\else\^{}\fi}\catcode`\%=\active\def%{\%}10.0}}%
\end{pgfscope}%
\begin{pgfscope}%
\pgfpathrectangle{\pgfqpoint{0.669309in}{0.529443in}}{\pgfqpoint{1.589120in}{1.745990in}}%
\pgfusepath{clip}%
\pgfsetroundcap%
\pgfsetroundjoin%
\pgfsetlinewidth{1.003750pt}%
\definecolor{currentstroke}{rgb}{0.800000,0.800000,0.800000}%
\pgfsetstrokecolor{currentstroke}%
\pgfsetdash{}{0pt}%
\pgfpathmoveto{\pgfqpoint{2.099119in}{0.529443in}}%
\pgfpathlineto{\pgfqpoint{2.099119in}{2.275433in}}%
\pgfusepath{stroke}%
\end{pgfscope}%
\begin{pgfscope}%
\definecolor{textcolor}{rgb}{0.150000,0.150000,0.150000}%
\pgfsetstrokecolor{textcolor}%
\pgfsetfillcolor{textcolor}%
\pgftext[x=2.099119in,y=0.397499in,,top]{\color{textcolor}{\rmfamily\fontsize{8.000000}{9.600000}\selectfont\catcode`\^=\active\def^{\ifmmode\sp\else\^{}\fi}\catcode`\%=\active\def%{\%}12.5}}%
\end{pgfscope}%
\begin{pgfscope}%
\definecolor{textcolor}{rgb}{0.150000,0.150000,0.150000}%
\pgfsetstrokecolor{textcolor}%
\pgfsetfillcolor{textcolor}%
\pgftext[x=1.463869in,y=0.234413in,,top]{\color{textcolor}{\rmfamily\fontsize{10.000000}{12.000000}\selectfont\catcode`\^=\active\def^{\ifmmode\sp\else\^{}\fi}\catcode`\%=\active\def%{\%}UMAP 1}}%
\end{pgfscope}%
\begin{pgfscope}%
\pgfpathrectangle{\pgfqpoint{0.669309in}{0.529443in}}{\pgfqpoint{1.589120in}{1.745990in}}%
\pgfusepath{clip}%
\pgfsetroundcap%
\pgfsetroundjoin%
\pgfsetlinewidth{1.003750pt}%
\definecolor{currentstroke}{rgb}{0.800000,0.800000,0.800000}%
\pgfsetstrokecolor{currentstroke}%
\pgfsetdash{}{0pt}%
\pgfpathmoveto{\pgfqpoint{0.669309in}{0.626151in}}%
\pgfpathlineto{\pgfqpoint{2.258429in}{0.626151in}}%
\pgfusepath{stroke}%
\end{pgfscope}%
\begin{pgfscope}%
\definecolor{textcolor}{rgb}{0.150000,0.150000,0.150000}%
\pgfsetstrokecolor{textcolor}%
\pgfsetfillcolor{textcolor}%
\pgftext[x=0.289968in, y=0.583942in, left, base]{\color{textcolor}{\rmfamily\fontsize{8.000000}{9.600000}\selectfont\catcode`\^=\active\def^{\ifmmode\sp\else\^{}\fi}\catcode`\%=\active\def%{\%}10.0}}%
\end{pgfscope}%
\begin{pgfscope}%
\pgfpathrectangle{\pgfqpoint{0.669309in}{0.529443in}}{\pgfqpoint{1.589120in}{1.745990in}}%
\pgfusepath{clip}%
\pgfsetroundcap%
\pgfsetroundjoin%
\pgfsetlinewidth{1.003750pt}%
\definecolor{currentstroke}{rgb}{0.800000,0.800000,0.800000}%
\pgfsetstrokecolor{currentstroke}%
\pgfsetdash{}{0pt}%
\pgfpathmoveto{\pgfqpoint{0.669309in}{0.913660in}}%
\pgfpathlineto{\pgfqpoint{2.258429in}{0.913660in}}%
\pgfusepath{stroke}%
\end{pgfscope}%
\begin{pgfscope}%
\definecolor{textcolor}{rgb}{0.150000,0.150000,0.150000}%
\pgfsetstrokecolor{textcolor}%
\pgfsetfillcolor{textcolor}%
\pgftext[x=0.289968in, y=0.871451in, left, base]{\color{textcolor}{\rmfamily\fontsize{8.000000}{9.600000}\selectfont\catcode`\^=\active\def^{\ifmmode\sp\else\^{}\fi}\catcode`\%=\active\def%{\%}10.5}}%
\end{pgfscope}%
\begin{pgfscope}%
\pgfpathrectangle{\pgfqpoint{0.669309in}{0.529443in}}{\pgfqpoint{1.589120in}{1.745990in}}%
\pgfusepath{clip}%
\pgfsetroundcap%
\pgfsetroundjoin%
\pgfsetlinewidth{1.003750pt}%
\definecolor{currentstroke}{rgb}{0.800000,0.800000,0.800000}%
\pgfsetstrokecolor{currentstroke}%
\pgfsetdash{}{0pt}%
\pgfpathmoveto{\pgfqpoint{0.669309in}{1.201169in}}%
\pgfpathlineto{\pgfqpoint{2.258429in}{1.201169in}}%
\pgfusepath{stroke}%
\end{pgfscope}%
\begin{pgfscope}%
\definecolor{textcolor}{rgb}{0.150000,0.150000,0.150000}%
\pgfsetstrokecolor{textcolor}%
\pgfsetfillcolor{textcolor}%
\pgftext[x=0.289968in, y=1.158960in, left, base]{\color{textcolor}{\rmfamily\fontsize{8.000000}{9.600000}\selectfont\catcode`\^=\active\def^{\ifmmode\sp\else\^{}\fi}\catcode`\%=\active\def%{\%}11.0}}%
\end{pgfscope}%
\begin{pgfscope}%
\pgfpathrectangle{\pgfqpoint{0.669309in}{0.529443in}}{\pgfqpoint{1.589120in}{1.745990in}}%
\pgfusepath{clip}%
\pgfsetroundcap%
\pgfsetroundjoin%
\pgfsetlinewidth{1.003750pt}%
\definecolor{currentstroke}{rgb}{0.800000,0.800000,0.800000}%
\pgfsetstrokecolor{currentstroke}%
\pgfsetdash{}{0pt}%
\pgfpathmoveto{\pgfqpoint{0.669309in}{1.488678in}}%
\pgfpathlineto{\pgfqpoint{2.258429in}{1.488678in}}%
\pgfusepath{stroke}%
\end{pgfscope}%
\begin{pgfscope}%
\definecolor{textcolor}{rgb}{0.150000,0.150000,0.150000}%
\pgfsetstrokecolor{textcolor}%
\pgfsetfillcolor{textcolor}%
\pgftext[x=0.289968in, y=1.446468in, left, base]{\color{textcolor}{\rmfamily\fontsize{8.000000}{9.600000}\selectfont\catcode`\^=\active\def^{\ifmmode\sp\else\^{}\fi}\catcode`\%=\active\def%{\%}11.5}}%
\end{pgfscope}%
\begin{pgfscope}%
\pgfpathrectangle{\pgfqpoint{0.669309in}{0.529443in}}{\pgfqpoint{1.589120in}{1.745990in}}%
\pgfusepath{clip}%
\pgfsetroundcap%
\pgfsetroundjoin%
\pgfsetlinewidth{1.003750pt}%
\definecolor{currentstroke}{rgb}{0.800000,0.800000,0.800000}%
\pgfsetstrokecolor{currentstroke}%
\pgfsetdash{}{0pt}%
\pgfpathmoveto{\pgfqpoint{0.669309in}{1.776186in}}%
\pgfpathlineto{\pgfqpoint{2.258429in}{1.776186in}}%
\pgfusepath{stroke}%
\end{pgfscope}%
\begin{pgfscope}%
\definecolor{textcolor}{rgb}{0.150000,0.150000,0.150000}%
\pgfsetstrokecolor{textcolor}%
\pgfsetfillcolor{textcolor}%
\pgftext[x=0.289968in, y=1.733977in, left, base]{\color{textcolor}{\rmfamily\fontsize{8.000000}{9.600000}\selectfont\catcode`\^=\active\def^{\ifmmode\sp\else\^{}\fi}\catcode`\%=\active\def%{\%}12.0}}%
\end{pgfscope}%
\begin{pgfscope}%
\pgfpathrectangle{\pgfqpoint{0.669309in}{0.529443in}}{\pgfqpoint{1.589120in}{1.745990in}}%
\pgfusepath{clip}%
\pgfsetroundcap%
\pgfsetroundjoin%
\pgfsetlinewidth{1.003750pt}%
\definecolor{currentstroke}{rgb}{0.800000,0.800000,0.800000}%
\pgfsetstrokecolor{currentstroke}%
\pgfsetdash{}{0pt}%
\pgfpathmoveto{\pgfqpoint{0.669309in}{2.063695in}}%
\pgfpathlineto{\pgfqpoint{2.258429in}{2.063695in}}%
\pgfusepath{stroke}%
\end{pgfscope}%
\begin{pgfscope}%
\definecolor{textcolor}{rgb}{0.150000,0.150000,0.150000}%
\pgfsetstrokecolor{textcolor}%
\pgfsetfillcolor{textcolor}%
\pgftext[x=0.289968in, y=2.021486in, left, base]{\color{textcolor}{\rmfamily\fontsize{8.000000}{9.600000}\selectfont\catcode`\^=\active\def^{\ifmmode\sp\else\^{}\fi}\catcode`\%=\active\def%{\%}12.5}}%
\end{pgfscope}%
\begin{pgfscope}%
\definecolor{textcolor}{rgb}{0.150000,0.150000,0.150000}%
\pgfsetstrokecolor{textcolor}%
\pgfsetfillcolor{textcolor}%
\pgftext[x=0.234413in,y=1.402438in,,bottom,rotate=90.000000]{\color{textcolor}{\rmfamily\fontsize{10.000000}{12.000000}\selectfont\catcode`\^=\active\def^{\ifmmode\sp\else\^{}\fi}\catcode`\%=\active\def%{\%}UMAP 2}}%
\end{pgfscope}%
\begin{pgfscope}%
\pgfpathrectangle{\pgfqpoint{0.669309in}{0.529443in}}{\pgfqpoint{1.589120in}{1.745990in}}%
\pgfusepath{clip}%
\pgfsetbuttcap%
\pgfsetroundjoin%
\definecolor{currentfill}{rgb}{0.298039,0.447059,0.690196}%
\pgfsetfillcolor{currentfill}%
\pgfsetfillopacity{0.900000}%
\pgfsetlinewidth{0.507862pt}%
\definecolor{currentstroke}{rgb}{1.000000,1.000000,1.000000}%
\pgfsetstrokecolor{currentstroke}%
\pgfsetstrokeopacity{0.900000}%
\pgfsetdash{}{0pt}%
\pgfpathmoveto{\pgfqpoint{0.855198in}{1.129265in}}%
\pgfpathlineto{\pgfqpoint{0.877158in}{1.151225in}}%
\pgfpathlineto{\pgfqpoint{0.899118in}{1.129265in}}%
\pgfpathlineto{\pgfqpoint{0.921078in}{1.151225in}}%
\pgfpathlineto{\pgfqpoint{0.899118in}{1.173186in}}%
\pgfpathlineto{\pgfqpoint{0.921078in}{1.195146in}}%
\pgfpathlineto{\pgfqpoint{0.899118in}{1.217106in}}%
\pgfpathlineto{\pgfqpoint{0.877158in}{1.195146in}}%
\pgfpathlineto{\pgfqpoint{0.855198in}{1.217106in}}%
\pgfpathlineto{\pgfqpoint{0.833237in}{1.195146in}}%
\pgfpathlineto{\pgfqpoint{0.855198in}{1.173186in}}%
\pgfpathlineto{\pgfqpoint{0.833237in}{1.151225in}}%
\pgfpathlineto{\pgfqpoint{0.855198in}{1.129265in}}%
\pgfpathclose%
\pgfusepath{stroke,fill}%
\end{pgfscope}%
\begin{pgfscope}%
\pgfpathrectangle{\pgfqpoint{0.669309in}{0.529443in}}{\pgfqpoint{1.589120in}{1.745990in}}%
\pgfusepath{clip}%
\pgfsetbuttcap%
\pgfsetroundjoin%
\definecolor{currentfill}{rgb}{0.866667,0.517647,0.321569}%
\pgfsetfillcolor{currentfill}%
\pgfsetfillopacity{0.900000}%
\pgfsetlinewidth{0.507862pt}%
\definecolor{currentstroke}{rgb}{1.000000,1.000000,1.000000}%
\pgfsetstrokecolor{currentstroke}%
\pgfsetstrokeopacity{0.900000}%
\pgfsetdash{}{0pt}%
\pgfpathmoveto{\pgfqpoint{2.095843in}{0.886872in}}%
\pgfpathcurveto{\pgfqpoint{2.107491in}{0.886872in}}{\pgfqpoint{2.118663in}{0.891500in}}{\pgfqpoint{2.126900in}{0.899736in}}%
\pgfpathcurveto{\pgfqpoint{2.135136in}{0.907972in}}{\pgfqpoint{2.139764in}{0.919145in}}{\pgfqpoint{2.139764in}{0.930793in}}%
\pgfpathcurveto{\pgfqpoint{2.139764in}{0.942441in}}{\pgfqpoint{2.135136in}{0.953613in}}{\pgfqpoint{2.126900in}{0.961849in}}%
\pgfpathcurveto{\pgfqpoint{2.118663in}{0.970085in}}{\pgfqpoint{2.107491in}{0.974713in}}{\pgfqpoint{2.095843in}{0.974713in}}%
\pgfpathcurveto{\pgfqpoint{2.084195in}{0.974713in}}{\pgfqpoint{2.073023in}{0.970085in}}{\pgfqpoint{2.064787in}{0.961849in}}%
\pgfpathcurveto{\pgfqpoint{2.056550in}{0.953613in}}{\pgfqpoint{2.051923in}{0.942441in}}{\pgfqpoint{2.051923in}{0.930793in}}%
\pgfpathcurveto{\pgfqpoint{2.051923in}{0.919145in}}{\pgfqpoint{2.056550in}{0.907972in}}{\pgfqpoint{2.064787in}{0.899736in}}%
\pgfpathcurveto{\pgfqpoint{2.073023in}{0.891500in}}{\pgfqpoint{2.084195in}{0.886872in}}{\pgfqpoint{2.095843in}{0.886872in}}%
\pgfpathlineto{\pgfqpoint{2.095843in}{0.886872in}}%
\pgfpathclose%
\pgfusepath{stroke,fill}%
\end{pgfscope}%
\begin{pgfscope}%
\pgfpathrectangle{\pgfqpoint{0.669309in}{0.529443in}}{\pgfqpoint{1.589120in}{1.745990in}}%
\pgfusepath{clip}%
\pgfsetbuttcap%
\pgfsetroundjoin%
\definecolor{currentfill}{rgb}{0.333333,0.658824,0.407843}%
\pgfsetfillcolor{currentfill}%
\pgfsetfillopacity{0.900000}%
\pgfsetlinewidth{0.507862pt}%
\definecolor{currentstroke}{rgb}{1.000000,1.000000,1.000000}%
\pgfsetstrokecolor{currentstroke}%
\pgfsetstrokeopacity{0.900000}%
\pgfsetdash{}{0pt}%
\pgfpathmoveto{\pgfqpoint{0.719581in}{0.727170in}}%
\pgfpathlineto{\pgfqpoint{0.741541in}{0.749131in}}%
\pgfpathlineto{\pgfqpoint{0.763502in}{0.727170in}}%
\pgfpathlineto{\pgfqpoint{0.785462in}{0.749131in}}%
\pgfpathlineto{\pgfqpoint{0.763502in}{0.771091in}}%
\pgfpathlineto{\pgfqpoint{0.785462in}{0.793051in}}%
\pgfpathlineto{\pgfqpoint{0.763502in}{0.815011in}}%
\pgfpathlineto{\pgfqpoint{0.741541in}{0.793051in}}%
\pgfpathlineto{\pgfqpoint{0.719581in}{0.815011in}}%
\pgfpathlineto{\pgfqpoint{0.697621in}{0.793051in}}%
\pgfpathlineto{\pgfqpoint{0.719581in}{0.771091in}}%
\pgfpathlineto{\pgfqpoint{0.697621in}{0.749131in}}%
\pgfpathlineto{\pgfqpoint{0.719581in}{0.727170in}}%
\pgfpathclose%
\pgfusepath{stroke,fill}%
\end{pgfscope}%
\begin{pgfscope}%
\pgfpathrectangle{\pgfqpoint{0.669309in}{0.529443in}}{\pgfqpoint{1.589120in}{1.745990in}}%
\pgfusepath{clip}%
\pgfsetbuttcap%
\pgfsetroundjoin%
\definecolor{currentfill}{rgb}{0.298039,0.447059,0.690196}%
\pgfsetfillcolor{currentfill}%
\pgfsetfillopacity{0.900000}%
\pgfsetlinewidth{0.507862pt}%
\definecolor{currentstroke}{rgb}{1.000000,1.000000,1.000000}%
\pgfsetstrokecolor{currentstroke}%
\pgfsetstrokeopacity{0.900000}%
\pgfsetdash{}{0pt}%
\pgfpathmoveto{\pgfqpoint{1.412163in}{1.940564in}}%
\pgfpathlineto{\pgfqpoint{1.412163in}{1.878451in}}%
\pgfpathlineto{\pgfqpoint{1.474276in}{1.878451in}}%
\pgfpathlineto{\pgfqpoint{1.474276in}{1.940564in}}%
\pgfpathlineto{\pgfqpoint{1.412163in}{1.940564in}}%
\pgfpathclose%
\pgfusepath{stroke,fill}%
\end{pgfscope}%
\begin{pgfscope}%
\pgfpathrectangle{\pgfqpoint{0.669309in}{0.529443in}}{\pgfqpoint{1.589120in}{1.745990in}}%
\pgfusepath{clip}%
\pgfsetbuttcap%
\pgfsetroundjoin%
\definecolor{currentfill}{rgb}{0.298039,0.447059,0.690196}%
\pgfsetfillcolor{currentfill}%
\pgfsetfillopacity{0.900000}%
\pgfsetlinewidth{0.507862pt}%
\definecolor{currentstroke}{rgb}{1.000000,1.000000,1.000000}%
\pgfsetstrokecolor{currentstroke}%
\pgfsetstrokeopacity{0.900000}%
\pgfsetdash{}{0pt}%
\pgfpathmoveto{\pgfqpoint{1.428196in}{2.173706in}}%
\pgfpathlineto{\pgfqpoint{1.428196in}{2.111593in}}%
\pgfpathlineto{\pgfqpoint{1.490309in}{2.111593in}}%
\pgfpathlineto{\pgfqpoint{1.490309in}{2.173706in}}%
\pgfpathlineto{\pgfqpoint{1.428196in}{2.173706in}}%
\pgfpathclose%
\pgfusepath{stroke,fill}%
\end{pgfscope}%
\begin{pgfscope}%
\pgfpathrectangle{\pgfqpoint{0.669309in}{0.529443in}}{\pgfqpoint{1.589120in}{1.745990in}}%
\pgfusepath{clip}%
\pgfsetbuttcap%
\pgfsetroundjoin%
\definecolor{currentfill}{rgb}{0.866667,0.517647,0.321569}%
\pgfsetfillcolor{currentfill}%
\pgfsetfillopacity{0.900000}%
\pgfsetlinewidth{0.507862pt}%
\definecolor{currentstroke}{rgb}{1.000000,1.000000,1.000000}%
\pgfsetstrokecolor{currentstroke}%
\pgfsetstrokeopacity{0.900000}%
\pgfsetdash{}{0pt}%
\pgfpathmoveto{\pgfqpoint{2.132284in}{1.274611in}}%
\pgfpathcurveto{\pgfqpoint{2.143932in}{1.274611in}}{\pgfqpoint{2.155105in}{1.279239in}}{\pgfqpoint{2.163341in}{1.287475in}}%
\pgfpathcurveto{\pgfqpoint{2.171577in}{1.295712in}}{\pgfqpoint{2.176205in}{1.306884in}}{\pgfqpoint{2.176205in}{1.318532in}}%
\pgfpathcurveto{\pgfqpoint{2.176205in}{1.330180in}}{\pgfqpoint{2.171577in}{1.341352in}}{\pgfqpoint{2.163341in}{1.349588in}}%
\pgfpathcurveto{\pgfqpoint{2.155105in}{1.357825in}}{\pgfqpoint{2.143932in}{1.362452in}}{\pgfqpoint{2.132284in}{1.362452in}}%
\pgfpathcurveto{\pgfqpoint{2.120636in}{1.362452in}}{\pgfqpoint{2.109464in}{1.357825in}}{\pgfqpoint{2.101228in}{1.349588in}}%
\pgfpathcurveto{\pgfqpoint{2.092992in}{1.341352in}}{\pgfqpoint{2.088364in}{1.330180in}}{\pgfqpoint{2.088364in}{1.318532in}}%
\pgfpathcurveto{\pgfqpoint{2.088364in}{1.306884in}}{\pgfqpoint{2.092992in}{1.295712in}}{\pgfqpoint{2.101228in}{1.287475in}}%
\pgfpathcurveto{\pgfqpoint{2.109464in}{1.279239in}}{\pgfqpoint{2.120636in}{1.274611in}}{\pgfqpoint{2.132284in}{1.274611in}}%
\pgfpathlineto{\pgfqpoint{2.132284in}{1.274611in}}%
\pgfpathclose%
\pgfusepath{stroke,fill}%
\end{pgfscope}%
\begin{pgfscope}%
\pgfpathrectangle{\pgfqpoint{0.669309in}{0.529443in}}{\pgfqpoint{1.589120in}{1.745990in}}%
\pgfusepath{clip}%
\pgfsetbuttcap%
\pgfsetroundjoin%
\definecolor{currentfill}{rgb}{0.333333,0.658824,0.407843}%
\pgfsetfillcolor{currentfill}%
\pgfsetfillopacity{0.900000}%
\pgfsetlinewidth{0.507862pt}%
\definecolor{currentstroke}{rgb}{1.000000,1.000000,1.000000}%
\pgfsetstrokecolor{currentstroke}%
\pgfsetstrokeopacity{0.900000}%
\pgfsetdash{}{0pt}%
\pgfpathmoveto{\pgfqpoint{1.931569in}{1.137717in}}%
\pgfpathcurveto{\pgfqpoint{1.943217in}{1.137717in}}{\pgfqpoint{1.954390in}{1.142344in}}{\pgfqpoint{1.962626in}{1.150581in}}%
\pgfpathcurveto{\pgfqpoint{1.970862in}{1.158817in}}{\pgfqpoint{1.975490in}{1.169989in}}{\pgfqpoint{1.975490in}{1.181637in}}%
\pgfpathcurveto{\pgfqpoint{1.975490in}{1.193285in}}{\pgfqpoint{1.970862in}{1.204457in}}{\pgfqpoint{1.962626in}{1.212694in}}%
\pgfpathcurveto{\pgfqpoint{1.954390in}{1.220930in}}{\pgfqpoint{1.943217in}{1.225558in}}{\pgfqpoint{1.931569in}{1.225558in}}%
\pgfpathcurveto{\pgfqpoint{1.919922in}{1.225558in}}{\pgfqpoint{1.908749in}{1.220930in}}{\pgfqpoint{1.900513in}{1.212694in}}%
\pgfpathcurveto{\pgfqpoint{1.892277in}{1.204457in}}{\pgfqpoint{1.887649in}{1.193285in}}{\pgfqpoint{1.887649in}{1.181637in}}%
\pgfpathcurveto{\pgfqpoint{1.887649in}{1.169989in}}{\pgfqpoint{1.892277in}{1.158817in}}{\pgfqpoint{1.900513in}{1.150581in}}%
\pgfpathcurveto{\pgfqpoint{1.908749in}{1.142344in}}{\pgfqpoint{1.919922in}{1.137717in}}{\pgfqpoint{1.931569in}{1.137717in}}%
\pgfpathlineto{\pgfqpoint{1.931569in}{1.137717in}}%
\pgfpathclose%
\pgfusepath{stroke,fill}%
\end{pgfscope}%
\begin{pgfscope}%
\pgfpathrectangle{\pgfqpoint{0.669309in}{0.529443in}}{\pgfqpoint{1.589120in}{1.745990in}}%
\pgfusepath{clip}%
\pgfsetbuttcap%
\pgfsetroundjoin%
\definecolor{currentfill}{rgb}{0.333333,0.658824,0.407843}%
\pgfsetfillcolor{currentfill}%
\pgfsetfillopacity{0.900000}%
\pgfsetlinewidth{0.507862pt}%
\definecolor{currentstroke}{rgb}{1.000000,1.000000,1.000000}%
\pgfsetstrokecolor{currentstroke}%
\pgfsetstrokeopacity{0.900000}%
\pgfsetdash{}{0pt}%
\pgfpathmoveto{\pgfqpoint{1.351967in}{2.109627in}}%
\pgfpathlineto{\pgfqpoint{1.351967in}{2.047514in}}%
\pgfpathlineto{\pgfqpoint{1.414080in}{2.047514in}}%
\pgfpathlineto{\pgfqpoint{1.414080in}{2.109627in}}%
\pgfpathlineto{\pgfqpoint{1.351967in}{2.109627in}}%
\pgfpathclose%
\pgfusepath{stroke,fill}%
\end{pgfscope}%
\begin{pgfscope}%
\pgfpathrectangle{\pgfqpoint{0.669309in}{0.529443in}}{\pgfqpoint{1.589120in}{1.745990in}}%
\pgfusepath{clip}%
\pgfsetbuttcap%
\pgfsetroundjoin%
\definecolor{currentfill}{rgb}{0.298039,0.447059,0.690196}%
\pgfsetfillcolor{currentfill}%
\pgfsetfillopacity{0.900000}%
\pgfsetlinewidth{0.507862pt}%
\definecolor{currentstroke}{rgb}{1.000000,1.000000,1.000000}%
\pgfsetstrokecolor{currentstroke}%
\pgfsetstrokeopacity{0.900000}%
\pgfsetdash{}{0pt}%
\pgfpathmoveto{\pgfqpoint{0.840595in}{0.922041in}}%
\pgfpathlineto{\pgfqpoint{0.862555in}{0.944001in}}%
\pgfpathlineto{\pgfqpoint{0.884516in}{0.922041in}}%
\pgfpathlineto{\pgfqpoint{0.906476in}{0.944001in}}%
\pgfpathlineto{\pgfqpoint{0.884516in}{0.965961in}}%
\pgfpathlineto{\pgfqpoint{0.906476in}{0.987922in}}%
\pgfpathlineto{\pgfqpoint{0.884516in}{1.009882in}}%
\pgfpathlineto{\pgfqpoint{0.862555in}{0.987922in}}%
\pgfpathlineto{\pgfqpoint{0.840595in}{1.009882in}}%
\pgfpathlineto{\pgfqpoint{0.818635in}{0.987922in}}%
\pgfpathlineto{\pgfqpoint{0.840595in}{0.965961in}}%
\pgfpathlineto{\pgfqpoint{0.818635in}{0.944001in}}%
\pgfpathlineto{\pgfqpoint{0.840595in}{0.922041in}}%
\pgfpathclose%
\pgfusepath{stroke,fill}%
\end{pgfscope}%
\begin{pgfscope}%
\pgfpathrectangle{\pgfqpoint{0.669309in}{0.529443in}}{\pgfqpoint{1.589120in}{1.745990in}}%
\pgfusepath{clip}%
\pgfsetbuttcap%
\pgfsetroundjoin%
\definecolor{currentfill}{rgb}{0.333333,0.658824,0.407843}%
\pgfsetfillcolor{currentfill}%
\pgfsetfillopacity{0.900000}%
\pgfsetlinewidth{0.507862pt}%
\definecolor{currentstroke}{rgb}{1.000000,1.000000,1.000000}%
\pgfsetstrokecolor{currentstroke}%
\pgfsetstrokeopacity{0.900000}%
\pgfsetdash{}{0pt}%
\pgfpathmoveto{\pgfqpoint{1.999528in}{1.183209in}}%
\pgfpathcurveto{\pgfqpoint{2.011176in}{1.183209in}}{\pgfqpoint{2.022348in}{1.187836in}}{\pgfqpoint{2.030585in}{1.196073in}}%
\pgfpathcurveto{\pgfqpoint{2.038821in}{1.204309in}}{\pgfqpoint{2.043449in}{1.215481in}}{\pgfqpoint{2.043449in}{1.227129in}}%
\pgfpathcurveto{\pgfqpoint{2.043449in}{1.238777in}}{\pgfqpoint{2.038821in}{1.249949in}}{\pgfqpoint{2.030585in}{1.258186in}}%
\pgfpathcurveto{\pgfqpoint{2.022348in}{1.266422in}}{\pgfqpoint{2.011176in}{1.271050in}}{\pgfqpoint{1.999528in}{1.271050in}}%
\pgfpathcurveto{\pgfqpoint{1.987880in}{1.271050in}}{\pgfqpoint{1.976708in}{1.266422in}}{\pgfqpoint{1.968472in}{1.258186in}}%
\pgfpathcurveto{\pgfqpoint{1.960235in}{1.249949in}}{\pgfqpoint{1.955608in}{1.238777in}}{\pgfqpoint{1.955608in}{1.227129in}}%
\pgfpathcurveto{\pgfqpoint{1.955608in}{1.215481in}}{\pgfqpoint{1.960235in}{1.204309in}}{\pgfqpoint{1.968472in}{1.196073in}}%
\pgfpathcurveto{\pgfqpoint{1.976708in}{1.187836in}}{\pgfqpoint{1.987880in}{1.183209in}}{\pgfqpoint{1.999528in}{1.183209in}}%
\pgfpathlineto{\pgfqpoint{1.999528in}{1.183209in}}%
\pgfpathclose%
\pgfusepath{stroke,fill}%
\end{pgfscope}%
\begin{pgfscope}%
\pgfpathrectangle{\pgfqpoint{0.669309in}{0.529443in}}{\pgfqpoint{1.589120in}{1.745990in}}%
\pgfusepath{clip}%
\pgfsetbuttcap%
\pgfsetroundjoin%
\definecolor{currentfill}{rgb}{0.333333,0.658824,0.407843}%
\pgfsetfillcolor{currentfill}%
\pgfsetfillopacity{0.900000}%
\pgfsetlinewidth{0.507862pt}%
\definecolor{currentstroke}{rgb}{1.000000,1.000000,1.000000}%
\pgfsetstrokecolor{currentstroke}%
\pgfsetstrokeopacity{0.900000}%
\pgfsetdash{}{0pt}%
\pgfpathmoveto{\pgfqpoint{0.776139in}{1.002082in}}%
\pgfpathlineto{\pgfqpoint{0.798100in}{1.024042in}}%
\pgfpathlineto{\pgfqpoint{0.820060in}{1.002082in}}%
\pgfpathlineto{\pgfqpoint{0.842020in}{1.024042in}}%
\pgfpathlineto{\pgfqpoint{0.820060in}{1.046002in}}%
\pgfpathlineto{\pgfqpoint{0.842020in}{1.067962in}}%
\pgfpathlineto{\pgfqpoint{0.820060in}{1.089923in}}%
\pgfpathlineto{\pgfqpoint{0.798100in}{1.067962in}}%
\pgfpathlineto{\pgfqpoint{0.776139in}{1.089923in}}%
\pgfpathlineto{\pgfqpoint{0.754179in}{1.067962in}}%
\pgfpathlineto{\pgfqpoint{0.776139in}{1.046002in}}%
\pgfpathlineto{\pgfqpoint{0.754179in}{1.024042in}}%
\pgfpathlineto{\pgfqpoint{0.776139in}{1.002082in}}%
\pgfpathclose%
\pgfusepath{stroke,fill}%
\end{pgfscope}%
\begin{pgfscope}%
\pgfpathrectangle{\pgfqpoint{0.669309in}{0.529443in}}{\pgfqpoint{1.589120in}{1.745990in}}%
\pgfusepath{clip}%
\pgfsetbuttcap%
\pgfsetroundjoin%
\definecolor{currentfill}{rgb}{0.333333,0.658824,0.407843}%
\pgfsetfillcolor{currentfill}%
\pgfsetfillopacity{0.900000}%
\pgfsetlinewidth{0.507862pt}%
\definecolor{currentstroke}{rgb}{1.000000,1.000000,1.000000}%
\pgfsetstrokecolor{currentstroke}%
\pgfsetstrokeopacity{0.900000}%
\pgfsetdash{}{0pt}%
\pgfpathmoveto{\pgfqpoint{0.986286in}{1.400064in}}%
\pgfpathcurveto{\pgfqpoint{0.997934in}{1.400064in}}{\pgfqpoint{1.009106in}{1.404691in}}{\pgfqpoint{1.017342in}{1.412928in}}%
\pgfpathcurveto{\pgfqpoint{1.025579in}{1.421164in}}{\pgfqpoint{1.030206in}{1.432336in}}{\pgfqpoint{1.030206in}{1.443984in}}%
\pgfpathcurveto{\pgfqpoint{1.030206in}{1.455632in}}{\pgfqpoint{1.025579in}{1.466804in}}{\pgfqpoint{1.017342in}{1.475041in}}%
\pgfpathcurveto{\pgfqpoint{1.009106in}{1.483277in}}{\pgfqpoint{0.997934in}{1.487905in}}{\pgfqpoint{0.986286in}{1.487905in}}%
\pgfpathcurveto{\pgfqpoint{0.974638in}{1.487905in}}{\pgfqpoint{0.963466in}{1.483277in}}{\pgfqpoint{0.955229in}{1.475041in}}%
\pgfpathcurveto{\pgfqpoint{0.946993in}{1.466804in}}{\pgfqpoint{0.942365in}{1.455632in}}{\pgfqpoint{0.942365in}{1.443984in}}%
\pgfpathcurveto{\pgfqpoint{0.942365in}{1.432336in}}{\pgfqpoint{0.946993in}{1.421164in}}{\pgfqpoint{0.955229in}{1.412928in}}%
\pgfpathcurveto{\pgfqpoint{0.963466in}{1.404691in}}{\pgfqpoint{0.974638in}{1.400064in}}{\pgfqpoint{0.986286in}{1.400064in}}%
\pgfpathlineto{\pgfqpoint{0.986286in}{1.400064in}}%
\pgfpathclose%
\pgfusepath{stroke,fill}%
\end{pgfscope}%
\begin{pgfscope}%
\pgfpathrectangle{\pgfqpoint{0.669309in}{0.529443in}}{\pgfqpoint{1.589120in}{1.745990in}}%
\pgfusepath{clip}%
\pgfsetbuttcap%
\pgfsetroundjoin%
\definecolor{currentfill}{rgb}{0.298039,0.447059,0.690196}%
\pgfsetfillcolor{currentfill}%
\pgfsetfillopacity{0.900000}%
\pgfsetlinewidth{0.507862pt}%
\definecolor{currentstroke}{rgb}{1.000000,1.000000,1.000000}%
\pgfsetstrokecolor{currentstroke}%
\pgfsetstrokeopacity{0.900000}%
\pgfsetdash{}{0pt}%
\pgfpathmoveto{\pgfqpoint{1.123164in}{1.787757in}}%
\pgfpathcurveto{\pgfqpoint{1.134811in}{1.787757in}}{\pgfqpoint{1.145984in}{1.792385in}}{\pgfqpoint{1.154220in}{1.800621in}}%
\pgfpathcurveto{\pgfqpoint{1.162456in}{1.808857in}}{\pgfqpoint{1.167084in}{1.820030in}}{\pgfqpoint{1.167084in}{1.831677in}}%
\pgfpathcurveto{\pgfqpoint{1.167084in}{1.843325in}}{\pgfqpoint{1.162456in}{1.854498in}}{\pgfqpoint{1.154220in}{1.862734in}}%
\pgfpathcurveto{\pgfqpoint{1.145984in}{1.870970in}}{\pgfqpoint{1.134811in}{1.875598in}}{\pgfqpoint{1.123164in}{1.875598in}}%
\pgfpathcurveto{\pgfqpoint{1.111516in}{1.875598in}}{\pgfqpoint{1.100343in}{1.870970in}}{\pgfqpoint{1.092107in}{1.862734in}}%
\pgfpathcurveto{\pgfqpoint{1.083871in}{1.854498in}}{\pgfqpoint{1.079243in}{1.843325in}}{\pgfqpoint{1.079243in}{1.831677in}}%
\pgfpathcurveto{\pgfqpoint{1.079243in}{1.820030in}}{\pgfqpoint{1.083871in}{1.808857in}}{\pgfqpoint{1.092107in}{1.800621in}}%
\pgfpathcurveto{\pgfqpoint{1.100343in}{1.792385in}}{\pgfqpoint{1.111516in}{1.787757in}}{\pgfqpoint{1.123164in}{1.787757in}}%
\pgfpathlineto{\pgfqpoint{1.123164in}{1.787757in}}%
\pgfpathclose%
\pgfusepath{stroke,fill}%
\end{pgfscope}%
\begin{pgfscope}%
\pgfpathrectangle{\pgfqpoint{0.669309in}{0.529443in}}{\pgfqpoint{1.589120in}{1.745990in}}%
\pgfusepath{clip}%
\pgfsetbuttcap%
\pgfsetroundjoin%
\definecolor{currentfill}{rgb}{0.333333,0.658824,0.407843}%
\pgfsetfillcolor{currentfill}%
\pgfsetfillopacity{0.900000}%
\pgfsetlinewidth{0.507862pt}%
\definecolor{currentstroke}{rgb}{1.000000,1.000000,1.000000}%
\pgfsetstrokecolor{currentstroke}%
\pgfsetstrokeopacity{0.900000}%
\pgfsetdash{}{0pt}%
\pgfpathmoveto{\pgfqpoint{1.600972in}{1.712124in}}%
\pgfpathlineto{\pgfqpoint{1.600972in}{1.650011in}}%
\pgfpathlineto{\pgfqpoint{1.663085in}{1.650011in}}%
\pgfpathlineto{\pgfqpoint{1.663085in}{1.712124in}}%
\pgfpathlineto{\pgfqpoint{1.600972in}{1.712124in}}%
\pgfpathclose%
\pgfusepath{stroke,fill}%
\end{pgfscope}%
\begin{pgfscope}%
\pgfpathrectangle{\pgfqpoint{0.669309in}{0.529443in}}{\pgfqpoint{1.589120in}{1.745990in}}%
\pgfusepath{clip}%
\pgfsetbuttcap%
\pgfsetroundjoin%
\definecolor{currentfill}{rgb}{0.866667,0.517647,0.321569}%
\pgfsetfillcolor{currentfill}%
\pgfsetfillopacity{0.900000}%
\pgfsetlinewidth{0.507862pt}%
\definecolor{currentstroke}{rgb}{1.000000,1.000000,1.000000}%
\pgfsetstrokecolor{currentstroke}%
\pgfsetstrokeopacity{0.900000}%
\pgfsetdash{}{0pt}%
\pgfpathmoveto{\pgfqpoint{2.171556in}{1.441336in}}%
\pgfpathlineto{\pgfqpoint{2.200836in}{1.441336in}}%
\pgfpathlineto{\pgfqpoint{2.200836in}{1.470616in}}%
\pgfpathlineto{\pgfqpoint{2.230117in}{1.470616in}}%
\pgfpathlineto{\pgfqpoint{2.230117in}{1.499896in}}%
\pgfpathlineto{\pgfqpoint{2.200836in}{1.499896in}}%
\pgfpathlineto{\pgfqpoint{2.200836in}{1.529177in}}%
\pgfpathlineto{\pgfqpoint{2.171556in}{1.529177in}}%
\pgfpathlineto{\pgfqpoint{2.171556in}{1.499896in}}%
\pgfpathlineto{\pgfqpoint{2.142276in}{1.499896in}}%
\pgfpathlineto{\pgfqpoint{2.142276in}{1.470616in}}%
\pgfpathlineto{\pgfqpoint{2.171556in}{1.470616in}}%
\pgfpathlineto{\pgfqpoint{2.171556in}{1.441336in}}%
\pgfpathclose%
\pgfusepath{stroke,fill}%
\end{pgfscope}%
\begin{pgfscope}%
\pgfpathrectangle{\pgfqpoint{0.669309in}{0.529443in}}{\pgfqpoint{1.589120in}{1.745990in}}%
\pgfusepath{clip}%
\pgfsetbuttcap%
\pgfsetroundjoin%
\definecolor{currentfill}{rgb}{0.298039,0.447059,0.690196}%
\pgfsetfillcolor{currentfill}%
\pgfsetfillopacity{0.900000}%
\pgfsetlinewidth{0.507862pt}%
\definecolor{currentstroke}{rgb}{1.000000,1.000000,1.000000}%
\pgfsetstrokecolor{currentstroke}%
\pgfsetstrokeopacity{0.900000}%
\pgfsetdash{}{0pt}%
\pgfpathmoveto{\pgfqpoint{2.107425in}{0.985138in}}%
\pgfpathlineto{\pgfqpoint{2.136705in}{0.985138in}}%
\pgfpathlineto{\pgfqpoint{2.136705in}{1.014419in}}%
\pgfpathlineto{\pgfqpoint{2.165985in}{1.014419in}}%
\pgfpathlineto{\pgfqpoint{2.165985in}{1.043699in}}%
\pgfpathlineto{\pgfqpoint{2.136705in}{1.043699in}}%
\pgfpathlineto{\pgfqpoint{2.136705in}{1.072979in}}%
\pgfpathlineto{\pgfqpoint{2.107425in}{1.072979in}}%
\pgfpathlineto{\pgfqpoint{2.107425in}{1.043699in}}%
\pgfpathlineto{\pgfqpoint{2.078144in}{1.043699in}}%
\pgfpathlineto{\pgfqpoint{2.078144in}{1.014419in}}%
\pgfpathlineto{\pgfqpoint{2.107425in}{1.014419in}}%
\pgfpathlineto{\pgfqpoint{2.107425in}{0.985138in}}%
\pgfpathclose%
\pgfusepath{stroke,fill}%
\end{pgfscope}%
\begin{pgfscope}%
\pgfpathrectangle{\pgfqpoint{0.669309in}{0.529443in}}{\pgfqpoint{1.589120in}{1.745990in}}%
\pgfusepath{clip}%
\pgfsetbuttcap%
\pgfsetroundjoin%
\definecolor{currentfill}{rgb}{0.298039,0.447059,0.690196}%
\pgfsetfillcolor{currentfill}%
\pgfsetfillopacity{0.900000}%
\pgfsetlinewidth{0.507862pt}%
\definecolor{currentstroke}{rgb}{1.000000,1.000000,1.000000}%
\pgfsetstrokecolor{currentstroke}%
\pgfsetstrokeopacity{0.900000}%
\pgfsetdash{}{0pt}%
\pgfpathmoveto{\pgfqpoint{0.724955in}{0.615136in}}%
\pgfpathlineto{\pgfqpoint{0.746915in}{0.637096in}}%
\pgfpathlineto{\pgfqpoint{0.768875in}{0.615136in}}%
\pgfpathlineto{\pgfqpoint{0.790835in}{0.637096in}}%
\pgfpathlineto{\pgfqpoint{0.768875in}{0.659056in}}%
\pgfpathlineto{\pgfqpoint{0.790835in}{0.681017in}}%
\pgfpathlineto{\pgfqpoint{0.768875in}{0.702977in}}%
\pgfpathlineto{\pgfqpoint{0.746915in}{0.681017in}}%
\pgfpathlineto{\pgfqpoint{0.724955in}{0.702977in}}%
\pgfpathlineto{\pgfqpoint{0.702994in}{0.681017in}}%
\pgfpathlineto{\pgfqpoint{0.724955in}{0.659056in}}%
\pgfpathlineto{\pgfqpoint{0.702994in}{0.637096in}}%
\pgfpathlineto{\pgfqpoint{0.724955in}{0.615136in}}%
\pgfpathclose%
\pgfusepath{stroke,fill}%
\end{pgfscope}%
\begin{pgfscope}%
\pgfpathrectangle{\pgfqpoint{0.669309in}{0.529443in}}{\pgfqpoint{1.589120in}{1.745990in}}%
\pgfusepath{clip}%
\pgfsetbuttcap%
\pgfsetroundjoin%
\definecolor{currentfill}{rgb}{0.866667,0.517647,0.321569}%
\pgfsetfillcolor{currentfill}%
\pgfsetfillopacity{0.900000}%
\pgfsetlinewidth{0.507862pt}%
\definecolor{currentstroke}{rgb}{1.000000,1.000000,1.000000}%
\pgfsetstrokecolor{currentstroke}%
\pgfsetstrokeopacity{0.900000}%
\pgfsetdash{}{0pt}%
\pgfpathmoveto{\pgfqpoint{1.487204in}{2.227126in}}%
\pgfpathlineto{\pgfqpoint{1.487204in}{2.165013in}}%
\pgfpathlineto{\pgfqpoint{1.549317in}{2.165013in}}%
\pgfpathlineto{\pgfqpoint{1.549317in}{2.227126in}}%
\pgfpathlineto{\pgfqpoint{1.487204in}{2.227126in}}%
\pgfpathclose%
\pgfusepath{stroke,fill}%
\end{pgfscope}%
\begin{pgfscope}%
\pgfpathrectangle{\pgfqpoint{0.669309in}{0.529443in}}{\pgfqpoint{1.589120in}{1.745990in}}%
\pgfusepath{clip}%
\pgfsetbuttcap%
\pgfsetroundjoin%
\definecolor{currentfill}{rgb}{0.298039,0.447059,0.690196}%
\pgfsetfillcolor{currentfill}%
\pgfsetfillopacity{0.900000}%
\pgfsetlinewidth{0.507862pt}%
\definecolor{currentstroke}{rgb}{1.000000,1.000000,1.000000}%
\pgfsetstrokecolor{currentstroke}%
\pgfsetstrokeopacity{0.900000}%
\pgfsetdash{}{0pt}%
\pgfpathmoveto{\pgfqpoint{2.043670in}{1.191572in}}%
\pgfpathlineto{\pgfqpoint{2.072950in}{1.191572in}}%
\pgfpathlineto{\pgfqpoint{2.072950in}{1.220852in}}%
\pgfpathlineto{\pgfqpoint{2.102230in}{1.220852in}}%
\pgfpathlineto{\pgfqpoint{2.102230in}{1.250133in}}%
\pgfpathlineto{\pgfqpoint{2.072950in}{1.250133in}}%
\pgfpathlineto{\pgfqpoint{2.072950in}{1.279413in}}%
\pgfpathlineto{\pgfqpoint{2.043670in}{1.279413in}}%
\pgfpathlineto{\pgfqpoint{2.043670in}{1.250133in}}%
\pgfpathlineto{\pgfqpoint{2.014389in}{1.250133in}}%
\pgfpathlineto{\pgfqpoint{2.014389in}{1.220852in}}%
\pgfpathlineto{\pgfqpoint{2.043670in}{1.220852in}}%
\pgfpathlineto{\pgfqpoint{2.043670in}{1.191572in}}%
\pgfpathclose%
\pgfusepath{stroke,fill}%
\end{pgfscope}%
\begin{pgfscope}%
\pgfpathrectangle{\pgfqpoint{0.669309in}{0.529443in}}{\pgfqpoint{1.589120in}{1.745990in}}%
\pgfusepath{clip}%
\pgfsetbuttcap%
\pgfsetroundjoin%
\definecolor{currentfill}{rgb}{0.866667,0.517647,0.321569}%
\pgfsetfillcolor{currentfill}%
\pgfsetfillopacity{0.900000}%
\pgfsetlinewidth{0.507862pt}%
\definecolor{currentstroke}{rgb}{1.000000,1.000000,1.000000}%
\pgfsetstrokecolor{currentstroke}%
\pgfsetstrokeopacity{0.900000}%
\pgfsetdash{}{0pt}%
\pgfpathmoveto{\pgfqpoint{0.784251in}{0.564886in}}%
\pgfpathlineto{\pgfqpoint{0.806211in}{0.586846in}}%
\pgfpathlineto{\pgfqpoint{0.828171in}{0.564886in}}%
\pgfpathlineto{\pgfqpoint{0.850132in}{0.586846in}}%
\pgfpathlineto{\pgfqpoint{0.828171in}{0.608806in}}%
\pgfpathlineto{\pgfqpoint{0.850132in}{0.630766in}}%
\pgfpathlineto{\pgfqpoint{0.828171in}{0.652727in}}%
\pgfpathlineto{\pgfqpoint{0.806211in}{0.630766in}}%
\pgfpathlineto{\pgfqpoint{0.784251in}{0.652727in}}%
\pgfpathlineto{\pgfqpoint{0.762291in}{0.630766in}}%
\pgfpathlineto{\pgfqpoint{0.784251in}{0.608806in}}%
\pgfpathlineto{\pgfqpoint{0.762291in}{0.586846in}}%
\pgfpathlineto{\pgfqpoint{0.784251in}{0.564886in}}%
\pgfpathclose%
\pgfusepath{stroke,fill}%
\end{pgfscope}%
\begin{pgfscope}%
\pgfpathrectangle{\pgfqpoint{0.669309in}{0.529443in}}{\pgfqpoint{1.589120in}{1.745990in}}%
\pgfusepath{clip}%
\pgfsetbuttcap%
\pgfsetroundjoin%
\definecolor{currentfill}{rgb}{0.866667,0.517647,0.321569}%
\pgfsetfillcolor{currentfill}%
\pgfsetfillopacity{0.900000}%
\pgfsetlinewidth{0.507862pt}%
\definecolor{currentstroke}{rgb}{1.000000,1.000000,1.000000}%
\pgfsetstrokecolor{currentstroke}%
\pgfsetstrokeopacity{0.900000}%
\pgfsetdash{}{0pt}%
\pgfpathmoveto{\pgfqpoint{1.732141in}{1.839650in}}%
\pgfpathcurveto{\pgfqpoint{1.743789in}{1.839650in}}{\pgfqpoint{1.754961in}{1.844278in}}{\pgfqpoint{1.763197in}{1.852514in}}%
\pgfpathcurveto{\pgfqpoint{1.771434in}{1.860750in}}{\pgfqpoint{1.776061in}{1.871923in}}{\pgfqpoint{1.776061in}{1.883571in}}%
\pgfpathcurveto{\pgfqpoint{1.776061in}{1.895218in}}{\pgfqpoint{1.771434in}{1.906391in}}{\pgfqpoint{1.763197in}{1.914627in}}%
\pgfpathcurveto{\pgfqpoint{1.754961in}{1.922863in}}{\pgfqpoint{1.743789in}{1.927491in}}{\pgfqpoint{1.732141in}{1.927491in}}%
\pgfpathcurveto{\pgfqpoint{1.720493in}{1.927491in}}{\pgfqpoint{1.709321in}{1.922863in}}{\pgfqpoint{1.701084in}{1.914627in}}%
\pgfpathcurveto{\pgfqpoint{1.692848in}{1.906391in}}{\pgfqpoint{1.688220in}{1.895218in}}{\pgfqpoint{1.688220in}{1.883571in}}%
\pgfpathcurveto{\pgfqpoint{1.688220in}{1.871923in}}{\pgfqpoint{1.692848in}{1.860750in}}{\pgfqpoint{1.701084in}{1.852514in}}%
\pgfpathcurveto{\pgfqpoint{1.709321in}{1.844278in}}{\pgfqpoint{1.720493in}{1.839650in}}{\pgfqpoint{1.732141in}{1.839650in}}%
\pgfpathlineto{\pgfqpoint{1.732141in}{1.839650in}}%
\pgfpathclose%
\pgfusepath{stroke,fill}%
\end{pgfscope}%
\begin{pgfscope}%
\pgfpathrectangle{\pgfqpoint{0.669309in}{0.529443in}}{\pgfqpoint{1.589120in}{1.745990in}}%
\pgfusepath{clip}%
\pgfsetbuttcap%
\pgfsetroundjoin%
\definecolor{currentfill}{rgb}{0.333333,0.658824,0.407843}%
\pgfsetfillcolor{currentfill}%
\pgfsetfillopacity{0.900000}%
\pgfsetlinewidth{0.507862pt}%
\definecolor{currentstroke}{rgb}{1.000000,1.000000,1.000000}%
\pgfsetstrokecolor{currentstroke}%
\pgfsetstrokeopacity{0.900000}%
\pgfsetdash{}{0pt}%
\pgfpathmoveto{\pgfqpoint{1.538078in}{1.704520in}}%
\pgfpathlineto{\pgfqpoint{1.538078in}{1.642407in}}%
\pgfpathlineto{\pgfqpoint{1.600191in}{1.642407in}}%
\pgfpathlineto{\pgfqpoint{1.600191in}{1.704520in}}%
\pgfpathlineto{\pgfqpoint{1.538078in}{1.704520in}}%
\pgfpathclose%
\pgfusepath{stroke,fill}%
\end{pgfscope}%
\begin{pgfscope}%
\pgfpathrectangle{\pgfqpoint{0.669309in}{0.529443in}}{\pgfqpoint{1.589120in}{1.745990in}}%
\pgfusepath{clip}%
\pgfsetbuttcap%
\pgfsetroundjoin%
\definecolor{currentfill}{rgb}{0.866667,0.517647,0.321569}%
\pgfsetfillcolor{currentfill}%
\pgfsetfillopacity{0.900000}%
\pgfsetlinewidth{0.507862pt}%
\definecolor{currentstroke}{rgb}{1.000000,1.000000,1.000000}%
\pgfsetstrokecolor{currentstroke}%
\pgfsetstrokeopacity{0.900000}%
\pgfsetdash{}{0pt}%
\pgfpathmoveto{\pgfqpoint{2.110468in}{1.507054in}}%
\pgfpathlineto{\pgfqpoint{2.139749in}{1.507054in}}%
\pgfpathlineto{\pgfqpoint{2.139749in}{1.536334in}}%
\pgfpathlineto{\pgfqpoint{2.169029in}{1.536334in}}%
\pgfpathlineto{\pgfqpoint{2.169029in}{1.565615in}}%
\pgfpathlineto{\pgfqpoint{2.139749in}{1.565615in}}%
\pgfpathlineto{\pgfqpoint{2.139749in}{1.594895in}}%
\pgfpathlineto{\pgfqpoint{2.110468in}{1.594895in}}%
\pgfpathlineto{\pgfqpoint{2.110468in}{1.565615in}}%
\pgfpathlineto{\pgfqpoint{2.081188in}{1.565615in}}%
\pgfpathlineto{\pgfqpoint{2.081188in}{1.536334in}}%
\pgfpathlineto{\pgfqpoint{2.110468in}{1.536334in}}%
\pgfpathlineto{\pgfqpoint{2.110468in}{1.507054in}}%
\pgfpathclose%
\pgfusepath{stroke,fill}%
\end{pgfscope}%
\begin{pgfscope}%
\pgfpathrectangle{\pgfqpoint{0.669309in}{0.529443in}}{\pgfqpoint{1.589120in}{1.745990in}}%
\pgfusepath{clip}%
\pgfsetbuttcap%
\pgfsetroundjoin%
\definecolor{currentfill}{rgb}{0.298039,0.447059,0.690196}%
\pgfsetfillcolor{currentfill}%
\pgfsetfillopacity{0.900000}%
\pgfsetlinewidth{0.507862pt}%
\definecolor{currentstroke}{rgb}{1.000000,1.000000,1.000000}%
\pgfsetstrokecolor{currentstroke}%
\pgfsetstrokeopacity{0.900000}%
\pgfsetdash{}{0pt}%
\pgfpathmoveto{\pgfqpoint{1.957892in}{1.007412in}}%
\pgfpathlineto{\pgfqpoint{1.987172in}{1.007412in}}%
\pgfpathlineto{\pgfqpoint{1.987172in}{1.036692in}}%
\pgfpathlineto{\pgfqpoint{2.016452in}{1.036692in}}%
\pgfpathlineto{\pgfqpoint{2.016452in}{1.065973in}}%
\pgfpathlineto{\pgfqpoint{1.987172in}{1.065973in}}%
\pgfpathlineto{\pgfqpoint{1.987172in}{1.095253in}}%
\pgfpathlineto{\pgfqpoint{1.957892in}{1.095253in}}%
\pgfpathlineto{\pgfqpoint{1.957892in}{1.065973in}}%
\pgfpathlineto{\pgfqpoint{1.928611in}{1.065973in}}%
\pgfpathlineto{\pgfqpoint{1.928611in}{1.036692in}}%
\pgfpathlineto{\pgfqpoint{1.957892in}{1.036692in}}%
\pgfpathlineto{\pgfqpoint{1.957892in}{1.007412in}}%
\pgfpathclose%
\pgfusepath{stroke,fill}%
\end{pgfscope}%
\begin{pgfscope}%
\pgfpathrectangle{\pgfqpoint{0.669309in}{0.529443in}}{\pgfqpoint{1.589120in}{1.745990in}}%
\pgfusepath{clip}%
\pgfsetbuttcap%
\pgfsetroundjoin%
\definecolor{currentfill}{rgb}{0.866667,0.517647,0.321569}%
\pgfsetfillcolor{currentfill}%
\pgfsetfillopacity{0.900000}%
\pgfsetlinewidth{0.507862pt}%
\definecolor{currentstroke}{rgb}{1.000000,1.000000,1.000000}%
\pgfsetstrokecolor{currentstroke}%
\pgfsetstrokeopacity{0.900000}%
\pgfsetdash{}{0pt}%
\pgfpathmoveto{\pgfqpoint{1.579290in}{1.927390in}}%
\pgfpathlineto{\pgfqpoint{1.579290in}{1.865277in}}%
\pgfpathlineto{\pgfqpoint{1.641403in}{1.865277in}}%
\pgfpathlineto{\pgfqpoint{1.641403in}{1.927390in}}%
\pgfpathlineto{\pgfqpoint{1.579290in}{1.927390in}}%
\pgfpathclose%
\pgfusepath{stroke,fill}%
\end{pgfscope}%
\begin{pgfscope}%
\pgfpathrectangle{\pgfqpoint{0.669309in}{0.529443in}}{\pgfqpoint{1.589120in}{1.745990in}}%
\pgfusepath{clip}%
\pgfsetbuttcap%
\pgfsetroundjoin%
\definecolor{currentfill}{rgb}{0.866667,0.517647,0.321569}%
\pgfsetfillcolor{currentfill}%
\pgfsetfillopacity{0.900000}%
\pgfsetlinewidth{0.507862pt}%
\definecolor{currentstroke}{rgb}{1.000000,1.000000,1.000000}%
\pgfsetstrokecolor{currentstroke}%
\pgfsetstrokeopacity{0.900000}%
\pgfsetdash{}{0pt}%
\pgfpathmoveto{\pgfqpoint{1.234435in}{1.830391in}}%
\pgfpathcurveto{\pgfqpoint{1.246082in}{1.830391in}}{\pgfqpoint{1.257255in}{1.835018in}}{\pgfqpoint{1.265491in}{1.843255in}}%
\pgfpathcurveto{\pgfqpoint{1.273727in}{1.851491in}}{\pgfqpoint{1.278355in}{1.862663in}}{\pgfqpoint{1.278355in}{1.874311in}}%
\pgfpathcurveto{\pgfqpoint{1.278355in}{1.885959in}}{\pgfqpoint{1.273727in}{1.897131in}}{\pgfqpoint{1.265491in}{1.905368in}}%
\pgfpathcurveto{\pgfqpoint{1.257255in}{1.913604in}}{\pgfqpoint{1.246082in}{1.918232in}}{\pgfqpoint{1.234435in}{1.918232in}}%
\pgfpathcurveto{\pgfqpoint{1.222787in}{1.918232in}}{\pgfqpoint{1.211614in}{1.913604in}}{\pgfqpoint{1.203378in}{1.905368in}}%
\pgfpathcurveto{\pgfqpoint{1.195142in}{1.897131in}}{\pgfqpoint{1.190514in}{1.885959in}}{\pgfqpoint{1.190514in}{1.874311in}}%
\pgfpathcurveto{\pgfqpoint{1.190514in}{1.862663in}}{\pgfqpoint{1.195142in}{1.851491in}}{\pgfqpoint{1.203378in}{1.843255in}}%
\pgfpathcurveto{\pgfqpoint{1.211614in}{1.835018in}}{\pgfqpoint{1.222787in}{1.830391in}}{\pgfqpoint{1.234435in}{1.830391in}}%
\pgfpathlineto{\pgfqpoint{1.234435in}{1.830391in}}%
\pgfpathclose%
\pgfusepath{stroke,fill}%
\end{pgfscope}%
\begin{pgfscope}%
\pgfpathrectangle{\pgfqpoint{0.669309in}{0.529443in}}{\pgfqpoint{1.589120in}{1.745990in}}%
\pgfusepath{clip}%
\pgfsetbuttcap%
\pgfsetroundjoin%
\definecolor{currentfill}{rgb}{0.333333,0.658824,0.407843}%
\pgfsetfillcolor{currentfill}%
\pgfsetfillopacity{0.900000}%
\pgfsetlinewidth{0.507862pt}%
\definecolor{currentstroke}{rgb}{1.000000,1.000000,1.000000}%
\pgfsetstrokecolor{currentstroke}%
\pgfsetstrokeopacity{0.900000}%
\pgfsetdash{}{0pt}%
\pgfpathmoveto{\pgfqpoint{1.994916in}{0.765837in}}%
\pgfpathlineto{\pgfqpoint{2.024196in}{0.765837in}}%
\pgfpathlineto{\pgfqpoint{2.024196in}{0.795117in}}%
\pgfpathlineto{\pgfqpoint{2.053477in}{0.795117in}}%
\pgfpathlineto{\pgfqpoint{2.053477in}{0.824397in}}%
\pgfpathlineto{\pgfqpoint{2.024196in}{0.824397in}}%
\pgfpathlineto{\pgfqpoint{2.024196in}{0.853678in}}%
\pgfpathlineto{\pgfqpoint{1.994916in}{0.853678in}}%
\pgfpathlineto{\pgfqpoint{1.994916in}{0.824397in}}%
\pgfpathlineto{\pgfqpoint{1.965636in}{0.824397in}}%
\pgfpathlineto{\pgfqpoint{1.965636in}{0.795117in}}%
\pgfpathlineto{\pgfqpoint{1.994916in}{0.795117in}}%
\pgfpathlineto{\pgfqpoint{1.994916in}{0.765837in}}%
\pgfpathclose%
\pgfusepath{stroke,fill}%
\end{pgfscope}%
\begin{pgfscope}%
\pgfpathrectangle{\pgfqpoint{0.669309in}{0.529443in}}{\pgfqpoint{1.589120in}{1.745990in}}%
\pgfusepath{clip}%
\pgfsetbuttcap%
\pgfsetroundjoin%
\definecolor{currentfill}{rgb}{0.333333,0.658824,0.407843}%
\pgfsetfillcolor{currentfill}%
\pgfsetfillopacity{0.900000}%
\pgfsetlinewidth{0.507862pt}%
\definecolor{currentstroke}{rgb}{1.000000,1.000000,1.000000}%
\pgfsetstrokecolor{currentstroke}%
\pgfsetstrokeopacity{0.900000}%
\pgfsetdash{}{0pt}%
\pgfpathmoveto{\pgfqpoint{1.180903in}{1.690619in}}%
\pgfpathcurveto{\pgfqpoint{1.192551in}{1.690619in}}{\pgfqpoint{1.203723in}{1.695247in}}{\pgfqpoint{1.211960in}{1.703483in}}%
\pgfpathcurveto{\pgfqpoint{1.220196in}{1.711719in}}{\pgfqpoint{1.224824in}{1.722892in}}{\pgfqpoint{1.224824in}{1.734540in}}%
\pgfpathcurveto{\pgfqpoint{1.224824in}{1.746188in}}{\pgfqpoint{1.220196in}{1.757360in}}{\pgfqpoint{1.211960in}{1.765596in}}%
\pgfpathcurveto{\pgfqpoint{1.203723in}{1.773832in}}{\pgfqpoint{1.192551in}{1.778460in}}{\pgfqpoint{1.180903in}{1.778460in}}%
\pgfpathcurveto{\pgfqpoint{1.169255in}{1.778460in}}{\pgfqpoint{1.158083in}{1.773832in}}{\pgfqpoint{1.149847in}{1.765596in}}%
\pgfpathcurveto{\pgfqpoint{1.141610in}{1.757360in}}{\pgfqpoint{1.136983in}{1.746188in}}{\pgfqpoint{1.136983in}{1.734540in}}%
\pgfpathcurveto{\pgfqpoint{1.136983in}{1.722892in}}{\pgfqpoint{1.141610in}{1.711719in}}{\pgfqpoint{1.149847in}{1.703483in}}%
\pgfpathcurveto{\pgfqpoint{1.158083in}{1.695247in}}{\pgfqpoint{1.169255in}{1.690619in}}{\pgfqpoint{1.180903in}{1.690619in}}%
\pgfpathlineto{\pgfqpoint{1.180903in}{1.690619in}}%
\pgfpathclose%
\pgfusepath{stroke,fill}%
\end{pgfscope}%
\begin{pgfscope}%
\pgfpathrectangle{\pgfqpoint{0.669309in}{0.529443in}}{\pgfqpoint{1.589120in}{1.745990in}}%
\pgfusepath{clip}%
\pgfsetbuttcap%
\pgfsetroundjoin%
\definecolor{currentfill}{rgb}{0.298039,0.447059,0.690196}%
\pgfsetfillcolor{currentfill}%
\pgfsetfillopacity{0.900000}%
\pgfsetlinewidth{0.507862pt}%
\definecolor{currentstroke}{rgb}{1.000000,1.000000,1.000000}%
\pgfsetstrokecolor{currentstroke}%
\pgfsetstrokeopacity{0.900000}%
\pgfsetdash{}{0pt}%
\pgfpathmoveto{\pgfqpoint{2.109135in}{1.653023in}}%
\pgfpathcurveto{\pgfqpoint{2.120783in}{1.653023in}}{\pgfqpoint{2.131955in}{1.657651in}}{\pgfqpoint{2.140192in}{1.665887in}}%
\pgfpathcurveto{\pgfqpoint{2.148428in}{1.674124in}}{\pgfqpoint{2.153056in}{1.685296in}}{\pgfqpoint{2.153056in}{1.696944in}}%
\pgfpathcurveto{\pgfqpoint{2.153056in}{1.708592in}}{\pgfqpoint{2.148428in}{1.719764in}}{\pgfqpoint{2.140192in}{1.728000in}}%
\pgfpathcurveto{\pgfqpoint{2.131955in}{1.736237in}}{\pgfqpoint{2.120783in}{1.740864in}}{\pgfqpoint{2.109135in}{1.740864in}}%
\pgfpathcurveto{\pgfqpoint{2.097487in}{1.740864in}}{\pgfqpoint{2.086315in}{1.736237in}}{\pgfqpoint{2.078079in}{1.728000in}}%
\pgfpathcurveto{\pgfqpoint{2.069842in}{1.719764in}}{\pgfqpoint{2.065215in}{1.708592in}}{\pgfqpoint{2.065215in}{1.696944in}}%
\pgfpathcurveto{\pgfqpoint{2.065215in}{1.685296in}}{\pgfqpoint{2.069842in}{1.674124in}}{\pgfqpoint{2.078079in}{1.665887in}}%
\pgfpathcurveto{\pgfqpoint{2.086315in}{1.657651in}}{\pgfqpoint{2.097487in}{1.653023in}}{\pgfqpoint{2.109135in}{1.653023in}}%
\pgfpathlineto{\pgfqpoint{2.109135in}{1.653023in}}%
\pgfpathclose%
\pgfusepath{stroke,fill}%
\end{pgfscope}%
\begin{pgfscope}%
\pgfpathrectangle{\pgfqpoint{0.669309in}{0.529443in}}{\pgfqpoint{1.589120in}{1.745990in}}%
\pgfusepath{clip}%
\pgfsetbuttcap%
\pgfsetroundjoin%
\definecolor{currentfill}{rgb}{0.866667,0.517647,0.321569}%
\pgfsetfillcolor{currentfill}%
\pgfsetfillopacity{0.900000}%
\pgfsetlinewidth{0.507862pt}%
\definecolor{currentstroke}{rgb}{1.000000,1.000000,1.000000}%
\pgfsetstrokecolor{currentstroke}%
\pgfsetstrokeopacity{0.900000}%
\pgfsetdash{}{0pt}%
\pgfpathmoveto{\pgfqpoint{0.924465in}{1.495467in}}%
\pgfpathcurveto{\pgfqpoint{0.936112in}{1.495467in}}{\pgfqpoint{0.947285in}{1.500095in}}{\pgfqpoint{0.955521in}{1.508331in}}%
\pgfpathcurveto{\pgfqpoint{0.963757in}{1.516567in}}{\pgfqpoint{0.968385in}{1.527740in}}{\pgfqpoint{0.968385in}{1.539387in}}%
\pgfpathcurveto{\pgfqpoint{0.968385in}{1.551035in}}{\pgfqpoint{0.963757in}{1.562208in}}{\pgfqpoint{0.955521in}{1.570444in}}%
\pgfpathcurveto{\pgfqpoint{0.947285in}{1.578680in}}{\pgfqpoint{0.936112in}{1.583308in}}{\pgfqpoint{0.924465in}{1.583308in}}%
\pgfpathcurveto{\pgfqpoint{0.912817in}{1.583308in}}{\pgfqpoint{0.901644in}{1.578680in}}{\pgfqpoint{0.893408in}{1.570444in}}%
\pgfpathcurveto{\pgfqpoint{0.885172in}{1.562208in}}{\pgfqpoint{0.880544in}{1.551035in}}{\pgfqpoint{0.880544in}{1.539387in}}%
\pgfpathcurveto{\pgfqpoint{0.880544in}{1.527740in}}{\pgfqpoint{0.885172in}{1.516567in}}{\pgfqpoint{0.893408in}{1.508331in}}%
\pgfpathcurveto{\pgfqpoint{0.901644in}{1.500095in}}{\pgfqpoint{0.912817in}{1.495467in}}{\pgfqpoint{0.924465in}{1.495467in}}%
\pgfpathlineto{\pgfqpoint{0.924465in}{1.495467in}}%
\pgfpathclose%
\pgfusepath{stroke,fill}%
\end{pgfscope}%
\begin{pgfscope}%
\pgfpathrectangle{\pgfqpoint{0.669309in}{0.529443in}}{\pgfqpoint{1.589120in}{1.745990in}}%
\pgfusepath{clip}%
\pgfsetbuttcap%
\pgfsetroundjoin%
\definecolor{currentfill}{rgb}{0.298039,0.447059,0.690196}%
\pgfsetfillcolor{currentfill}%
\pgfsetfillopacity{0.900000}%
\pgfsetlinewidth{0.507862pt}%
\definecolor{currentstroke}{rgb}{1.000000,1.000000,1.000000}%
\pgfsetstrokecolor{currentstroke}%
\pgfsetstrokeopacity{0.900000}%
\pgfsetdash{}{0pt}%
\pgfpathmoveto{\pgfqpoint{1.480996in}{1.939838in}}%
\pgfpathlineto{\pgfqpoint{1.480996in}{1.877725in}}%
\pgfpathlineto{\pgfqpoint{1.543109in}{1.877725in}}%
\pgfpathlineto{\pgfqpoint{1.543109in}{1.939838in}}%
\pgfpathlineto{\pgfqpoint{1.480996in}{1.939838in}}%
\pgfpathclose%
\pgfusepath{stroke,fill}%
\end{pgfscope}%
\begin{pgfscope}%
\pgfpathrectangle{\pgfqpoint{0.669309in}{0.529443in}}{\pgfqpoint{1.589120in}{1.745990in}}%
\pgfusepath{clip}%
\pgfsetbuttcap%
\pgfsetroundjoin%
\definecolor{currentfill}{rgb}{0.333333,0.658824,0.407843}%
\pgfsetfillcolor{currentfill}%
\pgfsetfillopacity{0.900000}%
\pgfsetlinewidth{0.507862pt}%
\definecolor{currentstroke}{rgb}{1.000000,1.000000,1.000000}%
\pgfsetstrokecolor{currentstroke}%
\pgfsetstrokeopacity{0.900000}%
\pgfsetdash{}{0pt}%
\pgfpathmoveto{\pgfqpoint{0.896536in}{1.279658in}}%
\pgfpathlineto{\pgfqpoint{0.918496in}{1.301618in}}%
\pgfpathlineto{\pgfqpoint{0.940456in}{1.279658in}}%
\pgfpathlineto{\pgfqpoint{0.962417in}{1.301618in}}%
\pgfpathlineto{\pgfqpoint{0.940456in}{1.323579in}}%
\pgfpathlineto{\pgfqpoint{0.962417in}{1.345539in}}%
\pgfpathlineto{\pgfqpoint{0.940456in}{1.367499in}}%
\pgfpathlineto{\pgfqpoint{0.918496in}{1.345539in}}%
\pgfpathlineto{\pgfqpoint{0.896536in}{1.367499in}}%
\pgfpathlineto{\pgfqpoint{0.874576in}{1.345539in}}%
\pgfpathlineto{\pgfqpoint{0.896536in}{1.323579in}}%
\pgfpathlineto{\pgfqpoint{0.874576in}{1.301618in}}%
\pgfpathlineto{\pgfqpoint{0.896536in}{1.279658in}}%
\pgfpathclose%
\pgfusepath{stroke,fill}%
\end{pgfscope}%
\begin{pgfscope}%
\pgfpathrectangle{\pgfqpoint{0.669309in}{0.529443in}}{\pgfqpoint{1.589120in}{1.745990in}}%
\pgfusepath{clip}%
\pgfsetbuttcap%
\pgfsetroundjoin%
\definecolor{currentfill}{rgb}{0.333333,0.658824,0.407843}%
\pgfsetfillcolor{currentfill}%
\pgfsetfillopacity{0.900000}%
\pgfsetlinewidth{0.507862pt}%
\definecolor{currentstroke}{rgb}{1.000000,1.000000,1.000000}%
\pgfsetstrokecolor{currentstroke}%
\pgfsetstrokeopacity{0.900000}%
\pgfsetdash{}{0pt}%
\pgfpathmoveto{\pgfqpoint{1.646837in}{2.022872in}}%
\pgfpathcurveto{\pgfqpoint{1.658485in}{2.022872in}}{\pgfqpoint{1.669657in}{2.027500in}}{\pgfqpoint{1.677894in}{2.035736in}}%
\pgfpathcurveto{\pgfqpoint{1.686130in}{2.043973in}}{\pgfqpoint{1.690758in}{2.055145in}}{\pgfqpoint{1.690758in}{2.066793in}}%
\pgfpathcurveto{\pgfqpoint{1.690758in}{2.078441in}}{\pgfqpoint{1.686130in}{2.089613in}}{\pgfqpoint{1.677894in}{2.097849in}}%
\pgfpathcurveto{\pgfqpoint{1.669657in}{2.106086in}}{\pgfqpoint{1.658485in}{2.110713in}}{\pgfqpoint{1.646837in}{2.110713in}}%
\pgfpathcurveto{\pgfqpoint{1.635189in}{2.110713in}}{\pgfqpoint{1.624017in}{2.106086in}}{\pgfqpoint{1.615781in}{2.097849in}}%
\pgfpathcurveto{\pgfqpoint{1.607544in}{2.089613in}}{\pgfqpoint{1.602917in}{2.078441in}}{\pgfqpoint{1.602917in}{2.066793in}}%
\pgfpathcurveto{\pgfqpoint{1.602917in}{2.055145in}}{\pgfqpoint{1.607544in}{2.043973in}}{\pgfqpoint{1.615781in}{2.035736in}}%
\pgfpathcurveto{\pgfqpoint{1.624017in}{2.027500in}}{\pgfqpoint{1.635189in}{2.022872in}}{\pgfqpoint{1.646837in}{2.022872in}}%
\pgfpathlineto{\pgfqpoint{1.646837in}{2.022872in}}%
\pgfpathclose%
\pgfusepath{stroke,fill}%
\end{pgfscope}%
\begin{pgfscope}%
\pgfpathrectangle{\pgfqpoint{0.669309in}{0.529443in}}{\pgfqpoint{1.589120in}{1.745990in}}%
\pgfusepath{clip}%
\pgfsetbuttcap%
\pgfsetroundjoin%
\definecolor{currentfill}{rgb}{0.333333,0.658824,0.407843}%
\pgfsetfillcolor{currentfill}%
\pgfsetfillopacity{0.900000}%
\pgfsetlinewidth{0.507862pt}%
\definecolor{currentstroke}{rgb}{1.000000,1.000000,1.000000}%
\pgfsetstrokecolor{currentstroke}%
\pgfsetstrokeopacity{0.900000}%
\pgfsetdash{}{0pt}%
\pgfpathmoveto{\pgfqpoint{0.764777in}{0.822349in}}%
\pgfpathlineto{\pgfqpoint{0.786737in}{0.844310in}}%
\pgfpathlineto{\pgfqpoint{0.808697in}{0.822349in}}%
\pgfpathlineto{\pgfqpoint{0.830658in}{0.844310in}}%
\pgfpathlineto{\pgfqpoint{0.808697in}{0.866270in}}%
\pgfpathlineto{\pgfqpoint{0.830658in}{0.888230in}}%
\pgfpathlineto{\pgfqpoint{0.808697in}{0.910190in}}%
\pgfpathlineto{\pgfqpoint{0.786737in}{0.888230in}}%
\pgfpathlineto{\pgfqpoint{0.764777in}{0.910190in}}%
\pgfpathlineto{\pgfqpoint{0.742817in}{0.888230in}}%
\pgfpathlineto{\pgfqpoint{0.764777in}{0.866270in}}%
\pgfpathlineto{\pgfqpoint{0.742817in}{0.844310in}}%
\pgfpathlineto{\pgfqpoint{0.764777in}{0.822349in}}%
\pgfpathclose%
\pgfusepath{stroke,fill}%
\end{pgfscope}%
\begin{pgfscope}%
\pgfpathrectangle{\pgfqpoint{0.669309in}{0.529443in}}{\pgfqpoint{1.589120in}{1.745990in}}%
\pgfusepath{clip}%
\pgfsetbuttcap%
\pgfsetroundjoin%
\definecolor{currentfill}{rgb}{0.298039,0.447059,0.690196}%
\pgfsetfillcolor{currentfill}%
\pgfsetfillopacity{0.900000}%
\pgfsetlinewidth{0.507862pt}%
\definecolor{currentstroke}{rgb}{1.000000,1.000000,1.000000}%
\pgfsetstrokecolor{currentstroke}%
\pgfsetstrokeopacity{0.900000}%
\pgfsetdash{}{0pt}%
\pgfpathmoveto{\pgfqpoint{2.029416in}{1.408923in}}%
\pgfpathlineto{\pgfqpoint{2.058697in}{1.408923in}}%
\pgfpathlineto{\pgfqpoint{2.058697in}{1.438203in}}%
\pgfpathlineto{\pgfqpoint{2.087977in}{1.438203in}}%
\pgfpathlineto{\pgfqpoint{2.087977in}{1.467483in}}%
\pgfpathlineto{\pgfqpoint{2.058697in}{1.467483in}}%
\pgfpathlineto{\pgfqpoint{2.058697in}{1.496764in}}%
\pgfpathlineto{\pgfqpoint{2.029416in}{1.496764in}}%
\pgfpathlineto{\pgfqpoint{2.029416in}{1.467483in}}%
\pgfpathlineto{\pgfqpoint{2.000136in}{1.467483in}}%
\pgfpathlineto{\pgfqpoint{2.000136in}{1.438203in}}%
\pgfpathlineto{\pgfqpoint{2.029416in}{1.438203in}}%
\pgfpathlineto{\pgfqpoint{2.029416in}{1.408923in}}%
\pgfpathclose%
\pgfusepath{stroke,fill}%
\end{pgfscope}%
\begin{pgfscope}%
\pgfpathrectangle{\pgfqpoint{0.669309in}{0.529443in}}{\pgfqpoint{1.589120in}{1.745990in}}%
\pgfusepath{clip}%
\pgfsetbuttcap%
\pgfsetroundjoin%
\definecolor{currentfill}{rgb}{0.333333,0.658824,0.407843}%
\pgfsetfillcolor{currentfill}%
\pgfsetfillopacity{0.900000}%
\pgfsetlinewidth{0.507862pt}%
\definecolor{currentstroke}{rgb}{1.000000,1.000000,1.000000}%
\pgfsetstrokecolor{currentstroke}%
\pgfsetstrokeopacity{0.900000}%
\pgfsetdash{}{0pt}%
\pgfpathmoveto{\pgfqpoint{1.291636in}{1.999629in}}%
\pgfpathlineto{\pgfqpoint{1.291636in}{1.937516in}}%
\pgfpathlineto{\pgfqpoint{1.353749in}{1.937516in}}%
\pgfpathlineto{\pgfqpoint{1.353749in}{1.999629in}}%
\pgfpathlineto{\pgfqpoint{1.291636in}{1.999629in}}%
\pgfpathclose%
\pgfusepath{stroke,fill}%
\end{pgfscope}%
\begin{pgfscope}%
\pgfsetrectcap%
\pgfsetmiterjoin%
\pgfsetlinewidth{1.254687pt}%
\definecolor{currentstroke}{rgb}{0.800000,0.800000,0.800000}%
\pgfsetstrokecolor{currentstroke}%
\pgfsetdash{}{0pt}%
\pgfpathmoveto{\pgfqpoint{0.669309in}{0.529443in}}%
\pgfpathlineto{\pgfqpoint{0.669309in}{2.275433in}}%
\pgfusepath{stroke}%
\end{pgfscope}%
\begin{pgfscope}%
\pgfsetrectcap%
\pgfsetmiterjoin%
\pgfsetlinewidth{1.254687pt}%
\definecolor{currentstroke}{rgb}{0.800000,0.800000,0.800000}%
\pgfsetstrokecolor{currentstroke}%
\pgfsetdash{}{0pt}%
\pgfpathmoveto{\pgfqpoint{2.258429in}{0.529443in}}%
\pgfpathlineto{\pgfqpoint{2.258429in}{2.275433in}}%
\pgfusepath{stroke}%
\end{pgfscope}%
\begin{pgfscope}%
\pgfsetrectcap%
\pgfsetmiterjoin%
\pgfsetlinewidth{1.254687pt}%
\definecolor{currentstroke}{rgb}{0.800000,0.800000,0.800000}%
\pgfsetstrokecolor{currentstroke}%
\pgfsetdash{}{0pt}%
\pgfpathmoveto{\pgfqpoint{0.669309in}{0.529443in}}%
\pgfpathlineto{\pgfqpoint{2.258429in}{0.529443in}}%
\pgfusepath{stroke}%
\end{pgfscope}%
\begin{pgfscope}%
\pgfsetrectcap%
\pgfsetmiterjoin%
\pgfsetlinewidth{1.254687pt}%
\definecolor{currentstroke}{rgb}{0.800000,0.800000,0.800000}%
\pgfsetstrokecolor{currentstroke}%
\pgfsetdash{}{0pt}%
\pgfpathmoveto{\pgfqpoint{0.669309in}{2.275433in}}%
\pgfpathlineto{\pgfqpoint{2.258429in}{2.275433in}}%
\pgfusepath{stroke}%
\end{pgfscope}%
\begin{pgfscope}%
\definecolor{textcolor}{rgb}{0.150000,0.150000,0.150000}%
\pgfsetstrokecolor{textcolor}%
\pgfsetfillcolor{textcolor}%
\pgftext[x=1.463869in,y=2.358766in,,base]{\color{textcolor}{\rmfamily\fontsize{11.000000}{13.200000}\selectfont\catcode`\^=\active\def^{\ifmmode\sp\else\^{}\fi}\catcode`\%=\active\def%{\%}Graph2Vec}}%
\end{pgfscope}%
\begin{pgfscope}%
\pgfsetbuttcap%
\pgfsetmiterjoin%
\definecolor{currentfill}{rgb}{1.000000,1.000000,1.000000}%
\pgfsetfillcolor{currentfill}%
\pgfsetlinewidth{0.000000pt}%
\definecolor{currentstroke}{rgb}{0.000000,0.000000,0.000000}%
\pgfsetstrokecolor{currentstroke}%
\pgfsetstrokeopacity{0.000000}%
\pgfsetdash{}{0pt}%
\pgfpathmoveto{\pgfqpoint{2.811623in}{0.529443in}}%
\pgfpathlineto{\pgfqpoint{4.400744in}{0.529443in}}%
\pgfpathlineto{\pgfqpoint{4.400744in}{2.275433in}}%
\pgfpathlineto{\pgfqpoint{2.811623in}{2.275433in}}%
\pgfpathlineto{\pgfqpoint{2.811623in}{0.529443in}}%
\pgfpathclose%
\pgfusepath{fill}%
\end{pgfscope}%
\begin{pgfscope}%
\pgfpathrectangle{\pgfqpoint{2.811623in}{0.529443in}}{\pgfqpoint{1.589120in}{1.745990in}}%
\pgfusepath{clip}%
\pgfsetroundcap%
\pgfsetroundjoin%
\pgfsetlinewidth{1.003750pt}%
\definecolor{currentstroke}{rgb}{0.800000,0.800000,0.800000}%
\pgfsetstrokecolor{currentstroke}%
\pgfsetdash{}{0pt}%
\pgfpathmoveto{\pgfqpoint{3.131113in}{0.529443in}}%
\pgfpathlineto{\pgfqpoint{3.131113in}{2.275433in}}%
\pgfusepath{stroke}%
\end{pgfscope}%
\begin{pgfscope}%
\definecolor{textcolor}{rgb}{0.150000,0.150000,0.150000}%
\pgfsetstrokecolor{textcolor}%
\pgfsetfillcolor{textcolor}%
\pgftext[x=3.131113in,y=0.397499in,,top]{\color{textcolor}{\rmfamily\fontsize{8.000000}{9.600000}\selectfont\catcode`\^=\active\def^{\ifmmode\sp\else\^{}\fi}\catcode`\%=\active\def%{\%}\ensuremath{-}5}}%
\end{pgfscope}%
\begin{pgfscope}%
\pgfpathrectangle{\pgfqpoint{2.811623in}{0.529443in}}{\pgfqpoint{1.589120in}{1.745990in}}%
\pgfusepath{clip}%
\pgfsetroundcap%
\pgfsetroundjoin%
\pgfsetlinewidth{1.003750pt}%
\definecolor{currentstroke}{rgb}{0.800000,0.800000,0.800000}%
\pgfsetstrokecolor{currentstroke}%
\pgfsetdash{}{0pt}%
\pgfpathmoveto{\pgfqpoint{3.721279in}{0.529443in}}%
\pgfpathlineto{\pgfqpoint{3.721279in}{2.275433in}}%
\pgfusepath{stroke}%
\end{pgfscope}%
\begin{pgfscope}%
\definecolor{textcolor}{rgb}{0.150000,0.150000,0.150000}%
\pgfsetstrokecolor{textcolor}%
\pgfsetfillcolor{textcolor}%
\pgftext[x=3.721279in,y=0.397499in,,top]{\color{textcolor}{\rmfamily\fontsize{8.000000}{9.600000}\selectfont\catcode`\^=\active\def^{\ifmmode\sp\else\^{}\fi}\catcode`\%=\active\def%{\%}0}}%
\end{pgfscope}%
\begin{pgfscope}%
\pgfpathrectangle{\pgfqpoint{2.811623in}{0.529443in}}{\pgfqpoint{1.589120in}{1.745990in}}%
\pgfusepath{clip}%
\pgfsetroundcap%
\pgfsetroundjoin%
\pgfsetlinewidth{1.003750pt}%
\definecolor{currentstroke}{rgb}{0.800000,0.800000,0.800000}%
\pgfsetstrokecolor{currentstroke}%
\pgfsetdash{}{0pt}%
\pgfpathmoveto{\pgfqpoint{4.311445in}{0.529443in}}%
\pgfpathlineto{\pgfqpoint{4.311445in}{2.275433in}}%
\pgfusepath{stroke}%
\end{pgfscope}%
\begin{pgfscope}%
\definecolor{textcolor}{rgb}{0.150000,0.150000,0.150000}%
\pgfsetstrokecolor{textcolor}%
\pgfsetfillcolor{textcolor}%
\pgftext[x=4.311445in,y=0.397499in,,top]{\color{textcolor}{\rmfamily\fontsize{8.000000}{9.600000}\selectfont\catcode`\^=\active\def^{\ifmmode\sp\else\^{}\fi}\catcode`\%=\active\def%{\%}5}}%
\end{pgfscope}%
\begin{pgfscope}%
\definecolor{textcolor}{rgb}{0.150000,0.150000,0.150000}%
\pgfsetstrokecolor{textcolor}%
\pgfsetfillcolor{textcolor}%
\pgftext[x=3.606184in,y=0.234413in,,top]{\color{textcolor}{\rmfamily\fontsize{10.000000}{12.000000}\selectfont\catcode`\^=\active\def^{\ifmmode\sp\else\^{}\fi}\catcode`\%=\active\def%{\%}UMAP 1}}%
\end{pgfscope}%
\begin{pgfscope}%
\pgfpathrectangle{\pgfqpoint{2.811623in}{0.529443in}}{\pgfqpoint{1.589120in}{1.745990in}}%
\pgfusepath{clip}%
\pgfsetroundcap%
\pgfsetroundjoin%
\pgfsetlinewidth{1.003750pt}%
\definecolor{currentstroke}{rgb}{0.800000,0.800000,0.800000}%
\pgfsetstrokecolor{currentstroke}%
\pgfsetdash{}{0pt}%
\pgfpathmoveto{\pgfqpoint{2.811623in}{0.763133in}}%
\pgfpathlineto{\pgfqpoint{4.400744in}{0.763133in}}%
\pgfusepath{stroke}%
\end{pgfscope}%
\begin{pgfscope}%
\definecolor{textcolor}{rgb}{0.150000,0.150000,0.150000}%
\pgfsetstrokecolor{textcolor}%
\pgfsetfillcolor{textcolor}%
\pgftext[x=2.517164in, y=0.720923in, left, base]{\color{textcolor}{\rmfamily\fontsize{8.000000}{9.600000}\selectfont\catcode`\^=\active\def^{\ifmmode\sp\else\^{}\fi}\catcode`\%=\active\def%{\%}\ensuremath{-}5}}%
\end{pgfscope}%
\begin{pgfscope}%
\pgfpathrectangle{\pgfqpoint{2.811623in}{0.529443in}}{\pgfqpoint{1.589120in}{1.745990in}}%
\pgfusepath{clip}%
\pgfsetroundcap%
\pgfsetroundjoin%
\pgfsetlinewidth{1.003750pt}%
\definecolor{currentstroke}{rgb}{0.800000,0.800000,0.800000}%
\pgfsetstrokecolor{currentstroke}%
\pgfsetdash{}{0pt}%
\pgfpathmoveto{\pgfqpoint{2.811623in}{1.153263in}}%
\pgfpathlineto{\pgfqpoint{4.400744in}{1.153263in}}%
\pgfusepath{stroke}%
\end{pgfscope}%
\begin{pgfscope}%
\definecolor{textcolor}{rgb}{0.150000,0.150000,0.150000}%
\pgfsetstrokecolor{textcolor}%
\pgfsetfillcolor{textcolor}%
\pgftext[x=2.608987in, y=1.111054in, left, base]{\color{textcolor}{\rmfamily\fontsize{8.000000}{9.600000}\selectfont\catcode`\^=\active\def^{\ifmmode\sp\else\^{}\fi}\catcode`\%=\active\def%{\%}0}}%
\end{pgfscope}%
\begin{pgfscope}%
\pgfpathrectangle{\pgfqpoint{2.811623in}{0.529443in}}{\pgfqpoint{1.589120in}{1.745990in}}%
\pgfusepath{clip}%
\pgfsetroundcap%
\pgfsetroundjoin%
\pgfsetlinewidth{1.003750pt}%
\definecolor{currentstroke}{rgb}{0.800000,0.800000,0.800000}%
\pgfsetstrokecolor{currentstroke}%
\pgfsetdash{}{0pt}%
\pgfpathmoveto{\pgfqpoint{2.811623in}{1.543394in}}%
\pgfpathlineto{\pgfqpoint{4.400744in}{1.543394in}}%
\pgfusepath{stroke}%
\end{pgfscope}%
\begin{pgfscope}%
\definecolor{textcolor}{rgb}{0.150000,0.150000,0.150000}%
\pgfsetstrokecolor{textcolor}%
\pgfsetfillcolor{textcolor}%
\pgftext[x=2.608987in, y=1.501185in, left, base]{\color{textcolor}{\rmfamily\fontsize{8.000000}{9.600000}\selectfont\catcode`\^=\active\def^{\ifmmode\sp\else\^{}\fi}\catcode`\%=\active\def%{\%}5}}%
\end{pgfscope}%
\begin{pgfscope}%
\pgfpathrectangle{\pgfqpoint{2.811623in}{0.529443in}}{\pgfqpoint{1.589120in}{1.745990in}}%
\pgfusepath{clip}%
\pgfsetroundcap%
\pgfsetroundjoin%
\pgfsetlinewidth{1.003750pt}%
\definecolor{currentstroke}{rgb}{0.800000,0.800000,0.800000}%
\pgfsetstrokecolor{currentstroke}%
\pgfsetdash{}{0pt}%
\pgfpathmoveto{\pgfqpoint{2.811623in}{1.933525in}}%
\pgfpathlineto{\pgfqpoint{4.400744in}{1.933525in}}%
\pgfusepath{stroke}%
\end{pgfscope}%
\begin{pgfscope}%
\definecolor{textcolor}{rgb}{0.150000,0.150000,0.150000}%
\pgfsetstrokecolor{textcolor}%
\pgfsetfillcolor{textcolor}%
\pgftext[x=2.538294in, y=1.891315in, left, base]{\color{textcolor}{\rmfamily\fontsize{8.000000}{9.600000}\selectfont\catcode`\^=\active\def^{\ifmmode\sp\else\^{}\fi}\catcode`\%=\active\def%{\%}10}}%
\end{pgfscope}%
\begin{pgfscope}%
\pgfpathrectangle{\pgfqpoint{2.811623in}{0.529443in}}{\pgfqpoint{1.589120in}{1.745990in}}%
\pgfusepath{clip}%
\pgfsetbuttcap%
\pgfsetroundjoin%
\definecolor{currentfill}{rgb}{0.298039,0.447059,0.690196}%
\pgfsetfillcolor{currentfill}%
\pgfsetfillopacity{0.900000}%
\pgfsetlinewidth{0.507862pt}%
\definecolor{currentstroke}{rgb}{1.000000,1.000000,1.000000}%
\pgfsetstrokecolor{currentstroke}%
\pgfsetstrokeopacity{0.900000}%
\pgfsetdash{}{0pt}%
\pgfpathmoveto{\pgfqpoint{4.231125in}{2.130684in}}%
\pgfpathlineto{\pgfqpoint{4.253086in}{2.152645in}}%
\pgfpathlineto{\pgfqpoint{4.275046in}{2.130684in}}%
\pgfpathlineto{\pgfqpoint{4.297006in}{2.152645in}}%
\pgfpathlineto{\pgfqpoint{4.275046in}{2.174605in}}%
\pgfpathlineto{\pgfqpoint{4.297006in}{2.196565in}}%
\pgfpathlineto{\pgfqpoint{4.275046in}{2.218525in}}%
\pgfpathlineto{\pgfqpoint{4.253086in}{2.196565in}}%
\pgfpathlineto{\pgfqpoint{4.231125in}{2.218525in}}%
\pgfpathlineto{\pgfqpoint{4.209165in}{2.196565in}}%
\pgfpathlineto{\pgfqpoint{4.231125in}{2.174605in}}%
\pgfpathlineto{\pgfqpoint{4.209165in}{2.152645in}}%
\pgfpathlineto{\pgfqpoint{4.231125in}{2.130684in}}%
\pgfpathclose%
\pgfusepath{stroke,fill}%
\end{pgfscope}%
\begin{pgfscope}%
\pgfpathrectangle{\pgfqpoint{2.811623in}{0.529443in}}{\pgfqpoint{1.589120in}{1.745990in}}%
\pgfusepath{clip}%
\pgfsetbuttcap%
\pgfsetroundjoin%
\definecolor{currentfill}{rgb}{0.866667,0.517647,0.321569}%
\pgfsetfillcolor{currentfill}%
\pgfsetfillopacity{0.900000}%
\pgfsetlinewidth{0.507862pt}%
\definecolor{currentstroke}{rgb}{1.000000,1.000000,1.000000}%
\pgfsetstrokecolor{currentstroke}%
\pgfsetstrokeopacity{0.900000}%
\pgfsetdash{}{0pt}%
\pgfpathmoveto{\pgfqpoint{3.015003in}{0.677541in}}%
\pgfpathlineto{\pgfqpoint{3.015003in}{0.615428in}}%
\pgfpathlineto{\pgfqpoint{3.077116in}{0.615428in}}%
\pgfpathlineto{\pgfqpoint{3.077116in}{0.677541in}}%
\pgfpathlineto{\pgfqpoint{3.015003in}{0.677541in}}%
\pgfpathclose%
\pgfusepath{stroke,fill}%
\end{pgfscope}%
\begin{pgfscope}%
\pgfpathrectangle{\pgfqpoint{2.811623in}{0.529443in}}{\pgfqpoint{1.589120in}{1.745990in}}%
\pgfusepath{clip}%
\pgfsetbuttcap%
\pgfsetroundjoin%
\definecolor{currentfill}{rgb}{0.333333,0.658824,0.407843}%
\pgfsetfillcolor{currentfill}%
\pgfsetfillopacity{0.900000}%
\pgfsetlinewidth{0.507862pt}%
\definecolor{currentstroke}{rgb}{1.000000,1.000000,1.000000}%
\pgfsetstrokecolor{currentstroke}%
\pgfsetstrokeopacity{0.900000}%
\pgfsetdash{}{0pt}%
\pgfpathmoveto{\pgfqpoint{4.203282in}{2.118090in}}%
\pgfpathlineto{\pgfqpoint{4.225242in}{2.140050in}}%
\pgfpathlineto{\pgfqpoint{4.247203in}{2.118090in}}%
\pgfpathlineto{\pgfqpoint{4.269163in}{2.140050in}}%
\pgfpathlineto{\pgfqpoint{4.247203in}{2.162010in}}%
\pgfpathlineto{\pgfqpoint{4.269163in}{2.183971in}}%
\pgfpathlineto{\pgfqpoint{4.247203in}{2.205931in}}%
\pgfpathlineto{\pgfqpoint{4.225242in}{2.183971in}}%
\pgfpathlineto{\pgfqpoint{4.203282in}{2.205931in}}%
\pgfpathlineto{\pgfqpoint{4.181322in}{2.183971in}}%
\pgfpathlineto{\pgfqpoint{4.203282in}{2.162010in}}%
\pgfpathlineto{\pgfqpoint{4.181322in}{2.140050in}}%
\pgfpathlineto{\pgfqpoint{4.203282in}{2.118090in}}%
\pgfpathclose%
\pgfusepath{stroke,fill}%
\end{pgfscope}%
\begin{pgfscope}%
\pgfpathrectangle{\pgfqpoint{2.811623in}{0.529443in}}{\pgfqpoint{1.589120in}{1.745990in}}%
\pgfusepath{clip}%
\pgfsetbuttcap%
\pgfsetroundjoin%
\definecolor{currentfill}{rgb}{0.298039,0.447059,0.690196}%
\pgfsetfillcolor{currentfill}%
\pgfsetfillopacity{0.900000}%
\pgfsetlinewidth{0.507862pt}%
\definecolor{currentstroke}{rgb}{1.000000,1.000000,1.000000}%
\pgfsetstrokecolor{currentstroke}%
\pgfsetstrokeopacity{0.900000}%
\pgfsetdash{}{0pt}%
\pgfpathmoveto{\pgfqpoint{3.050207in}{0.701444in}}%
\pgfpathlineto{\pgfqpoint{3.050207in}{0.639331in}}%
\pgfpathlineto{\pgfqpoint{3.112320in}{0.639331in}}%
\pgfpathlineto{\pgfqpoint{3.112320in}{0.701444in}}%
\pgfpathlineto{\pgfqpoint{3.050207in}{0.701444in}}%
\pgfpathclose%
\pgfusepath{stroke,fill}%
\end{pgfscope}%
\begin{pgfscope}%
\pgfpathrectangle{\pgfqpoint{2.811623in}{0.529443in}}{\pgfqpoint{1.589120in}{1.745990in}}%
\pgfusepath{clip}%
\pgfsetbuttcap%
\pgfsetroundjoin%
\definecolor{currentfill}{rgb}{0.298039,0.447059,0.690196}%
\pgfsetfillcolor{currentfill}%
\pgfsetfillopacity{0.900000}%
\pgfsetlinewidth{0.507862pt}%
\definecolor{currentstroke}{rgb}{1.000000,1.000000,1.000000}%
\pgfsetstrokecolor{currentstroke}%
\pgfsetstrokeopacity{0.900000}%
\pgfsetdash{}{0pt}%
\pgfpathmoveto{\pgfqpoint{3.011126in}{0.722543in}}%
\pgfpathlineto{\pgfqpoint{3.011126in}{0.660430in}}%
\pgfpathlineto{\pgfqpoint{3.073239in}{0.660430in}}%
\pgfpathlineto{\pgfqpoint{3.073239in}{0.722543in}}%
\pgfpathlineto{\pgfqpoint{3.011126in}{0.722543in}}%
\pgfpathclose%
\pgfusepath{stroke,fill}%
\end{pgfscope}%
\begin{pgfscope}%
\pgfpathrectangle{\pgfqpoint{2.811623in}{0.529443in}}{\pgfqpoint{1.589120in}{1.745990in}}%
\pgfusepath{clip}%
\pgfsetbuttcap%
\pgfsetroundjoin%
\definecolor{currentfill}{rgb}{0.866667,0.517647,0.321569}%
\pgfsetfillcolor{currentfill}%
\pgfsetfillopacity{0.900000}%
\pgfsetlinewidth{0.507862pt}%
\definecolor{currentstroke}{rgb}{1.000000,1.000000,1.000000}%
\pgfsetstrokecolor{currentstroke}%
\pgfsetstrokeopacity{0.900000}%
\pgfsetdash{}{0pt}%
\pgfpathmoveto{\pgfqpoint{2.913290in}{0.697293in}}%
\pgfpathlineto{\pgfqpoint{2.913290in}{0.635180in}}%
\pgfpathlineto{\pgfqpoint{2.975403in}{0.635180in}}%
\pgfpathlineto{\pgfqpoint{2.975403in}{0.697293in}}%
\pgfpathlineto{\pgfqpoint{2.913290in}{0.697293in}}%
\pgfpathclose%
\pgfusepath{stroke,fill}%
\end{pgfscope}%
\begin{pgfscope}%
\pgfpathrectangle{\pgfqpoint{2.811623in}{0.529443in}}{\pgfqpoint{1.589120in}{1.745990in}}%
\pgfusepath{clip}%
\pgfsetbuttcap%
\pgfsetroundjoin%
\definecolor{currentfill}{rgb}{0.333333,0.658824,0.407843}%
\pgfsetfillcolor{currentfill}%
\pgfsetfillopacity{0.900000}%
\pgfsetlinewidth{0.507862pt}%
\definecolor{currentstroke}{rgb}{1.000000,1.000000,1.000000}%
\pgfsetstrokecolor{currentstroke}%
\pgfsetstrokeopacity{0.900000}%
\pgfsetdash{}{0pt}%
\pgfpathmoveto{\pgfqpoint{2.977478in}{0.657426in}}%
\pgfpathlineto{\pgfqpoint{2.977478in}{0.595313in}}%
\pgfpathlineto{\pgfqpoint{3.039591in}{0.595313in}}%
\pgfpathlineto{\pgfqpoint{3.039591in}{0.657426in}}%
\pgfpathlineto{\pgfqpoint{2.977478in}{0.657426in}}%
\pgfpathclose%
\pgfusepath{stroke,fill}%
\end{pgfscope}%
\begin{pgfscope}%
\pgfpathrectangle{\pgfqpoint{2.811623in}{0.529443in}}{\pgfqpoint{1.589120in}{1.745990in}}%
\pgfusepath{clip}%
\pgfsetbuttcap%
\pgfsetroundjoin%
\definecolor{currentfill}{rgb}{0.333333,0.658824,0.407843}%
\pgfsetfillcolor{currentfill}%
\pgfsetfillopacity{0.900000}%
\pgfsetlinewidth{0.507862pt}%
\definecolor{currentstroke}{rgb}{1.000000,1.000000,1.000000}%
\pgfsetstrokecolor{currentstroke}%
\pgfsetstrokeopacity{0.900000}%
\pgfsetdash{}{0pt}%
\pgfpathmoveto{\pgfqpoint{2.982580in}{0.826113in}}%
\pgfpathlineto{\pgfqpoint{2.982580in}{0.764000in}}%
\pgfpathlineto{\pgfqpoint{3.044693in}{0.764000in}}%
\pgfpathlineto{\pgfqpoint{3.044693in}{0.826113in}}%
\pgfpathlineto{\pgfqpoint{2.982580in}{0.826113in}}%
\pgfpathclose%
\pgfusepath{stroke,fill}%
\end{pgfscope}%
\begin{pgfscope}%
\pgfpathrectangle{\pgfqpoint{2.811623in}{0.529443in}}{\pgfqpoint{1.589120in}{1.745990in}}%
\pgfusepath{clip}%
\pgfsetbuttcap%
\pgfsetroundjoin%
\definecolor{currentfill}{rgb}{0.298039,0.447059,0.690196}%
\pgfsetfillcolor{currentfill}%
\pgfsetfillopacity{0.900000}%
\pgfsetlinewidth{0.507862pt}%
\definecolor{currentstroke}{rgb}{1.000000,1.000000,1.000000}%
\pgfsetstrokecolor{currentstroke}%
\pgfsetstrokeopacity{0.900000}%
\pgfsetdash{}{0pt}%
\pgfpathmoveto{\pgfqpoint{4.306551in}{2.126415in}}%
\pgfpathlineto{\pgfqpoint{4.328511in}{2.148375in}}%
\pgfpathlineto{\pgfqpoint{4.350471in}{2.126415in}}%
\pgfpathlineto{\pgfqpoint{4.372432in}{2.148375in}}%
\pgfpathlineto{\pgfqpoint{4.350471in}{2.170335in}}%
\pgfpathlineto{\pgfqpoint{4.372432in}{2.192296in}}%
\pgfpathlineto{\pgfqpoint{4.350471in}{2.214256in}}%
\pgfpathlineto{\pgfqpoint{4.328511in}{2.192296in}}%
\pgfpathlineto{\pgfqpoint{4.306551in}{2.214256in}}%
\pgfpathlineto{\pgfqpoint{4.284591in}{2.192296in}}%
\pgfpathlineto{\pgfqpoint{4.306551in}{2.170335in}}%
\pgfpathlineto{\pgfqpoint{4.284591in}{2.148375in}}%
\pgfpathlineto{\pgfqpoint{4.306551in}{2.126415in}}%
\pgfpathclose%
\pgfusepath{stroke,fill}%
\end{pgfscope}%
\begin{pgfscope}%
\pgfpathrectangle{\pgfqpoint{2.811623in}{0.529443in}}{\pgfqpoint{1.589120in}{1.745990in}}%
\pgfusepath{clip}%
\pgfsetbuttcap%
\pgfsetroundjoin%
\definecolor{currentfill}{rgb}{0.333333,0.658824,0.407843}%
\pgfsetfillcolor{currentfill}%
\pgfsetfillopacity{0.900000}%
\pgfsetlinewidth{0.507862pt}%
\definecolor{currentstroke}{rgb}{1.000000,1.000000,1.000000}%
\pgfsetstrokecolor{currentstroke}%
\pgfsetstrokeopacity{0.900000}%
\pgfsetdash{}{0pt}%
\pgfpathmoveto{\pgfqpoint{2.979228in}{0.700581in}}%
\pgfpathlineto{\pgfqpoint{2.979228in}{0.638468in}}%
\pgfpathlineto{\pgfqpoint{3.041341in}{0.638468in}}%
\pgfpathlineto{\pgfqpoint{3.041341in}{0.700581in}}%
\pgfpathlineto{\pgfqpoint{2.979228in}{0.700581in}}%
\pgfpathclose%
\pgfusepath{stroke,fill}%
\end{pgfscope}%
\begin{pgfscope}%
\pgfpathrectangle{\pgfqpoint{2.811623in}{0.529443in}}{\pgfqpoint{1.589120in}{1.745990in}}%
\pgfusepath{clip}%
\pgfsetbuttcap%
\pgfsetroundjoin%
\definecolor{currentfill}{rgb}{0.333333,0.658824,0.407843}%
\pgfsetfillcolor{currentfill}%
\pgfsetfillopacity{0.900000}%
\pgfsetlinewidth{0.507862pt}%
\definecolor{currentstroke}{rgb}{1.000000,1.000000,1.000000}%
\pgfsetstrokecolor{currentstroke}%
\pgfsetstrokeopacity{0.900000}%
\pgfsetdash{}{0pt}%
\pgfpathmoveto{\pgfqpoint{4.132878in}{2.087492in}}%
\pgfpathlineto{\pgfqpoint{4.154838in}{2.109453in}}%
\pgfpathlineto{\pgfqpoint{4.176798in}{2.087492in}}%
\pgfpathlineto{\pgfqpoint{4.198758in}{2.109453in}}%
\pgfpathlineto{\pgfqpoint{4.176798in}{2.131413in}}%
\pgfpathlineto{\pgfqpoint{4.198758in}{2.153373in}}%
\pgfpathlineto{\pgfqpoint{4.176798in}{2.175333in}}%
\pgfpathlineto{\pgfqpoint{4.154838in}{2.153373in}}%
\pgfpathlineto{\pgfqpoint{4.132878in}{2.175333in}}%
\pgfpathlineto{\pgfqpoint{4.110917in}{2.153373in}}%
\pgfpathlineto{\pgfqpoint{4.132878in}{2.131413in}}%
\pgfpathlineto{\pgfqpoint{4.110917in}{2.109453in}}%
\pgfpathlineto{\pgfqpoint{4.132878in}{2.087492in}}%
\pgfpathclose%
\pgfusepath{stroke,fill}%
\end{pgfscope}%
\begin{pgfscope}%
\pgfpathrectangle{\pgfqpoint{2.811623in}{0.529443in}}{\pgfqpoint{1.589120in}{1.745990in}}%
\pgfusepath{clip}%
\pgfsetbuttcap%
\pgfsetroundjoin%
\definecolor{currentfill}{rgb}{0.333333,0.658824,0.407843}%
\pgfsetfillcolor{currentfill}%
\pgfsetfillopacity{0.900000}%
\pgfsetlinewidth{0.507862pt}%
\definecolor{currentstroke}{rgb}{1.000000,1.000000,1.000000}%
\pgfsetstrokecolor{currentstroke}%
\pgfsetstrokeopacity{0.900000}%
\pgfsetdash{}{0pt}%
\pgfpathmoveto{\pgfqpoint{4.169479in}{2.070674in}}%
\pgfpathlineto{\pgfqpoint{4.191440in}{2.092635in}}%
\pgfpathlineto{\pgfqpoint{4.213400in}{2.070674in}}%
\pgfpathlineto{\pgfqpoint{4.235360in}{2.092635in}}%
\pgfpathlineto{\pgfqpoint{4.213400in}{2.114595in}}%
\pgfpathlineto{\pgfqpoint{4.235360in}{2.136555in}}%
\pgfpathlineto{\pgfqpoint{4.213400in}{2.158515in}}%
\pgfpathlineto{\pgfqpoint{4.191440in}{2.136555in}}%
\pgfpathlineto{\pgfqpoint{4.169479in}{2.158515in}}%
\pgfpathlineto{\pgfqpoint{4.147519in}{2.136555in}}%
\pgfpathlineto{\pgfqpoint{4.169479in}{2.114595in}}%
\pgfpathlineto{\pgfqpoint{4.147519in}{2.092635in}}%
\pgfpathlineto{\pgfqpoint{4.169479in}{2.070674in}}%
\pgfpathclose%
\pgfusepath{stroke,fill}%
\end{pgfscope}%
\begin{pgfscope}%
\pgfpathrectangle{\pgfqpoint{2.811623in}{0.529443in}}{\pgfqpoint{1.589120in}{1.745990in}}%
\pgfusepath{clip}%
\pgfsetbuttcap%
\pgfsetroundjoin%
\definecolor{currentfill}{rgb}{0.298039,0.447059,0.690196}%
\pgfsetfillcolor{currentfill}%
\pgfsetfillopacity{0.900000}%
\pgfsetlinewidth{0.507862pt}%
\definecolor{currentstroke}{rgb}{1.000000,1.000000,1.000000}%
\pgfsetstrokecolor{currentstroke}%
\pgfsetstrokeopacity{0.900000}%
\pgfsetdash{}{0pt}%
\pgfpathmoveto{\pgfqpoint{4.051480in}{2.006405in}}%
\pgfpathcurveto{\pgfqpoint{4.063128in}{2.006405in}}{\pgfqpoint{4.074300in}{2.011033in}}{\pgfqpoint{4.082537in}{2.019269in}}%
\pgfpathcurveto{\pgfqpoint{4.090773in}{2.027505in}}{\pgfqpoint{4.095401in}{2.038677in}}{\pgfqpoint{4.095401in}{2.050325in}}%
\pgfpathcurveto{\pgfqpoint{4.095401in}{2.061973in}}{\pgfqpoint{4.090773in}{2.073146in}}{\pgfqpoint{4.082537in}{2.081382in}}%
\pgfpathcurveto{\pgfqpoint{4.074300in}{2.089618in}}{\pgfqpoint{4.063128in}{2.094246in}}{\pgfqpoint{4.051480in}{2.094246in}}%
\pgfpathcurveto{\pgfqpoint{4.039832in}{2.094246in}}{\pgfqpoint{4.028660in}{2.089618in}}{\pgfqpoint{4.020424in}{2.081382in}}%
\pgfpathcurveto{\pgfqpoint{4.012187in}{2.073146in}}{\pgfqpoint{4.007560in}{2.061973in}}{\pgfqpoint{4.007560in}{2.050325in}}%
\pgfpathcurveto{\pgfqpoint{4.007560in}{2.038677in}}{\pgfqpoint{4.012187in}{2.027505in}}{\pgfqpoint{4.020424in}{2.019269in}}%
\pgfpathcurveto{\pgfqpoint{4.028660in}{2.011033in}}{\pgfqpoint{4.039832in}{2.006405in}}{\pgfqpoint{4.051480in}{2.006405in}}%
\pgfpathlineto{\pgfqpoint{4.051480in}{2.006405in}}%
\pgfpathclose%
\pgfusepath{stroke,fill}%
\end{pgfscope}%
\begin{pgfscope}%
\pgfpathrectangle{\pgfqpoint{2.811623in}{0.529443in}}{\pgfqpoint{1.589120in}{1.745990in}}%
\pgfusepath{clip}%
\pgfsetbuttcap%
\pgfsetroundjoin%
\definecolor{currentfill}{rgb}{0.333333,0.658824,0.407843}%
\pgfsetfillcolor{currentfill}%
\pgfsetfillopacity{0.900000}%
\pgfsetlinewidth{0.507862pt}%
\definecolor{currentstroke}{rgb}{1.000000,1.000000,1.000000}%
\pgfsetstrokecolor{currentstroke}%
\pgfsetstrokeopacity{0.900000}%
\pgfsetdash{}{0pt}%
\pgfpathmoveto{\pgfqpoint{2.955145in}{0.767621in}}%
\pgfpathlineto{\pgfqpoint{2.955145in}{0.705508in}}%
\pgfpathlineto{\pgfqpoint{3.017258in}{0.705508in}}%
\pgfpathlineto{\pgfqpoint{3.017258in}{0.767621in}}%
\pgfpathlineto{\pgfqpoint{2.955145in}{0.767621in}}%
\pgfpathclose%
\pgfusepath{stroke,fill}%
\end{pgfscope}%
\begin{pgfscope}%
\pgfpathrectangle{\pgfqpoint{2.811623in}{0.529443in}}{\pgfqpoint{1.589120in}{1.745990in}}%
\pgfusepath{clip}%
\pgfsetbuttcap%
\pgfsetroundjoin%
\definecolor{currentfill}{rgb}{0.866667,0.517647,0.321569}%
\pgfsetfillcolor{currentfill}%
\pgfsetfillopacity{0.900000}%
\pgfsetlinewidth{0.507862pt}%
\definecolor{currentstroke}{rgb}{1.000000,1.000000,1.000000}%
\pgfsetstrokecolor{currentstroke}%
\pgfsetstrokeopacity{0.900000}%
\pgfsetdash{}{0pt}%
\pgfpathmoveto{\pgfqpoint{2.880322in}{0.761150in}}%
\pgfpathlineto{\pgfqpoint{2.880322in}{0.699037in}}%
\pgfpathlineto{\pgfqpoint{2.942435in}{0.699037in}}%
\pgfpathlineto{\pgfqpoint{2.942435in}{0.761150in}}%
\pgfpathlineto{\pgfqpoint{2.880322in}{0.761150in}}%
\pgfpathclose%
\pgfusepath{stroke,fill}%
\end{pgfscope}%
\begin{pgfscope}%
\pgfpathrectangle{\pgfqpoint{2.811623in}{0.529443in}}{\pgfqpoint{1.589120in}{1.745990in}}%
\pgfusepath{clip}%
\pgfsetbuttcap%
\pgfsetroundjoin%
\definecolor{currentfill}{rgb}{0.298039,0.447059,0.690196}%
\pgfsetfillcolor{currentfill}%
\pgfsetfillopacity{0.900000}%
\pgfsetlinewidth{0.507862pt}%
\definecolor{currentstroke}{rgb}{1.000000,1.000000,1.000000}%
\pgfsetstrokecolor{currentstroke}%
\pgfsetstrokeopacity{0.900000}%
\pgfsetdash{}{0pt}%
\pgfpathmoveto{\pgfqpoint{3.016153in}{0.764924in}}%
\pgfpathlineto{\pgfqpoint{3.016153in}{0.702811in}}%
\pgfpathlineto{\pgfqpoint{3.078266in}{0.702811in}}%
\pgfpathlineto{\pgfqpoint{3.078266in}{0.764924in}}%
\pgfpathlineto{\pgfqpoint{3.016153in}{0.764924in}}%
\pgfpathclose%
\pgfusepath{stroke,fill}%
\end{pgfscope}%
\begin{pgfscope}%
\pgfpathrectangle{\pgfqpoint{2.811623in}{0.529443in}}{\pgfqpoint{1.589120in}{1.745990in}}%
\pgfusepath{clip}%
\pgfsetbuttcap%
\pgfsetroundjoin%
\definecolor{currentfill}{rgb}{0.298039,0.447059,0.690196}%
\pgfsetfillcolor{currentfill}%
\pgfsetfillopacity{0.900000}%
\pgfsetlinewidth{0.507862pt}%
\definecolor{currentstroke}{rgb}{1.000000,1.000000,1.000000}%
\pgfsetstrokecolor{currentstroke}%
\pgfsetstrokeopacity{0.900000}%
\pgfsetdash{}{0pt}%
\pgfpathmoveto{\pgfqpoint{4.261042in}{2.144974in}}%
\pgfpathlineto{\pgfqpoint{4.283002in}{2.166935in}}%
\pgfpathlineto{\pgfqpoint{4.304962in}{2.144974in}}%
\pgfpathlineto{\pgfqpoint{4.326923in}{2.166935in}}%
\pgfpathlineto{\pgfqpoint{4.304962in}{2.188895in}}%
\pgfpathlineto{\pgfqpoint{4.326923in}{2.210855in}}%
\pgfpathlineto{\pgfqpoint{4.304962in}{2.232815in}}%
\pgfpathlineto{\pgfqpoint{4.283002in}{2.210855in}}%
\pgfpathlineto{\pgfqpoint{4.261042in}{2.232815in}}%
\pgfpathlineto{\pgfqpoint{4.239082in}{2.210855in}}%
\pgfpathlineto{\pgfqpoint{4.261042in}{2.188895in}}%
\pgfpathlineto{\pgfqpoint{4.239082in}{2.166935in}}%
\pgfpathlineto{\pgfqpoint{4.261042in}{2.144974in}}%
\pgfpathclose%
\pgfusepath{stroke,fill}%
\end{pgfscope}%
\begin{pgfscope}%
\pgfpathrectangle{\pgfqpoint{2.811623in}{0.529443in}}{\pgfqpoint{1.589120in}{1.745990in}}%
\pgfusepath{clip}%
\pgfsetbuttcap%
\pgfsetroundjoin%
\definecolor{currentfill}{rgb}{0.866667,0.517647,0.321569}%
\pgfsetfillcolor{currentfill}%
\pgfsetfillopacity{0.900000}%
\pgfsetlinewidth{0.507862pt}%
\definecolor{currentstroke}{rgb}{1.000000,1.000000,1.000000}%
\pgfsetstrokecolor{currentstroke}%
\pgfsetstrokeopacity{0.900000}%
\pgfsetdash{}{0pt}%
\pgfpathmoveto{\pgfqpoint{2.930221in}{0.716343in}}%
\pgfpathlineto{\pgfqpoint{2.930221in}{0.654230in}}%
\pgfpathlineto{\pgfqpoint{2.992334in}{0.654230in}}%
\pgfpathlineto{\pgfqpoint{2.992334in}{0.716343in}}%
\pgfpathlineto{\pgfqpoint{2.930221in}{0.716343in}}%
\pgfpathclose%
\pgfusepath{stroke,fill}%
\end{pgfscope}%
\begin{pgfscope}%
\pgfpathrectangle{\pgfqpoint{2.811623in}{0.529443in}}{\pgfqpoint{1.589120in}{1.745990in}}%
\pgfusepath{clip}%
\pgfsetbuttcap%
\pgfsetroundjoin%
\definecolor{currentfill}{rgb}{0.298039,0.447059,0.690196}%
\pgfsetfillcolor{currentfill}%
\pgfsetfillopacity{0.900000}%
\pgfsetlinewidth{0.507862pt}%
\definecolor{currentstroke}{rgb}{1.000000,1.000000,1.000000}%
\pgfsetstrokecolor{currentstroke}%
\pgfsetstrokeopacity{0.900000}%
\pgfsetdash{}{0pt}%
\pgfpathmoveto{\pgfqpoint{2.852800in}{0.708736in}}%
\pgfpathlineto{\pgfqpoint{2.852800in}{0.646623in}}%
\pgfpathlineto{\pgfqpoint{2.914913in}{0.646623in}}%
\pgfpathlineto{\pgfqpoint{2.914913in}{0.708736in}}%
\pgfpathlineto{\pgfqpoint{2.852800in}{0.708736in}}%
\pgfpathclose%
\pgfusepath{stroke,fill}%
\end{pgfscope}%
\begin{pgfscope}%
\pgfpathrectangle{\pgfqpoint{2.811623in}{0.529443in}}{\pgfqpoint{1.589120in}{1.745990in}}%
\pgfusepath{clip}%
\pgfsetbuttcap%
\pgfsetroundjoin%
\definecolor{currentfill}{rgb}{0.866667,0.517647,0.321569}%
\pgfsetfillcolor{currentfill}%
\pgfsetfillopacity{0.900000}%
\pgfsetlinewidth{0.507862pt}%
\definecolor{currentstroke}{rgb}{1.000000,1.000000,1.000000}%
\pgfsetstrokecolor{currentstroke}%
\pgfsetstrokeopacity{0.900000}%
\pgfsetdash{}{0pt}%
\pgfpathmoveto{\pgfqpoint{4.283519in}{2.106126in}}%
\pgfpathlineto{\pgfqpoint{4.305479in}{2.128086in}}%
\pgfpathlineto{\pgfqpoint{4.327440in}{2.106126in}}%
\pgfpathlineto{\pgfqpoint{4.349400in}{2.128086in}}%
\pgfpathlineto{\pgfqpoint{4.327440in}{2.150046in}}%
\pgfpathlineto{\pgfqpoint{4.349400in}{2.172007in}}%
\pgfpathlineto{\pgfqpoint{4.327440in}{2.193967in}}%
\pgfpathlineto{\pgfqpoint{4.305479in}{2.172007in}}%
\pgfpathlineto{\pgfqpoint{4.283519in}{2.193967in}}%
\pgfpathlineto{\pgfqpoint{4.261559in}{2.172007in}}%
\pgfpathlineto{\pgfqpoint{4.283519in}{2.150046in}}%
\pgfpathlineto{\pgfqpoint{4.261559in}{2.128086in}}%
\pgfpathlineto{\pgfqpoint{4.283519in}{2.106126in}}%
\pgfpathclose%
\pgfusepath{stroke,fill}%
\end{pgfscope}%
\begin{pgfscope}%
\pgfpathrectangle{\pgfqpoint{2.811623in}{0.529443in}}{\pgfqpoint{1.589120in}{1.745990in}}%
\pgfusepath{clip}%
\pgfsetbuttcap%
\pgfsetroundjoin%
\definecolor{currentfill}{rgb}{0.866667,0.517647,0.321569}%
\pgfsetfillcolor{currentfill}%
\pgfsetfillopacity{0.900000}%
\pgfsetlinewidth{0.507862pt}%
\definecolor{currentstroke}{rgb}{1.000000,1.000000,1.000000}%
\pgfsetstrokecolor{currentstroke}%
\pgfsetstrokeopacity{0.900000}%
\pgfsetdash{}{0pt}%
\pgfpathmoveto{\pgfqpoint{2.884881in}{0.731584in}}%
\pgfpathlineto{\pgfqpoint{2.884881in}{0.669471in}}%
\pgfpathlineto{\pgfqpoint{2.946994in}{0.669471in}}%
\pgfpathlineto{\pgfqpoint{2.946994in}{0.731584in}}%
\pgfpathlineto{\pgfqpoint{2.884881in}{0.731584in}}%
\pgfpathclose%
\pgfusepath{stroke,fill}%
\end{pgfscope}%
\begin{pgfscope}%
\pgfpathrectangle{\pgfqpoint{2.811623in}{0.529443in}}{\pgfqpoint{1.589120in}{1.745990in}}%
\pgfusepath{clip}%
\pgfsetbuttcap%
\pgfsetroundjoin%
\definecolor{currentfill}{rgb}{0.333333,0.658824,0.407843}%
\pgfsetfillcolor{currentfill}%
\pgfsetfillopacity{0.900000}%
\pgfsetlinewidth{0.507862pt}%
\definecolor{currentstroke}{rgb}{1.000000,1.000000,1.000000}%
\pgfsetstrokecolor{currentstroke}%
\pgfsetstrokeopacity{0.900000}%
\pgfsetdash{}{0pt}%
\pgfpathmoveto{\pgfqpoint{2.916684in}{0.639863in}}%
\pgfpathlineto{\pgfqpoint{2.916684in}{0.577750in}}%
\pgfpathlineto{\pgfqpoint{2.978797in}{0.577750in}}%
\pgfpathlineto{\pgfqpoint{2.978797in}{0.639863in}}%
\pgfpathlineto{\pgfqpoint{2.916684in}{0.639863in}}%
\pgfpathclose%
\pgfusepath{stroke,fill}%
\end{pgfscope}%
\begin{pgfscope}%
\pgfpathrectangle{\pgfqpoint{2.811623in}{0.529443in}}{\pgfqpoint{1.589120in}{1.745990in}}%
\pgfusepath{clip}%
\pgfsetbuttcap%
\pgfsetroundjoin%
\definecolor{currentfill}{rgb}{0.866667,0.517647,0.321569}%
\pgfsetfillcolor{currentfill}%
\pgfsetfillopacity{0.900000}%
\pgfsetlinewidth{0.507862pt}%
\definecolor{currentstroke}{rgb}{1.000000,1.000000,1.000000}%
\pgfsetstrokecolor{currentstroke}%
\pgfsetstrokeopacity{0.900000}%
\pgfsetdash{}{0pt}%
\pgfpathmoveto{\pgfqpoint{2.938787in}{0.672118in}}%
\pgfpathlineto{\pgfqpoint{2.938787in}{0.610005in}}%
\pgfpathlineto{\pgfqpoint{3.000900in}{0.610005in}}%
\pgfpathlineto{\pgfqpoint{3.000900in}{0.672118in}}%
\pgfpathlineto{\pgfqpoint{2.938787in}{0.672118in}}%
\pgfpathclose%
\pgfusepath{stroke,fill}%
\end{pgfscope}%
\begin{pgfscope}%
\pgfpathrectangle{\pgfqpoint{2.811623in}{0.529443in}}{\pgfqpoint{1.589120in}{1.745990in}}%
\pgfusepath{clip}%
\pgfsetbuttcap%
\pgfsetroundjoin%
\definecolor{currentfill}{rgb}{0.298039,0.447059,0.690196}%
\pgfsetfillcolor{currentfill}%
\pgfsetfillopacity{0.900000}%
\pgfsetlinewidth{0.507862pt}%
\definecolor{currentstroke}{rgb}{1.000000,1.000000,1.000000}%
\pgfsetstrokecolor{currentstroke}%
\pgfsetstrokeopacity{0.900000}%
\pgfsetdash{}{0pt}%
\pgfpathmoveto{\pgfqpoint{3.068931in}{0.733838in}}%
\pgfpathlineto{\pgfqpoint{3.068931in}{0.671725in}}%
\pgfpathlineto{\pgfqpoint{3.131044in}{0.671725in}}%
\pgfpathlineto{\pgfqpoint{3.131044in}{0.733838in}}%
\pgfpathlineto{\pgfqpoint{3.068931in}{0.733838in}}%
\pgfpathclose%
\pgfusepath{stroke,fill}%
\end{pgfscope}%
\begin{pgfscope}%
\pgfpathrectangle{\pgfqpoint{2.811623in}{0.529443in}}{\pgfqpoint{1.589120in}{1.745990in}}%
\pgfusepath{clip}%
\pgfsetbuttcap%
\pgfsetroundjoin%
\definecolor{currentfill}{rgb}{0.866667,0.517647,0.321569}%
\pgfsetfillcolor{currentfill}%
\pgfsetfillopacity{0.900000}%
\pgfsetlinewidth{0.507862pt}%
\definecolor{currentstroke}{rgb}{1.000000,1.000000,1.000000}%
\pgfsetstrokecolor{currentstroke}%
\pgfsetstrokeopacity{0.900000}%
\pgfsetdash{}{0pt}%
\pgfpathmoveto{\pgfqpoint{4.146156in}{2.152149in}}%
\pgfpathlineto{\pgfqpoint{4.168116in}{2.174109in}}%
\pgfpathlineto{\pgfqpoint{4.190076in}{2.152149in}}%
\pgfpathlineto{\pgfqpoint{4.212036in}{2.174109in}}%
\pgfpathlineto{\pgfqpoint{4.190076in}{2.196069in}}%
\pgfpathlineto{\pgfqpoint{4.212036in}{2.218030in}}%
\pgfpathlineto{\pgfqpoint{4.190076in}{2.239990in}}%
\pgfpathlineto{\pgfqpoint{4.168116in}{2.218030in}}%
\pgfpathlineto{\pgfqpoint{4.146156in}{2.239990in}}%
\pgfpathlineto{\pgfqpoint{4.124195in}{2.218030in}}%
\pgfpathlineto{\pgfqpoint{4.146156in}{2.196069in}}%
\pgfpathlineto{\pgfqpoint{4.124195in}{2.174109in}}%
\pgfpathlineto{\pgfqpoint{4.146156in}{2.152149in}}%
\pgfpathclose%
\pgfusepath{stroke,fill}%
\end{pgfscope}%
\begin{pgfscope}%
\pgfpathrectangle{\pgfqpoint{2.811623in}{0.529443in}}{\pgfqpoint{1.589120in}{1.745990in}}%
\pgfusepath{clip}%
\pgfsetbuttcap%
\pgfsetroundjoin%
\definecolor{currentfill}{rgb}{0.866667,0.517647,0.321569}%
\pgfsetfillcolor{currentfill}%
\pgfsetfillopacity{0.900000}%
\pgfsetlinewidth{0.507862pt}%
\definecolor{currentstroke}{rgb}{1.000000,1.000000,1.000000}%
\pgfsetstrokecolor{currentstroke}%
\pgfsetstrokeopacity{0.900000}%
\pgfsetdash{}{0pt}%
\pgfpathmoveto{\pgfqpoint{4.144760in}{2.117297in}}%
\pgfpathlineto{\pgfqpoint{4.166720in}{2.139257in}}%
\pgfpathlineto{\pgfqpoint{4.188680in}{2.117297in}}%
\pgfpathlineto{\pgfqpoint{4.210641in}{2.139257in}}%
\pgfpathlineto{\pgfqpoint{4.188680in}{2.161217in}}%
\pgfpathlineto{\pgfqpoint{4.210641in}{2.183178in}}%
\pgfpathlineto{\pgfqpoint{4.188680in}{2.205138in}}%
\pgfpathlineto{\pgfqpoint{4.166720in}{2.183178in}}%
\pgfpathlineto{\pgfqpoint{4.144760in}{2.205138in}}%
\pgfpathlineto{\pgfqpoint{4.122800in}{2.183178in}}%
\pgfpathlineto{\pgfqpoint{4.144760in}{2.161217in}}%
\pgfpathlineto{\pgfqpoint{4.122800in}{2.139257in}}%
\pgfpathlineto{\pgfqpoint{4.144760in}{2.117297in}}%
\pgfpathclose%
\pgfusepath{stroke,fill}%
\end{pgfscope}%
\begin{pgfscope}%
\pgfpathrectangle{\pgfqpoint{2.811623in}{0.529443in}}{\pgfqpoint{1.589120in}{1.745990in}}%
\pgfusepath{clip}%
\pgfsetbuttcap%
\pgfsetroundjoin%
\definecolor{currentfill}{rgb}{0.333333,0.658824,0.407843}%
\pgfsetfillcolor{currentfill}%
\pgfsetfillopacity{0.900000}%
\pgfsetlinewidth{0.507862pt}%
\definecolor{currentstroke}{rgb}{1.000000,1.000000,1.000000}%
\pgfsetstrokecolor{currentstroke}%
\pgfsetstrokeopacity{0.900000}%
\pgfsetdash{}{0pt}%
\pgfpathmoveto{\pgfqpoint{2.954952in}{0.731113in}}%
\pgfpathlineto{\pgfqpoint{2.954952in}{0.669000in}}%
\pgfpathlineto{\pgfqpoint{3.017065in}{0.669000in}}%
\pgfpathlineto{\pgfqpoint{3.017065in}{0.731113in}}%
\pgfpathlineto{\pgfqpoint{2.954952in}{0.731113in}}%
\pgfpathclose%
\pgfusepath{stroke,fill}%
\end{pgfscope}%
\begin{pgfscope}%
\pgfpathrectangle{\pgfqpoint{2.811623in}{0.529443in}}{\pgfqpoint{1.589120in}{1.745990in}}%
\pgfusepath{clip}%
\pgfsetbuttcap%
\pgfsetroundjoin%
\definecolor{currentfill}{rgb}{0.333333,0.658824,0.407843}%
\pgfsetfillcolor{currentfill}%
\pgfsetfillopacity{0.900000}%
\pgfsetlinewidth{0.507862pt}%
\definecolor{currentstroke}{rgb}{1.000000,1.000000,1.000000}%
\pgfsetstrokecolor{currentstroke}%
\pgfsetstrokeopacity{0.900000}%
\pgfsetdash{}{0pt}%
\pgfpathmoveto{\pgfqpoint{4.077395in}{2.099194in}}%
\pgfpathlineto{\pgfqpoint{4.099356in}{2.121154in}}%
\pgfpathlineto{\pgfqpoint{4.121316in}{2.099194in}}%
\pgfpathlineto{\pgfqpoint{4.143276in}{2.121154in}}%
\pgfpathlineto{\pgfqpoint{4.121316in}{2.143115in}}%
\pgfpathlineto{\pgfqpoint{4.143276in}{2.165075in}}%
\pgfpathlineto{\pgfqpoint{4.121316in}{2.187035in}}%
\pgfpathlineto{\pgfqpoint{4.099356in}{2.165075in}}%
\pgfpathlineto{\pgfqpoint{4.077395in}{2.187035in}}%
\pgfpathlineto{\pgfqpoint{4.055435in}{2.165075in}}%
\pgfpathlineto{\pgfqpoint{4.077395in}{2.143115in}}%
\pgfpathlineto{\pgfqpoint{4.055435in}{2.121154in}}%
\pgfpathlineto{\pgfqpoint{4.077395in}{2.099194in}}%
\pgfpathclose%
\pgfusepath{stroke,fill}%
\end{pgfscope}%
\begin{pgfscope}%
\pgfpathrectangle{\pgfqpoint{2.811623in}{0.529443in}}{\pgfqpoint{1.589120in}{1.745990in}}%
\pgfusepath{clip}%
\pgfsetbuttcap%
\pgfsetroundjoin%
\definecolor{currentfill}{rgb}{0.298039,0.447059,0.690196}%
\pgfsetfillcolor{currentfill}%
\pgfsetfillopacity{0.900000}%
\pgfsetlinewidth{0.507862pt}%
\definecolor{currentstroke}{rgb}{1.000000,1.000000,1.000000}%
\pgfsetstrokecolor{currentstroke}%
\pgfsetstrokeopacity{0.900000}%
\pgfsetdash{}{0pt}%
\pgfpathmoveto{\pgfqpoint{2.865032in}{0.668560in}}%
\pgfpathlineto{\pgfqpoint{2.865032in}{0.606447in}}%
\pgfpathlineto{\pgfqpoint{2.927145in}{0.606447in}}%
\pgfpathlineto{\pgfqpoint{2.927145in}{0.668560in}}%
\pgfpathlineto{\pgfqpoint{2.865032in}{0.668560in}}%
\pgfpathclose%
\pgfusepath{stroke,fill}%
\end{pgfscope}%
\begin{pgfscope}%
\pgfpathrectangle{\pgfqpoint{2.811623in}{0.529443in}}{\pgfqpoint{1.589120in}{1.745990in}}%
\pgfusepath{clip}%
\pgfsetbuttcap%
\pgfsetroundjoin%
\definecolor{currentfill}{rgb}{0.866667,0.517647,0.321569}%
\pgfsetfillcolor{currentfill}%
\pgfsetfillopacity{0.900000}%
\pgfsetlinewidth{0.507862pt}%
\definecolor{currentstroke}{rgb}{1.000000,1.000000,1.000000}%
\pgfsetstrokecolor{currentstroke}%
\pgfsetstrokeopacity{0.900000}%
\pgfsetdash{}{0pt}%
\pgfpathmoveto{\pgfqpoint{4.255635in}{2.097574in}}%
\pgfpathlineto{\pgfqpoint{4.277595in}{2.119534in}}%
\pgfpathlineto{\pgfqpoint{4.299556in}{2.097574in}}%
\pgfpathlineto{\pgfqpoint{4.321516in}{2.119534in}}%
\pgfpathlineto{\pgfqpoint{4.299556in}{2.141494in}}%
\pgfpathlineto{\pgfqpoint{4.321516in}{2.163454in}}%
\pgfpathlineto{\pgfqpoint{4.299556in}{2.185415in}}%
\pgfpathlineto{\pgfqpoint{4.277595in}{2.163454in}}%
\pgfpathlineto{\pgfqpoint{4.255635in}{2.185415in}}%
\pgfpathlineto{\pgfqpoint{4.233675in}{2.163454in}}%
\pgfpathlineto{\pgfqpoint{4.255635in}{2.141494in}}%
\pgfpathlineto{\pgfqpoint{4.233675in}{2.119534in}}%
\pgfpathlineto{\pgfqpoint{4.255635in}{2.097574in}}%
\pgfpathclose%
\pgfusepath{stroke,fill}%
\end{pgfscope}%
\begin{pgfscope}%
\pgfpathrectangle{\pgfqpoint{2.811623in}{0.529443in}}{\pgfqpoint{1.589120in}{1.745990in}}%
\pgfusepath{clip}%
\pgfsetbuttcap%
\pgfsetroundjoin%
\definecolor{currentfill}{rgb}{0.298039,0.447059,0.690196}%
\pgfsetfillcolor{currentfill}%
\pgfsetfillopacity{0.900000}%
\pgfsetlinewidth{0.507862pt}%
\definecolor{currentstroke}{rgb}{1.000000,1.000000,1.000000}%
\pgfsetstrokecolor{currentstroke}%
\pgfsetstrokeopacity{0.900000}%
\pgfsetdash{}{0pt}%
\pgfpathmoveto{\pgfqpoint{4.100129in}{2.072427in}}%
\pgfpathlineto{\pgfqpoint{4.122090in}{2.094387in}}%
\pgfpathlineto{\pgfqpoint{4.144050in}{2.072427in}}%
\pgfpathlineto{\pgfqpoint{4.166010in}{2.094387in}}%
\pgfpathlineto{\pgfqpoint{4.144050in}{2.116347in}}%
\pgfpathlineto{\pgfqpoint{4.166010in}{2.138308in}}%
\pgfpathlineto{\pgfqpoint{4.144050in}{2.160268in}}%
\pgfpathlineto{\pgfqpoint{4.122090in}{2.138308in}}%
\pgfpathlineto{\pgfqpoint{4.100129in}{2.160268in}}%
\pgfpathlineto{\pgfqpoint{4.078169in}{2.138308in}}%
\pgfpathlineto{\pgfqpoint{4.100129in}{2.116347in}}%
\pgfpathlineto{\pgfqpoint{4.078169in}{2.094387in}}%
\pgfpathlineto{\pgfqpoint{4.100129in}{2.072427in}}%
\pgfpathclose%
\pgfusepath{stroke,fill}%
\end{pgfscope}%
\begin{pgfscope}%
\pgfpathrectangle{\pgfqpoint{2.811623in}{0.529443in}}{\pgfqpoint{1.589120in}{1.745990in}}%
\pgfusepath{clip}%
\pgfsetbuttcap%
\pgfsetroundjoin%
\definecolor{currentfill}{rgb}{0.333333,0.658824,0.407843}%
\pgfsetfillcolor{currentfill}%
\pgfsetfillopacity{0.900000}%
\pgfsetlinewidth{0.507862pt}%
\definecolor{currentstroke}{rgb}{1.000000,1.000000,1.000000}%
\pgfsetstrokecolor{currentstroke}%
\pgfsetstrokeopacity{0.900000}%
\pgfsetdash{}{0pt}%
\pgfpathmoveto{\pgfqpoint{4.043330in}{2.111815in}}%
\pgfpathlineto{\pgfqpoint{4.043330in}{2.049702in}}%
\pgfpathlineto{\pgfqpoint{4.105443in}{2.049702in}}%
\pgfpathlineto{\pgfqpoint{4.105443in}{2.111815in}}%
\pgfpathlineto{\pgfqpoint{4.043330in}{2.111815in}}%
\pgfpathclose%
\pgfusepath{stroke,fill}%
\end{pgfscope}%
\begin{pgfscope}%
\pgfpathrectangle{\pgfqpoint{2.811623in}{0.529443in}}{\pgfqpoint{1.589120in}{1.745990in}}%
\pgfusepath{clip}%
\pgfsetbuttcap%
\pgfsetroundjoin%
\definecolor{currentfill}{rgb}{0.333333,0.658824,0.407843}%
\pgfsetfillcolor{currentfill}%
\pgfsetfillopacity{0.900000}%
\pgfsetlinewidth{0.507862pt}%
\definecolor{currentstroke}{rgb}{1.000000,1.000000,1.000000}%
\pgfsetstrokecolor{currentstroke}%
\pgfsetstrokeopacity{0.900000}%
\pgfsetdash{}{0pt}%
\pgfpathmoveto{\pgfqpoint{3.061998in}{0.655785in}}%
\pgfpathlineto{\pgfqpoint{3.061998in}{0.593672in}}%
\pgfpathlineto{\pgfqpoint{3.124111in}{0.593672in}}%
\pgfpathlineto{\pgfqpoint{3.124111in}{0.655785in}}%
\pgfpathlineto{\pgfqpoint{3.061998in}{0.655785in}}%
\pgfpathclose%
\pgfusepath{stroke,fill}%
\end{pgfscope}%
\begin{pgfscope}%
\pgfpathrectangle{\pgfqpoint{2.811623in}{0.529443in}}{\pgfqpoint{1.589120in}{1.745990in}}%
\pgfusepath{clip}%
\pgfsetbuttcap%
\pgfsetroundjoin%
\definecolor{currentfill}{rgb}{0.333333,0.658824,0.407843}%
\pgfsetfillcolor{currentfill}%
\pgfsetfillopacity{0.900000}%
\pgfsetlinewidth{0.507862pt}%
\definecolor{currentstroke}{rgb}{1.000000,1.000000,1.000000}%
\pgfsetstrokecolor{currentstroke}%
\pgfsetstrokeopacity{0.900000}%
\pgfsetdash{}{0pt}%
\pgfpathmoveto{\pgfqpoint{4.178165in}{2.149392in}}%
\pgfpathlineto{\pgfqpoint{4.200125in}{2.171352in}}%
\pgfpathlineto{\pgfqpoint{4.222085in}{2.149392in}}%
\pgfpathlineto{\pgfqpoint{4.244045in}{2.171352in}}%
\pgfpathlineto{\pgfqpoint{4.222085in}{2.193312in}}%
\pgfpathlineto{\pgfqpoint{4.244045in}{2.215272in}}%
\pgfpathlineto{\pgfqpoint{4.222085in}{2.237233in}}%
\pgfpathlineto{\pgfqpoint{4.200125in}{2.215272in}}%
\pgfpathlineto{\pgfqpoint{4.178165in}{2.237233in}}%
\pgfpathlineto{\pgfqpoint{4.156204in}{2.215272in}}%
\pgfpathlineto{\pgfqpoint{4.178165in}{2.193312in}}%
\pgfpathlineto{\pgfqpoint{4.156204in}{2.171352in}}%
\pgfpathlineto{\pgfqpoint{4.178165in}{2.149392in}}%
\pgfpathclose%
\pgfusepath{stroke,fill}%
\end{pgfscope}%
\begin{pgfscope}%
\pgfpathrectangle{\pgfqpoint{2.811623in}{0.529443in}}{\pgfqpoint{1.589120in}{1.745990in}}%
\pgfusepath{clip}%
\pgfsetbuttcap%
\pgfsetroundjoin%
\definecolor{currentfill}{rgb}{0.298039,0.447059,0.690196}%
\pgfsetfillcolor{currentfill}%
\pgfsetfillopacity{0.900000}%
\pgfsetlinewidth{0.507862pt}%
\definecolor{currentstroke}{rgb}{1.000000,1.000000,1.000000}%
\pgfsetstrokecolor{currentstroke}%
\pgfsetstrokeopacity{0.900000}%
\pgfsetdash{}{0pt}%
\pgfpathmoveto{\pgfqpoint{2.905609in}{0.795147in}}%
\pgfpathlineto{\pgfqpoint{2.905609in}{0.733034in}}%
\pgfpathlineto{\pgfqpoint{2.967722in}{0.733034in}}%
\pgfpathlineto{\pgfqpoint{2.967722in}{0.795147in}}%
\pgfpathlineto{\pgfqpoint{2.905609in}{0.795147in}}%
\pgfpathclose%
\pgfusepath{stroke,fill}%
\end{pgfscope}%
\begin{pgfscope}%
\pgfpathrectangle{\pgfqpoint{2.811623in}{0.529443in}}{\pgfqpoint{1.589120in}{1.745990in}}%
\pgfusepath{clip}%
\pgfsetbuttcap%
\pgfsetroundjoin%
\definecolor{currentfill}{rgb}{0.333333,0.658824,0.407843}%
\pgfsetfillcolor{currentfill}%
\pgfsetfillopacity{0.900000}%
\pgfsetlinewidth{0.507862pt}%
\definecolor{currentstroke}{rgb}{1.000000,1.000000,1.000000}%
\pgfsetstrokecolor{currentstroke}%
\pgfsetstrokeopacity{0.900000}%
\pgfsetdash{}{0pt}%
\pgfpathmoveto{\pgfqpoint{3.014039in}{0.846741in}}%
\pgfpathlineto{\pgfqpoint{3.014039in}{0.784628in}}%
\pgfpathlineto{\pgfqpoint{3.076152in}{0.784628in}}%
\pgfpathlineto{\pgfqpoint{3.076152in}{0.846741in}}%
\pgfpathlineto{\pgfqpoint{3.014039in}{0.846741in}}%
\pgfpathclose%
\pgfusepath{stroke,fill}%
\end{pgfscope}%
\begin{pgfscope}%
\pgfsetrectcap%
\pgfsetmiterjoin%
\pgfsetlinewidth{1.254687pt}%
\definecolor{currentstroke}{rgb}{0.800000,0.800000,0.800000}%
\pgfsetstrokecolor{currentstroke}%
\pgfsetdash{}{0pt}%
\pgfpathmoveto{\pgfqpoint{2.811623in}{0.529443in}}%
\pgfpathlineto{\pgfqpoint{2.811623in}{2.275433in}}%
\pgfusepath{stroke}%
\end{pgfscope}%
\begin{pgfscope}%
\pgfsetrectcap%
\pgfsetmiterjoin%
\pgfsetlinewidth{1.254687pt}%
\definecolor{currentstroke}{rgb}{0.800000,0.800000,0.800000}%
\pgfsetstrokecolor{currentstroke}%
\pgfsetdash{}{0pt}%
\pgfpathmoveto{\pgfqpoint{4.400744in}{0.529443in}}%
\pgfpathlineto{\pgfqpoint{4.400744in}{2.275433in}}%
\pgfusepath{stroke}%
\end{pgfscope}%
\begin{pgfscope}%
\pgfsetrectcap%
\pgfsetmiterjoin%
\pgfsetlinewidth{1.254687pt}%
\definecolor{currentstroke}{rgb}{0.800000,0.800000,0.800000}%
\pgfsetstrokecolor{currentstroke}%
\pgfsetdash{}{0pt}%
\pgfpathmoveto{\pgfqpoint{2.811623in}{0.529443in}}%
\pgfpathlineto{\pgfqpoint{4.400744in}{0.529443in}}%
\pgfusepath{stroke}%
\end{pgfscope}%
\begin{pgfscope}%
\pgfsetrectcap%
\pgfsetmiterjoin%
\pgfsetlinewidth{1.254687pt}%
\definecolor{currentstroke}{rgb}{0.800000,0.800000,0.800000}%
\pgfsetstrokecolor{currentstroke}%
\pgfsetdash{}{0pt}%
\pgfpathmoveto{\pgfqpoint{2.811623in}{2.275433in}}%
\pgfpathlineto{\pgfqpoint{4.400744in}{2.275433in}}%
\pgfusepath{stroke}%
\end{pgfscope}%
\begin{pgfscope}%
\definecolor{textcolor}{rgb}{0.150000,0.150000,0.150000}%
\pgfsetstrokecolor{textcolor}%
\pgfsetfillcolor{textcolor}%
\pgftext[x=3.606184in,y=2.358766in,,base]{\color{textcolor}{\rmfamily\fontsize{11.000000}{13.200000}\selectfont\catcode`\^=\active\def^{\ifmmode\sp\else\^{}\fi}\catcode`\%=\active\def%{\%}Bag-Of-Subgraphs}}%
\end{pgfscope}%
\begin{pgfscope}%
\pgfsetbuttcap%
\pgfsetmiterjoin%
\definecolor{currentfill}{rgb}{1.000000,1.000000,1.000000}%
\pgfsetfillcolor{currentfill}%
\pgfsetlinewidth{0.000000pt}%
\definecolor{currentstroke}{rgb}{0.000000,0.000000,0.000000}%
\pgfsetstrokecolor{currentstroke}%
\pgfsetstrokeopacity{0.000000}%
\pgfsetdash{}{0pt}%
\pgfpathmoveto{\pgfqpoint{4.953938in}{0.529443in}}%
\pgfpathlineto{\pgfqpoint{6.543059in}{0.529443in}}%
\pgfpathlineto{\pgfqpoint{6.543059in}{2.275433in}}%
\pgfpathlineto{\pgfqpoint{4.953938in}{2.275433in}}%
\pgfpathlineto{\pgfqpoint{4.953938in}{0.529443in}}%
\pgfpathclose%
\pgfusepath{fill}%
\end{pgfscope}%
\begin{pgfscope}%
\pgfpathrectangle{\pgfqpoint{4.953938in}{0.529443in}}{\pgfqpoint{1.589120in}{1.745990in}}%
\pgfusepath{clip}%
\pgfsetroundcap%
\pgfsetroundjoin%
\pgfsetlinewidth{1.003750pt}%
\definecolor{currentstroke}{rgb}{0.800000,0.800000,0.800000}%
\pgfsetstrokecolor{currentstroke}%
\pgfsetdash{}{0pt}%
\pgfpathmoveto{\pgfqpoint{5.237055in}{0.529443in}}%
\pgfpathlineto{\pgfqpoint{5.237055in}{2.275433in}}%
\pgfusepath{stroke}%
\end{pgfscope}%
\begin{pgfscope}%
\definecolor{textcolor}{rgb}{0.150000,0.150000,0.150000}%
\pgfsetstrokecolor{textcolor}%
\pgfsetfillcolor{textcolor}%
\pgftext[x=5.237055in,y=0.397499in,,top]{\color{textcolor}{\rmfamily\fontsize{8.000000}{9.600000}\selectfont\catcode`\^=\active\def^{\ifmmode\sp\else\^{}\fi}\catcode`\%=\active\def%{\%}10}}%
\end{pgfscope}%
\begin{pgfscope}%
\pgfpathrectangle{\pgfqpoint{4.953938in}{0.529443in}}{\pgfqpoint{1.589120in}{1.745990in}}%
\pgfusepath{clip}%
\pgfsetroundcap%
\pgfsetroundjoin%
\pgfsetlinewidth{1.003750pt}%
\definecolor{currentstroke}{rgb}{0.800000,0.800000,0.800000}%
\pgfsetstrokecolor{currentstroke}%
\pgfsetdash{}{0pt}%
\pgfpathmoveto{\pgfqpoint{5.683505in}{0.529443in}}%
\pgfpathlineto{\pgfqpoint{5.683505in}{2.275433in}}%
\pgfusepath{stroke}%
\end{pgfscope}%
\begin{pgfscope}%
\definecolor{textcolor}{rgb}{0.150000,0.150000,0.150000}%
\pgfsetstrokecolor{textcolor}%
\pgfsetfillcolor{textcolor}%
\pgftext[x=5.683505in,y=0.397499in,,top]{\color{textcolor}{\rmfamily\fontsize{8.000000}{9.600000}\selectfont\catcode`\^=\active\def^{\ifmmode\sp\else\^{}\fi}\catcode`\%=\active\def%{\%}15}}%
\end{pgfscope}%
\begin{pgfscope}%
\pgfpathrectangle{\pgfqpoint{4.953938in}{0.529443in}}{\pgfqpoint{1.589120in}{1.745990in}}%
\pgfusepath{clip}%
\pgfsetroundcap%
\pgfsetroundjoin%
\pgfsetlinewidth{1.003750pt}%
\definecolor{currentstroke}{rgb}{0.800000,0.800000,0.800000}%
\pgfsetstrokecolor{currentstroke}%
\pgfsetdash{}{0pt}%
\pgfpathmoveto{\pgfqpoint{6.129956in}{0.529443in}}%
\pgfpathlineto{\pgfqpoint{6.129956in}{2.275433in}}%
\pgfusepath{stroke}%
\end{pgfscope}%
\begin{pgfscope}%
\definecolor{textcolor}{rgb}{0.150000,0.150000,0.150000}%
\pgfsetstrokecolor{textcolor}%
\pgfsetfillcolor{textcolor}%
\pgftext[x=6.129956in,y=0.397499in,,top]{\color{textcolor}{\rmfamily\fontsize{8.000000}{9.600000}\selectfont\catcode`\^=\active\def^{\ifmmode\sp\else\^{}\fi}\catcode`\%=\active\def%{\%}20}}%
\end{pgfscope}%
\begin{pgfscope}%
\definecolor{textcolor}{rgb}{0.150000,0.150000,0.150000}%
\pgfsetstrokecolor{textcolor}%
\pgfsetfillcolor{textcolor}%
\pgftext[x=5.748498in,y=0.234413in,,top]{\color{textcolor}{\rmfamily\fontsize{10.000000}{12.000000}\selectfont\catcode`\^=\active\def^{\ifmmode\sp\else\^{}\fi}\catcode`\%=\active\def%{\%}UMAP 1}}%
\end{pgfscope}%
\begin{pgfscope}%
\pgfpathrectangle{\pgfqpoint{4.953938in}{0.529443in}}{\pgfqpoint{1.589120in}{1.745990in}}%
\pgfusepath{clip}%
\pgfsetroundcap%
\pgfsetroundjoin%
\pgfsetlinewidth{1.003750pt}%
\definecolor{currentstroke}{rgb}{0.800000,0.800000,0.800000}%
\pgfsetstrokecolor{currentstroke}%
\pgfsetdash{}{0pt}%
\pgfpathmoveto{\pgfqpoint{4.953938in}{0.724636in}}%
\pgfpathlineto{\pgfqpoint{6.543059in}{0.724636in}}%
\pgfusepath{stroke}%
\end{pgfscope}%
\begin{pgfscope}%
\definecolor{textcolor}{rgb}{0.150000,0.150000,0.150000}%
\pgfsetstrokecolor{textcolor}%
\pgfsetfillcolor{textcolor}%
\pgftext[x=4.659479in, y=0.682427in, left, base]{\color{textcolor}{\rmfamily\fontsize{8.000000}{9.600000}\selectfont\catcode`\^=\active\def^{\ifmmode\sp\else\^{}\fi}\catcode`\%=\active\def%{\%}\ensuremath{-}5}}%
\end{pgfscope}%
\begin{pgfscope}%
\pgfpathrectangle{\pgfqpoint{4.953938in}{0.529443in}}{\pgfqpoint{1.589120in}{1.745990in}}%
\pgfusepath{clip}%
\pgfsetroundcap%
\pgfsetroundjoin%
\pgfsetlinewidth{1.003750pt}%
\definecolor{currentstroke}{rgb}{0.800000,0.800000,0.800000}%
\pgfsetstrokecolor{currentstroke}%
\pgfsetdash{}{0pt}%
\pgfpathmoveto{\pgfqpoint{4.953938in}{1.010608in}}%
\pgfpathlineto{\pgfqpoint{6.543059in}{1.010608in}}%
\pgfusepath{stroke}%
\end{pgfscope}%
\begin{pgfscope}%
\definecolor{textcolor}{rgb}{0.150000,0.150000,0.150000}%
\pgfsetstrokecolor{textcolor}%
\pgfsetfillcolor{textcolor}%
\pgftext[x=4.751302in, y=0.968398in, left, base]{\color{textcolor}{\rmfamily\fontsize{8.000000}{9.600000}\selectfont\catcode`\^=\active\def^{\ifmmode\sp\else\^{}\fi}\catcode`\%=\active\def%{\%}0}}%
\end{pgfscope}%
\begin{pgfscope}%
\pgfpathrectangle{\pgfqpoint{4.953938in}{0.529443in}}{\pgfqpoint{1.589120in}{1.745990in}}%
\pgfusepath{clip}%
\pgfsetroundcap%
\pgfsetroundjoin%
\pgfsetlinewidth{1.003750pt}%
\definecolor{currentstroke}{rgb}{0.800000,0.800000,0.800000}%
\pgfsetstrokecolor{currentstroke}%
\pgfsetdash{}{0pt}%
\pgfpathmoveto{\pgfqpoint{4.953938in}{1.296579in}}%
\pgfpathlineto{\pgfqpoint{6.543059in}{1.296579in}}%
\pgfusepath{stroke}%
\end{pgfscope}%
\begin{pgfscope}%
\definecolor{textcolor}{rgb}{0.150000,0.150000,0.150000}%
\pgfsetstrokecolor{textcolor}%
\pgfsetfillcolor{textcolor}%
\pgftext[x=4.751302in, y=1.254369in, left, base]{\color{textcolor}{\rmfamily\fontsize{8.000000}{9.600000}\selectfont\catcode`\^=\active\def^{\ifmmode\sp\else\^{}\fi}\catcode`\%=\active\def%{\%}5}}%
\end{pgfscope}%
\begin{pgfscope}%
\pgfpathrectangle{\pgfqpoint{4.953938in}{0.529443in}}{\pgfqpoint{1.589120in}{1.745990in}}%
\pgfusepath{clip}%
\pgfsetroundcap%
\pgfsetroundjoin%
\pgfsetlinewidth{1.003750pt}%
\definecolor{currentstroke}{rgb}{0.800000,0.800000,0.800000}%
\pgfsetstrokecolor{currentstroke}%
\pgfsetdash{}{0pt}%
\pgfpathmoveto{\pgfqpoint{4.953938in}{1.582550in}}%
\pgfpathlineto{\pgfqpoint{6.543059in}{1.582550in}}%
\pgfusepath{stroke}%
\end{pgfscope}%
\begin{pgfscope}%
\definecolor{textcolor}{rgb}{0.150000,0.150000,0.150000}%
\pgfsetstrokecolor{textcolor}%
\pgfsetfillcolor{textcolor}%
\pgftext[x=4.680609in, y=1.540341in, left, base]{\color{textcolor}{\rmfamily\fontsize{8.000000}{9.600000}\selectfont\catcode`\^=\active\def^{\ifmmode\sp\else\^{}\fi}\catcode`\%=\active\def%{\%}10}}%
\end{pgfscope}%
\begin{pgfscope}%
\pgfpathrectangle{\pgfqpoint{4.953938in}{0.529443in}}{\pgfqpoint{1.589120in}{1.745990in}}%
\pgfusepath{clip}%
\pgfsetroundcap%
\pgfsetroundjoin%
\pgfsetlinewidth{1.003750pt}%
\definecolor{currentstroke}{rgb}{0.800000,0.800000,0.800000}%
\pgfsetstrokecolor{currentstroke}%
\pgfsetdash{}{0pt}%
\pgfpathmoveto{\pgfqpoint{4.953938in}{1.868521in}}%
\pgfpathlineto{\pgfqpoint{6.543059in}{1.868521in}}%
\pgfusepath{stroke}%
\end{pgfscope}%
\begin{pgfscope}%
\definecolor{textcolor}{rgb}{0.150000,0.150000,0.150000}%
\pgfsetstrokecolor{textcolor}%
\pgfsetfillcolor{textcolor}%
\pgftext[x=4.680609in, y=1.826312in, left, base]{\color{textcolor}{\rmfamily\fontsize{8.000000}{9.600000}\selectfont\catcode`\^=\active\def^{\ifmmode\sp\else\^{}\fi}\catcode`\%=\active\def%{\%}15}}%
\end{pgfscope}%
\begin{pgfscope}%
\pgfpathrectangle{\pgfqpoint{4.953938in}{0.529443in}}{\pgfqpoint{1.589120in}{1.745990in}}%
\pgfusepath{clip}%
\pgfsetroundcap%
\pgfsetroundjoin%
\pgfsetlinewidth{1.003750pt}%
\definecolor{currentstroke}{rgb}{0.800000,0.800000,0.800000}%
\pgfsetstrokecolor{currentstroke}%
\pgfsetdash{}{0pt}%
\pgfpathmoveto{\pgfqpoint{4.953938in}{2.154492in}}%
\pgfpathlineto{\pgfqpoint{6.543059in}{2.154492in}}%
\pgfusepath{stroke}%
\end{pgfscope}%
\begin{pgfscope}%
\definecolor{textcolor}{rgb}{0.150000,0.150000,0.150000}%
\pgfsetstrokecolor{textcolor}%
\pgfsetfillcolor{textcolor}%
\pgftext[x=4.680609in, y=2.112283in, left, base]{\color{textcolor}{\rmfamily\fontsize{8.000000}{9.600000}\selectfont\catcode`\^=\active\def^{\ifmmode\sp\else\^{}\fi}\catcode`\%=\active\def%{\%}20}}%
\end{pgfscope}%
\begin{pgfscope}%
\pgfpathrectangle{\pgfqpoint{4.953938in}{0.529443in}}{\pgfqpoint{1.589120in}{1.745990in}}%
\pgfusepath{clip}%
\pgfsetbuttcap%
\pgfsetroundjoin%
\definecolor{currentfill}{rgb}{0.298039,0.447059,0.690196}%
\pgfsetfillcolor{currentfill}%
\pgfsetfillopacity{0.900000}%
\pgfsetlinewidth{0.507862pt}%
\definecolor{currentstroke}{rgb}{1.000000,1.000000,1.000000}%
\pgfsetstrokecolor{currentstroke}%
\pgfsetstrokeopacity{0.900000}%
\pgfsetdash{}{0pt}%
\pgfpathmoveto{\pgfqpoint{5.029542in}{0.656423in}}%
\pgfpathlineto{\pgfqpoint{5.051502in}{0.678384in}}%
\pgfpathlineto{\pgfqpoint{5.073462in}{0.656423in}}%
\pgfpathlineto{\pgfqpoint{5.095423in}{0.678384in}}%
\pgfpathlineto{\pgfqpoint{5.073462in}{0.700344in}}%
\pgfpathlineto{\pgfqpoint{5.095423in}{0.722304in}}%
\pgfpathlineto{\pgfqpoint{5.073462in}{0.744264in}}%
\pgfpathlineto{\pgfqpoint{5.051502in}{0.722304in}}%
\pgfpathlineto{\pgfqpoint{5.029542in}{0.744264in}}%
\pgfpathlineto{\pgfqpoint{5.007582in}{0.722304in}}%
\pgfpathlineto{\pgfqpoint{5.029542in}{0.700344in}}%
\pgfpathlineto{\pgfqpoint{5.007582in}{0.678384in}}%
\pgfpathlineto{\pgfqpoint{5.029542in}{0.656423in}}%
\pgfpathclose%
\pgfusepath{stroke,fill}%
\end{pgfscope}%
\begin{pgfscope}%
\pgfpathrectangle{\pgfqpoint{4.953938in}{0.529443in}}{\pgfqpoint{1.589120in}{1.745990in}}%
\pgfusepath{clip}%
\pgfsetbuttcap%
\pgfsetroundjoin%
\definecolor{currentfill}{rgb}{0.866667,0.517647,0.321569}%
\pgfsetfillcolor{currentfill}%
\pgfsetfillopacity{0.900000}%
\pgfsetlinewidth{0.507862pt}%
\definecolor{currentstroke}{rgb}{1.000000,1.000000,1.000000}%
\pgfsetstrokecolor{currentstroke}%
\pgfsetstrokeopacity{0.900000}%
\pgfsetdash{}{0pt}%
\pgfpathmoveto{\pgfqpoint{6.296383in}{2.199867in}}%
\pgfpathlineto{\pgfqpoint{6.296383in}{2.137754in}}%
\pgfpathlineto{\pgfqpoint{6.358496in}{2.137754in}}%
\pgfpathlineto{\pgfqpoint{6.358496in}{2.199867in}}%
\pgfpathlineto{\pgfqpoint{6.296383in}{2.199867in}}%
\pgfpathclose%
\pgfusepath{stroke,fill}%
\end{pgfscope}%
\begin{pgfscope}%
\pgfpathrectangle{\pgfqpoint{4.953938in}{0.529443in}}{\pgfqpoint{1.589120in}{1.745990in}}%
\pgfusepath{clip}%
\pgfsetbuttcap%
\pgfsetroundjoin%
\definecolor{currentfill}{rgb}{0.333333,0.658824,0.407843}%
\pgfsetfillcolor{currentfill}%
\pgfsetfillopacity{0.900000}%
\pgfsetlinewidth{0.507862pt}%
\definecolor{currentstroke}{rgb}{1.000000,1.000000,1.000000}%
\pgfsetstrokecolor{currentstroke}%
\pgfsetstrokeopacity{0.900000}%
\pgfsetdash{}{0pt}%
\pgfpathmoveto{\pgfqpoint{5.064114in}{0.694271in}}%
\pgfpathlineto{\pgfqpoint{5.086074in}{0.716231in}}%
\pgfpathlineto{\pgfqpoint{5.108035in}{0.694271in}}%
\pgfpathlineto{\pgfqpoint{5.129995in}{0.716231in}}%
\pgfpathlineto{\pgfqpoint{5.108035in}{0.738192in}}%
\pgfpathlineto{\pgfqpoint{5.129995in}{0.760152in}}%
\pgfpathlineto{\pgfqpoint{5.108035in}{0.782112in}}%
\pgfpathlineto{\pgfqpoint{5.086074in}{0.760152in}}%
\pgfpathlineto{\pgfqpoint{5.064114in}{0.782112in}}%
\pgfpathlineto{\pgfqpoint{5.042154in}{0.760152in}}%
\pgfpathlineto{\pgfqpoint{5.064114in}{0.738192in}}%
\pgfpathlineto{\pgfqpoint{5.042154in}{0.716231in}}%
\pgfpathlineto{\pgfqpoint{5.064114in}{0.694271in}}%
\pgfpathclose%
\pgfusepath{stroke,fill}%
\end{pgfscope}%
\begin{pgfscope}%
\pgfpathrectangle{\pgfqpoint{4.953938in}{0.529443in}}{\pgfqpoint{1.589120in}{1.745990in}}%
\pgfusepath{clip}%
\pgfsetbuttcap%
\pgfsetroundjoin%
\definecolor{currentfill}{rgb}{0.298039,0.447059,0.690196}%
\pgfsetfillcolor{currentfill}%
\pgfsetfillopacity{0.900000}%
\pgfsetlinewidth{0.507862pt}%
\definecolor{currentstroke}{rgb}{1.000000,1.000000,1.000000}%
\pgfsetstrokecolor{currentstroke}%
\pgfsetstrokeopacity{0.900000}%
\pgfsetdash{}{0pt}%
\pgfpathmoveto{\pgfqpoint{6.298681in}{2.177707in}}%
\pgfpathlineto{\pgfqpoint{6.298681in}{2.115594in}}%
\pgfpathlineto{\pgfqpoint{6.360794in}{2.115594in}}%
\pgfpathlineto{\pgfqpoint{6.360794in}{2.177707in}}%
\pgfpathlineto{\pgfqpoint{6.298681in}{2.177707in}}%
\pgfpathclose%
\pgfusepath{stroke,fill}%
\end{pgfscope}%
\begin{pgfscope}%
\pgfpathrectangle{\pgfqpoint{4.953938in}{0.529443in}}{\pgfqpoint{1.589120in}{1.745990in}}%
\pgfusepath{clip}%
\pgfsetbuttcap%
\pgfsetroundjoin%
\definecolor{currentfill}{rgb}{0.298039,0.447059,0.690196}%
\pgfsetfillcolor{currentfill}%
\pgfsetfillopacity{0.900000}%
\pgfsetlinewidth{0.507862pt}%
\definecolor{currentstroke}{rgb}{1.000000,1.000000,1.000000}%
\pgfsetstrokecolor{currentstroke}%
\pgfsetstrokeopacity{0.900000}%
\pgfsetdash{}{0pt}%
\pgfpathmoveto{\pgfqpoint{6.308722in}{2.221164in}}%
\pgfpathlineto{\pgfqpoint{6.308722in}{2.159051in}}%
\pgfpathlineto{\pgfqpoint{6.370835in}{2.159051in}}%
\pgfpathlineto{\pgfqpoint{6.370835in}{2.221164in}}%
\pgfpathlineto{\pgfqpoint{6.308722in}{2.221164in}}%
\pgfpathclose%
\pgfusepath{stroke,fill}%
\end{pgfscope}%
\begin{pgfscope}%
\pgfpathrectangle{\pgfqpoint{4.953938in}{0.529443in}}{\pgfqpoint{1.589120in}{1.745990in}}%
\pgfusepath{clip}%
\pgfsetbuttcap%
\pgfsetroundjoin%
\definecolor{currentfill}{rgb}{0.866667,0.517647,0.321569}%
\pgfsetfillcolor{currentfill}%
\pgfsetfillopacity{0.900000}%
\pgfsetlinewidth{0.507862pt}%
\definecolor{currentstroke}{rgb}{1.000000,1.000000,1.000000}%
\pgfsetstrokecolor{currentstroke}%
\pgfsetstrokeopacity{0.900000}%
\pgfsetdash{}{0pt}%
\pgfpathmoveto{\pgfqpoint{6.354983in}{2.180968in}}%
\pgfpathlineto{\pgfqpoint{6.354983in}{2.118855in}}%
\pgfpathlineto{\pgfqpoint{6.417096in}{2.118855in}}%
\pgfpathlineto{\pgfqpoint{6.417096in}{2.180968in}}%
\pgfpathlineto{\pgfqpoint{6.354983in}{2.180968in}}%
\pgfpathclose%
\pgfusepath{stroke,fill}%
\end{pgfscope}%
\begin{pgfscope}%
\pgfpathrectangle{\pgfqpoint{4.953938in}{0.529443in}}{\pgfqpoint{1.589120in}{1.745990in}}%
\pgfusepath{clip}%
\pgfsetbuttcap%
\pgfsetroundjoin%
\definecolor{currentfill}{rgb}{0.333333,0.658824,0.407843}%
\pgfsetfillcolor{currentfill}%
\pgfsetfillopacity{0.900000}%
\pgfsetlinewidth{0.507862pt}%
\definecolor{currentstroke}{rgb}{1.000000,1.000000,1.000000}%
\pgfsetstrokecolor{currentstroke}%
\pgfsetstrokeopacity{0.900000}%
\pgfsetdash{}{0pt}%
\pgfpathmoveto{\pgfqpoint{6.357667in}{2.158788in}}%
\pgfpathlineto{\pgfqpoint{6.357667in}{2.096675in}}%
\pgfpathlineto{\pgfqpoint{6.419780in}{2.096675in}}%
\pgfpathlineto{\pgfqpoint{6.419780in}{2.158788in}}%
\pgfpathlineto{\pgfqpoint{6.357667in}{2.158788in}}%
\pgfpathclose%
\pgfusepath{stroke,fill}%
\end{pgfscope}%
\begin{pgfscope}%
\pgfpathrectangle{\pgfqpoint{4.953938in}{0.529443in}}{\pgfqpoint{1.589120in}{1.745990in}}%
\pgfusepath{clip}%
\pgfsetbuttcap%
\pgfsetroundjoin%
\definecolor{currentfill}{rgb}{0.333333,0.658824,0.407843}%
\pgfsetfillcolor{currentfill}%
\pgfsetfillopacity{0.900000}%
\pgfsetlinewidth{0.507862pt}%
\definecolor{currentstroke}{rgb}{1.000000,1.000000,1.000000}%
\pgfsetstrokecolor{currentstroke}%
\pgfsetstrokeopacity{0.900000}%
\pgfsetdash{}{0pt}%
\pgfpathmoveto{\pgfqpoint{5.029727in}{0.592260in}}%
\pgfpathlineto{\pgfqpoint{5.051687in}{0.614220in}}%
\pgfpathlineto{\pgfqpoint{5.073647in}{0.592260in}}%
\pgfpathlineto{\pgfqpoint{5.095608in}{0.614220in}}%
\pgfpathlineto{\pgfqpoint{5.073647in}{0.636180in}}%
\pgfpathlineto{\pgfqpoint{5.095608in}{0.658140in}}%
\pgfpathlineto{\pgfqpoint{5.073647in}{0.680101in}}%
\pgfpathlineto{\pgfqpoint{5.051687in}{0.658140in}}%
\pgfpathlineto{\pgfqpoint{5.029727in}{0.680101in}}%
\pgfpathlineto{\pgfqpoint{5.007767in}{0.658140in}}%
\pgfpathlineto{\pgfqpoint{5.029727in}{0.636180in}}%
\pgfpathlineto{\pgfqpoint{5.007767in}{0.614220in}}%
\pgfpathlineto{\pgfqpoint{5.029727in}{0.592260in}}%
\pgfpathclose%
\pgfusepath{stroke,fill}%
\end{pgfscope}%
\begin{pgfscope}%
\pgfpathrectangle{\pgfqpoint{4.953938in}{0.529443in}}{\pgfqpoint{1.589120in}{1.745990in}}%
\pgfusepath{clip}%
\pgfsetbuttcap%
\pgfsetroundjoin%
\definecolor{currentfill}{rgb}{0.298039,0.447059,0.690196}%
\pgfsetfillcolor{currentfill}%
\pgfsetfillopacity{0.900000}%
\pgfsetlinewidth{0.507862pt}%
\definecolor{currentstroke}{rgb}{1.000000,1.000000,1.000000}%
\pgfsetstrokecolor{currentstroke}%
\pgfsetstrokeopacity{0.900000}%
\pgfsetdash{}{0pt}%
\pgfpathmoveto{\pgfqpoint{5.064564in}{0.614173in}}%
\pgfpathlineto{\pgfqpoint{5.086524in}{0.636134in}}%
\pgfpathlineto{\pgfqpoint{5.108484in}{0.614173in}}%
\pgfpathlineto{\pgfqpoint{5.130445in}{0.636134in}}%
\pgfpathlineto{\pgfqpoint{5.108484in}{0.658094in}}%
\pgfpathlineto{\pgfqpoint{5.130445in}{0.680054in}}%
\pgfpathlineto{\pgfqpoint{5.108484in}{0.702014in}}%
\pgfpathlineto{\pgfqpoint{5.086524in}{0.680054in}}%
\pgfpathlineto{\pgfqpoint{5.064564in}{0.702014in}}%
\pgfpathlineto{\pgfqpoint{5.042604in}{0.680054in}}%
\pgfpathlineto{\pgfqpoint{5.064564in}{0.658094in}}%
\pgfpathlineto{\pgfqpoint{5.042604in}{0.636134in}}%
\pgfpathlineto{\pgfqpoint{5.064564in}{0.614173in}}%
\pgfpathclose%
\pgfusepath{stroke,fill}%
\end{pgfscope}%
\begin{pgfscope}%
\pgfpathrectangle{\pgfqpoint{4.953938in}{0.529443in}}{\pgfqpoint{1.589120in}{1.745990in}}%
\pgfusepath{clip}%
\pgfsetbuttcap%
\pgfsetroundjoin%
\definecolor{currentfill}{rgb}{0.333333,0.658824,0.407843}%
\pgfsetfillcolor{currentfill}%
\pgfsetfillopacity{0.900000}%
\pgfsetlinewidth{0.507862pt}%
\definecolor{currentstroke}{rgb}{1.000000,1.000000,1.000000}%
\pgfsetstrokecolor{currentstroke}%
\pgfsetstrokeopacity{0.900000}%
\pgfsetdash{}{0pt}%
\pgfpathmoveto{\pgfqpoint{6.331540in}{2.192718in}}%
\pgfpathlineto{\pgfqpoint{6.331540in}{2.130605in}}%
\pgfpathlineto{\pgfqpoint{6.393653in}{2.130605in}}%
\pgfpathlineto{\pgfqpoint{6.393653in}{2.192718in}}%
\pgfpathlineto{\pgfqpoint{6.331540in}{2.192718in}}%
\pgfpathclose%
\pgfusepath{stroke,fill}%
\end{pgfscope}%
\begin{pgfscope}%
\pgfpathrectangle{\pgfqpoint{4.953938in}{0.529443in}}{\pgfqpoint{1.589120in}{1.745990in}}%
\pgfusepath{clip}%
\pgfsetbuttcap%
\pgfsetroundjoin%
\definecolor{currentfill}{rgb}{0.333333,0.658824,0.407843}%
\pgfsetfillcolor{currentfill}%
\pgfsetfillopacity{0.900000}%
\pgfsetlinewidth{0.507862pt}%
\definecolor{currentstroke}{rgb}{1.000000,1.000000,1.000000}%
\pgfsetstrokecolor{currentstroke}%
\pgfsetstrokeopacity{0.900000}%
\pgfsetdash{}{0pt}%
\pgfpathmoveto{\pgfqpoint{5.023046in}{0.799957in}}%
\pgfpathlineto{\pgfqpoint{5.023046in}{0.737844in}}%
\pgfpathlineto{\pgfqpoint{5.085159in}{0.737844in}}%
\pgfpathlineto{\pgfqpoint{5.085159in}{0.799957in}}%
\pgfpathlineto{\pgfqpoint{5.023046in}{0.799957in}}%
\pgfpathclose%
\pgfusepath{stroke,fill}%
\end{pgfscope}%
\begin{pgfscope}%
\pgfpathrectangle{\pgfqpoint{4.953938in}{0.529443in}}{\pgfqpoint{1.589120in}{1.745990in}}%
\pgfusepath{clip}%
\pgfsetbuttcap%
\pgfsetroundjoin%
\definecolor{currentfill}{rgb}{0.333333,0.658824,0.407843}%
\pgfsetfillcolor{currentfill}%
\pgfsetfillopacity{0.900000}%
\pgfsetlinewidth{0.507862pt}%
\definecolor{currentstroke}{rgb}{1.000000,1.000000,1.000000}%
\pgfsetstrokecolor{currentstroke}%
\pgfsetstrokeopacity{0.900000}%
\pgfsetdash{}{0pt}%
\pgfpathmoveto{\pgfqpoint{6.309237in}{2.122807in}}%
\pgfpathlineto{\pgfqpoint{6.309237in}{2.060694in}}%
\pgfpathlineto{\pgfqpoint{6.371350in}{2.060694in}}%
\pgfpathlineto{\pgfqpoint{6.371350in}{2.122807in}}%
\pgfpathlineto{\pgfqpoint{6.309237in}{2.122807in}}%
\pgfpathclose%
\pgfusepath{stroke,fill}%
\end{pgfscope}%
\begin{pgfscope}%
\pgfpathrectangle{\pgfqpoint{4.953938in}{0.529443in}}{\pgfqpoint{1.589120in}{1.745990in}}%
\pgfusepath{clip}%
\pgfsetbuttcap%
\pgfsetroundjoin%
\definecolor{currentfill}{rgb}{0.298039,0.447059,0.690196}%
\pgfsetfillcolor{currentfill}%
\pgfsetfillopacity{0.900000}%
\pgfsetlinewidth{0.507862pt}%
\definecolor{currentstroke}{rgb}{1.000000,1.000000,1.000000}%
\pgfsetstrokecolor{currentstroke}%
\pgfsetstrokeopacity{0.900000}%
\pgfsetdash{}{0pt}%
\pgfpathmoveto{\pgfqpoint{6.274758in}{2.027395in}}%
\pgfpathcurveto{\pgfqpoint{6.286406in}{2.027395in}}{\pgfqpoint{6.297578in}{2.032023in}}{\pgfqpoint{6.305815in}{2.040259in}}%
\pgfpathcurveto{\pgfqpoint{6.314051in}{2.048496in}}{\pgfqpoint{6.318679in}{2.059668in}}{\pgfqpoint{6.318679in}{2.071316in}}%
\pgfpathcurveto{\pgfqpoint{6.318679in}{2.082964in}}{\pgfqpoint{6.314051in}{2.094136in}}{\pgfqpoint{6.305815in}{2.102372in}}%
\pgfpathcurveto{\pgfqpoint{6.297578in}{2.110609in}}{\pgfqpoint{6.286406in}{2.115236in}}{\pgfqpoint{6.274758in}{2.115236in}}%
\pgfpathcurveto{\pgfqpoint{6.263110in}{2.115236in}}{\pgfqpoint{6.251938in}{2.110609in}}{\pgfqpoint{6.243702in}{2.102372in}}%
\pgfpathcurveto{\pgfqpoint{6.235465in}{2.094136in}}{\pgfqpoint{6.230838in}{2.082964in}}{\pgfqpoint{6.230838in}{2.071316in}}%
\pgfpathcurveto{\pgfqpoint{6.230838in}{2.059668in}}{\pgfqpoint{6.235465in}{2.048496in}}{\pgfqpoint{6.243702in}{2.040259in}}%
\pgfpathcurveto{\pgfqpoint{6.251938in}{2.032023in}}{\pgfqpoint{6.263110in}{2.027395in}}{\pgfqpoint{6.274758in}{2.027395in}}%
\pgfpathlineto{\pgfqpoint{6.274758in}{2.027395in}}%
\pgfpathclose%
\pgfusepath{stroke,fill}%
\end{pgfscope}%
\begin{pgfscope}%
\pgfpathrectangle{\pgfqpoint{4.953938in}{0.529443in}}{\pgfqpoint{1.589120in}{1.745990in}}%
\pgfusepath{clip}%
\pgfsetbuttcap%
\pgfsetroundjoin%
\definecolor{currentfill}{rgb}{0.333333,0.658824,0.407843}%
\pgfsetfillcolor{currentfill}%
\pgfsetfillopacity{0.900000}%
\pgfsetlinewidth{0.507862pt}%
\definecolor{currentstroke}{rgb}{1.000000,1.000000,1.000000}%
\pgfsetstrokecolor{currentstroke}%
\pgfsetstrokeopacity{0.900000}%
\pgfsetdash{}{0pt}%
\pgfpathmoveto{\pgfqpoint{6.351496in}{2.227126in}}%
\pgfpathlineto{\pgfqpoint{6.351496in}{2.165013in}}%
\pgfpathlineto{\pgfqpoint{6.413609in}{2.165013in}}%
\pgfpathlineto{\pgfqpoint{6.413609in}{2.227126in}}%
\pgfpathlineto{\pgfqpoint{6.351496in}{2.227126in}}%
\pgfpathclose%
\pgfusepath{stroke,fill}%
\end{pgfscope}%
\begin{pgfscope}%
\pgfpathrectangle{\pgfqpoint{4.953938in}{0.529443in}}{\pgfqpoint{1.589120in}{1.745990in}}%
\pgfusepath{clip}%
\pgfsetbuttcap%
\pgfsetroundjoin%
\definecolor{currentfill}{rgb}{0.866667,0.517647,0.321569}%
\pgfsetfillcolor{currentfill}%
\pgfsetfillopacity{0.900000}%
\pgfsetlinewidth{0.507862pt}%
\definecolor{currentstroke}{rgb}{1.000000,1.000000,1.000000}%
\pgfsetstrokecolor{currentstroke}%
\pgfsetstrokeopacity{0.900000}%
\pgfsetdash{}{0pt}%
\pgfpathmoveto{\pgfqpoint{6.400309in}{2.152980in}}%
\pgfpathlineto{\pgfqpoint{6.400309in}{2.090867in}}%
\pgfpathlineto{\pgfqpoint{6.462422in}{2.090867in}}%
\pgfpathlineto{\pgfqpoint{6.462422in}{2.152980in}}%
\pgfpathlineto{\pgfqpoint{6.400309in}{2.152980in}}%
\pgfpathclose%
\pgfusepath{stroke,fill}%
\end{pgfscope}%
\begin{pgfscope}%
\pgfpathrectangle{\pgfqpoint{4.953938in}{0.529443in}}{\pgfqpoint{1.589120in}{1.745990in}}%
\pgfusepath{clip}%
\pgfsetbuttcap%
\pgfsetroundjoin%
\definecolor{currentfill}{rgb}{0.298039,0.447059,0.690196}%
\pgfsetfillcolor{currentfill}%
\pgfsetfillopacity{0.900000}%
\pgfsetlinewidth{0.507862pt}%
\definecolor{currentstroke}{rgb}{1.000000,1.000000,1.000000}%
\pgfsetstrokecolor{currentstroke}%
\pgfsetstrokeopacity{0.900000}%
\pgfsetdash{}{0pt}%
\pgfpathmoveto{\pgfqpoint{6.392433in}{2.174256in}}%
\pgfpathlineto{\pgfqpoint{6.392433in}{2.112143in}}%
\pgfpathlineto{\pgfqpoint{6.454546in}{2.112143in}}%
\pgfpathlineto{\pgfqpoint{6.454546in}{2.174256in}}%
\pgfpathlineto{\pgfqpoint{6.392433in}{2.174256in}}%
\pgfpathclose%
\pgfusepath{stroke,fill}%
\end{pgfscope}%
\begin{pgfscope}%
\pgfpathrectangle{\pgfqpoint{4.953938in}{0.529443in}}{\pgfqpoint{1.589120in}{1.745990in}}%
\pgfusepath{clip}%
\pgfsetbuttcap%
\pgfsetroundjoin%
\definecolor{currentfill}{rgb}{0.298039,0.447059,0.690196}%
\pgfsetfillcolor{currentfill}%
\pgfsetfillopacity{0.900000}%
\pgfsetlinewidth{0.507862pt}%
\definecolor{currentstroke}{rgb}{1.000000,1.000000,1.000000}%
\pgfsetstrokecolor{currentstroke}%
\pgfsetstrokeopacity{0.900000}%
\pgfsetdash{}{0pt}%
\pgfpathmoveto{\pgfqpoint{5.057684in}{0.572497in}}%
\pgfpathlineto{\pgfqpoint{5.079644in}{0.594457in}}%
\pgfpathlineto{\pgfqpoint{5.101604in}{0.572497in}}%
\pgfpathlineto{\pgfqpoint{5.123565in}{0.594457in}}%
\pgfpathlineto{\pgfqpoint{5.101604in}{0.616418in}}%
\pgfpathlineto{\pgfqpoint{5.123565in}{0.638378in}}%
\pgfpathlineto{\pgfqpoint{5.101604in}{0.660338in}}%
\pgfpathlineto{\pgfqpoint{5.079644in}{0.638378in}}%
\pgfpathlineto{\pgfqpoint{5.057684in}{0.660338in}}%
\pgfpathlineto{\pgfqpoint{5.035724in}{0.638378in}}%
\pgfpathlineto{\pgfqpoint{5.057684in}{0.616418in}}%
\pgfpathlineto{\pgfqpoint{5.035724in}{0.594457in}}%
\pgfpathlineto{\pgfqpoint{5.057684in}{0.572497in}}%
\pgfpathclose%
\pgfusepath{stroke,fill}%
\end{pgfscope}%
\begin{pgfscope}%
\pgfpathrectangle{\pgfqpoint{4.953938in}{0.529443in}}{\pgfqpoint{1.589120in}{1.745990in}}%
\pgfusepath{clip}%
\pgfsetbuttcap%
\pgfsetroundjoin%
\definecolor{currentfill}{rgb}{0.866667,0.517647,0.321569}%
\pgfsetfillcolor{currentfill}%
\pgfsetfillopacity{0.900000}%
\pgfsetlinewidth{0.507862pt}%
\definecolor{currentstroke}{rgb}{1.000000,1.000000,1.000000}%
\pgfsetstrokecolor{currentstroke}%
\pgfsetstrokeopacity{0.900000}%
\pgfsetdash{}{0pt}%
\pgfpathmoveto{\pgfqpoint{6.294233in}{2.147594in}}%
\pgfpathlineto{\pgfqpoint{6.294233in}{2.085481in}}%
\pgfpathlineto{\pgfqpoint{6.356346in}{2.085481in}}%
\pgfpathlineto{\pgfqpoint{6.356346in}{2.147594in}}%
\pgfpathlineto{\pgfqpoint{6.294233in}{2.147594in}}%
\pgfpathclose%
\pgfusepath{stroke,fill}%
\end{pgfscope}%
\begin{pgfscope}%
\pgfpathrectangle{\pgfqpoint{4.953938in}{0.529443in}}{\pgfqpoint{1.589120in}{1.745990in}}%
\pgfusepath{clip}%
\pgfsetbuttcap%
\pgfsetroundjoin%
\definecolor{currentfill}{rgb}{0.298039,0.447059,0.690196}%
\pgfsetfillcolor{currentfill}%
\pgfsetfillopacity{0.900000}%
\pgfsetlinewidth{0.507862pt}%
\definecolor{currentstroke}{rgb}{1.000000,1.000000,1.000000}%
\pgfsetstrokecolor{currentstroke}%
\pgfsetstrokeopacity{0.900000}%
\pgfsetdash{}{0pt}%
\pgfpathmoveto{\pgfqpoint{6.397467in}{2.224595in}}%
\pgfpathlineto{\pgfqpoint{6.397467in}{2.162482in}}%
\pgfpathlineto{\pgfqpoint{6.459580in}{2.162482in}}%
\pgfpathlineto{\pgfqpoint{6.459580in}{2.224595in}}%
\pgfpathlineto{\pgfqpoint{6.397467in}{2.224595in}}%
\pgfpathclose%
\pgfusepath{stroke,fill}%
\end{pgfscope}%
\begin{pgfscope}%
\pgfpathrectangle{\pgfqpoint{4.953938in}{0.529443in}}{\pgfqpoint{1.589120in}{1.745990in}}%
\pgfusepath{clip}%
\pgfsetbuttcap%
\pgfsetroundjoin%
\definecolor{currentfill}{rgb}{0.866667,0.517647,0.321569}%
\pgfsetfillcolor{currentfill}%
\pgfsetfillopacity{0.900000}%
\pgfsetlinewidth{0.507862pt}%
\definecolor{currentstroke}{rgb}{1.000000,1.000000,1.000000}%
\pgfsetstrokecolor{currentstroke}%
\pgfsetstrokeopacity{0.900000}%
\pgfsetdash{}{0pt}%
\pgfpathmoveto{\pgfqpoint{5.084624in}{0.589694in}}%
\pgfpathlineto{\pgfqpoint{5.106584in}{0.611654in}}%
\pgfpathlineto{\pgfqpoint{5.128544in}{0.589694in}}%
\pgfpathlineto{\pgfqpoint{5.150504in}{0.611654in}}%
\pgfpathlineto{\pgfqpoint{5.128544in}{0.633614in}}%
\pgfpathlineto{\pgfqpoint{5.150504in}{0.655574in}}%
\pgfpathlineto{\pgfqpoint{5.128544in}{0.677535in}}%
\pgfpathlineto{\pgfqpoint{5.106584in}{0.655574in}}%
\pgfpathlineto{\pgfqpoint{5.084624in}{0.677535in}}%
\pgfpathlineto{\pgfqpoint{5.062663in}{0.655574in}}%
\pgfpathlineto{\pgfqpoint{5.084624in}{0.633614in}}%
\pgfpathlineto{\pgfqpoint{5.062663in}{0.611654in}}%
\pgfpathlineto{\pgfqpoint{5.084624in}{0.589694in}}%
\pgfpathclose%
\pgfusepath{stroke,fill}%
\end{pgfscope}%
\begin{pgfscope}%
\pgfpathrectangle{\pgfqpoint{4.953938in}{0.529443in}}{\pgfqpoint{1.589120in}{1.745990in}}%
\pgfusepath{clip}%
\pgfsetbuttcap%
\pgfsetroundjoin%
\definecolor{currentfill}{rgb}{0.866667,0.517647,0.321569}%
\pgfsetfillcolor{currentfill}%
\pgfsetfillopacity{0.900000}%
\pgfsetlinewidth{0.507862pt}%
\definecolor{currentstroke}{rgb}{1.000000,1.000000,1.000000}%
\pgfsetstrokecolor{currentstroke}%
\pgfsetstrokeopacity{0.900000}%
\pgfsetdash{}{0pt}%
\pgfpathmoveto{\pgfqpoint{6.253410in}{2.209173in}}%
\pgfpathlineto{\pgfqpoint{6.253410in}{2.147060in}}%
\pgfpathlineto{\pgfqpoint{6.315523in}{2.147060in}}%
\pgfpathlineto{\pgfqpoint{6.315523in}{2.209173in}}%
\pgfpathlineto{\pgfqpoint{6.253410in}{2.209173in}}%
\pgfpathclose%
\pgfusepath{stroke,fill}%
\end{pgfscope}%
\begin{pgfscope}%
\pgfpathrectangle{\pgfqpoint{4.953938in}{0.529443in}}{\pgfqpoint{1.589120in}{1.745990in}}%
\pgfusepath{clip}%
\pgfsetbuttcap%
\pgfsetroundjoin%
\definecolor{currentfill}{rgb}{0.333333,0.658824,0.407843}%
\pgfsetfillcolor{currentfill}%
\pgfsetfillopacity{0.900000}%
\pgfsetlinewidth{0.507862pt}%
\definecolor{currentstroke}{rgb}{1.000000,1.000000,1.000000}%
\pgfsetstrokecolor{currentstroke}%
\pgfsetstrokeopacity{0.900000}%
\pgfsetdash{}{0pt}%
\pgfpathmoveto{\pgfqpoint{5.022415in}{0.568486in}}%
\pgfpathlineto{\pgfqpoint{5.044375in}{0.590446in}}%
\pgfpathlineto{\pgfqpoint{5.066335in}{0.568486in}}%
\pgfpathlineto{\pgfqpoint{5.088296in}{0.590446in}}%
\pgfpathlineto{\pgfqpoint{5.066335in}{0.612407in}}%
\pgfpathlineto{\pgfqpoint{5.088296in}{0.634367in}}%
\pgfpathlineto{\pgfqpoint{5.066335in}{0.656327in}}%
\pgfpathlineto{\pgfqpoint{5.044375in}{0.634367in}}%
\pgfpathlineto{\pgfqpoint{5.022415in}{0.656327in}}%
\pgfpathlineto{\pgfqpoint{5.000455in}{0.634367in}}%
\pgfpathlineto{\pgfqpoint{5.022415in}{0.612407in}}%
\pgfpathlineto{\pgfqpoint{5.000455in}{0.590446in}}%
\pgfpathlineto{\pgfqpoint{5.022415in}{0.568486in}}%
\pgfpathclose%
\pgfusepath{stroke,fill}%
\end{pgfscope}%
\begin{pgfscope}%
\pgfpathrectangle{\pgfqpoint{4.953938in}{0.529443in}}{\pgfqpoint{1.589120in}{1.745990in}}%
\pgfusepath{clip}%
\pgfsetbuttcap%
\pgfsetroundjoin%
\definecolor{currentfill}{rgb}{0.866667,0.517647,0.321569}%
\pgfsetfillcolor{currentfill}%
\pgfsetfillopacity{0.900000}%
\pgfsetlinewidth{0.507862pt}%
\definecolor{currentstroke}{rgb}{1.000000,1.000000,1.000000}%
\pgfsetstrokecolor{currentstroke}%
\pgfsetstrokeopacity{0.900000}%
\pgfsetdash{}{0pt}%
\pgfpathmoveto{\pgfqpoint{6.381108in}{2.127125in}}%
\pgfpathlineto{\pgfqpoint{6.381108in}{2.065012in}}%
\pgfpathlineto{\pgfqpoint{6.443221in}{2.065012in}}%
\pgfpathlineto{\pgfqpoint{6.443221in}{2.127125in}}%
\pgfpathlineto{\pgfqpoint{6.381108in}{2.127125in}}%
\pgfpathclose%
\pgfusepath{stroke,fill}%
\end{pgfscope}%
\begin{pgfscope}%
\pgfpathrectangle{\pgfqpoint{4.953938in}{0.529443in}}{\pgfqpoint{1.589120in}{1.745990in}}%
\pgfusepath{clip}%
\pgfsetbuttcap%
\pgfsetroundjoin%
\definecolor{currentfill}{rgb}{0.298039,0.447059,0.690196}%
\pgfsetfillcolor{currentfill}%
\pgfsetfillopacity{0.900000}%
\pgfsetlinewidth{0.507862pt}%
\definecolor{currentstroke}{rgb}{1.000000,1.000000,1.000000}%
\pgfsetstrokecolor{currentstroke}%
\pgfsetstrokeopacity{0.900000}%
\pgfsetdash{}{0pt}%
\pgfpathmoveto{\pgfqpoint{6.380112in}{2.199028in}}%
\pgfpathlineto{\pgfqpoint{6.380112in}{2.136915in}}%
\pgfpathlineto{\pgfqpoint{6.442225in}{2.136915in}}%
\pgfpathlineto{\pgfqpoint{6.442225in}{2.199028in}}%
\pgfpathlineto{\pgfqpoint{6.380112in}{2.199028in}}%
\pgfpathclose%
\pgfusepath{stroke,fill}%
\end{pgfscope}%
\begin{pgfscope}%
\pgfpathrectangle{\pgfqpoint{4.953938in}{0.529443in}}{\pgfqpoint{1.589120in}{1.745990in}}%
\pgfusepath{clip}%
\pgfsetbuttcap%
\pgfsetroundjoin%
\definecolor{currentfill}{rgb}{0.866667,0.517647,0.321569}%
\pgfsetfillcolor{currentfill}%
\pgfsetfillopacity{0.900000}%
\pgfsetlinewidth{0.507862pt}%
\definecolor{currentstroke}{rgb}{1.000000,1.000000,1.000000}%
\pgfsetstrokecolor{currentstroke}%
\pgfsetstrokeopacity{0.900000}%
\pgfsetdash{}{0pt}%
\pgfpathmoveto{\pgfqpoint{6.344946in}{2.134410in}}%
\pgfpathlineto{\pgfqpoint{6.344946in}{2.072297in}}%
\pgfpathlineto{\pgfqpoint{6.407059in}{2.072297in}}%
\pgfpathlineto{\pgfqpoint{6.407059in}{2.134410in}}%
\pgfpathlineto{\pgfqpoint{6.344946in}{2.134410in}}%
\pgfpathclose%
\pgfusepath{stroke,fill}%
\end{pgfscope}%
\begin{pgfscope}%
\pgfpathrectangle{\pgfqpoint{4.953938in}{0.529443in}}{\pgfqpoint{1.589120in}{1.745990in}}%
\pgfusepath{clip}%
\pgfsetbuttcap%
\pgfsetroundjoin%
\definecolor{currentfill}{rgb}{0.866667,0.517647,0.321569}%
\pgfsetfillcolor{currentfill}%
\pgfsetfillopacity{0.900000}%
\pgfsetlinewidth{0.507862pt}%
\definecolor{currentstroke}{rgb}{1.000000,1.000000,1.000000}%
\pgfsetstrokecolor{currentstroke}%
\pgfsetstrokeopacity{0.900000}%
\pgfsetdash{}{0pt}%
\pgfpathmoveto{\pgfqpoint{5.085150in}{0.669741in}}%
\pgfpathcurveto{\pgfqpoint{5.096798in}{0.669741in}}{\pgfqpoint{5.107970in}{0.674369in}}{\pgfqpoint{5.116206in}{0.682605in}}%
\pgfpathcurveto{\pgfqpoint{5.124443in}{0.690841in}}{\pgfqpoint{5.129070in}{0.702014in}}{\pgfqpoint{5.129070in}{0.713661in}}%
\pgfpathcurveto{\pgfqpoint{5.129070in}{0.725309in}}{\pgfqpoint{5.124443in}{0.736482in}}{\pgfqpoint{5.116206in}{0.744718in}}%
\pgfpathcurveto{\pgfqpoint{5.107970in}{0.752954in}}{\pgfqpoint{5.096798in}{0.757582in}}{\pgfqpoint{5.085150in}{0.757582in}}%
\pgfpathcurveto{\pgfqpoint{5.073502in}{0.757582in}}{\pgfqpoint{5.062330in}{0.752954in}}{\pgfqpoint{5.054093in}{0.744718in}}%
\pgfpathcurveto{\pgfqpoint{5.045857in}{0.736482in}}{\pgfqpoint{5.041229in}{0.725309in}}{\pgfqpoint{5.041229in}{0.713661in}}%
\pgfpathcurveto{\pgfqpoint{5.041229in}{0.702014in}}{\pgfqpoint{5.045857in}{0.690841in}}{\pgfqpoint{5.054093in}{0.682605in}}%
\pgfpathcurveto{\pgfqpoint{5.062330in}{0.674369in}}{\pgfqpoint{5.073502in}{0.669741in}}{\pgfqpoint{5.085150in}{0.669741in}}%
\pgfpathlineto{\pgfqpoint{5.085150in}{0.669741in}}%
\pgfpathclose%
\pgfusepath{stroke,fill}%
\end{pgfscope}%
\begin{pgfscope}%
\pgfpathrectangle{\pgfqpoint{4.953938in}{0.529443in}}{\pgfqpoint{1.589120in}{1.745990in}}%
\pgfusepath{clip}%
\pgfsetbuttcap%
\pgfsetroundjoin%
\definecolor{currentfill}{rgb}{0.333333,0.658824,0.407843}%
\pgfsetfillcolor{currentfill}%
\pgfsetfillopacity{0.900000}%
\pgfsetlinewidth{0.507862pt}%
\definecolor{currentstroke}{rgb}{1.000000,1.000000,1.000000}%
\pgfsetstrokecolor{currentstroke}%
\pgfsetstrokeopacity{0.900000}%
\pgfsetdash{}{0pt}%
\pgfpathmoveto{\pgfqpoint{6.198798in}{2.181251in}}%
\pgfpathlineto{\pgfqpoint{6.198798in}{2.119138in}}%
\pgfpathlineto{\pgfqpoint{6.260911in}{2.119138in}}%
\pgfpathlineto{\pgfqpoint{6.260911in}{2.181251in}}%
\pgfpathlineto{\pgfqpoint{6.198798in}{2.181251in}}%
\pgfpathclose%
\pgfusepath{stroke,fill}%
\end{pgfscope}%
\begin{pgfscope}%
\pgfpathrectangle{\pgfqpoint{4.953938in}{0.529443in}}{\pgfqpoint{1.589120in}{1.745990in}}%
\pgfusepath{clip}%
\pgfsetbuttcap%
\pgfsetroundjoin%
\definecolor{currentfill}{rgb}{0.333333,0.658824,0.407843}%
\pgfsetfillcolor{currentfill}%
\pgfsetfillopacity{0.900000}%
\pgfsetlinewidth{0.507862pt}%
\definecolor{currentstroke}{rgb}{1.000000,1.000000,1.000000}%
\pgfsetstrokecolor{currentstroke}%
\pgfsetstrokeopacity{0.900000}%
\pgfsetdash{}{0pt}%
\pgfpathmoveto{\pgfqpoint{5.060264in}{0.564886in}}%
\pgfpathlineto{\pgfqpoint{5.082224in}{0.586846in}}%
\pgfpathlineto{\pgfqpoint{5.104184in}{0.564886in}}%
\pgfpathlineto{\pgfqpoint{5.126144in}{0.586846in}}%
\pgfpathlineto{\pgfqpoint{5.104184in}{0.608806in}}%
\pgfpathlineto{\pgfqpoint{5.126144in}{0.630766in}}%
\pgfpathlineto{\pgfqpoint{5.104184in}{0.652727in}}%
\pgfpathlineto{\pgfqpoint{5.082224in}{0.630766in}}%
\pgfpathlineto{\pgfqpoint{5.060264in}{0.652727in}}%
\pgfpathlineto{\pgfqpoint{5.038303in}{0.630766in}}%
\pgfpathlineto{\pgfqpoint{5.060264in}{0.608806in}}%
\pgfpathlineto{\pgfqpoint{5.038303in}{0.586846in}}%
\pgfpathlineto{\pgfqpoint{5.060264in}{0.564886in}}%
\pgfpathclose%
\pgfusepath{stroke,fill}%
\end{pgfscope}%
\begin{pgfscope}%
\pgfpathrectangle{\pgfqpoint{4.953938in}{0.529443in}}{\pgfqpoint{1.589120in}{1.745990in}}%
\pgfusepath{clip}%
\pgfsetbuttcap%
\pgfsetroundjoin%
\definecolor{currentfill}{rgb}{0.298039,0.447059,0.690196}%
\pgfsetfillcolor{currentfill}%
\pgfsetfillopacity{0.900000}%
\pgfsetlinewidth{0.507862pt}%
\definecolor{currentstroke}{rgb}{1.000000,1.000000,1.000000}%
\pgfsetstrokecolor{currentstroke}%
\pgfsetstrokeopacity{0.900000}%
\pgfsetdash{}{0pt}%
\pgfpathmoveto{\pgfqpoint{6.439769in}{2.182669in}}%
\pgfpathlineto{\pgfqpoint{6.439769in}{2.120556in}}%
\pgfpathlineto{\pgfqpoint{6.501882in}{2.120556in}}%
\pgfpathlineto{\pgfqpoint{6.501882in}{2.182669in}}%
\pgfpathlineto{\pgfqpoint{6.439769in}{2.182669in}}%
\pgfpathclose%
\pgfusepath{stroke,fill}%
\end{pgfscope}%
\begin{pgfscope}%
\pgfpathrectangle{\pgfqpoint{4.953938in}{0.529443in}}{\pgfqpoint{1.589120in}{1.745990in}}%
\pgfusepath{clip}%
\pgfsetbuttcap%
\pgfsetroundjoin%
\definecolor{currentfill}{rgb}{0.866667,0.517647,0.321569}%
\pgfsetfillcolor{currentfill}%
\pgfsetfillopacity{0.900000}%
\pgfsetlinewidth{0.507862pt}%
\definecolor{currentstroke}{rgb}{1.000000,1.000000,1.000000}%
\pgfsetstrokecolor{currentstroke}%
\pgfsetstrokeopacity{0.900000}%
\pgfsetdash{}{0pt}%
\pgfpathmoveto{\pgfqpoint{5.087921in}{0.630775in}}%
\pgfpathlineto{\pgfqpoint{5.109882in}{0.652735in}}%
\pgfpathlineto{\pgfqpoint{5.131842in}{0.630775in}}%
\pgfpathlineto{\pgfqpoint{5.153802in}{0.652735in}}%
\pgfpathlineto{\pgfqpoint{5.131842in}{0.674695in}}%
\pgfpathlineto{\pgfqpoint{5.153802in}{0.696656in}}%
\pgfpathlineto{\pgfqpoint{5.131842in}{0.718616in}}%
\pgfpathlineto{\pgfqpoint{5.109882in}{0.696656in}}%
\pgfpathlineto{\pgfqpoint{5.087921in}{0.718616in}}%
\pgfpathlineto{\pgfqpoint{5.065961in}{0.696656in}}%
\pgfpathlineto{\pgfqpoint{5.087921in}{0.674695in}}%
\pgfpathlineto{\pgfqpoint{5.065961in}{0.652735in}}%
\pgfpathlineto{\pgfqpoint{5.087921in}{0.630775in}}%
\pgfpathclose%
\pgfusepath{stroke,fill}%
\end{pgfscope}%
\begin{pgfscope}%
\pgfpathrectangle{\pgfqpoint{4.953938in}{0.529443in}}{\pgfqpoint{1.589120in}{1.745990in}}%
\pgfusepath{clip}%
\pgfsetbuttcap%
\pgfsetroundjoin%
\definecolor{currentfill}{rgb}{0.298039,0.447059,0.690196}%
\pgfsetfillcolor{currentfill}%
\pgfsetfillopacity{0.900000}%
\pgfsetlinewidth{0.507862pt}%
\definecolor{currentstroke}{rgb}{1.000000,1.000000,1.000000}%
\pgfsetstrokecolor{currentstroke}%
\pgfsetstrokeopacity{0.900000}%
\pgfsetdash{}{0pt}%
\pgfpathmoveto{\pgfqpoint{6.318053in}{2.156378in}}%
\pgfpathlineto{\pgfqpoint{6.318053in}{2.094265in}}%
\pgfpathlineto{\pgfqpoint{6.380166in}{2.094265in}}%
\pgfpathlineto{\pgfqpoint{6.380166in}{2.156378in}}%
\pgfpathlineto{\pgfqpoint{6.318053in}{2.156378in}}%
\pgfpathclose%
\pgfusepath{stroke,fill}%
\end{pgfscope}%
\begin{pgfscope}%
\pgfpathrectangle{\pgfqpoint{4.953938in}{0.529443in}}{\pgfqpoint{1.589120in}{1.745990in}}%
\pgfusepath{clip}%
\pgfsetbuttcap%
\pgfsetroundjoin%
\definecolor{currentfill}{rgb}{0.333333,0.658824,0.407843}%
\pgfsetfillcolor{currentfill}%
\pgfsetfillopacity{0.900000}%
\pgfsetlinewidth{0.507862pt}%
\definecolor{currentstroke}{rgb}{1.000000,1.000000,1.000000}%
\pgfsetstrokecolor{currentstroke}%
\pgfsetstrokeopacity{0.900000}%
\pgfsetdash{}{0pt}%
\pgfpathmoveto{\pgfqpoint{5.033966in}{0.690222in}}%
\pgfpathlineto{\pgfqpoint{5.055926in}{0.712182in}}%
\pgfpathlineto{\pgfqpoint{5.077886in}{0.690222in}}%
\pgfpathlineto{\pgfqpoint{5.099846in}{0.712182in}}%
\pgfpathlineto{\pgfqpoint{5.077886in}{0.734142in}}%
\pgfpathlineto{\pgfqpoint{5.099846in}{0.756102in}}%
\pgfpathlineto{\pgfqpoint{5.077886in}{0.778063in}}%
\pgfpathlineto{\pgfqpoint{5.055926in}{0.756102in}}%
\pgfpathlineto{\pgfqpoint{5.033966in}{0.778063in}}%
\pgfpathlineto{\pgfqpoint{5.012005in}{0.756102in}}%
\pgfpathlineto{\pgfqpoint{5.033966in}{0.734142in}}%
\pgfpathlineto{\pgfqpoint{5.012005in}{0.712182in}}%
\pgfpathlineto{\pgfqpoint{5.033966in}{0.690222in}}%
\pgfpathclose%
\pgfusepath{stroke,fill}%
\end{pgfscope}%
\begin{pgfscope}%
\pgfpathrectangle{\pgfqpoint{4.953938in}{0.529443in}}{\pgfqpoint{1.589120in}{1.745990in}}%
\pgfusepath{clip}%
\pgfsetbuttcap%
\pgfsetroundjoin%
\definecolor{currentfill}{rgb}{0.333333,0.658824,0.407843}%
\pgfsetfillcolor{currentfill}%
\pgfsetfillopacity{0.900000}%
\pgfsetlinewidth{0.507862pt}%
\definecolor{currentstroke}{rgb}{1.000000,1.000000,1.000000}%
\pgfsetstrokecolor{currentstroke}%
\pgfsetstrokeopacity{0.900000}%
\pgfsetdash{}{0pt}%
\pgfpathmoveto{\pgfqpoint{6.435849in}{2.145851in}}%
\pgfpathlineto{\pgfqpoint{6.435849in}{2.083738in}}%
\pgfpathlineto{\pgfqpoint{6.497962in}{2.083738in}}%
\pgfpathlineto{\pgfqpoint{6.497962in}{2.145851in}}%
\pgfpathlineto{\pgfqpoint{6.435849in}{2.145851in}}%
\pgfpathclose%
\pgfusepath{stroke,fill}%
\end{pgfscope}%
\begin{pgfscope}%
\pgfpathrectangle{\pgfqpoint{4.953938in}{0.529443in}}{\pgfqpoint{1.589120in}{1.745990in}}%
\pgfusepath{clip}%
\pgfsetbuttcap%
\pgfsetroundjoin%
\definecolor{currentfill}{rgb}{0.333333,0.658824,0.407843}%
\pgfsetfillcolor{currentfill}%
\pgfsetfillopacity{0.900000}%
\pgfsetlinewidth{0.507862pt}%
\definecolor{currentstroke}{rgb}{1.000000,1.000000,1.000000}%
\pgfsetstrokecolor{currentstroke}%
\pgfsetstrokeopacity{0.900000}%
\pgfsetdash{}{0pt}%
\pgfpathmoveto{\pgfqpoint{5.004211in}{0.585327in}}%
\pgfpathlineto{\pgfqpoint{5.026171in}{0.607288in}}%
\pgfpathlineto{\pgfqpoint{5.048131in}{0.585327in}}%
\pgfpathlineto{\pgfqpoint{5.070092in}{0.607288in}}%
\pgfpathlineto{\pgfqpoint{5.048131in}{0.629248in}}%
\pgfpathlineto{\pgfqpoint{5.070092in}{0.651208in}}%
\pgfpathlineto{\pgfqpoint{5.048131in}{0.673169in}}%
\pgfpathlineto{\pgfqpoint{5.026171in}{0.651208in}}%
\pgfpathlineto{\pgfqpoint{5.004211in}{0.673169in}}%
\pgfpathlineto{\pgfqpoint{4.982250in}{0.651208in}}%
\pgfpathlineto{\pgfqpoint{5.004211in}{0.629248in}}%
\pgfpathlineto{\pgfqpoint{4.982250in}{0.607288in}}%
\pgfpathlineto{\pgfqpoint{5.004211in}{0.585327in}}%
\pgfpathclose%
\pgfusepath{stroke,fill}%
\end{pgfscope}%
\begin{pgfscope}%
\pgfpathrectangle{\pgfqpoint{4.953938in}{0.529443in}}{\pgfqpoint{1.589120in}{1.745990in}}%
\pgfusepath{clip}%
\pgfsetbuttcap%
\pgfsetroundjoin%
\definecolor{currentfill}{rgb}{0.298039,0.447059,0.690196}%
\pgfsetfillcolor{currentfill}%
\pgfsetfillopacity{0.900000}%
\pgfsetlinewidth{0.507862pt}%
\definecolor{currentstroke}{rgb}{1.000000,1.000000,1.000000}%
\pgfsetstrokecolor{currentstroke}%
\pgfsetstrokeopacity{0.900000}%
\pgfsetdash{}{0pt}%
\pgfpathmoveto{\pgfqpoint{6.417674in}{2.199847in}}%
\pgfpathlineto{\pgfqpoint{6.417674in}{2.137734in}}%
\pgfpathlineto{\pgfqpoint{6.479787in}{2.137734in}}%
\pgfpathlineto{\pgfqpoint{6.479787in}{2.199847in}}%
\pgfpathlineto{\pgfqpoint{6.417674in}{2.199847in}}%
\pgfpathclose%
\pgfusepath{stroke,fill}%
\end{pgfscope}%
\begin{pgfscope}%
\pgfpathrectangle{\pgfqpoint{4.953938in}{0.529443in}}{\pgfqpoint{1.589120in}{1.745990in}}%
\pgfusepath{clip}%
\pgfsetbuttcap%
\pgfsetroundjoin%
\definecolor{currentfill}{rgb}{0.333333,0.658824,0.407843}%
\pgfsetfillcolor{currentfill}%
\pgfsetfillopacity{0.900000}%
\pgfsetlinewidth{0.507862pt}%
\definecolor{currentstroke}{rgb}{1.000000,1.000000,1.000000}%
\pgfsetstrokecolor{currentstroke}%
\pgfsetstrokeopacity{0.900000}%
\pgfsetdash{}{0pt}%
\pgfpathmoveto{\pgfqpoint{5.004859in}{0.621006in}}%
\pgfpathlineto{\pgfqpoint{5.026819in}{0.642966in}}%
\pgfpathlineto{\pgfqpoint{5.048780in}{0.621006in}}%
\pgfpathlineto{\pgfqpoint{5.070740in}{0.642966in}}%
\pgfpathlineto{\pgfqpoint{5.048780in}{0.664927in}}%
\pgfpathlineto{\pgfqpoint{5.070740in}{0.686887in}}%
\pgfpathlineto{\pgfqpoint{5.048780in}{0.708847in}}%
\pgfpathlineto{\pgfqpoint{5.026819in}{0.686887in}}%
\pgfpathlineto{\pgfqpoint{5.004859in}{0.708847in}}%
\pgfpathlineto{\pgfqpoint{4.982899in}{0.686887in}}%
\pgfpathlineto{\pgfqpoint{5.004859in}{0.664927in}}%
\pgfpathlineto{\pgfqpoint{4.982899in}{0.642966in}}%
\pgfpathlineto{\pgfqpoint{5.004859in}{0.621006in}}%
\pgfpathclose%
\pgfusepath{stroke,fill}%
\end{pgfscope}%
\begin{pgfscope}%
\pgfsetrectcap%
\pgfsetmiterjoin%
\pgfsetlinewidth{1.254687pt}%
\definecolor{currentstroke}{rgb}{0.800000,0.800000,0.800000}%
\pgfsetstrokecolor{currentstroke}%
\pgfsetdash{}{0pt}%
\pgfpathmoveto{\pgfqpoint{4.953938in}{0.529443in}}%
\pgfpathlineto{\pgfqpoint{4.953938in}{2.275433in}}%
\pgfusepath{stroke}%
\end{pgfscope}%
\begin{pgfscope}%
\pgfsetrectcap%
\pgfsetmiterjoin%
\pgfsetlinewidth{1.254687pt}%
\definecolor{currentstroke}{rgb}{0.800000,0.800000,0.800000}%
\pgfsetstrokecolor{currentstroke}%
\pgfsetdash{}{0pt}%
\pgfpathmoveto{\pgfqpoint{6.543059in}{0.529443in}}%
\pgfpathlineto{\pgfqpoint{6.543059in}{2.275433in}}%
\pgfusepath{stroke}%
\end{pgfscope}%
\begin{pgfscope}%
\pgfsetrectcap%
\pgfsetmiterjoin%
\pgfsetlinewidth{1.254687pt}%
\definecolor{currentstroke}{rgb}{0.800000,0.800000,0.800000}%
\pgfsetstrokecolor{currentstroke}%
\pgfsetdash{}{0pt}%
\pgfpathmoveto{\pgfqpoint{4.953938in}{0.529443in}}%
\pgfpathlineto{\pgfqpoint{6.543059in}{0.529443in}}%
\pgfusepath{stroke}%
\end{pgfscope}%
\begin{pgfscope}%
\pgfsetrectcap%
\pgfsetmiterjoin%
\pgfsetlinewidth{1.254687pt}%
\definecolor{currentstroke}{rgb}{0.800000,0.800000,0.800000}%
\pgfsetstrokecolor{currentstroke}%
\pgfsetdash{}{0pt}%
\pgfpathmoveto{\pgfqpoint{4.953938in}{2.275433in}}%
\pgfpathlineto{\pgfqpoint{6.543059in}{2.275433in}}%
\pgfusepath{stroke}%
\end{pgfscope}%
\begin{pgfscope}%
\definecolor{textcolor}{rgb}{0.150000,0.150000,0.150000}%
\pgfsetstrokecolor{textcolor}%
\pgfsetfillcolor{textcolor}%
\pgftext[x=5.748498in,y=2.358766in,,base]{\color{textcolor}{\rmfamily\fontsize{11.000000}{13.200000}\selectfont\catcode`\^=\active\def^{\ifmmode\sp\else\^{}\fi}\catcode`\%=\active\def%{\%}Sentiment Bag-Of-Subgraphs}}%
\end{pgfscope}%
\begin{pgfscope}%
\definecolor{textcolor}{rgb}{0.150000,0.150000,0.150000}%
\pgfsetstrokecolor{textcolor}%
\pgfsetfillcolor{textcolor}%
\pgftext[x=3.318059in,y=2.885276in,,top]{\color{textcolor}{\rmfamily\fontsize{12.000000}{14.400000}\selectfont\catcode`\^=\active\def^{\ifmmode\sp\else\^{}\fi}\catcode`\%=\active\def%{\%}User graph embeddings with HDBSCAN clustering}}%
\end{pgfscope}%
\end{pgfpicture}%
\makeatother%
\endgroup%
\\
        %% Creator: Matplotlib, PGF backend
%%
%% To include the figure in your LaTeX document, write
%%   \input{<filename>.pgf}
%%
%% Make sure the required packages are loaded in your preamble
%%   \usepackage{pgf}
%%
%% Also ensure that all the required font packages are loaded; for instance,
%% the lmodern package is sometimes necessary when using math font.
%%   \usepackage{lmodern}
%%
%% Figures using additional raster images can only be included by \input if
%% they are in the same directory as the main LaTeX file. For loading figures
%% from other directories you can use the `import` package
%%   \usepackage{import}
%%
%% and then include the figures with
%%   \import{<path to file>}{<filename>.pgf}
%%
%% Matplotlib used the following preamble
%%   \def\mathdefault#1{#1}
%%   \everymath=\expandafter{\the\everymath\displaystyle}
%%   
%%   \ifdefined\pdftexversion\else  % non-pdftex case.
%%     \usepackage{fontspec}
%%     \setmainfont{DejaVuSerif.ttf}[Path=\detokenize{/home/mahf/.local/lib/python3.10/site-packages/matplotlib/mpl-data/fonts/ttf/}]
%%     \setsansfont{DejaVuSans.ttf}[Path=\detokenize{/home/mahf/.local/lib/python3.10/site-packages/matplotlib/mpl-data/fonts/ttf/}]
%%     \setmonofont{DejaVuSansMono.ttf}[Path=\detokenize{/home/mahf/.local/lib/python3.10/site-packages/matplotlib/mpl-data/fonts/ttf/}]
%%   \fi
%%   \makeatletter\@ifpackageloaded{underscore}{}{\usepackage[strings]{underscore}}\makeatother
%%
\begingroup%
\makeatletter%
\begin{pgfpicture}%
\pgfpathrectangle{\pgfpointorigin}{\pgfqpoint{2.614306in}{0.722591in}}%
\pgfusepath{use as bounding box, clip}%
\begin{pgfscope}%
\pgfsetbuttcap%
\pgfsetmiterjoin%
\definecolor{currentfill}{rgb}{1.000000,1.000000,1.000000}%
\pgfsetfillcolor{currentfill}%
\pgfsetlinewidth{0.000000pt}%
\definecolor{currentstroke}{rgb}{1.000000,1.000000,1.000000}%
\pgfsetstrokecolor{currentstroke}%
\pgfsetdash{}{0pt}%
\pgfpathmoveto{\pgfqpoint{0.000000in}{0.000000in}}%
\pgfpathlineto{\pgfqpoint{2.614306in}{0.000000in}}%
\pgfpathlineto{\pgfqpoint{2.614306in}{0.722591in}}%
\pgfpathlineto{\pgfqpoint{0.000000in}{0.722591in}}%
\pgfpathlineto{\pgfqpoint{0.000000in}{0.000000in}}%
\pgfpathclose%
\pgfusepath{fill}%
\end{pgfscope}%
\begin{pgfscope}%
\pgfsetbuttcap%
\pgfsetroundjoin%
\definecolor{currentfill}{rgb}{0.298039,0.447059,0.690196}%
\pgfsetfillcolor{currentfill}%
\pgfsetfillopacity{0.900000}%
\pgfsetlinewidth{0.507862pt}%
\definecolor{currentstroke}{rgb}{1.000000,1.000000,1.000000}%
\pgfsetstrokecolor{currentstroke}%
\pgfsetstrokeopacity{0.900000}%
\pgfsetdash{}{0pt}%
\pgfsys@defobject{currentmarker}{\pgfqpoint{-0.043921in}{-0.043921in}}{\pgfqpoint{0.043921in}{0.043921in}}{%
\pgfpathmoveto{\pgfqpoint{0.000000in}{-0.043921in}}%
\pgfpathcurveto{\pgfqpoint{0.011648in}{-0.043921in}}{\pgfqpoint{0.022820in}{-0.039293in}}{\pgfqpoint{0.031056in}{-0.031056in}}%
\pgfpathcurveto{\pgfqpoint{0.039293in}{-0.022820in}}{\pgfqpoint{0.043921in}{-0.011648in}}{\pgfqpoint{0.043921in}{0.000000in}}%
\pgfpathcurveto{\pgfqpoint{0.043921in}{0.011648in}}{\pgfqpoint{0.039293in}{0.022820in}}{\pgfqpoint{0.031056in}{0.031056in}}%
\pgfpathcurveto{\pgfqpoint{0.022820in}{0.039293in}}{\pgfqpoint{0.011648in}{0.043921in}}{\pgfqpoint{0.000000in}{0.043921in}}%
\pgfpathcurveto{\pgfqpoint{-0.011648in}{0.043921in}}{\pgfqpoint{-0.022820in}{0.039293in}}{\pgfqpoint{-0.031056in}{0.031056in}}%
\pgfpathcurveto{\pgfqpoint{-0.039293in}{0.022820in}}{\pgfqpoint{-0.043921in}{0.011648in}}{\pgfqpoint{-0.043921in}{0.000000in}}%
\pgfpathcurveto{\pgfqpoint{-0.043921in}{-0.011648in}}{\pgfqpoint{-0.039293in}{-0.022820in}}{\pgfqpoint{-0.031056in}{-0.031056in}}%
\pgfpathcurveto{\pgfqpoint{-0.022820in}{-0.039293in}}{\pgfqpoint{-0.011648in}{-0.043921in}}{\pgfqpoint{0.000000in}{-0.043921in}}%
\pgfpathlineto{\pgfqpoint{0.000000in}{-0.043921in}}%
\pgfpathclose%
\pgfusepath{stroke,fill}%
}%
\begin{pgfscope}%
\pgfsys@transformshift{0.255556in}{0.532617in}%
\pgfsys@useobject{currentmarker}{}%
\end{pgfscope}%
\end{pgfscope}%
\begin{pgfscope}%
\definecolor{textcolor}{rgb}{0.150000,0.150000,0.150000}%
\pgfsetstrokecolor{textcolor}%
\pgfsetfillcolor{textcolor}%
\pgftext[x=0.455556in,y=0.493728in,left,base]{\color{textcolor}{\rmfamily\fontsize{8.000000}{9.600000}\selectfont\catcode`\^=\active\def^{\ifmmode\sp\else\^{}\fi}\catcode`\%=\active\def%{\%}Political}}%
\end{pgfscope}%
\begin{pgfscope}%
\pgfsetbuttcap%
\pgfsetroundjoin%
\definecolor{currentfill}{rgb}{0.866667,0.517647,0.321569}%
\pgfsetfillcolor{currentfill}%
\pgfsetfillopacity{0.900000}%
\pgfsetlinewidth{0.507862pt}%
\definecolor{currentstroke}{rgb}{1.000000,1.000000,1.000000}%
\pgfsetstrokecolor{currentstroke}%
\pgfsetstrokeopacity{0.900000}%
\pgfsetdash{}{0pt}%
\pgfsys@defobject{currentmarker}{\pgfqpoint{-0.043921in}{-0.043921in}}{\pgfqpoint{0.043921in}{0.043921in}}{%
\pgfpathmoveto{\pgfqpoint{0.000000in}{-0.043921in}}%
\pgfpathcurveto{\pgfqpoint{0.011648in}{-0.043921in}}{\pgfqpoint{0.022820in}{-0.039293in}}{\pgfqpoint{0.031056in}{-0.031056in}}%
\pgfpathcurveto{\pgfqpoint{0.039293in}{-0.022820in}}{\pgfqpoint{0.043921in}{-0.011648in}}{\pgfqpoint{0.043921in}{0.000000in}}%
\pgfpathcurveto{\pgfqpoint{0.043921in}{0.011648in}}{\pgfqpoint{0.039293in}{0.022820in}}{\pgfqpoint{0.031056in}{0.031056in}}%
\pgfpathcurveto{\pgfqpoint{0.022820in}{0.039293in}}{\pgfqpoint{0.011648in}{0.043921in}}{\pgfqpoint{0.000000in}{0.043921in}}%
\pgfpathcurveto{\pgfqpoint{-0.011648in}{0.043921in}}{\pgfqpoint{-0.022820in}{0.039293in}}{\pgfqpoint{-0.031056in}{0.031056in}}%
\pgfpathcurveto{\pgfqpoint{-0.039293in}{0.022820in}}{\pgfqpoint{-0.043921in}{0.011648in}}{\pgfqpoint{-0.043921in}{0.000000in}}%
\pgfpathcurveto{\pgfqpoint{-0.043921in}{-0.011648in}}{\pgfqpoint{-0.039293in}{-0.022820in}}{\pgfqpoint{-0.031056in}{-0.031056in}}%
\pgfpathcurveto{\pgfqpoint{-0.022820in}{-0.039293in}}{\pgfqpoint{-0.011648in}{-0.043921in}}{\pgfqpoint{0.000000in}{-0.043921in}}%
\pgfpathlineto{\pgfqpoint{0.000000in}{-0.043921in}}%
\pgfpathclose%
\pgfusepath{stroke,fill}%
}%
\begin{pgfscope}%
\pgfsys@transformshift{0.255556in}{0.369531in}%
\pgfsys@useobject{currentmarker}{}%
\end{pgfscope}%
\end{pgfscope}%
\begin{pgfscope}%
\definecolor{textcolor}{rgb}{0.150000,0.150000,0.150000}%
\pgfsetstrokecolor{textcolor}%
\pgfsetfillcolor{textcolor}%
\pgftext[x=0.455556in,y=0.330642in,left,base]{\color{textcolor}{\rmfamily\fontsize{8.000000}{9.600000}\selectfont\catcode`\^=\active\def^{\ifmmode\sp\else\^{}\fi}\catcode`\%=\active\def%{\%}Shared interest}}%
\end{pgfscope}%
\begin{pgfscope}%
\pgfsetbuttcap%
\pgfsetroundjoin%
\definecolor{currentfill}{rgb}{0.333333,0.658824,0.407843}%
\pgfsetfillcolor{currentfill}%
\pgfsetfillopacity{0.900000}%
\pgfsetlinewidth{0.507862pt}%
\definecolor{currentstroke}{rgb}{1.000000,1.000000,1.000000}%
\pgfsetstrokecolor{currentstroke}%
\pgfsetstrokeopacity{0.900000}%
\pgfsetdash{}{0pt}%
\pgfsys@defobject{currentmarker}{\pgfqpoint{-0.043921in}{-0.043921in}}{\pgfqpoint{0.043921in}{0.043921in}}{%
\pgfpathmoveto{\pgfqpoint{0.000000in}{-0.043921in}}%
\pgfpathcurveto{\pgfqpoint{0.011648in}{-0.043921in}}{\pgfqpoint{0.022820in}{-0.039293in}}{\pgfqpoint{0.031056in}{-0.031056in}}%
\pgfpathcurveto{\pgfqpoint{0.039293in}{-0.022820in}}{\pgfqpoint{0.043921in}{-0.011648in}}{\pgfqpoint{0.043921in}{0.000000in}}%
\pgfpathcurveto{\pgfqpoint{0.043921in}{0.011648in}}{\pgfqpoint{0.039293in}{0.022820in}}{\pgfqpoint{0.031056in}{0.031056in}}%
\pgfpathcurveto{\pgfqpoint{0.022820in}{0.039293in}}{\pgfqpoint{0.011648in}{0.043921in}}{\pgfqpoint{0.000000in}{0.043921in}}%
\pgfpathcurveto{\pgfqpoint{-0.011648in}{0.043921in}}{\pgfqpoint{-0.022820in}{0.039293in}}{\pgfqpoint{-0.031056in}{0.031056in}}%
\pgfpathcurveto{\pgfqpoint{-0.039293in}{0.022820in}}{\pgfqpoint{-0.043921in}{0.011648in}}{\pgfqpoint{-0.043921in}{0.000000in}}%
\pgfpathcurveto{\pgfqpoint{-0.043921in}{-0.011648in}}{\pgfqpoint{-0.039293in}{-0.022820in}}{\pgfqpoint{-0.031056in}{-0.031056in}}%
\pgfpathcurveto{\pgfqpoint{-0.022820in}{-0.039293in}}{\pgfqpoint{-0.011648in}{-0.043921in}}{\pgfqpoint{0.000000in}{-0.043921in}}%
\pgfpathlineto{\pgfqpoint{0.000000in}{-0.043921in}}%
\pgfpathclose%
\pgfusepath{stroke,fill}%
}%
\begin{pgfscope}%
\pgfsys@transformshift{0.255556in}{0.206445in}%
\pgfsys@useobject{currentmarker}{}%
\end{pgfscope}%
\end{pgfscope}%
\begin{pgfscope}%
\definecolor{textcolor}{rgb}{0.150000,0.150000,0.150000}%
\pgfsetstrokecolor{textcolor}%
\pgfsetfillcolor{textcolor}%
\pgftext[x=0.455556in,y=0.167556in,left,base]{\color{textcolor}{\rmfamily\fontsize{8.000000}{9.600000}\selectfont\catcode`\^=\active\def^{\ifmmode\sp\else\^{}\fi}\catcode`\%=\active\def%{\%}Entertainment}}%
\end{pgfscope}%
\begin{pgfscope}%
\pgfsetbuttcap%
\pgfsetroundjoin%
\definecolor{currentfill}{rgb}{0.200000,0.200000,0.200000}%
\pgfsetfillcolor{currentfill}%
\pgfsetfillopacity{0.900000}%
\pgfsetlinewidth{0.507862pt}%
\definecolor{currentstroke}{rgb}{1.000000,1.000000,1.000000}%
\pgfsetstrokecolor{currentstroke}%
\pgfsetstrokeopacity{0.900000}%
\pgfsetdash{}{0pt}%
\pgfsys@defobject{currentmarker}{\pgfqpoint{-0.043921in}{-0.043921in}}{\pgfqpoint{0.043921in}{0.043921in}}{%
\pgfpathmoveto{\pgfqpoint{0.000000in}{-0.043921in}}%
\pgfpathcurveto{\pgfqpoint{0.011648in}{-0.043921in}}{\pgfqpoint{0.022820in}{-0.039293in}}{\pgfqpoint{0.031056in}{-0.031056in}}%
\pgfpathcurveto{\pgfqpoint{0.039293in}{-0.022820in}}{\pgfqpoint{0.043921in}{-0.011648in}}{\pgfqpoint{0.043921in}{0.000000in}}%
\pgfpathcurveto{\pgfqpoint{0.043921in}{0.011648in}}{\pgfqpoint{0.039293in}{0.022820in}}{\pgfqpoint{0.031056in}{0.031056in}}%
\pgfpathcurveto{\pgfqpoint{0.022820in}{0.039293in}}{\pgfqpoint{0.011648in}{0.043921in}}{\pgfqpoint{0.000000in}{0.043921in}}%
\pgfpathcurveto{\pgfqpoint{-0.011648in}{0.043921in}}{\pgfqpoint{-0.022820in}{0.039293in}}{\pgfqpoint{-0.031056in}{0.031056in}}%
\pgfpathcurveto{\pgfqpoint{-0.039293in}{0.022820in}}{\pgfqpoint{-0.043921in}{0.011648in}}{\pgfqpoint{-0.043921in}{0.000000in}}%
\pgfpathcurveto{\pgfqpoint{-0.043921in}{-0.011648in}}{\pgfqpoint{-0.039293in}{-0.022820in}}{\pgfqpoint{-0.031056in}{-0.031056in}}%
\pgfpathcurveto{\pgfqpoint{-0.022820in}{-0.039293in}}{\pgfqpoint{-0.011648in}{-0.043921in}}{\pgfqpoint{0.000000in}{-0.043921in}}%
\pgfpathlineto{\pgfqpoint{0.000000in}{-0.043921in}}%
\pgfpathclose%
\pgfusepath{stroke,fill}%
}%
\begin{pgfscope}%
\pgfsys@transformshift{1.611884in}{0.532617in}%
\pgfsys@useobject{currentmarker}{}%
\end{pgfscope}%
\end{pgfscope}%
\begin{pgfscope}%
\definecolor{textcolor}{rgb}{0.150000,0.150000,0.150000}%
\pgfsetstrokecolor{textcolor}%
\pgfsetfillcolor{textcolor}%
\pgftext[x=1.811884in,y=0.493728in,left,base]{\color{textcolor}{\rmfamily\fontsize{8.000000}{9.600000}\selectfont\catcode`\^=\active\def^{\ifmmode\sp\else\^{}\fi}\catcode`\%=\active\def%{\%}-1}}%
\end{pgfscope}%
\begin{pgfscope}%
\pgfsetbuttcap%
\pgfsetmiterjoin%
\definecolor{currentfill}{rgb}{0.200000,0.200000,0.200000}%
\pgfsetfillcolor{currentfill}%
\pgfsetfillopacity{0.900000}%
\pgfsetlinewidth{0.507862pt}%
\definecolor{currentstroke}{rgb}{1.000000,1.000000,1.000000}%
\pgfsetstrokecolor{currentstroke}%
\pgfsetstrokeopacity{0.900000}%
\pgfsetdash{}{0pt}%
\pgfsys@defobject{currentmarker}{\pgfqpoint{-0.043921in}{-0.043921in}}{\pgfqpoint{0.043921in}{0.043921in}}{%
\pgfpathmoveto{\pgfqpoint{-0.021960in}{-0.043921in}}%
\pgfpathlineto{\pgfqpoint{0.000000in}{-0.021960in}}%
\pgfpathlineto{\pgfqpoint{0.021960in}{-0.043921in}}%
\pgfpathlineto{\pgfqpoint{0.043921in}{-0.021960in}}%
\pgfpathlineto{\pgfqpoint{0.021960in}{0.000000in}}%
\pgfpathlineto{\pgfqpoint{0.043921in}{0.021960in}}%
\pgfpathlineto{\pgfqpoint{0.021960in}{0.043921in}}%
\pgfpathlineto{\pgfqpoint{0.000000in}{0.021960in}}%
\pgfpathlineto{\pgfqpoint{-0.021960in}{0.043921in}}%
\pgfpathlineto{\pgfqpoint{-0.043921in}{0.021960in}}%
\pgfpathlineto{\pgfqpoint{-0.021960in}{0.000000in}}%
\pgfpathlineto{\pgfqpoint{-0.043921in}{-0.021960in}}%
\pgfpathlineto{\pgfqpoint{-0.021960in}{-0.043921in}}%
\pgfpathclose%
\pgfusepath{stroke,fill}%
}%
\begin{pgfscope}%
\pgfsys@transformshift{1.611884in}{0.369531in}%
\pgfsys@useobject{currentmarker}{}%
\end{pgfscope}%
\end{pgfscope}%
\begin{pgfscope}%
\definecolor{textcolor}{rgb}{0.150000,0.150000,0.150000}%
\pgfsetstrokecolor{textcolor}%
\pgfsetfillcolor{textcolor}%
\pgftext[x=1.811884in,y=0.330642in,left,base]{\color{textcolor}{\rmfamily\fontsize{8.000000}{9.600000}\selectfont\catcode`\^=\active\def^{\ifmmode\sp\else\^{}\fi}\catcode`\%=\active\def%{\%}0}}%
\end{pgfscope}%
\begin{pgfscope}%
\pgfsetbuttcap%
\pgfsetmiterjoin%
\definecolor{currentfill}{rgb}{0.200000,0.200000,0.200000}%
\pgfsetfillcolor{currentfill}%
\pgfsetfillopacity{0.900000}%
\pgfsetlinewidth{0.507862pt}%
\definecolor{currentstroke}{rgb}{1.000000,1.000000,1.000000}%
\pgfsetstrokecolor{currentstroke}%
\pgfsetstrokeopacity{0.900000}%
\pgfsetdash{}{0pt}%
\pgfsys@defobject{currentmarker}{\pgfqpoint{-0.031056in}{-0.031056in}}{\pgfqpoint{0.031056in}{0.031056in}}{%
\pgfpathmoveto{\pgfqpoint{-0.031056in}{0.031056in}}%
\pgfpathlineto{\pgfqpoint{-0.031056in}{-0.031056in}}%
\pgfpathlineto{\pgfqpoint{0.031056in}{-0.031056in}}%
\pgfpathlineto{\pgfqpoint{0.031056in}{0.031056in}}%
\pgfpathlineto{\pgfqpoint{-0.031056in}{0.031056in}}%
\pgfpathclose%
\pgfusepath{stroke,fill}%
}%
\begin{pgfscope}%
\pgfsys@transformshift{2.199169in}{0.532617in}%
\pgfsys@useobject{currentmarker}{}%
\end{pgfscope}%
\end{pgfscope}%
\begin{pgfscope}%
\definecolor{textcolor}{rgb}{0.150000,0.150000,0.150000}%
\pgfsetstrokecolor{textcolor}%
\pgfsetfillcolor{textcolor}%
\pgftext[x=2.399169in,y=0.493728in,left,base]{\color{textcolor}{\rmfamily\fontsize{8.000000}{9.600000}\selectfont\catcode`\^=\active\def^{\ifmmode\sp\else\^{}\fi}\catcode`\%=\active\def%{\%}1}}%
\end{pgfscope}%
\begin{pgfscope}%
\pgfsetbuttcap%
\pgfsetmiterjoin%
\definecolor{currentfill}{rgb}{0.200000,0.200000,0.200000}%
\pgfsetfillcolor{currentfill}%
\pgfsetfillopacity{0.900000}%
\pgfsetlinewidth{0.507862pt}%
\definecolor{currentstroke}{rgb}{1.000000,1.000000,1.000000}%
\pgfsetstrokecolor{currentstroke}%
\pgfsetstrokeopacity{0.900000}%
\pgfsetdash{}{0pt}%
\pgfsys@defobject{currentmarker}{\pgfqpoint{-0.043921in}{-0.043921in}}{\pgfqpoint{0.043921in}{0.043921in}}{%
\pgfpathmoveto{\pgfqpoint{-0.014640in}{-0.043921in}}%
\pgfpathlineto{\pgfqpoint{0.014640in}{-0.043921in}}%
\pgfpathlineto{\pgfqpoint{0.014640in}{-0.014640in}}%
\pgfpathlineto{\pgfqpoint{0.043921in}{-0.014640in}}%
\pgfpathlineto{\pgfqpoint{0.043921in}{0.014640in}}%
\pgfpathlineto{\pgfqpoint{0.014640in}{0.014640in}}%
\pgfpathlineto{\pgfqpoint{0.014640in}{0.043921in}}%
\pgfpathlineto{\pgfqpoint{-0.014640in}{0.043921in}}%
\pgfpathlineto{\pgfqpoint{-0.014640in}{0.014640in}}%
\pgfpathlineto{\pgfqpoint{-0.043921in}{0.014640in}}%
\pgfpathlineto{\pgfqpoint{-0.043921in}{-0.014640in}}%
\pgfpathlineto{\pgfqpoint{-0.014640in}{-0.014640in}}%
\pgfpathlineto{\pgfqpoint{-0.014640in}{-0.043921in}}%
\pgfpathclose%
\pgfusepath{stroke,fill}%
}%
\begin{pgfscope}%
\pgfsys@transformshift{2.199169in}{0.369531in}%
\pgfsys@useobject{currentmarker}{}%
\end{pgfscope}%
\end{pgfscope}%
\begin{pgfscope}%
\definecolor{textcolor}{rgb}{0.150000,0.150000,0.150000}%
\pgfsetstrokecolor{textcolor}%
\pgfsetfillcolor{textcolor}%
\pgftext[x=2.399169in,y=0.330642in,left,base]{\color{textcolor}{\rmfamily\fontsize{8.000000}{9.600000}\selectfont\catcode`\^=\active\def^{\ifmmode\sp\else\^{}\fi}\catcode`\%=\active\def%{\%}2}}%
\end{pgfscope}%
\end{pgfpicture}%
\makeatother%
\endgroup%

    \end{adjustwidth}
    \caption{Comparison of user graphs across three embedding spaces, labeled by assigned hashtag categories, clustered using HDBSCAN, and subsequently dimensionality reduced with UMAP. They reveal no separation between graphs corresponding to different categories. This observation is further reinforced by the minimal overlap between the assigned categories and the clusters identified by HDBSCAN.}
    \label{fig:emb_scatter}
\end{figure}

Inspecting the figure reveals no clear separation between graphs of different categories in any embedding space, indicating no relationship between the graph topologies and the categorization in the embedding spaces. Moreover, in the Graph-2Vec representation, three clusters are detected, whereas both Bag-Of-Subgraphs reveal only two clusters. Investigating the driving mechanism behind this partitioning, we find that the size of the graphs almost entirely explain the clustering. In the Graph2Vec space, the average graph size for cluster $0$ is $6.3$, $1$ is $55.9$, and $2$ is $230.7$, and the same trend is observed in the other two representations, showing a strong positive correlation between cluster assignment and graph size. These detected clusters show no overlap with the categorization. Calculating Normalized Mutual Information (NMI) between the categorization and the detected clusters supports this observation, with NMI values ranging from $0.001$ to $0.054$, further confirming the lack of overlap. Alternative embedding methods are presented in Appendix \ref{lab:alt_graph_embs} Figures \ref{fig:feathergraph_embeddings}, \ref{fig:directed_subgraph_embeddings}, and \ref{fig:directed_sentiment_subgraph_embeddings}.


\begin{figure}[!htbp]
    \centering
    \begin{adjustwidth}{-\textwidth}{-\textwidth}
        \centering
        %% Creator: Matplotlib, PGF backend
%%
%% To include the figure in your LaTeX document, write
%%   \input{<filename>.pgf}
%%
%% Make sure the required packages are loaded in your preamble
%%   \usepackage{pgf}
%%
%% Also ensure that all the required font packages are loaded; for instance,
%% the lmodern package is sometimes necessary when using math font.
%%   \usepackage{lmodern}
%%
%% Figures using additional raster images can only be included by \input if
%% they are in the same directory as the main LaTeX file. For loading figures
%% from other directories you can use the `import` package
%%   \usepackage{import}
%%
%% and then include the figures with
%%   \import{<path to file>}{<filename>.pgf}
%%
%% Matplotlib used the following preamble
%%   \def\mathdefault#1{#1}
%%   \everymath=\expandafter{\the\everymath\displaystyle}
%%   
%%   \ifdefined\pdftexversion\else  % non-pdftex case.
%%     \usepackage{fontspec}
%%     \setmainfont{DejaVuSerif.ttf}[Path=\detokenize{/home/mahf/.local/lib/python3.10/site-packages/matplotlib/mpl-data/fonts/ttf/}]
%%     \setsansfont{DejaVuSans.ttf}[Path=\detokenize{/home/mahf/.local/lib/python3.10/site-packages/matplotlib/mpl-data/fonts/ttf/}]
%%     \setmonofont{DejaVuSansMono.ttf}[Path=\detokenize{/home/mahf/.local/lib/python3.10/site-packages/matplotlib/mpl-data/fonts/ttf/}]
%%   \fi
%%   \makeatletter\@ifpackageloaded{underscore}{}{\usepackage[strings]{underscore}}\makeatother
%%
\begingroup%
\makeatletter%
\begin{pgfpicture}%
\pgfpathrectangle{\pgfpointorigin}{\pgfqpoint{6.922409in}{2.985276in}}%
\pgfusepath{use as bounding box, clip}%
\begin{pgfscope}%
\pgfsetbuttcap%
\pgfsetmiterjoin%
\definecolor{currentfill}{rgb}{1.000000,1.000000,1.000000}%
\pgfsetfillcolor{currentfill}%
\pgfsetlinewidth{0.000000pt}%
\definecolor{currentstroke}{rgb}{1.000000,1.000000,1.000000}%
\pgfsetstrokecolor{currentstroke}%
\pgfsetdash{}{0pt}%
\pgfpathmoveto{\pgfqpoint{0.000000in}{0.000000in}}%
\pgfpathlineto{\pgfqpoint{6.922409in}{0.000000in}}%
\pgfpathlineto{\pgfqpoint{6.922409in}{2.985276in}}%
\pgfpathlineto{\pgfqpoint{0.000000in}{2.985276in}}%
\pgfpathlineto{\pgfqpoint{0.000000in}{0.000000in}}%
\pgfpathclose%
\pgfusepath{fill}%
\end{pgfscope}%
\begin{pgfscope}%
\pgfsetbuttcap%
\pgfsetmiterjoin%
\definecolor{currentfill}{rgb}{1.000000,1.000000,1.000000}%
\pgfsetfillcolor{currentfill}%
\pgfsetlinewidth{0.000000pt}%
\definecolor{currentstroke}{rgb}{0.000000,0.000000,0.000000}%
\pgfsetstrokecolor{currentstroke}%
\pgfsetstrokeopacity{0.000000}%
\pgfsetdash{}{0pt}%
\pgfpathmoveto{\pgfqpoint{0.563297in}{0.529443in}}%
\pgfpathlineto{\pgfqpoint{2.199956in}{0.529443in}}%
\pgfpathlineto{\pgfqpoint{2.199956in}{2.275433in}}%
\pgfpathlineto{\pgfqpoint{0.563297in}{2.275433in}}%
\pgfpathlineto{\pgfqpoint{0.563297in}{0.529443in}}%
\pgfpathclose%
\pgfusepath{fill}%
\end{pgfscope}%
\begin{pgfscope}%
\pgfpathrectangle{\pgfqpoint{0.563297in}{0.529443in}}{\pgfqpoint{1.636659in}{1.745990in}}%
\pgfusepath{clip}%
\pgfsetroundcap%
\pgfsetroundjoin%
\pgfsetlinewidth{1.003750pt}%
\definecolor{currentstroke}{rgb}{0.800000,0.800000,0.800000}%
\pgfsetstrokecolor{currentstroke}%
\pgfsetdash{}{0pt}%
\pgfpathmoveto{\pgfqpoint{0.851247in}{0.529443in}}%
\pgfpathlineto{\pgfqpoint{0.851247in}{2.275433in}}%
\pgfusepath{stroke}%
\end{pgfscope}%
\begin{pgfscope}%
\definecolor{textcolor}{rgb}{0.150000,0.150000,0.150000}%
\pgfsetstrokecolor{textcolor}%
\pgfsetfillcolor{textcolor}%
\pgftext[x=0.851247in,y=0.397499in,,top]{\color{textcolor}{\rmfamily\fontsize{8.000000}{9.600000}\selectfont\catcode`\^=\active\def^{\ifmmode\sp\else\^{}\fi}\catcode`\%=\active\def%{\%}10}}%
\end{pgfscope}%
\begin{pgfscope}%
\pgfpathrectangle{\pgfqpoint{0.563297in}{0.529443in}}{\pgfqpoint{1.636659in}{1.745990in}}%
\pgfusepath{clip}%
\pgfsetroundcap%
\pgfsetroundjoin%
\pgfsetlinewidth{1.003750pt}%
\definecolor{currentstroke}{rgb}{0.800000,0.800000,0.800000}%
\pgfsetstrokecolor{currentstroke}%
\pgfsetdash{}{0pt}%
\pgfpathmoveto{\pgfqpoint{1.396500in}{0.529443in}}%
\pgfpathlineto{\pgfqpoint{1.396500in}{2.275433in}}%
\pgfusepath{stroke}%
\end{pgfscope}%
\begin{pgfscope}%
\definecolor{textcolor}{rgb}{0.150000,0.150000,0.150000}%
\pgfsetstrokecolor{textcolor}%
\pgfsetfillcolor{textcolor}%
\pgftext[x=1.396500in,y=0.397499in,,top]{\color{textcolor}{\rmfamily\fontsize{8.000000}{9.600000}\selectfont\catcode`\^=\active\def^{\ifmmode\sp\else\^{}\fi}\catcode`\%=\active\def%{\%}12}}%
\end{pgfscope}%
\begin{pgfscope}%
\pgfpathrectangle{\pgfqpoint{0.563297in}{0.529443in}}{\pgfqpoint{1.636659in}{1.745990in}}%
\pgfusepath{clip}%
\pgfsetroundcap%
\pgfsetroundjoin%
\pgfsetlinewidth{1.003750pt}%
\definecolor{currentstroke}{rgb}{0.800000,0.800000,0.800000}%
\pgfsetstrokecolor{currentstroke}%
\pgfsetdash{}{0pt}%
\pgfpathmoveto{\pgfqpoint{1.941752in}{0.529443in}}%
\pgfpathlineto{\pgfqpoint{1.941752in}{2.275433in}}%
\pgfusepath{stroke}%
\end{pgfscope}%
\begin{pgfscope}%
\definecolor{textcolor}{rgb}{0.150000,0.150000,0.150000}%
\pgfsetstrokecolor{textcolor}%
\pgfsetfillcolor{textcolor}%
\pgftext[x=1.941752in,y=0.397499in,,top]{\color{textcolor}{\rmfamily\fontsize{8.000000}{9.600000}\selectfont\catcode`\^=\active\def^{\ifmmode\sp\else\^{}\fi}\catcode`\%=\active\def%{\%}14}}%
\end{pgfscope}%
\begin{pgfscope}%
\definecolor{textcolor}{rgb}{0.150000,0.150000,0.150000}%
\pgfsetstrokecolor{textcolor}%
\pgfsetfillcolor{textcolor}%
\pgftext[x=1.381627in,y=0.234413in,,top]{\color{textcolor}{\rmfamily\fontsize{10.000000}{12.000000}\selectfont\catcode`\^=\active\def^{\ifmmode\sp\else\^{}\fi}\catcode`\%=\active\def%{\%}UMAP 1}}%
\end{pgfscope}%
\begin{pgfscope}%
\pgfpathrectangle{\pgfqpoint{0.563297in}{0.529443in}}{\pgfqpoint{1.636659in}{1.745990in}}%
\pgfusepath{clip}%
\pgfsetroundcap%
\pgfsetroundjoin%
\pgfsetlinewidth{1.003750pt}%
\definecolor{currentstroke}{rgb}{0.800000,0.800000,0.800000}%
\pgfsetstrokecolor{currentstroke}%
\pgfsetdash{}{0pt}%
\pgfpathmoveto{\pgfqpoint{0.563297in}{0.889650in}}%
\pgfpathlineto{\pgfqpoint{2.199956in}{0.889650in}}%
\pgfusepath{stroke}%
\end{pgfscope}%
\begin{pgfscope}%
\definecolor{textcolor}{rgb}{0.150000,0.150000,0.150000}%
\pgfsetstrokecolor{textcolor}%
\pgfsetfillcolor{textcolor}%
\pgftext[x=0.360661in, y=0.847441in, left, base]{\color{textcolor}{\rmfamily\fontsize{8.000000}{9.600000}\selectfont\catcode`\^=\active\def^{\ifmmode\sp\else\^{}\fi}\catcode`\%=\active\def%{\%}6}}%
\end{pgfscope}%
\begin{pgfscope}%
\pgfpathrectangle{\pgfqpoint{0.563297in}{0.529443in}}{\pgfqpoint{1.636659in}{1.745990in}}%
\pgfusepath{clip}%
\pgfsetroundcap%
\pgfsetroundjoin%
\pgfsetlinewidth{1.003750pt}%
\definecolor{currentstroke}{rgb}{0.800000,0.800000,0.800000}%
\pgfsetstrokecolor{currentstroke}%
\pgfsetdash{}{0pt}%
\pgfpathmoveto{\pgfqpoint{0.563297in}{1.347762in}}%
\pgfpathlineto{\pgfqpoint{2.199956in}{1.347762in}}%
\pgfusepath{stroke}%
\end{pgfscope}%
\begin{pgfscope}%
\definecolor{textcolor}{rgb}{0.150000,0.150000,0.150000}%
\pgfsetstrokecolor{textcolor}%
\pgfsetfillcolor{textcolor}%
\pgftext[x=0.360661in, y=1.305553in, left, base]{\color{textcolor}{\rmfamily\fontsize{8.000000}{9.600000}\selectfont\catcode`\^=\active\def^{\ifmmode\sp\else\^{}\fi}\catcode`\%=\active\def%{\%}8}}%
\end{pgfscope}%
\begin{pgfscope}%
\pgfpathrectangle{\pgfqpoint{0.563297in}{0.529443in}}{\pgfqpoint{1.636659in}{1.745990in}}%
\pgfusepath{clip}%
\pgfsetroundcap%
\pgfsetroundjoin%
\pgfsetlinewidth{1.003750pt}%
\definecolor{currentstroke}{rgb}{0.800000,0.800000,0.800000}%
\pgfsetstrokecolor{currentstroke}%
\pgfsetdash{}{0pt}%
\pgfpathmoveto{\pgfqpoint{0.563297in}{1.805874in}}%
\pgfpathlineto{\pgfqpoint{2.199956in}{1.805874in}}%
\pgfusepath{stroke}%
\end{pgfscope}%
\begin{pgfscope}%
\definecolor{textcolor}{rgb}{0.150000,0.150000,0.150000}%
\pgfsetstrokecolor{textcolor}%
\pgfsetfillcolor{textcolor}%
\pgftext[x=0.289968in, y=1.763665in, left, base]{\color{textcolor}{\rmfamily\fontsize{8.000000}{9.600000}\selectfont\catcode`\^=\active\def^{\ifmmode\sp\else\^{}\fi}\catcode`\%=\active\def%{\%}10}}%
\end{pgfscope}%
\begin{pgfscope}%
\pgfpathrectangle{\pgfqpoint{0.563297in}{0.529443in}}{\pgfqpoint{1.636659in}{1.745990in}}%
\pgfusepath{clip}%
\pgfsetroundcap%
\pgfsetroundjoin%
\pgfsetlinewidth{1.003750pt}%
\definecolor{currentstroke}{rgb}{0.800000,0.800000,0.800000}%
\pgfsetstrokecolor{currentstroke}%
\pgfsetdash{}{0pt}%
\pgfpathmoveto{\pgfqpoint{0.563297in}{2.263987in}}%
\pgfpathlineto{\pgfqpoint{2.199956in}{2.263987in}}%
\pgfusepath{stroke}%
\end{pgfscope}%
\begin{pgfscope}%
\definecolor{textcolor}{rgb}{0.150000,0.150000,0.150000}%
\pgfsetstrokecolor{textcolor}%
\pgfsetfillcolor{textcolor}%
\pgftext[x=0.289968in, y=2.221777in, left, base]{\color{textcolor}{\rmfamily\fontsize{8.000000}{9.600000}\selectfont\catcode`\^=\active\def^{\ifmmode\sp\else\^{}\fi}\catcode`\%=\active\def%{\%}12}}%
\end{pgfscope}%
\begin{pgfscope}%
\definecolor{textcolor}{rgb}{0.150000,0.150000,0.150000}%
\pgfsetstrokecolor{textcolor}%
\pgfsetfillcolor{textcolor}%
\pgftext[x=0.234413in,y=1.402438in,,bottom,rotate=90.000000]{\color{textcolor}{\rmfamily\fontsize{10.000000}{12.000000}\selectfont\catcode`\^=\active\def^{\ifmmode\sp\else\^{}\fi}\catcode`\%=\active\def%{\%}UMAP 2}}%
\end{pgfscope}%
\begin{pgfscope}%
\pgfpathrectangle{\pgfqpoint{0.563297in}{0.529443in}}{\pgfqpoint{1.636659in}{1.745990in}}%
\pgfusepath{clip}%
\pgfsetbuttcap%
\pgfsetroundjoin%
\definecolor{currentfill}{rgb}{0.003922,0.003922,0.003922}%
\pgfsetfillcolor{currentfill}%
\pgfsetfillopacity{0.900000}%
\pgfsetlinewidth{0.507862pt}%
\definecolor{currentstroke}{rgb}{1.000000,1.000000,1.000000}%
\pgfsetstrokecolor{currentstroke}%
\pgfsetstrokeopacity{0.900000}%
\pgfsetdash{}{0pt}%
\pgfpathmoveto{\pgfqpoint{0.790948in}{0.883313in}}%
\pgfpathcurveto{\pgfqpoint{0.802596in}{0.883313in}}{\pgfqpoint{0.813768in}{0.887941in}}{\pgfqpoint{0.822004in}{0.896177in}}%
\pgfpathcurveto{\pgfqpoint{0.830241in}{0.904414in}}{\pgfqpoint{0.834868in}{0.915586in}}{\pgfqpoint{0.834868in}{0.927234in}}%
\pgfpathcurveto{\pgfqpoint{0.834868in}{0.938882in}}{\pgfqpoint{0.830241in}{0.950054in}}{\pgfqpoint{0.822004in}{0.958290in}}%
\pgfpathcurveto{\pgfqpoint{0.813768in}{0.966527in}}{\pgfqpoint{0.802596in}{0.971154in}}{\pgfqpoint{0.790948in}{0.971154in}}%
\pgfpathcurveto{\pgfqpoint{0.779300in}{0.971154in}}{\pgfqpoint{0.768128in}{0.966527in}}{\pgfqpoint{0.759891in}{0.958290in}}%
\pgfpathcurveto{\pgfqpoint{0.751655in}{0.950054in}}{\pgfqpoint{0.747027in}{0.938882in}}{\pgfqpoint{0.747027in}{0.927234in}}%
\pgfpathcurveto{\pgfqpoint{0.747027in}{0.915586in}}{\pgfqpoint{0.751655in}{0.904414in}}{\pgfqpoint{0.759891in}{0.896177in}}%
\pgfpathcurveto{\pgfqpoint{0.768128in}{0.887941in}}{\pgfqpoint{0.779300in}{0.883313in}}{\pgfqpoint{0.790948in}{0.883313in}}%
\pgfpathlineto{\pgfqpoint{0.790948in}{0.883313in}}%
\pgfpathclose%
\pgfusepath{stroke,fill}%
\end{pgfscope}%
\begin{pgfscope}%
\pgfpathrectangle{\pgfqpoint{0.563297in}{0.529443in}}{\pgfqpoint{1.636659in}{1.745990in}}%
\pgfusepath{clip}%
\pgfsetbuttcap%
\pgfsetroundjoin%
\definecolor{currentfill}{rgb}{0.003922,0.003922,0.003922}%
\pgfsetfillcolor{currentfill}%
\pgfsetfillopacity{0.900000}%
\pgfsetlinewidth{0.507862pt}%
\definecolor{currentstroke}{rgb}{1.000000,1.000000,1.000000}%
\pgfsetstrokecolor{currentstroke}%
\pgfsetstrokeopacity{0.900000}%
\pgfsetdash{}{0pt}%
\pgfpathmoveto{\pgfqpoint{2.057572in}{1.773451in}}%
\pgfpathcurveto{\pgfqpoint{2.069220in}{1.773451in}}{\pgfqpoint{2.080392in}{1.778079in}}{\pgfqpoint{2.088628in}{1.786315in}}%
\pgfpathcurveto{\pgfqpoint{2.096865in}{1.794551in}}{\pgfqpoint{2.101492in}{1.805724in}}{\pgfqpoint{2.101492in}{1.817371in}}%
\pgfpathcurveto{\pgfqpoint{2.101492in}{1.829019in}}{\pgfqpoint{2.096865in}{1.840192in}}{\pgfqpoint{2.088628in}{1.848428in}}%
\pgfpathcurveto{\pgfqpoint{2.080392in}{1.856664in}}{\pgfqpoint{2.069220in}{1.861292in}}{\pgfqpoint{2.057572in}{1.861292in}}%
\pgfpathcurveto{\pgfqpoint{2.045924in}{1.861292in}}{\pgfqpoint{2.034752in}{1.856664in}}{\pgfqpoint{2.026515in}{1.848428in}}%
\pgfpathcurveto{\pgfqpoint{2.018279in}{1.840192in}}{\pgfqpoint{2.013651in}{1.829019in}}{\pgfqpoint{2.013651in}{1.817371in}}%
\pgfpathcurveto{\pgfqpoint{2.013651in}{1.805724in}}{\pgfqpoint{2.018279in}{1.794551in}}{\pgfqpoint{2.026515in}{1.786315in}}%
\pgfpathcurveto{\pgfqpoint{2.034752in}{1.778079in}}{\pgfqpoint{2.045924in}{1.773451in}}{\pgfqpoint{2.057572in}{1.773451in}}%
\pgfpathlineto{\pgfqpoint{2.057572in}{1.773451in}}%
\pgfpathclose%
\pgfusepath{stroke,fill}%
\end{pgfscope}%
\begin{pgfscope}%
\pgfpathrectangle{\pgfqpoint{0.563297in}{0.529443in}}{\pgfqpoint{1.636659in}{1.745990in}}%
\pgfusepath{clip}%
\pgfsetbuttcap%
\pgfsetroundjoin%
\definecolor{currentfill}{rgb}{0.003922,0.003922,0.003922}%
\pgfsetfillcolor{currentfill}%
\pgfsetfillopacity{0.900000}%
\pgfsetlinewidth{0.507862pt}%
\definecolor{currentstroke}{rgb}{1.000000,1.000000,1.000000}%
\pgfsetstrokecolor{currentstroke}%
\pgfsetstrokeopacity{0.900000}%
\pgfsetdash{}{0pt}%
\pgfpathmoveto{\pgfqpoint{0.777298in}{0.607874in}}%
\pgfpathcurveto{\pgfqpoint{0.788946in}{0.607874in}}{\pgfqpoint{0.800119in}{0.612502in}}{\pgfqpoint{0.808355in}{0.620738in}}%
\pgfpathcurveto{\pgfqpoint{0.816591in}{0.628975in}}{\pgfqpoint{0.821219in}{0.640147in}}{\pgfqpoint{0.821219in}{0.651795in}}%
\pgfpathcurveto{\pgfqpoint{0.821219in}{0.663443in}}{\pgfqpoint{0.816591in}{0.674615in}}{\pgfqpoint{0.808355in}{0.682851in}}%
\pgfpathcurveto{\pgfqpoint{0.800119in}{0.691088in}}{\pgfqpoint{0.788946in}{0.695716in}}{\pgfqpoint{0.777298in}{0.695716in}}%
\pgfpathcurveto{\pgfqpoint{0.765651in}{0.695716in}}{\pgfqpoint{0.754478in}{0.691088in}}{\pgfqpoint{0.746242in}{0.682851in}}%
\pgfpathcurveto{\pgfqpoint{0.738006in}{0.674615in}}{\pgfqpoint{0.733378in}{0.663443in}}{\pgfqpoint{0.733378in}{0.651795in}}%
\pgfpathcurveto{\pgfqpoint{0.733378in}{0.640147in}}{\pgfqpoint{0.738006in}{0.628975in}}{\pgfqpoint{0.746242in}{0.620738in}}%
\pgfpathcurveto{\pgfqpoint{0.754478in}{0.612502in}}{\pgfqpoint{0.765651in}{0.607874in}}{\pgfqpoint{0.777298in}{0.607874in}}%
\pgfpathlineto{\pgfqpoint{0.777298in}{0.607874in}}%
\pgfpathclose%
\pgfusepath{stroke,fill}%
\end{pgfscope}%
\begin{pgfscope}%
\pgfpathrectangle{\pgfqpoint{0.563297in}{0.529443in}}{\pgfqpoint{1.636659in}{1.745990in}}%
\pgfusepath{clip}%
\pgfsetbuttcap%
\pgfsetroundjoin%
\definecolor{currentfill}{rgb}{0.003922,0.003922,0.003922}%
\pgfsetfillcolor{currentfill}%
\pgfsetfillopacity{0.900000}%
\pgfsetlinewidth{0.507862pt}%
\definecolor{currentstroke}{rgb}{1.000000,1.000000,1.000000}%
\pgfsetstrokecolor{currentstroke}%
\pgfsetstrokeopacity{0.900000}%
\pgfsetdash{}{0pt}%
\pgfpathmoveto{\pgfqpoint{1.221211in}{1.436844in}}%
\pgfpathcurveto{\pgfqpoint{1.232859in}{1.436844in}}{\pgfqpoint{1.244031in}{1.441472in}}{\pgfqpoint{1.252267in}{1.449708in}}%
\pgfpathcurveto{\pgfqpoint{1.260504in}{1.457945in}}{\pgfqpoint{1.265131in}{1.469117in}}{\pgfqpoint{1.265131in}{1.480765in}}%
\pgfpathcurveto{\pgfqpoint{1.265131in}{1.492413in}}{\pgfqpoint{1.260504in}{1.503585in}}{\pgfqpoint{1.252267in}{1.511821in}}%
\pgfpathcurveto{\pgfqpoint{1.244031in}{1.520058in}}{\pgfqpoint{1.232859in}{1.524685in}}{\pgfqpoint{1.221211in}{1.524685in}}%
\pgfpathcurveto{\pgfqpoint{1.209563in}{1.524685in}}{\pgfqpoint{1.198391in}{1.520058in}}{\pgfqpoint{1.190154in}{1.511821in}}%
\pgfpathcurveto{\pgfqpoint{1.181918in}{1.503585in}}{\pgfqpoint{1.177290in}{1.492413in}}{\pgfqpoint{1.177290in}{1.480765in}}%
\pgfpathcurveto{\pgfqpoint{1.177290in}{1.469117in}}{\pgfqpoint{1.181918in}{1.457945in}}{\pgfqpoint{1.190154in}{1.449708in}}%
\pgfpathcurveto{\pgfqpoint{1.198391in}{1.441472in}}{\pgfqpoint{1.209563in}{1.436844in}}{\pgfqpoint{1.221211in}{1.436844in}}%
\pgfpathlineto{\pgfqpoint{1.221211in}{1.436844in}}%
\pgfpathclose%
\pgfusepath{stroke,fill}%
\end{pgfscope}%
\begin{pgfscope}%
\pgfpathrectangle{\pgfqpoint{0.563297in}{0.529443in}}{\pgfqpoint{1.636659in}{1.745990in}}%
\pgfusepath{clip}%
\pgfsetbuttcap%
\pgfsetroundjoin%
\definecolor{currentfill}{rgb}{0.003922,0.003922,0.003922}%
\pgfsetfillcolor{currentfill}%
\pgfsetfillopacity{0.900000}%
\pgfsetlinewidth{0.507862pt}%
\definecolor{currentstroke}{rgb}{1.000000,1.000000,1.000000}%
\pgfsetstrokecolor{currentstroke}%
\pgfsetstrokeopacity{0.900000}%
\pgfsetdash{}{0pt}%
\pgfpathmoveto{\pgfqpoint{1.337985in}{1.510326in}}%
\pgfpathcurveto{\pgfqpoint{1.349633in}{1.510326in}}{\pgfqpoint{1.360805in}{1.514954in}}{\pgfqpoint{1.369041in}{1.523190in}}%
\pgfpathcurveto{\pgfqpoint{1.377278in}{1.531427in}}{\pgfqpoint{1.381905in}{1.542599in}}{\pgfqpoint{1.381905in}{1.554247in}}%
\pgfpathcurveto{\pgfqpoint{1.381905in}{1.565895in}}{\pgfqpoint{1.377278in}{1.577067in}}{\pgfqpoint{1.369041in}{1.585303in}}%
\pgfpathcurveto{\pgfqpoint{1.360805in}{1.593540in}}{\pgfqpoint{1.349633in}{1.598167in}}{\pgfqpoint{1.337985in}{1.598167in}}%
\pgfpathcurveto{\pgfqpoint{1.326337in}{1.598167in}}{\pgfqpoint{1.315165in}{1.593540in}}{\pgfqpoint{1.306928in}{1.585303in}}%
\pgfpathcurveto{\pgfqpoint{1.298692in}{1.577067in}}{\pgfqpoint{1.294064in}{1.565895in}}{\pgfqpoint{1.294064in}{1.554247in}}%
\pgfpathcurveto{\pgfqpoint{1.294064in}{1.542599in}}{\pgfqpoint{1.298692in}{1.531427in}}{\pgfqpoint{1.306928in}{1.523190in}}%
\pgfpathcurveto{\pgfqpoint{1.315165in}{1.514954in}}{\pgfqpoint{1.326337in}{1.510326in}}{\pgfqpoint{1.337985in}{1.510326in}}%
\pgfpathlineto{\pgfqpoint{1.337985in}{1.510326in}}%
\pgfpathclose%
\pgfusepath{stroke,fill}%
\end{pgfscope}%
\begin{pgfscope}%
\pgfpathrectangle{\pgfqpoint{0.563297in}{0.529443in}}{\pgfqpoint{1.636659in}{1.745990in}}%
\pgfusepath{clip}%
\pgfsetbuttcap%
\pgfsetroundjoin%
\definecolor{currentfill}{rgb}{0.003922,0.003922,0.003922}%
\pgfsetfillcolor{currentfill}%
\pgfsetfillopacity{0.900000}%
\pgfsetlinewidth{0.507862pt}%
\definecolor{currentstroke}{rgb}{1.000000,1.000000,1.000000}%
\pgfsetstrokecolor{currentstroke}%
\pgfsetstrokeopacity{0.900000}%
\pgfsetdash{}{0pt}%
\pgfpathmoveto{\pgfqpoint{1.943724in}{1.868565in}}%
\pgfpathcurveto{\pgfqpoint{1.955372in}{1.868565in}}{\pgfqpoint{1.966544in}{1.873193in}}{\pgfqpoint{1.974780in}{1.881430in}}%
\pgfpathcurveto{\pgfqpoint{1.983017in}{1.889666in}}{\pgfqpoint{1.987644in}{1.900838in}}{\pgfqpoint{1.987644in}{1.912486in}}%
\pgfpathcurveto{\pgfqpoint{1.987644in}{1.924134in}}{\pgfqpoint{1.983017in}{1.935306in}}{\pgfqpoint{1.974780in}{1.943543in}}%
\pgfpathcurveto{\pgfqpoint{1.966544in}{1.951779in}}{\pgfqpoint{1.955372in}{1.956407in}}{\pgfqpoint{1.943724in}{1.956407in}}%
\pgfpathcurveto{\pgfqpoint{1.932076in}{1.956407in}}{\pgfqpoint{1.920904in}{1.951779in}}{\pgfqpoint{1.912667in}{1.943543in}}%
\pgfpathcurveto{\pgfqpoint{1.904431in}{1.935306in}}{\pgfqpoint{1.899803in}{1.924134in}}{\pgfqpoint{1.899803in}{1.912486in}}%
\pgfpathcurveto{\pgfqpoint{1.899803in}{1.900838in}}{\pgfqpoint{1.904431in}{1.889666in}}{\pgfqpoint{1.912667in}{1.881430in}}%
\pgfpathcurveto{\pgfqpoint{1.920904in}{1.873193in}}{\pgfqpoint{1.932076in}{1.868565in}}{\pgfqpoint{1.943724in}{1.868565in}}%
\pgfpathlineto{\pgfqpoint{1.943724in}{1.868565in}}%
\pgfpathclose%
\pgfusepath{stroke,fill}%
\end{pgfscope}%
\begin{pgfscope}%
\pgfpathrectangle{\pgfqpoint{0.563297in}{0.529443in}}{\pgfqpoint{1.636659in}{1.745990in}}%
\pgfusepath{clip}%
\pgfsetbuttcap%
\pgfsetroundjoin%
\definecolor{currentfill}{rgb}{0.003922,0.003922,0.003922}%
\pgfsetfillcolor{currentfill}%
\pgfsetfillopacity{0.900000}%
\pgfsetlinewidth{0.507862pt}%
\definecolor{currentstroke}{rgb}{1.000000,1.000000,1.000000}%
\pgfsetstrokecolor{currentstroke}%
\pgfsetstrokeopacity{0.900000}%
\pgfsetdash{}{0pt}%
\pgfpathmoveto{\pgfqpoint{2.023642in}{1.815527in}}%
\pgfpathcurveto{\pgfqpoint{2.035289in}{1.815527in}}{\pgfqpoint{2.046462in}{1.820155in}}{\pgfqpoint{2.054698in}{1.828392in}}%
\pgfpathcurveto{\pgfqpoint{2.062934in}{1.836628in}}{\pgfqpoint{2.067562in}{1.847800in}}{\pgfqpoint{2.067562in}{1.859448in}}%
\pgfpathcurveto{\pgfqpoint{2.067562in}{1.871096in}}{\pgfqpoint{2.062934in}{1.882268in}}{\pgfqpoint{2.054698in}{1.890505in}}%
\pgfpathcurveto{\pgfqpoint{2.046462in}{1.898741in}}{\pgfqpoint{2.035289in}{1.903369in}}{\pgfqpoint{2.023642in}{1.903369in}}%
\pgfpathcurveto{\pgfqpoint{2.011994in}{1.903369in}}{\pgfqpoint{2.000821in}{1.898741in}}{\pgfqpoint{1.992585in}{1.890505in}}%
\pgfpathcurveto{\pgfqpoint{1.984349in}{1.882268in}}{\pgfqpoint{1.979721in}{1.871096in}}{\pgfqpoint{1.979721in}{1.859448in}}%
\pgfpathcurveto{\pgfqpoint{1.979721in}{1.847800in}}{\pgfqpoint{1.984349in}{1.836628in}}{\pgfqpoint{1.992585in}{1.828392in}}%
\pgfpathcurveto{\pgfqpoint{2.000821in}{1.820155in}}{\pgfqpoint{2.011994in}{1.815527in}}{\pgfqpoint{2.023642in}{1.815527in}}%
\pgfpathlineto{\pgfqpoint{2.023642in}{1.815527in}}%
\pgfpathclose%
\pgfusepath{stroke,fill}%
\end{pgfscope}%
\begin{pgfscope}%
\pgfpathrectangle{\pgfqpoint{0.563297in}{0.529443in}}{\pgfqpoint{1.636659in}{1.745990in}}%
\pgfusepath{clip}%
\pgfsetbuttcap%
\pgfsetroundjoin%
\definecolor{currentfill}{rgb}{0.003922,0.003922,0.003922}%
\pgfsetfillcolor{currentfill}%
\pgfsetfillopacity{0.900000}%
\pgfsetlinewidth{0.507862pt}%
\definecolor{currentstroke}{rgb}{1.000000,1.000000,1.000000}%
\pgfsetstrokecolor{currentstroke}%
\pgfsetstrokeopacity{0.900000}%
\pgfsetdash{}{0pt}%
\pgfpathmoveto{\pgfqpoint{1.259991in}{1.514927in}}%
\pgfpathcurveto{\pgfqpoint{1.271639in}{1.514927in}}{\pgfqpoint{1.282812in}{1.519555in}}{\pgfqpoint{1.291048in}{1.527791in}}%
\pgfpathcurveto{\pgfqpoint{1.299284in}{1.536027in}}{\pgfqpoint{1.303912in}{1.547200in}}{\pgfqpoint{1.303912in}{1.558848in}}%
\pgfpathcurveto{\pgfqpoint{1.303912in}{1.570496in}}{\pgfqpoint{1.299284in}{1.581668in}}{\pgfqpoint{1.291048in}{1.589904in}}%
\pgfpathcurveto{\pgfqpoint{1.282812in}{1.598140in}}{\pgfqpoint{1.271639in}{1.602768in}}{\pgfqpoint{1.259991in}{1.602768in}}%
\pgfpathcurveto{\pgfqpoint{1.248344in}{1.602768in}}{\pgfqpoint{1.237171in}{1.598140in}}{\pgfqpoint{1.228935in}{1.589904in}}%
\pgfpathcurveto{\pgfqpoint{1.220699in}{1.581668in}}{\pgfqpoint{1.216071in}{1.570496in}}{\pgfqpoint{1.216071in}{1.558848in}}%
\pgfpathcurveto{\pgfqpoint{1.216071in}{1.547200in}}{\pgfqpoint{1.220699in}{1.536027in}}{\pgfqpoint{1.228935in}{1.527791in}}%
\pgfpathcurveto{\pgfqpoint{1.237171in}{1.519555in}}{\pgfqpoint{1.248344in}{1.514927in}}{\pgfqpoint{1.259991in}{1.514927in}}%
\pgfpathlineto{\pgfqpoint{1.259991in}{1.514927in}}%
\pgfpathclose%
\pgfusepath{stroke,fill}%
\end{pgfscope}%
\begin{pgfscope}%
\pgfpathrectangle{\pgfqpoint{0.563297in}{0.529443in}}{\pgfqpoint{1.636659in}{1.745990in}}%
\pgfusepath{clip}%
\pgfsetbuttcap%
\pgfsetroundjoin%
\definecolor{currentfill}{rgb}{0.003922,0.003922,0.003922}%
\pgfsetfillcolor{currentfill}%
\pgfsetfillopacity{0.900000}%
\pgfsetlinewidth{0.507862pt}%
\definecolor{currentstroke}{rgb}{1.000000,1.000000,1.000000}%
\pgfsetstrokecolor{currentstroke}%
\pgfsetstrokeopacity{0.900000}%
\pgfsetdash{}{0pt}%
\pgfpathmoveto{\pgfqpoint{0.747465in}{0.800947in}}%
\pgfpathcurveto{\pgfqpoint{0.759113in}{0.800947in}}{\pgfqpoint{0.770285in}{0.805575in}}{\pgfqpoint{0.778521in}{0.813811in}}%
\pgfpathcurveto{\pgfqpoint{0.786758in}{0.822048in}}{\pgfqpoint{0.791385in}{0.833220in}}{\pgfqpoint{0.791385in}{0.844868in}}%
\pgfpathcurveto{\pgfqpoint{0.791385in}{0.856516in}}{\pgfqpoint{0.786758in}{0.867688in}}{\pgfqpoint{0.778521in}{0.875924in}}%
\pgfpathcurveto{\pgfqpoint{0.770285in}{0.884161in}}{\pgfqpoint{0.759113in}{0.888788in}}{\pgfqpoint{0.747465in}{0.888788in}}%
\pgfpathcurveto{\pgfqpoint{0.735817in}{0.888788in}}{\pgfqpoint{0.724645in}{0.884161in}}{\pgfqpoint{0.716408in}{0.875924in}}%
\pgfpathcurveto{\pgfqpoint{0.708172in}{0.867688in}}{\pgfqpoint{0.703544in}{0.856516in}}{\pgfqpoint{0.703544in}{0.844868in}}%
\pgfpathcurveto{\pgfqpoint{0.703544in}{0.833220in}}{\pgfqpoint{0.708172in}{0.822048in}}{\pgfqpoint{0.716408in}{0.813811in}}%
\pgfpathcurveto{\pgfqpoint{0.724645in}{0.805575in}}{\pgfqpoint{0.735817in}{0.800947in}}{\pgfqpoint{0.747465in}{0.800947in}}%
\pgfpathlineto{\pgfqpoint{0.747465in}{0.800947in}}%
\pgfpathclose%
\pgfusepath{stroke,fill}%
\end{pgfscope}%
\begin{pgfscope}%
\pgfpathrectangle{\pgfqpoint{0.563297in}{0.529443in}}{\pgfqpoint{1.636659in}{1.745990in}}%
\pgfusepath{clip}%
\pgfsetbuttcap%
\pgfsetroundjoin%
\definecolor{currentfill}{rgb}{0.003922,0.003922,0.003922}%
\pgfsetfillcolor{currentfill}%
\pgfsetfillopacity{0.900000}%
\pgfsetlinewidth{0.507862pt}%
\definecolor{currentstroke}{rgb}{1.000000,1.000000,1.000000}%
\pgfsetstrokecolor{currentstroke}%
\pgfsetstrokeopacity{0.900000}%
\pgfsetdash{}{0pt}%
\pgfpathmoveto{\pgfqpoint{1.627944in}{2.098212in}}%
\pgfpathcurveto{\pgfqpoint{1.639592in}{2.098212in}}{\pgfqpoint{1.650764in}{2.102840in}}{\pgfqpoint{1.659000in}{2.111076in}}%
\pgfpathcurveto{\pgfqpoint{1.667237in}{2.119313in}}{\pgfqpoint{1.671864in}{2.130485in}}{\pgfqpoint{1.671864in}{2.142133in}}%
\pgfpathcurveto{\pgfqpoint{1.671864in}{2.153781in}}{\pgfqpoint{1.667237in}{2.164953in}}{\pgfqpoint{1.659000in}{2.173189in}}%
\pgfpathcurveto{\pgfqpoint{1.650764in}{2.181426in}}{\pgfqpoint{1.639592in}{2.186053in}}{\pgfqpoint{1.627944in}{2.186053in}}%
\pgfpathcurveto{\pgfqpoint{1.616296in}{2.186053in}}{\pgfqpoint{1.605124in}{2.181426in}}{\pgfqpoint{1.596887in}{2.173189in}}%
\pgfpathcurveto{\pgfqpoint{1.588651in}{2.164953in}}{\pgfqpoint{1.584023in}{2.153781in}}{\pgfqpoint{1.584023in}{2.142133in}}%
\pgfpathcurveto{\pgfqpoint{1.584023in}{2.130485in}}{\pgfqpoint{1.588651in}{2.119313in}}{\pgfqpoint{1.596887in}{2.111076in}}%
\pgfpathcurveto{\pgfqpoint{1.605124in}{2.102840in}}{\pgfqpoint{1.616296in}{2.098212in}}{\pgfqpoint{1.627944in}{2.098212in}}%
\pgfpathlineto{\pgfqpoint{1.627944in}{2.098212in}}%
\pgfpathclose%
\pgfusepath{stroke,fill}%
\end{pgfscope}%
\begin{pgfscope}%
\pgfpathrectangle{\pgfqpoint{0.563297in}{0.529443in}}{\pgfqpoint{1.636659in}{1.745990in}}%
\pgfusepath{clip}%
\pgfsetbuttcap%
\pgfsetroundjoin%
\definecolor{currentfill}{rgb}{0.003922,0.003922,0.003922}%
\pgfsetfillcolor{currentfill}%
\pgfsetfillopacity{0.900000}%
\pgfsetlinewidth{0.507862pt}%
\definecolor{currentstroke}{rgb}{1.000000,1.000000,1.000000}%
\pgfsetstrokecolor{currentstroke}%
\pgfsetstrokeopacity{0.900000}%
\pgfsetdash{}{0pt}%
\pgfpathmoveto{\pgfqpoint{0.809384in}{0.682455in}}%
\pgfpathcurveto{\pgfqpoint{0.821032in}{0.682455in}}{\pgfqpoint{0.832204in}{0.687082in}}{\pgfqpoint{0.840441in}{0.695319in}}%
\pgfpathcurveto{\pgfqpoint{0.848677in}{0.703555in}}{\pgfqpoint{0.853305in}{0.714727in}}{\pgfqpoint{0.853305in}{0.726375in}}%
\pgfpathcurveto{\pgfqpoint{0.853305in}{0.738023in}}{\pgfqpoint{0.848677in}{0.749195in}}{\pgfqpoint{0.840441in}{0.757432in}}%
\pgfpathcurveto{\pgfqpoint{0.832204in}{0.765668in}}{\pgfqpoint{0.821032in}{0.770296in}}{\pgfqpoint{0.809384in}{0.770296in}}%
\pgfpathcurveto{\pgfqpoint{0.797736in}{0.770296in}}{\pgfqpoint{0.786564in}{0.765668in}}{\pgfqpoint{0.778328in}{0.757432in}}%
\pgfpathcurveto{\pgfqpoint{0.770091in}{0.749195in}}{\pgfqpoint{0.765464in}{0.738023in}}{\pgfqpoint{0.765464in}{0.726375in}}%
\pgfpathcurveto{\pgfqpoint{0.765464in}{0.714727in}}{\pgfqpoint{0.770091in}{0.703555in}}{\pgfqpoint{0.778328in}{0.695319in}}%
\pgfpathcurveto{\pgfqpoint{0.786564in}{0.687082in}}{\pgfqpoint{0.797736in}{0.682455in}}{\pgfqpoint{0.809384in}{0.682455in}}%
\pgfpathlineto{\pgfqpoint{0.809384in}{0.682455in}}%
\pgfpathclose%
\pgfusepath{stroke,fill}%
\end{pgfscope}%
\begin{pgfscope}%
\pgfpathrectangle{\pgfqpoint{0.563297in}{0.529443in}}{\pgfqpoint{1.636659in}{1.745990in}}%
\pgfusepath{clip}%
\pgfsetbuttcap%
\pgfsetroundjoin%
\definecolor{currentfill}{rgb}{0.003922,0.003922,0.003922}%
\pgfsetfillcolor{currentfill}%
\pgfsetfillopacity{0.900000}%
\pgfsetlinewidth{0.507862pt}%
\definecolor{currentstroke}{rgb}{1.000000,1.000000,1.000000}%
\pgfsetstrokecolor{currentstroke}%
\pgfsetstrokeopacity{0.900000}%
\pgfsetdash{}{0pt}%
\pgfpathmoveto{\pgfqpoint{0.917144in}{1.101224in}}%
\pgfpathcurveto{\pgfqpoint{0.928792in}{1.101224in}}{\pgfqpoint{0.939965in}{1.105852in}}{\pgfqpoint{0.948201in}{1.114088in}}%
\pgfpathcurveto{\pgfqpoint{0.956437in}{1.122325in}}{\pgfqpoint{0.961065in}{1.133497in}}{\pgfqpoint{0.961065in}{1.145145in}}%
\pgfpathcurveto{\pgfqpoint{0.961065in}{1.156793in}}{\pgfqpoint{0.956437in}{1.167965in}}{\pgfqpoint{0.948201in}{1.176201in}}%
\pgfpathcurveto{\pgfqpoint{0.939965in}{1.184438in}}{\pgfqpoint{0.928792in}{1.189066in}}{\pgfqpoint{0.917144in}{1.189066in}}%
\pgfpathcurveto{\pgfqpoint{0.905496in}{1.189066in}}{\pgfqpoint{0.894324in}{1.184438in}}{\pgfqpoint{0.886088in}{1.176201in}}%
\pgfpathcurveto{\pgfqpoint{0.877852in}{1.167965in}}{\pgfqpoint{0.873224in}{1.156793in}}{\pgfqpoint{0.873224in}{1.145145in}}%
\pgfpathcurveto{\pgfqpoint{0.873224in}{1.133497in}}{\pgfqpoint{0.877852in}{1.122325in}}{\pgfqpoint{0.886088in}{1.114088in}}%
\pgfpathcurveto{\pgfqpoint{0.894324in}{1.105852in}}{\pgfqpoint{0.905496in}{1.101224in}}{\pgfqpoint{0.917144in}{1.101224in}}%
\pgfpathlineto{\pgfqpoint{0.917144in}{1.101224in}}%
\pgfpathclose%
\pgfusepath{stroke,fill}%
\end{pgfscope}%
\begin{pgfscope}%
\pgfpathrectangle{\pgfqpoint{0.563297in}{0.529443in}}{\pgfqpoint{1.636659in}{1.745990in}}%
\pgfusepath{clip}%
\pgfsetbuttcap%
\pgfsetroundjoin%
\definecolor{currentfill}{rgb}{0.003922,0.003922,0.003922}%
\pgfsetfillcolor{currentfill}%
\pgfsetfillopacity{0.900000}%
\pgfsetlinewidth{0.507862pt}%
\definecolor{currentstroke}{rgb}{1.000000,1.000000,1.000000}%
\pgfsetstrokecolor{currentstroke}%
\pgfsetstrokeopacity{0.900000}%
\pgfsetdash{}{0pt}%
\pgfpathmoveto{\pgfqpoint{0.890790in}{1.189562in}}%
\pgfpathcurveto{\pgfqpoint{0.902438in}{1.189562in}}{\pgfqpoint{0.913610in}{1.194190in}}{\pgfqpoint{0.921846in}{1.202426in}}%
\pgfpathcurveto{\pgfqpoint{0.930083in}{1.210662in}}{\pgfqpoint{0.934710in}{1.221835in}}{\pgfqpoint{0.934710in}{1.233482in}}%
\pgfpathcurveto{\pgfqpoint{0.934710in}{1.245130in}}{\pgfqpoint{0.930083in}{1.256303in}}{\pgfqpoint{0.921846in}{1.264539in}}%
\pgfpathcurveto{\pgfqpoint{0.913610in}{1.272775in}}{\pgfqpoint{0.902438in}{1.277403in}}{\pgfqpoint{0.890790in}{1.277403in}}%
\pgfpathcurveto{\pgfqpoint{0.879142in}{1.277403in}}{\pgfqpoint{0.867970in}{1.272775in}}{\pgfqpoint{0.859733in}{1.264539in}}%
\pgfpathcurveto{\pgfqpoint{0.851497in}{1.256303in}}{\pgfqpoint{0.846869in}{1.245130in}}{\pgfqpoint{0.846869in}{1.233482in}}%
\pgfpathcurveto{\pgfqpoint{0.846869in}{1.221835in}}{\pgfqpoint{0.851497in}{1.210662in}}{\pgfqpoint{0.859733in}{1.202426in}}%
\pgfpathcurveto{\pgfqpoint{0.867970in}{1.194190in}}{\pgfqpoint{0.879142in}{1.189562in}}{\pgfqpoint{0.890790in}{1.189562in}}%
\pgfpathlineto{\pgfqpoint{0.890790in}{1.189562in}}%
\pgfpathclose%
\pgfusepath{stroke,fill}%
\end{pgfscope}%
\begin{pgfscope}%
\pgfpathrectangle{\pgfqpoint{0.563297in}{0.529443in}}{\pgfqpoint{1.636659in}{1.745990in}}%
\pgfusepath{clip}%
\pgfsetbuttcap%
\pgfsetroundjoin%
\definecolor{currentfill}{rgb}{0.003922,0.003922,0.003922}%
\pgfsetfillcolor{currentfill}%
\pgfsetfillopacity{0.900000}%
\pgfsetlinewidth{0.507862pt}%
\definecolor{currentstroke}{rgb}{1.000000,1.000000,1.000000}%
\pgfsetstrokecolor{currentstroke}%
\pgfsetstrokeopacity{0.900000}%
\pgfsetdash{}{0pt}%
\pgfpathmoveto{\pgfqpoint{1.818473in}{1.742560in}}%
\pgfpathcurveto{\pgfqpoint{1.830121in}{1.742560in}}{\pgfqpoint{1.841293in}{1.747188in}}{\pgfqpoint{1.849529in}{1.755424in}}%
\pgfpathcurveto{\pgfqpoint{1.857766in}{1.763661in}}{\pgfqpoint{1.862393in}{1.774833in}}{\pgfqpoint{1.862393in}{1.786481in}}%
\pgfpathcurveto{\pgfqpoint{1.862393in}{1.798129in}}{\pgfqpoint{1.857766in}{1.809301in}}{\pgfqpoint{1.849529in}{1.817537in}}%
\pgfpathcurveto{\pgfqpoint{1.841293in}{1.825774in}}{\pgfqpoint{1.830121in}{1.830401in}}{\pgfqpoint{1.818473in}{1.830401in}}%
\pgfpathcurveto{\pgfqpoint{1.806825in}{1.830401in}}{\pgfqpoint{1.795653in}{1.825774in}}{\pgfqpoint{1.787416in}{1.817537in}}%
\pgfpathcurveto{\pgfqpoint{1.779180in}{1.809301in}}{\pgfqpoint{1.774552in}{1.798129in}}{\pgfqpoint{1.774552in}{1.786481in}}%
\pgfpathcurveto{\pgfqpoint{1.774552in}{1.774833in}}{\pgfqpoint{1.779180in}{1.763661in}}{\pgfqpoint{1.787416in}{1.755424in}}%
\pgfpathcurveto{\pgfqpoint{1.795653in}{1.747188in}}{\pgfqpoint{1.806825in}{1.742560in}}{\pgfqpoint{1.818473in}{1.742560in}}%
\pgfpathlineto{\pgfqpoint{1.818473in}{1.742560in}}%
\pgfpathclose%
\pgfusepath{stroke,fill}%
\end{pgfscope}%
\begin{pgfscope}%
\pgfpathrectangle{\pgfqpoint{0.563297in}{0.529443in}}{\pgfqpoint{1.636659in}{1.745990in}}%
\pgfusepath{clip}%
\pgfsetbuttcap%
\pgfsetroundjoin%
\definecolor{currentfill}{rgb}{0.003922,0.003922,0.003922}%
\pgfsetfillcolor{currentfill}%
\pgfsetfillopacity{0.900000}%
\pgfsetlinewidth{0.507862pt}%
\definecolor{currentstroke}{rgb}{1.000000,1.000000,1.000000}%
\pgfsetstrokecolor{currentstroke}%
\pgfsetstrokeopacity{0.900000}%
\pgfsetdash{}{0pt}%
\pgfpathmoveto{\pgfqpoint{1.918716in}{1.774024in}}%
\pgfpathcurveto{\pgfqpoint{1.930364in}{1.774024in}}{\pgfqpoint{1.941536in}{1.778651in}}{\pgfqpoint{1.949772in}{1.786888in}}%
\pgfpathcurveto{\pgfqpoint{1.958009in}{1.795124in}}{\pgfqpoint{1.962636in}{1.806296in}}{\pgfqpoint{1.962636in}{1.817944in}}%
\pgfpathcurveto{\pgfqpoint{1.962636in}{1.829592in}}{\pgfqpoint{1.958009in}{1.840764in}}{\pgfqpoint{1.949772in}{1.849001in}}%
\pgfpathcurveto{\pgfqpoint{1.941536in}{1.857237in}}{\pgfqpoint{1.930364in}{1.861865in}}{\pgfqpoint{1.918716in}{1.861865in}}%
\pgfpathcurveto{\pgfqpoint{1.907068in}{1.861865in}}{\pgfqpoint{1.895896in}{1.857237in}}{\pgfqpoint{1.887659in}{1.849001in}}%
\pgfpathcurveto{\pgfqpoint{1.879423in}{1.840764in}}{\pgfqpoint{1.874795in}{1.829592in}}{\pgfqpoint{1.874795in}{1.817944in}}%
\pgfpathcurveto{\pgfqpoint{1.874795in}{1.806296in}}{\pgfqpoint{1.879423in}{1.795124in}}{\pgfqpoint{1.887659in}{1.786888in}}%
\pgfpathcurveto{\pgfqpoint{1.895896in}{1.778651in}}{\pgfqpoint{1.907068in}{1.774024in}}{\pgfqpoint{1.918716in}{1.774024in}}%
\pgfpathlineto{\pgfqpoint{1.918716in}{1.774024in}}%
\pgfpathclose%
\pgfusepath{stroke,fill}%
\end{pgfscope}%
\begin{pgfscope}%
\pgfpathrectangle{\pgfqpoint{0.563297in}{0.529443in}}{\pgfqpoint{1.636659in}{1.745990in}}%
\pgfusepath{clip}%
\pgfsetbuttcap%
\pgfsetroundjoin%
\definecolor{currentfill}{rgb}{0.003922,0.003922,0.003922}%
\pgfsetfillcolor{currentfill}%
\pgfsetfillopacity{0.900000}%
\pgfsetlinewidth{0.507862pt}%
\definecolor{currentstroke}{rgb}{1.000000,1.000000,1.000000}%
\pgfsetstrokecolor{currentstroke}%
\pgfsetstrokeopacity{0.900000}%
\pgfsetdash{}{0pt}%
\pgfpathmoveto{\pgfqpoint{2.041416in}{1.930303in}}%
\pgfpathcurveto{\pgfqpoint{2.053064in}{1.930303in}}{\pgfqpoint{2.064236in}{1.934930in}}{\pgfqpoint{2.072472in}{1.943167in}}%
\pgfpathcurveto{\pgfqpoint{2.080709in}{1.951403in}}{\pgfqpoint{2.085337in}{1.962575in}}{\pgfqpoint{2.085337in}{1.974223in}}%
\pgfpathcurveto{\pgfqpoint{2.085337in}{1.985871in}}{\pgfqpoint{2.080709in}{1.997043in}}{\pgfqpoint{2.072472in}{2.005280in}}%
\pgfpathcurveto{\pgfqpoint{2.064236in}{2.013516in}}{\pgfqpoint{2.053064in}{2.018144in}}{\pgfqpoint{2.041416in}{2.018144in}}%
\pgfpathcurveto{\pgfqpoint{2.029768in}{2.018144in}}{\pgfqpoint{2.018596in}{2.013516in}}{\pgfqpoint{2.010359in}{2.005280in}}%
\pgfpathcurveto{\pgfqpoint{2.002123in}{1.997043in}}{\pgfqpoint{1.997495in}{1.985871in}}{\pgfqpoint{1.997495in}{1.974223in}}%
\pgfpathcurveto{\pgfqpoint{1.997495in}{1.962575in}}{\pgfqpoint{2.002123in}{1.951403in}}{\pgfqpoint{2.010359in}{1.943167in}}%
\pgfpathcurveto{\pgfqpoint{2.018596in}{1.934930in}}{\pgfqpoint{2.029768in}{1.930303in}}{\pgfqpoint{2.041416in}{1.930303in}}%
\pgfpathlineto{\pgfqpoint{2.041416in}{1.930303in}}%
\pgfpathclose%
\pgfusepath{stroke,fill}%
\end{pgfscope}%
\begin{pgfscope}%
\pgfpathrectangle{\pgfqpoint{0.563297in}{0.529443in}}{\pgfqpoint{1.636659in}{1.745990in}}%
\pgfusepath{clip}%
\pgfsetbuttcap%
\pgfsetroundjoin%
\definecolor{currentfill}{rgb}{0.003922,0.003922,0.003922}%
\pgfsetfillcolor{currentfill}%
\pgfsetfillopacity{0.900000}%
\pgfsetlinewidth{0.507862pt}%
\definecolor{currentstroke}{rgb}{1.000000,1.000000,1.000000}%
\pgfsetstrokecolor{currentstroke}%
\pgfsetstrokeopacity{0.900000}%
\pgfsetdash{}{0pt}%
\pgfpathmoveto{\pgfqpoint{0.697247in}{0.596425in}}%
\pgfpathcurveto{\pgfqpoint{0.708895in}{0.596425in}}{\pgfqpoint{0.720067in}{0.601053in}}{\pgfqpoint{0.728303in}{0.609289in}}%
\pgfpathcurveto{\pgfqpoint{0.736540in}{0.617525in}}{\pgfqpoint{0.741167in}{0.628698in}}{\pgfqpoint{0.741167in}{0.640346in}}%
\pgfpathcurveto{\pgfqpoint{0.741167in}{0.651993in}}{\pgfqpoint{0.736540in}{0.663166in}}{\pgfqpoint{0.728303in}{0.671402in}}%
\pgfpathcurveto{\pgfqpoint{0.720067in}{0.679638in}}{\pgfqpoint{0.708895in}{0.684266in}}{\pgfqpoint{0.697247in}{0.684266in}}%
\pgfpathcurveto{\pgfqpoint{0.685599in}{0.684266in}}{\pgfqpoint{0.674427in}{0.679638in}}{\pgfqpoint{0.666190in}{0.671402in}}%
\pgfpathcurveto{\pgfqpoint{0.657954in}{0.663166in}}{\pgfqpoint{0.653326in}{0.651993in}}{\pgfqpoint{0.653326in}{0.640346in}}%
\pgfpathcurveto{\pgfqpoint{0.653326in}{0.628698in}}{\pgfqpoint{0.657954in}{0.617525in}}{\pgfqpoint{0.666190in}{0.609289in}}%
\pgfpathcurveto{\pgfqpoint{0.674427in}{0.601053in}}{\pgfqpoint{0.685599in}{0.596425in}}{\pgfqpoint{0.697247in}{0.596425in}}%
\pgfpathlineto{\pgfqpoint{0.697247in}{0.596425in}}%
\pgfpathclose%
\pgfusepath{stroke,fill}%
\end{pgfscope}%
\begin{pgfscope}%
\pgfpathrectangle{\pgfqpoint{0.563297in}{0.529443in}}{\pgfqpoint{1.636659in}{1.745990in}}%
\pgfusepath{clip}%
\pgfsetbuttcap%
\pgfsetroundjoin%
\definecolor{currentfill}{rgb}{0.003922,0.003922,0.003922}%
\pgfsetfillcolor{currentfill}%
\pgfsetfillopacity{0.900000}%
\pgfsetlinewidth{0.507862pt}%
\definecolor{currentstroke}{rgb}{1.000000,1.000000,1.000000}%
\pgfsetstrokecolor{currentstroke}%
\pgfsetstrokeopacity{0.900000}%
\pgfsetdash{}{0pt}%
\pgfpathmoveto{\pgfqpoint{1.448948in}{1.673730in}}%
\pgfpathcurveto{\pgfqpoint{1.460596in}{1.673730in}}{\pgfqpoint{1.471769in}{1.678357in}}{\pgfqpoint{1.480005in}{1.686594in}}%
\pgfpathcurveto{\pgfqpoint{1.488241in}{1.694830in}}{\pgfqpoint{1.492869in}{1.706002in}}{\pgfqpoint{1.492869in}{1.717650in}}%
\pgfpathcurveto{\pgfqpoint{1.492869in}{1.729298in}}{\pgfqpoint{1.488241in}{1.740470in}}{\pgfqpoint{1.480005in}{1.748707in}}%
\pgfpathcurveto{\pgfqpoint{1.471769in}{1.756943in}}{\pgfqpoint{1.460596in}{1.761571in}}{\pgfqpoint{1.448948in}{1.761571in}}%
\pgfpathcurveto{\pgfqpoint{1.437300in}{1.761571in}}{\pgfqpoint{1.426128in}{1.756943in}}{\pgfqpoint{1.417892in}{1.748707in}}%
\pgfpathcurveto{\pgfqpoint{1.409656in}{1.740470in}}{\pgfqpoint{1.405028in}{1.729298in}}{\pgfqpoint{1.405028in}{1.717650in}}%
\pgfpathcurveto{\pgfqpoint{1.405028in}{1.706002in}}{\pgfqpoint{1.409656in}{1.694830in}}{\pgfqpoint{1.417892in}{1.686594in}}%
\pgfpathcurveto{\pgfqpoint{1.426128in}{1.678357in}}{\pgfqpoint{1.437300in}{1.673730in}}{\pgfqpoint{1.448948in}{1.673730in}}%
\pgfpathlineto{\pgfqpoint{1.448948in}{1.673730in}}%
\pgfpathclose%
\pgfusepath{stroke,fill}%
\end{pgfscope}%
\begin{pgfscope}%
\pgfpathrectangle{\pgfqpoint{0.563297in}{0.529443in}}{\pgfqpoint{1.636659in}{1.745990in}}%
\pgfusepath{clip}%
\pgfsetbuttcap%
\pgfsetroundjoin%
\definecolor{currentfill}{rgb}{0.003922,0.003922,0.003922}%
\pgfsetfillcolor{currentfill}%
\pgfsetfillopacity{0.900000}%
\pgfsetlinewidth{0.507862pt}%
\definecolor{currentstroke}{rgb}{1.000000,1.000000,1.000000}%
\pgfsetstrokecolor{currentstroke}%
\pgfsetstrokeopacity{0.900000}%
\pgfsetdash{}{0pt}%
\pgfpathmoveto{\pgfqpoint{1.826983in}{1.882543in}}%
\pgfpathcurveto{\pgfqpoint{1.838631in}{1.882543in}}{\pgfqpoint{1.849803in}{1.887171in}}{\pgfqpoint{1.858039in}{1.895407in}}%
\pgfpathcurveto{\pgfqpoint{1.866276in}{1.903644in}}{\pgfqpoint{1.870903in}{1.914816in}}{\pgfqpoint{1.870903in}{1.926464in}}%
\pgfpathcurveto{\pgfqpoint{1.870903in}{1.938112in}}{\pgfqpoint{1.866276in}{1.949284in}}{\pgfqpoint{1.858039in}{1.957520in}}%
\pgfpathcurveto{\pgfqpoint{1.849803in}{1.965757in}}{\pgfqpoint{1.838631in}{1.970384in}}{\pgfqpoint{1.826983in}{1.970384in}}%
\pgfpathcurveto{\pgfqpoint{1.815335in}{1.970384in}}{\pgfqpoint{1.804163in}{1.965757in}}{\pgfqpoint{1.795926in}{1.957520in}}%
\pgfpathcurveto{\pgfqpoint{1.787690in}{1.949284in}}{\pgfqpoint{1.783062in}{1.938112in}}{\pgfqpoint{1.783062in}{1.926464in}}%
\pgfpathcurveto{\pgfqpoint{1.783062in}{1.914816in}}{\pgfqpoint{1.787690in}{1.903644in}}{\pgfqpoint{1.795926in}{1.895407in}}%
\pgfpathcurveto{\pgfqpoint{1.804163in}{1.887171in}}{\pgfqpoint{1.815335in}{1.882543in}}{\pgfqpoint{1.826983in}{1.882543in}}%
\pgfpathlineto{\pgfqpoint{1.826983in}{1.882543in}}%
\pgfpathclose%
\pgfusepath{stroke,fill}%
\end{pgfscope}%
\begin{pgfscope}%
\pgfpathrectangle{\pgfqpoint{0.563297in}{0.529443in}}{\pgfqpoint{1.636659in}{1.745990in}}%
\pgfusepath{clip}%
\pgfsetbuttcap%
\pgfsetroundjoin%
\definecolor{currentfill}{rgb}{0.003922,0.003922,0.003922}%
\pgfsetfillcolor{currentfill}%
\pgfsetfillopacity{0.900000}%
\pgfsetlinewidth{0.507862pt}%
\definecolor{currentstroke}{rgb}{1.000000,1.000000,1.000000}%
\pgfsetstrokecolor{currentstroke}%
\pgfsetstrokeopacity{0.900000}%
\pgfsetdash{}{0pt}%
\pgfpathmoveto{\pgfqpoint{0.637691in}{0.646799in}}%
\pgfpathcurveto{\pgfqpoint{0.649339in}{0.646799in}}{\pgfqpoint{0.660511in}{0.651426in}}{\pgfqpoint{0.668747in}{0.659663in}}%
\pgfpathcurveto{\pgfqpoint{0.676984in}{0.667899in}}{\pgfqpoint{0.681611in}{0.679071in}}{\pgfqpoint{0.681611in}{0.690719in}}%
\pgfpathcurveto{\pgfqpoint{0.681611in}{0.702367in}}{\pgfqpoint{0.676984in}{0.713539in}}{\pgfqpoint{0.668747in}{0.721776in}}%
\pgfpathcurveto{\pgfqpoint{0.660511in}{0.730012in}}{\pgfqpoint{0.649339in}{0.734640in}}{\pgfqpoint{0.637691in}{0.734640in}}%
\pgfpathcurveto{\pgfqpoint{0.626043in}{0.734640in}}{\pgfqpoint{0.614871in}{0.730012in}}{\pgfqpoint{0.606634in}{0.721776in}}%
\pgfpathcurveto{\pgfqpoint{0.598398in}{0.713539in}}{\pgfqpoint{0.593770in}{0.702367in}}{\pgfqpoint{0.593770in}{0.690719in}}%
\pgfpathcurveto{\pgfqpoint{0.593770in}{0.679071in}}{\pgfqpoint{0.598398in}{0.667899in}}{\pgfqpoint{0.606634in}{0.659663in}}%
\pgfpathcurveto{\pgfqpoint{0.614871in}{0.651426in}}{\pgfqpoint{0.626043in}{0.646799in}}{\pgfqpoint{0.637691in}{0.646799in}}%
\pgfpathlineto{\pgfqpoint{0.637691in}{0.646799in}}%
\pgfpathclose%
\pgfusepath{stroke,fill}%
\end{pgfscope}%
\begin{pgfscope}%
\pgfpathrectangle{\pgfqpoint{0.563297in}{0.529443in}}{\pgfqpoint{1.636659in}{1.745990in}}%
\pgfusepath{clip}%
\pgfsetbuttcap%
\pgfsetroundjoin%
\definecolor{currentfill}{rgb}{0.003922,0.003922,0.003922}%
\pgfsetfillcolor{currentfill}%
\pgfsetfillopacity{0.900000}%
\pgfsetlinewidth{0.507862pt}%
\definecolor{currentstroke}{rgb}{1.000000,1.000000,1.000000}%
\pgfsetstrokecolor{currentstroke}%
\pgfsetstrokeopacity{0.900000}%
\pgfsetdash{}{0pt}%
\pgfpathmoveto{\pgfqpoint{1.617628in}{1.575810in}}%
\pgfpathcurveto{\pgfqpoint{1.629276in}{1.575810in}}{\pgfqpoint{1.640448in}{1.580437in}}{\pgfqpoint{1.648684in}{1.588674in}}%
\pgfpathcurveto{\pgfqpoint{1.656921in}{1.596910in}}{\pgfqpoint{1.661548in}{1.608082in}}{\pgfqpoint{1.661548in}{1.619730in}}%
\pgfpathcurveto{\pgfqpoint{1.661548in}{1.631378in}}{\pgfqpoint{1.656921in}{1.642550in}}{\pgfqpoint{1.648684in}{1.650787in}}%
\pgfpathcurveto{\pgfqpoint{1.640448in}{1.659023in}}{\pgfqpoint{1.629276in}{1.663651in}}{\pgfqpoint{1.617628in}{1.663651in}}%
\pgfpathcurveto{\pgfqpoint{1.605980in}{1.663651in}}{\pgfqpoint{1.594808in}{1.659023in}}{\pgfqpoint{1.586571in}{1.650787in}}%
\pgfpathcurveto{\pgfqpoint{1.578335in}{1.642550in}}{\pgfqpoint{1.573707in}{1.631378in}}{\pgfqpoint{1.573707in}{1.619730in}}%
\pgfpathcurveto{\pgfqpoint{1.573707in}{1.608082in}}{\pgfqpoint{1.578335in}{1.596910in}}{\pgfqpoint{1.586571in}{1.588674in}}%
\pgfpathcurveto{\pgfqpoint{1.594808in}{1.580437in}}{\pgfqpoint{1.605980in}{1.575810in}}{\pgfqpoint{1.617628in}{1.575810in}}%
\pgfpathlineto{\pgfqpoint{1.617628in}{1.575810in}}%
\pgfpathclose%
\pgfusepath{stroke,fill}%
\end{pgfscope}%
\begin{pgfscope}%
\pgfpathrectangle{\pgfqpoint{0.563297in}{0.529443in}}{\pgfqpoint{1.636659in}{1.745990in}}%
\pgfusepath{clip}%
\pgfsetbuttcap%
\pgfsetroundjoin%
\definecolor{currentfill}{rgb}{0.003922,0.003922,0.003922}%
\pgfsetfillcolor{currentfill}%
\pgfsetfillopacity{0.900000}%
\pgfsetlinewidth{0.507862pt}%
\definecolor{currentstroke}{rgb}{1.000000,1.000000,1.000000}%
\pgfsetstrokecolor{currentstroke}%
\pgfsetstrokeopacity{0.900000}%
\pgfsetdash{}{0pt}%
\pgfpathmoveto{\pgfqpoint{1.409596in}{1.588268in}}%
\pgfpathcurveto{\pgfqpoint{1.421243in}{1.588268in}}{\pgfqpoint{1.432416in}{1.592895in}}{\pgfqpoint{1.440652in}{1.601132in}}%
\pgfpathcurveto{\pgfqpoint{1.448888in}{1.609368in}}{\pgfqpoint{1.453516in}{1.620540in}}{\pgfqpoint{1.453516in}{1.632188in}}%
\pgfpathcurveto{\pgfqpoint{1.453516in}{1.643836in}}{\pgfqpoint{1.448888in}{1.655008in}}{\pgfqpoint{1.440652in}{1.663245in}}%
\pgfpathcurveto{\pgfqpoint{1.432416in}{1.671481in}}{\pgfqpoint{1.421243in}{1.676109in}}{\pgfqpoint{1.409596in}{1.676109in}}%
\pgfpathcurveto{\pgfqpoint{1.397948in}{1.676109in}}{\pgfqpoint{1.386775in}{1.671481in}}{\pgfqpoint{1.378539in}{1.663245in}}%
\pgfpathcurveto{\pgfqpoint{1.370303in}{1.655008in}}{\pgfqpoint{1.365675in}{1.643836in}}{\pgfqpoint{1.365675in}{1.632188in}}%
\pgfpathcurveto{\pgfqpoint{1.365675in}{1.620540in}}{\pgfqpoint{1.370303in}{1.609368in}}{\pgfqpoint{1.378539in}{1.601132in}}%
\pgfpathcurveto{\pgfqpoint{1.386775in}{1.592895in}}{\pgfqpoint{1.397948in}{1.588268in}}{\pgfqpoint{1.409596in}{1.588268in}}%
\pgfpathlineto{\pgfqpoint{1.409596in}{1.588268in}}%
\pgfpathclose%
\pgfusepath{stroke,fill}%
\end{pgfscope}%
\begin{pgfscope}%
\pgfpathrectangle{\pgfqpoint{0.563297in}{0.529443in}}{\pgfqpoint{1.636659in}{1.745990in}}%
\pgfusepath{clip}%
\pgfsetbuttcap%
\pgfsetroundjoin%
\definecolor{currentfill}{rgb}{0.003922,0.003922,0.003922}%
\pgfsetfillcolor{currentfill}%
\pgfsetfillopacity{0.900000}%
\pgfsetlinewidth{0.507862pt}%
\definecolor{currentstroke}{rgb}{1.000000,1.000000,1.000000}%
\pgfsetstrokecolor{currentstroke}%
\pgfsetstrokeopacity{0.900000}%
\pgfsetdash{}{0pt}%
\pgfpathmoveto{\pgfqpoint{2.084978in}{2.013742in}}%
\pgfpathcurveto{\pgfqpoint{2.096626in}{2.013742in}}{\pgfqpoint{2.107798in}{2.018369in}}{\pgfqpoint{2.116034in}{2.026606in}}%
\pgfpathcurveto{\pgfqpoint{2.124271in}{2.034842in}}{\pgfqpoint{2.128898in}{2.046014in}}{\pgfqpoint{2.128898in}{2.057662in}}%
\pgfpathcurveto{\pgfqpoint{2.128898in}{2.069310in}}{\pgfqpoint{2.124271in}{2.080482in}}{\pgfqpoint{2.116034in}{2.088719in}}%
\pgfpathcurveto{\pgfqpoint{2.107798in}{2.096955in}}{\pgfqpoint{2.096626in}{2.101583in}}{\pgfqpoint{2.084978in}{2.101583in}}%
\pgfpathcurveto{\pgfqpoint{2.073330in}{2.101583in}}{\pgfqpoint{2.062158in}{2.096955in}}{\pgfqpoint{2.053921in}{2.088719in}}%
\pgfpathcurveto{\pgfqpoint{2.045685in}{2.080482in}}{\pgfqpoint{2.041057in}{2.069310in}}{\pgfqpoint{2.041057in}{2.057662in}}%
\pgfpathcurveto{\pgfqpoint{2.041057in}{2.046014in}}{\pgfqpoint{2.045685in}{2.034842in}}{\pgfqpoint{2.053921in}{2.026606in}}%
\pgfpathcurveto{\pgfqpoint{2.062158in}{2.018369in}}{\pgfqpoint{2.073330in}{2.013742in}}{\pgfqpoint{2.084978in}{2.013742in}}%
\pgfpathlineto{\pgfqpoint{2.084978in}{2.013742in}}%
\pgfpathclose%
\pgfusepath{stroke,fill}%
\end{pgfscope}%
\begin{pgfscope}%
\pgfpathrectangle{\pgfqpoint{0.563297in}{0.529443in}}{\pgfqpoint{1.636659in}{1.745990in}}%
\pgfusepath{clip}%
\pgfsetbuttcap%
\pgfsetroundjoin%
\definecolor{currentfill}{rgb}{0.003922,0.003922,0.003922}%
\pgfsetfillcolor{currentfill}%
\pgfsetfillopacity{0.900000}%
\pgfsetlinewidth{0.507862pt}%
\definecolor{currentstroke}{rgb}{1.000000,1.000000,1.000000}%
\pgfsetstrokecolor{currentstroke}%
\pgfsetstrokeopacity{0.900000}%
\pgfsetdash{}{0pt}%
\pgfpathmoveto{\pgfqpoint{1.587251in}{1.917390in}}%
\pgfpathcurveto{\pgfqpoint{1.598899in}{1.917390in}}{\pgfqpoint{1.610071in}{1.922017in}}{\pgfqpoint{1.618307in}{1.930254in}}%
\pgfpathcurveto{\pgfqpoint{1.626544in}{1.938490in}}{\pgfqpoint{1.631171in}{1.949662in}}{\pgfqpoint{1.631171in}{1.961310in}}%
\pgfpathcurveto{\pgfqpoint{1.631171in}{1.972958in}}{\pgfqpoint{1.626544in}{1.984130in}}{\pgfqpoint{1.618307in}{1.992367in}}%
\pgfpathcurveto{\pgfqpoint{1.610071in}{2.000603in}}{\pgfqpoint{1.598899in}{2.005231in}}{\pgfqpoint{1.587251in}{2.005231in}}%
\pgfpathcurveto{\pgfqpoint{1.575603in}{2.005231in}}{\pgfqpoint{1.564431in}{2.000603in}}{\pgfqpoint{1.556194in}{1.992367in}}%
\pgfpathcurveto{\pgfqpoint{1.547958in}{1.984130in}}{\pgfqpoint{1.543330in}{1.972958in}}{\pgfqpoint{1.543330in}{1.961310in}}%
\pgfpathcurveto{\pgfqpoint{1.543330in}{1.949662in}}{\pgfqpoint{1.547958in}{1.938490in}}{\pgfqpoint{1.556194in}{1.930254in}}%
\pgfpathcurveto{\pgfqpoint{1.564431in}{1.922017in}}{\pgfqpoint{1.575603in}{1.917390in}}{\pgfqpoint{1.587251in}{1.917390in}}%
\pgfpathlineto{\pgfqpoint{1.587251in}{1.917390in}}%
\pgfpathclose%
\pgfusepath{stroke,fill}%
\end{pgfscope}%
\begin{pgfscope}%
\pgfpathrectangle{\pgfqpoint{0.563297in}{0.529443in}}{\pgfqpoint{1.636659in}{1.745990in}}%
\pgfusepath{clip}%
\pgfsetbuttcap%
\pgfsetroundjoin%
\definecolor{currentfill}{rgb}{0.003922,0.003922,0.003922}%
\pgfsetfillcolor{currentfill}%
\pgfsetfillopacity{0.900000}%
\pgfsetlinewidth{0.507862pt}%
\definecolor{currentstroke}{rgb}{1.000000,1.000000,1.000000}%
\pgfsetstrokecolor{currentstroke}%
\pgfsetstrokeopacity{0.900000}%
\pgfsetdash{}{0pt}%
\pgfpathmoveto{\pgfqpoint{1.596984in}{1.647331in}}%
\pgfpathcurveto{\pgfqpoint{1.608632in}{1.647331in}}{\pgfqpoint{1.619805in}{1.651958in}}{\pgfqpoint{1.628041in}{1.660195in}}%
\pgfpathcurveto{\pgfqpoint{1.636277in}{1.668431in}}{\pgfqpoint{1.640905in}{1.679603in}}{\pgfqpoint{1.640905in}{1.691251in}}%
\pgfpathcurveto{\pgfqpoint{1.640905in}{1.702899in}}{\pgfqpoint{1.636277in}{1.714071in}}{\pgfqpoint{1.628041in}{1.722308in}}%
\pgfpathcurveto{\pgfqpoint{1.619805in}{1.730544in}}{\pgfqpoint{1.608632in}{1.735172in}}{\pgfqpoint{1.596984in}{1.735172in}}%
\pgfpathcurveto{\pgfqpoint{1.585337in}{1.735172in}}{\pgfqpoint{1.574164in}{1.730544in}}{\pgfqpoint{1.565928in}{1.722308in}}%
\pgfpathcurveto{\pgfqpoint{1.557692in}{1.714071in}}{\pgfqpoint{1.553064in}{1.702899in}}{\pgfqpoint{1.553064in}{1.691251in}}%
\pgfpathcurveto{\pgfqpoint{1.553064in}{1.679603in}}{\pgfqpoint{1.557692in}{1.668431in}}{\pgfqpoint{1.565928in}{1.660195in}}%
\pgfpathcurveto{\pgfqpoint{1.574164in}{1.651958in}}{\pgfqpoint{1.585337in}{1.647331in}}{\pgfqpoint{1.596984in}{1.647331in}}%
\pgfpathlineto{\pgfqpoint{1.596984in}{1.647331in}}%
\pgfpathclose%
\pgfusepath{stroke,fill}%
\end{pgfscope}%
\begin{pgfscope}%
\pgfpathrectangle{\pgfqpoint{0.563297in}{0.529443in}}{\pgfqpoint{1.636659in}{1.745990in}}%
\pgfusepath{clip}%
\pgfsetbuttcap%
\pgfsetroundjoin%
\definecolor{currentfill}{rgb}{0.003922,0.003922,0.003922}%
\pgfsetfillcolor{currentfill}%
\pgfsetfillopacity{0.900000}%
\pgfsetlinewidth{0.507862pt}%
\definecolor{currentstroke}{rgb}{1.000000,1.000000,1.000000}%
\pgfsetstrokecolor{currentstroke}%
\pgfsetstrokeopacity{0.900000}%
\pgfsetdash{}{0pt}%
\pgfpathmoveto{\pgfqpoint{1.017376in}{1.265000in}}%
\pgfpathcurveto{\pgfqpoint{1.029024in}{1.265000in}}{\pgfqpoint{1.040196in}{1.269628in}}{\pgfqpoint{1.048433in}{1.277864in}}%
\pgfpathcurveto{\pgfqpoint{1.056669in}{1.286100in}}{\pgfqpoint{1.061297in}{1.297273in}}{\pgfqpoint{1.061297in}{1.308921in}}%
\pgfpathcurveto{\pgfqpoint{1.061297in}{1.320569in}}{\pgfqpoint{1.056669in}{1.331741in}}{\pgfqpoint{1.048433in}{1.339977in}}%
\pgfpathcurveto{\pgfqpoint{1.040196in}{1.348213in}}{\pgfqpoint{1.029024in}{1.352841in}}{\pgfqpoint{1.017376in}{1.352841in}}%
\pgfpathcurveto{\pgfqpoint{1.005728in}{1.352841in}}{\pgfqpoint{0.994556in}{1.348213in}}{\pgfqpoint{0.986320in}{1.339977in}}%
\pgfpathcurveto{\pgfqpoint{0.978083in}{1.331741in}}{\pgfqpoint{0.973456in}{1.320569in}}{\pgfqpoint{0.973456in}{1.308921in}}%
\pgfpathcurveto{\pgfqpoint{0.973456in}{1.297273in}}{\pgfqpoint{0.978083in}{1.286100in}}{\pgfqpoint{0.986320in}{1.277864in}}%
\pgfpathcurveto{\pgfqpoint{0.994556in}{1.269628in}}{\pgfqpoint{1.005728in}{1.265000in}}{\pgfqpoint{1.017376in}{1.265000in}}%
\pgfpathlineto{\pgfqpoint{1.017376in}{1.265000in}}%
\pgfpathclose%
\pgfusepath{stroke,fill}%
\end{pgfscope}%
\begin{pgfscope}%
\pgfpathrectangle{\pgfqpoint{0.563297in}{0.529443in}}{\pgfqpoint{1.636659in}{1.745990in}}%
\pgfusepath{clip}%
\pgfsetbuttcap%
\pgfsetroundjoin%
\definecolor{currentfill}{rgb}{0.003922,0.003922,0.003922}%
\pgfsetfillcolor{currentfill}%
\pgfsetfillopacity{0.900000}%
\pgfsetlinewidth{0.507862pt}%
\definecolor{currentstroke}{rgb}{1.000000,1.000000,1.000000}%
\pgfsetstrokecolor{currentstroke}%
\pgfsetstrokeopacity{0.900000}%
\pgfsetdash{}{0pt}%
\pgfpathmoveto{\pgfqpoint{2.125563in}{1.873923in}}%
\pgfpathcurveto{\pgfqpoint{2.137211in}{1.873923in}}{\pgfqpoint{2.148383in}{1.878550in}}{\pgfqpoint{2.156619in}{1.886787in}}%
\pgfpathcurveto{\pgfqpoint{2.164856in}{1.895023in}}{\pgfqpoint{2.169483in}{1.906195in}}{\pgfqpoint{2.169483in}{1.917843in}}%
\pgfpathcurveto{\pgfqpoint{2.169483in}{1.929491in}}{\pgfqpoint{2.164856in}{1.940663in}}{\pgfqpoint{2.156619in}{1.948900in}}%
\pgfpathcurveto{\pgfqpoint{2.148383in}{1.957136in}}{\pgfqpoint{2.137211in}{1.961764in}}{\pgfqpoint{2.125563in}{1.961764in}}%
\pgfpathcurveto{\pgfqpoint{2.113915in}{1.961764in}}{\pgfqpoint{2.102743in}{1.957136in}}{\pgfqpoint{2.094506in}{1.948900in}}%
\pgfpathcurveto{\pgfqpoint{2.086270in}{1.940663in}}{\pgfqpoint{2.081642in}{1.929491in}}{\pgfqpoint{2.081642in}{1.917843in}}%
\pgfpathcurveto{\pgfqpoint{2.081642in}{1.906195in}}{\pgfqpoint{2.086270in}{1.895023in}}{\pgfqpoint{2.094506in}{1.886787in}}%
\pgfpathcurveto{\pgfqpoint{2.102743in}{1.878550in}}{\pgfqpoint{2.113915in}{1.873923in}}{\pgfqpoint{2.125563in}{1.873923in}}%
\pgfpathlineto{\pgfqpoint{2.125563in}{1.873923in}}%
\pgfpathclose%
\pgfusepath{stroke,fill}%
\end{pgfscope}%
\begin{pgfscope}%
\pgfpathrectangle{\pgfqpoint{0.563297in}{0.529443in}}{\pgfqpoint{1.636659in}{1.745990in}}%
\pgfusepath{clip}%
\pgfsetbuttcap%
\pgfsetroundjoin%
\definecolor{currentfill}{rgb}{0.003922,0.003922,0.003922}%
\pgfsetfillcolor{currentfill}%
\pgfsetfillopacity{0.900000}%
\pgfsetlinewidth{0.507862pt}%
\definecolor{currentstroke}{rgb}{1.000000,1.000000,1.000000}%
\pgfsetstrokecolor{currentstroke}%
\pgfsetstrokeopacity{0.900000}%
\pgfsetdash{}{0pt}%
\pgfpathmoveto{\pgfqpoint{0.945799in}{1.213120in}}%
\pgfpathcurveto{\pgfqpoint{0.957447in}{1.213120in}}{\pgfqpoint{0.968619in}{1.217748in}}{\pgfqpoint{0.976856in}{1.225984in}}%
\pgfpathcurveto{\pgfqpoint{0.985092in}{1.234220in}}{\pgfqpoint{0.989720in}{1.245393in}}{\pgfqpoint{0.989720in}{1.257041in}}%
\pgfpathcurveto{\pgfqpoint{0.989720in}{1.268688in}}{\pgfqpoint{0.985092in}{1.279861in}}{\pgfqpoint{0.976856in}{1.288097in}}%
\pgfpathcurveto{\pgfqpoint{0.968619in}{1.296333in}}{\pgfqpoint{0.957447in}{1.300961in}}{\pgfqpoint{0.945799in}{1.300961in}}%
\pgfpathcurveto{\pgfqpoint{0.934151in}{1.300961in}}{\pgfqpoint{0.922979in}{1.296333in}}{\pgfqpoint{0.914743in}{1.288097in}}%
\pgfpathcurveto{\pgfqpoint{0.906506in}{1.279861in}}{\pgfqpoint{0.901879in}{1.268688in}}{\pgfqpoint{0.901879in}{1.257041in}}%
\pgfpathcurveto{\pgfqpoint{0.901879in}{1.245393in}}{\pgfqpoint{0.906506in}{1.234220in}}{\pgfqpoint{0.914743in}{1.225984in}}%
\pgfpathcurveto{\pgfqpoint{0.922979in}{1.217748in}}{\pgfqpoint{0.934151in}{1.213120in}}{\pgfqpoint{0.945799in}{1.213120in}}%
\pgfpathlineto{\pgfqpoint{0.945799in}{1.213120in}}%
\pgfpathclose%
\pgfusepath{stroke,fill}%
\end{pgfscope}%
\begin{pgfscope}%
\pgfpathrectangle{\pgfqpoint{0.563297in}{0.529443in}}{\pgfqpoint{1.636659in}{1.745990in}}%
\pgfusepath{clip}%
\pgfsetbuttcap%
\pgfsetroundjoin%
\definecolor{currentfill}{rgb}{0.003922,0.003922,0.003922}%
\pgfsetfillcolor{currentfill}%
\pgfsetfillopacity{0.900000}%
\pgfsetlinewidth{0.507862pt}%
\definecolor{currentstroke}{rgb}{1.000000,1.000000,1.000000}%
\pgfsetstrokecolor{currentstroke}%
\pgfsetstrokeopacity{0.900000}%
\pgfsetdash{}{0pt}%
\pgfpathmoveto{\pgfqpoint{2.004303in}{2.005781in}}%
\pgfpathcurveto{\pgfqpoint{2.015951in}{2.005781in}}{\pgfqpoint{2.027124in}{2.010409in}}{\pgfqpoint{2.035360in}{2.018645in}}%
\pgfpathcurveto{\pgfqpoint{2.043596in}{2.026882in}}{\pgfqpoint{2.048224in}{2.038054in}}{\pgfqpoint{2.048224in}{2.049702in}}%
\pgfpathcurveto{\pgfqpoint{2.048224in}{2.061350in}}{\pgfqpoint{2.043596in}{2.072522in}}{\pgfqpoint{2.035360in}{2.080758in}}%
\pgfpathcurveto{\pgfqpoint{2.027124in}{2.088995in}}{\pgfqpoint{2.015951in}{2.093622in}}{\pgfqpoint{2.004303in}{2.093622in}}%
\pgfpathcurveto{\pgfqpoint{1.992655in}{2.093622in}}{\pgfqpoint{1.981483in}{2.088995in}}{\pgfqpoint{1.973247in}{2.080758in}}%
\pgfpathcurveto{\pgfqpoint{1.965011in}{2.072522in}}{\pgfqpoint{1.960383in}{2.061350in}}{\pgfqpoint{1.960383in}{2.049702in}}%
\pgfpathcurveto{\pgfqpoint{1.960383in}{2.038054in}}{\pgfqpoint{1.965011in}{2.026882in}}{\pgfqpoint{1.973247in}{2.018645in}}%
\pgfpathcurveto{\pgfqpoint{1.981483in}{2.010409in}}{\pgfqpoint{1.992655in}{2.005781in}}{\pgfqpoint{2.004303in}{2.005781in}}%
\pgfpathlineto{\pgfqpoint{2.004303in}{2.005781in}}%
\pgfpathclose%
\pgfusepath{stroke,fill}%
\end{pgfscope}%
\begin{pgfscope}%
\pgfpathrectangle{\pgfqpoint{0.563297in}{0.529443in}}{\pgfqpoint{1.636659in}{1.745990in}}%
\pgfusepath{clip}%
\pgfsetbuttcap%
\pgfsetroundjoin%
\definecolor{currentfill}{rgb}{0.003922,0.003922,0.003922}%
\pgfsetfillcolor{currentfill}%
\pgfsetfillopacity{0.900000}%
\pgfsetlinewidth{0.507862pt}%
\definecolor{currentstroke}{rgb}{1.000000,1.000000,1.000000}%
\pgfsetstrokecolor{currentstroke}%
\pgfsetstrokeopacity{0.900000}%
\pgfsetdash{}{0pt}%
\pgfpathmoveto{\pgfqpoint{0.803143in}{0.990230in}}%
\pgfpathcurveto{\pgfqpoint{0.814790in}{0.990230in}}{\pgfqpoint{0.825963in}{0.994858in}}{\pgfqpoint{0.834199in}{1.003094in}}%
\pgfpathcurveto{\pgfqpoint{0.842435in}{1.011330in}}{\pgfqpoint{0.847063in}{1.022503in}}{\pgfqpoint{0.847063in}{1.034151in}}%
\pgfpathcurveto{\pgfqpoint{0.847063in}{1.045798in}}{\pgfqpoint{0.842435in}{1.056971in}}{\pgfqpoint{0.834199in}{1.065207in}}%
\pgfpathcurveto{\pgfqpoint{0.825963in}{1.073443in}}{\pgfqpoint{0.814790in}{1.078071in}}{\pgfqpoint{0.803143in}{1.078071in}}%
\pgfpathcurveto{\pgfqpoint{0.791495in}{1.078071in}}{\pgfqpoint{0.780322in}{1.073443in}}{\pgfqpoint{0.772086in}{1.065207in}}%
\pgfpathcurveto{\pgfqpoint{0.763850in}{1.056971in}}{\pgfqpoint{0.759222in}{1.045798in}}{\pgfqpoint{0.759222in}{1.034151in}}%
\pgfpathcurveto{\pgfqpoint{0.759222in}{1.022503in}}{\pgfqpoint{0.763850in}{1.011330in}}{\pgfqpoint{0.772086in}{1.003094in}}%
\pgfpathcurveto{\pgfqpoint{0.780322in}{0.994858in}}{\pgfqpoint{0.791495in}{0.990230in}}{\pgfqpoint{0.803143in}{0.990230in}}%
\pgfpathlineto{\pgfqpoint{0.803143in}{0.990230in}}%
\pgfpathclose%
\pgfusepath{stroke,fill}%
\end{pgfscope}%
\begin{pgfscope}%
\pgfpathrectangle{\pgfqpoint{0.563297in}{0.529443in}}{\pgfqpoint{1.636659in}{1.745990in}}%
\pgfusepath{clip}%
\pgfsetbuttcap%
\pgfsetroundjoin%
\definecolor{currentfill}{rgb}{0.003922,0.003922,0.003922}%
\pgfsetfillcolor{currentfill}%
\pgfsetfillopacity{0.900000}%
\pgfsetlinewidth{0.507862pt}%
\definecolor{currentstroke}{rgb}{1.000000,1.000000,1.000000}%
\pgfsetstrokecolor{currentstroke}%
\pgfsetstrokeopacity{0.900000}%
\pgfsetdash{}{0pt}%
\pgfpathmoveto{\pgfqpoint{1.648506in}{1.740302in}}%
\pgfpathcurveto{\pgfqpoint{1.660154in}{1.740302in}}{\pgfqpoint{1.671326in}{1.744930in}}{\pgfqpoint{1.679562in}{1.753167in}}%
\pgfpathcurveto{\pgfqpoint{1.687799in}{1.761403in}}{\pgfqpoint{1.692426in}{1.772575in}}{\pgfqpoint{1.692426in}{1.784223in}}%
\pgfpathcurveto{\pgfqpoint{1.692426in}{1.795871in}}{\pgfqpoint{1.687799in}{1.807043in}}{\pgfqpoint{1.679562in}{1.815280in}}%
\pgfpathcurveto{\pgfqpoint{1.671326in}{1.823516in}}{\pgfqpoint{1.660154in}{1.828144in}}{\pgfqpoint{1.648506in}{1.828144in}}%
\pgfpathcurveto{\pgfqpoint{1.636858in}{1.828144in}}{\pgfqpoint{1.625686in}{1.823516in}}{\pgfqpoint{1.617449in}{1.815280in}}%
\pgfpathcurveto{\pgfqpoint{1.609213in}{1.807043in}}{\pgfqpoint{1.604585in}{1.795871in}}{\pgfqpoint{1.604585in}{1.784223in}}%
\pgfpathcurveto{\pgfqpoint{1.604585in}{1.772575in}}{\pgfqpoint{1.609213in}{1.761403in}}{\pgfqpoint{1.617449in}{1.753167in}}%
\pgfpathcurveto{\pgfqpoint{1.625686in}{1.744930in}}{\pgfqpoint{1.636858in}{1.740302in}}{\pgfqpoint{1.648506in}{1.740302in}}%
\pgfpathlineto{\pgfqpoint{1.648506in}{1.740302in}}%
\pgfpathclose%
\pgfusepath{stroke,fill}%
\end{pgfscope}%
\begin{pgfscope}%
\pgfpathrectangle{\pgfqpoint{0.563297in}{0.529443in}}{\pgfqpoint{1.636659in}{1.745990in}}%
\pgfusepath{clip}%
\pgfsetbuttcap%
\pgfsetroundjoin%
\definecolor{currentfill}{rgb}{0.003922,0.003922,0.003922}%
\pgfsetfillcolor{currentfill}%
\pgfsetfillopacity{0.900000}%
\pgfsetlinewidth{0.507862pt}%
\definecolor{currentstroke}{rgb}{1.000000,1.000000,1.000000}%
\pgfsetstrokecolor{currentstroke}%
\pgfsetstrokeopacity{0.900000}%
\pgfsetdash{}{0pt}%
\pgfpathmoveto{\pgfqpoint{0.729061in}{0.892299in}}%
\pgfpathcurveto{\pgfqpoint{0.740709in}{0.892299in}}{\pgfqpoint{0.751882in}{0.896926in}}{\pgfqpoint{0.760118in}{0.905163in}}%
\pgfpathcurveto{\pgfqpoint{0.768354in}{0.913399in}}{\pgfqpoint{0.772982in}{0.924571in}}{\pgfqpoint{0.772982in}{0.936219in}}%
\pgfpathcurveto{\pgfqpoint{0.772982in}{0.947867in}}{\pgfqpoint{0.768354in}{0.959039in}}{\pgfqpoint{0.760118in}{0.967276in}}%
\pgfpathcurveto{\pgfqpoint{0.751882in}{0.975512in}}{\pgfqpoint{0.740709in}{0.980140in}}{\pgfqpoint{0.729061in}{0.980140in}}%
\pgfpathcurveto{\pgfqpoint{0.717413in}{0.980140in}}{\pgfqpoint{0.706241in}{0.975512in}}{\pgfqpoint{0.698005in}{0.967276in}}%
\pgfpathcurveto{\pgfqpoint{0.689769in}{0.959039in}}{\pgfqpoint{0.685141in}{0.947867in}}{\pgfqpoint{0.685141in}{0.936219in}}%
\pgfpathcurveto{\pgfqpoint{0.685141in}{0.924571in}}{\pgfqpoint{0.689769in}{0.913399in}}{\pgfqpoint{0.698005in}{0.905163in}}%
\pgfpathcurveto{\pgfqpoint{0.706241in}{0.896926in}}{\pgfqpoint{0.717413in}{0.892299in}}{\pgfqpoint{0.729061in}{0.892299in}}%
\pgfpathlineto{\pgfqpoint{0.729061in}{0.892299in}}%
\pgfpathclose%
\pgfusepath{stroke,fill}%
\end{pgfscope}%
\begin{pgfscope}%
\pgfpathrectangle{\pgfqpoint{0.563297in}{0.529443in}}{\pgfqpoint{1.636659in}{1.745990in}}%
\pgfusepath{clip}%
\pgfsetbuttcap%
\pgfsetroundjoin%
\definecolor{currentfill}{rgb}{0.003922,0.003922,0.003922}%
\pgfsetfillcolor{currentfill}%
\pgfsetfillopacity{0.900000}%
\pgfsetlinewidth{0.507862pt}%
\definecolor{currentstroke}{rgb}{1.000000,1.000000,1.000000}%
\pgfsetstrokecolor{currentstroke}%
\pgfsetstrokeopacity{0.900000}%
\pgfsetdash{}{0pt}%
\pgfpathmoveto{\pgfqpoint{1.506715in}{1.544508in}}%
\pgfpathcurveto{\pgfqpoint{1.518363in}{1.544508in}}{\pgfqpoint{1.529535in}{1.549136in}}{\pgfqpoint{1.537772in}{1.557372in}}%
\pgfpathcurveto{\pgfqpoint{1.546008in}{1.565608in}}{\pgfqpoint{1.550636in}{1.576781in}}{\pgfqpoint{1.550636in}{1.588428in}}%
\pgfpathcurveto{\pgfqpoint{1.550636in}{1.600076in}}{\pgfqpoint{1.546008in}{1.611249in}}{\pgfqpoint{1.537772in}{1.619485in}}%
\pgfpathcurveto{\pgfqpoint{1.529535in}{1.627721in}}{\pgfqpoint{1.518363in}{1.632349in}}{\pgfqpoint{1.506715in}{1.632349in}}%
\pgfpathcurveto{\pgfqpoint{1.495067in}{1.632349in}}{\pgfqpoint{1.483895in}{1.627721in}}{\pgfqpoint{1.475659in}{1.619485in}}%
\pgfpathcurveto{\pgfqpoint{1.467422in}{1.611249in}}{\pgfqpoint{1.462795in}{1.600076in}}{\pgfqpoint{1.462795in}{1.588428in}}%
\pgfpathcurveto{\pgfqpoint{1.462795in}{1.576781in}}{\pgfqpoint{1.467422in}{1.565608in}}{\pgfqpoint{1.475659in}{1.557372in}}%
\pgfpathcurveto{\pgfqpoint{1.483895in}{1.549136in}}{\pgfqpoint{1.495067in}{1.544508in}}{\pgfqpoint{1.506715in}{1.544508in}}%
\pgfpathlineto{\pgfqpoint{1.506715in}{1.544508in}}%
\pgfpathclose%
\pgfusepath{stroke,fill}%
\end{pgfscope}%
\begin{pgfscope}%
\pgfpathrectangle{\pgfqpoint{0.563297in}{0.529443in}}{\pgfqpoint{1.636659in}{1.745990in}}%
\pgfusepath{clip}%
\pgfsetbuttcap%
\pgfsetroundjoin%
\definecolor{currentfill}{rgb}{0.003922,0.003922,0.003922}%
\pgfsetfillcolor{currentfill}%
\pgfsetfillopacity{0.900000}%
\pgfsetlinewidth{0.507862pt}%
\definecolor{currentstroke}{rgb}{1.000000,1.000000,1.000000}%
\pgfsetstrokecolor{currentstroke}%
\pgfsetstrokeopacity{0.900000}%
\pgfsetdash{}{0pt}%
\pgfpathmoveto{\pgfqpoint{0.718473in}{0.564886in}}%
\pgfpathcurveto{\pgfqpoint{0.730121in}{0.564886in}}{\pgfqpoint{0.741294in}{0.569513in}}{\pgfqpoint{0.749530in}{0.577750in}}%
\pgfpathcurveto{\pgfqpoint{0.757766in}{0.585986in}}{\pgfqpoint{0.762394in}{0.597158in}}{\pgfqpoint{0.762394in}{0.608806in}}%
\pgfpathcurveto{\pgfqpoint{0.762394in}{0.620454in}}{\pgfqpoint{0.757766in}{0.631626in}}{\pgfqpoint{0.749530in}{0.639863in}}%
\pgfpathcurveto{\pgfqpoint{0.741294in}{0.648099in}}{\pgfqpoint{0.730121in}{0.652727in}}{\pgfqpoint{0.718473in}{0.652727in}}%
\pgfpathcurveto{\pgfqpoint{0.706826in}{0.652727in}}{\pgfqpoint{0.695653in}{0.648099in}}{\pgfqpoint{0.687417in}{0.639863in}}%
\pgfpathcurveto{\pgfqpoint{0.679181in}{0.631626in}}{\pgfqpoint{0.674553in}{0.620454in}}{\pgfqpoint{0.674553in}{0.608806in}}%
\pgfpathcurveto{\pgfqpoint{0.674553in}{0.597158in}}{\pgfqpoint{0.679181in}{0.585986in}}{\pgfqpoint{0.687417in}{0.577750in}}%
\pgfpathcurveto{\pgfqpoint{0.695653in}{0.569513in}}{\pgfqpoint{0.706826in}{0.564886in}}{\pgfqpoint{0.718473in}{0.564886in}}%
\pgfpathlineto{\pgfqpoint{0.718473in}{0.564886in}}%
\pgfpathclose%
\pgfusepath{stroke,fill}%
\end{pgfscope}%
\begin{pgfscope}%
\pgfpathrectangle{\pgfqpoint{0.563297in}{0.529443in}}{\pgfqpoint{1.636659in}{1.745990in}}%
\pgfusepath{clip}%
\pgfsetbuttcap%
\pgfsetroundjoin%
\definecolor{currentfill}{rgb}{0.003922,0.003922,0.003922}%
\pgfsetfillcolor{currentfill}%
\pgfsetfillopacity{0.900000}%
\pgfsetlinewidth{0.507862pt}%
\definecolor{currentstroke}{rgb}{1.000000,1.000000,1.000000}%
\pgfsetstrokecolor{currentstroke}%
\pgfsetstrokeopacity{0.900000}%
\pgfsetdash{}{0pt}%
\pgfpathmoveto{\pgfqpoint{1.720188in}{1.907224in}}%
\pgfpathcurveto{\pgfqpoint{1.731836in}{1.907224in}}{\pgfqpoint{1.743008in}{1.911852in}}{\pgfqpoint{1.751244in}{1.920088in}}%
\pgfpathcurveto{\pgfqpoint{1.759480in}{1.928324in}}{\pgfqpoint{1.764108in}{1.939497in}}{\pgfqpoint{1.764108in}{1.951145in}}%
\pgfpathcurveto{\pgfqpoint{1.764108in}{1.962793in}}{\pgfqpoint{1.759480in}{1.973965in}}{\pgfqpoint{1.751244in}{1.982201in}}%
\pgfpathcurveto{\pgfqpoint{1.743008in}{1.990437in}}{\pgfqpoint{1.731836in}{1.995065in}}{\pgfqpoint{1.720188in}{1.995065in}}%
\pgfpathcurveto{\pgfqpoint{1.708540in}{1.995065in}}{\pgfqpoint{1.697367in}{1.990437in}}{\pgfqpoint{1.689131in}{1.982201in}}%
\pgfpathcurveto{\pgfqpoint{1.680895in}{1.973965in}}{\pgfqpoint{1.676267in}{1.962793in}}{\pgfqpoint{1.676267in}{1.951145in}}%
\pgfpathcurveto{\pgfqpoint{1.676267in}{1.939497in}}{\pgfqpoint{1.680895in}{1.928324in}}{\pgfqpoint{1.689131in}{1.920088in}}%
\pgfpathcurveto{\pgfqpoint{1.697367in}{1.911852in}}{\pgfqpoint{1.708540in}{1.907224in}}{\pgfqpoint{1.720188in}{1.907224in}}%
\pgfpathlineto{\pgfqpoint{1.720188in}{1.907224in}}%
\pgfpathclose%
\pgfusepath{stroke,fill}%
\end{pgfscope}%
\begin{pgfscope}%
\pgfpathrectangle{\pgfqpoint{0.563297in}{0.529443in}}{\pgfqpoint{1.636659in}{1.745990in}}%
\pgfusepath{clip}%
\pgfsetbuttcap%
\pgfsetroundjoin%
\definecolor{currentfill}{rgb}{0.003922,0.003922,0.003922}%
\pgfsetfillcolor{currentfill}%
\pgfsetfillopacity{0.900000}%
\pgfsetlinewidth{0.507862pt}%
\definecolor{currentstroke}{rgb}{1.000000,1.000000,1.000000}%
\pgfsetstrokecolor{currentstroke}%
\pgfsetstrokeopacity{0.900000}%
\pgfsetdash{}{0pt}%
\pgfpathmoveto{\pgfqpoint{1.172409in}{1.410284in}}%
\pgfpathcurveto{\pgfqpoint{1.184057in}{1.410284in}}{\pgfqpoint{1.195229in}{1.414911in}}{\pgfqpoint{1.203465in}{1.423148in}}%
\pgfpathcurveto{\pgfqpoint{1.211701in}{1.431384in}}{\pgfqpoint{1.216329in}{1.442556in}}{\pgfqpoint{1.216329in}{1.454204in}}%
\pgfpathcurveto{\pgfqpoint{1.216329in}{1.465852in}}{\pgfqpoint{1.211701in}{1.477024in}}{\pgfqpoint{1.203465in}{1.485261in}}%
\pgfpathcurveto{\pgfqpoint{1.195229in}{1.493497in}}{\pgfqpoint{1.184057in}{1.498125in}}{\pgfqpoint{1.172409in}{1.498125in}}%
\pgfpathcurveto{\pgfqpoint{1.160761in}{1.498125in}}{\pgfqpoint{1.149588in}{1.493497in}}{\pgfqpoint{1.141352in}{1.485261in}}%
\pgfpathcurveto{\pgfqpoint{1.133116in}{1.477024in}}{\pgfqpoint{1.128488in}{1.465852in}}{\pgfqpoint{1.128488in}{1.454204in}}%
\pgfpathcurveto{\pgfqpoint{1.128488in}{1.442556in}}{\pgfqpoint{1.133116in}{1.431384in}}{\pgfqpoint{1.141352in}{1.423148in}}%
\pgfpathcurveto{\pgfqpoint{1.149588in}{1.414911in}}{\pgfqpoint{1.160761in}{1.410284in}}{\pgfqpoint{1.172409in}{1.410284in}}%
\pgfpathlineto{\pgfqpoint{1.172409in}{1.410284in}}%
\pgfpathclose%
\pgfusepath{stroke,fill}%
\end{pgfscope}%
\begin{pgfscope}%
\pgfpathrectangle{\pgfqpoint{0.563297in}{0.529443in}}{\pgfqpoint{1.636659in}{1.745990in}}%
\pgfusepath{clip}%
\pgfsetbuttcap%
\pgfsetroundjoin%
\definecolor{currentfill}{rgb}{0.031373,0.627451,0.913725}%
\pgfsetfillcolor{currentfill}%
\pgfsetfillopacity{0.900000}%
\pgfsetlinewidth{0.507862pt}%
\definecolor{currentstroke}{rgb}{1.000000,1.000000,1.000000}%
\pgfsetstrokecolor{currentstroke}%
\pgfsetstrokeopacity{0.900000}%
\pgfsetdash{}{0pt}%
\pgfpathmoveto{\pgfqpoint{0.815726in}{1.119557in}}%
\pgfpathcurveto{\pgfqpoint{0.827374in}{1.119557in}}{\pgfqpoint{0.838546in}{1.124185in}}{\pgfqpoint{0.846782in}{1.132421in}}%
\pgfpathcurveto{\pgfqpoint{0.855019in}{1.140658in}}{\pgfqpoint{0.859646in}{1.151830in}}{\pgfqpoint{0.859646in}{1.163478in}}%
\pgfpathcurveto{\pgfqpoint{0.859646in}{1.175126in}}{\pgfqpoint{0.855019in}{1.186298in}}{\pgfqpoint{0.846782in}{1.194534in}}%
\pgfpathcurveto{\pgfqpoint{0.838546in}{1.202771in}}{\pgfqpoint{0.827374in}{1.207399in}}{\pgfqpoint{0.815726in}{1.207399in}}%
\pgfpathcurveto{\pgfqpoint{0.804078in}{1.207399in}}{\pgfqpoint{0.792906in}{1.202771in}}{\pgfqpoint{0.784669in}{1.194534in}}%
\pgfpathcurveto{\pgfqpoint{0.776433in}{1.186298in}}{\pgfqpoint{0.771805in}{1.175126in}}{\pgfqpoint{0.771805in}{1.163478in}}%
\pgfpathcurveto{\pgfqpoint{0.771805in}{1.151830in}}{\pgfqpoint{0.776433in}{1.140658in}}{\pgfqpoint{0.784669in}{1.132421in}}%
\pgfpathcurveto{\pgfqpoint{0.792906in}{1.124185in}}{\pgfqpoint{0.804078in}{1.119557in}}{\pgfqpoint{0.815726in}{1.119557in}}%
\pgfpathlineto{\pgfqpoint{0.815726in}{1.119557in}}%
\pgfpathclose%
\pgfusepath{stroke,fill}%
\end{pgfscope}%
\begin{pgfscope}%
\pgfpathrectangle{\pgfqpoint{0.563297in}{0.529443in}}{\pgfqpoint{1.636659in}{1.745990in}}%
\pgfusepath{clip}%
\pgfsetbuttcap%
\pgfsetroundjoin%
\definecolor{currentfill}{rgb}{0.031373,0.627451,0.913725}%
\pgfsetfillcolor{currentfill}%
\pgfsetfillopacity{0.900000}%
\pgfsetlinewidth{0.507862pt}%
\definecolor{currentstroke}{rgb}{1.000000,1.000000,1.000000}%
\pgfsetstrokecolor{currentstroke}%
\pgfsetstrokeopacity{0.900000}%
\pgfsetdash{}{0pt}%
\pgfpathmoveto{\pgfqpoint{0.786232in}{0.735174in}}%
\pgfpathcurveto{\pgfqpoint{0.797880in}{0.735174in}}{\pgfqpoint{0.809052in}{0.739802in}}{\pgfqpoint{0.817288in}{0.748038in}}%
\pgfpathcurveto{\pgfqpoint{0.825524in}{0.756274in}}{\pgfqpoint{0.830152in}{0.767447in}}{\pgfqpoint{0.830152in}{0.779094in}}%
\pgfpathcurveto{\pgfqpoint{0.830152in}{0.790742in}}{\pgfqpoint{0.825524in}{0.801915in}}{\pgfqpoint{0.817288in}{0.810151in}}%
\pgfpathcurveto{\pgfqpoint{0.809052in}{0.818387in}}{\pgfqpoint{0.797880in}{0.823015in}}{\pgfqpoint{0.786232in}{0.823015in}}%
\pgfpathcurveto{\pgfqpoint{0.774584in}{0.823015in}}{\pgfqpoint{0.763411in}{0.818387in}}{\pgfqpoint{0.755175in}{0.810151in}}%
\pgfpathcurveto{\pgfqpoint{0.746939in}{0.801915in}}{\pgfqpoint{0.742311in}{0.790742in}}{\pgfqpoint{0.742311in}{0.779094in}}%
\pgfpathcurveto{\pgfqpoint{0.742311in}{0.767447in}}{\pgfqpoint{0.746939in}{0.756274in}}{\pgfqpoint{0.755175in}{0.748038in}}%
\pgfpathcurveto{\pgfqpoint{0.763411in}{0.739802in}}{\pgfqpoint{0.774584in}{0.735174in}}{\pgfqpoint{0.786232in}{0.735174in}}%
\pgfpathlineto{\pgfqpoint{0.786232in}{0.735174in}}%
\pgfpathclose%
\pgfusepath{stroke,fill}%
\end{pgfscope}%
\begin{pgfscope}%
\pgfpathrectangle{\pgfqpoint{0.563297in}{0.529443in}}{\pgfqpoint{1.636659in}{1.745990in}}%
\pgfusepath{clip}%
\pgfsetbuttcap%
\pgfsetroundjoin%
\definecolor{currentfill}{rgb}{0.031373,0.627451,0.913725}%
\pgfsetfillcolor{currentfill}%
\pgfsetfillopacity{0.900000}%
\pgfsetlinewidth{0.507862pt}%
\definecolor{currentstroke}{rgb}{1.000000,1.000000,1.000000}%
\pgfsetstrokecolor{currentstroke}%
\pgfsetstrokeopacity{0.900000}%
\pgfsetdash{}{0pt}%
\pgfpathmoveto{\pgfqpoint{1.666928in}{2.096160in}}%
\pgfpathcurveto{\pgfqpoint{1.678576in}{2.096160in}}{\pgfqpoint{1.689748in}{2.100787in}}{\pgfqpoint{1.697985in}{2.109024in}}%
\pgfpathcurveto{\pgfqpoint{1.706221in}{2.117260in}}{\pgfqpoint{1.710849in}{2.128432in}}{\pgfqpoint{1.710849in}{2.140080in}}%
\pgfpathcurveto{\pgfqpoint{1.710849in}{2.151728in}}{\pgfqpoint{1.706221in}{2.162900in}}{\pgfqpoint{1.697985in}{2.171137in}}%
\pgfpathcurveto{\pgfqpoint{1.689748in}{2.179373in}}{\pgfqpoint{1.678576in}{2.184001in}}{\pgfqpoint{1.666928in}{2.184001in}}%
\pgfpathcurveto{\pgfqpoint{1.655280in}{2.184001in}}{\pgfqpoint{1.644108in}{2.179373in}}{\pgfqpoint{1.635872in}{2.171137in}}%
\pgfpathcurveto{\pgfqpoint{1.627635in}{2.162900in}}{\pgfqpoint{1.623008in}{2.151728in}}{\pgfqpoint{1.623008in}{2.140080in}}%
\pgfpathcurveto{\pgfqpoint{1.623008in}{2.128432in}}{\pgfqpoint{1.627635in}{2.117260in}}{\pgfqpoint{1.635872in}{2.109024in}}%
\pgfpathcurveto{\pgfqpoint{1.644108in}{2.100787in}}{\pgfqpoint{1.655280in}{2.096160in}}{\pgfqpoint{1.666928in}{2.096160in}}%
\pgfpathlineto{\pgfqpoint{1.666928in}{2.096160in}}%
\pgfpathclose%
\pgfusepath{stroke,fill}%
\end{pgfscope}%
\begin{pgfscope}%
\pgfpathrectangle{\pgfqpoint{0.563297in}{0.529443in}}{\pgfqpoint{1.636659in}{1.745990in}}%
\pgfusepath{clip}%
\pgfsetbuttcap%
\pgfsetroundjoin%
\definecolor{currentfill}{rgb}{0.031373,0.627451,0.913725}%
\pgfsetfillcolor{currentfill}%
\pgfsetfillopacity{0.900000}%
\pgfsetlinewidth{0.507862pt}%
\definecolor{currentstroke}{rgb}{1.000000,1.000000,1.000000}%
\pgfsetstrokecolor{currentstroke}%
\pgfsetstrokeopacity{0.900000}%
\pgfsetdash{}{0pt}%
\pgfpathmoveto{\pgfqpoint{1.517631in}{2.005405in}}%
\pgfpathcurveto{\pgfqpoint{1.529279in}{2.005405in}}{\pgfqpoint{1.540451in}{2.010032in}}{\pgfqpoint{1.548688in}{2.018269in}}%
\pgfpathcurveto{\pgfqpoint{1.556924in}{2.026505in}}{\pgfqpoint{1.561552in}{2.037677in}}{\pgfqpoint{1.561552in}{2.049325in}}%
\pgfpathcurveto{\pgfqpoint{1.561552in}{2.060973in}}{\pgfqpoint{1.556924in}{2.072145in}}{\pgfqpoint{1.548688in}{2.080382in}}%
\pgfpathcurveto{\pgfqpoint{1.540451in}{2.088618in}}{\pgfqpoint{1.529279in}{2.093246in}}{\pgfqpoint{1.517631in}{2.093246in}}%
\pgfpathcurveto{\pgfqpoint{1.505983in}{2.093246in}}{\pgfqpoint{1.494811in}{2.088618in}}{\pgfqpoint{1.486575in}{2.080382in}}%
\pgfpathcurveto{\pgfqpoint{1.478338in}{2.072145in}}{\pgfqpoint{1.473711in}{2.060973in}}{\pgfqpoint{1.473711in}{2.049325in}}%
\pgfpathcurveto{\pgfqpoint{1.473711in}{2.037677in}}{\pgfqpoint{1.478338in}{2.026505in}}{\pgfqpoint{1.486575in}{2.018269in}}%
\pgfpathcurveto{\pgfqpoint{1.494811in}{2.010032in}}{\pgfqpoint{1.505983in}{2.005405in}}{\pgfqpoint{1.517631in}{2.005405in}}%
\pgfpathlineto{\pgfqpoint{1.517631in}{2.005405in}}%
\pgfpathclose%
\pgfusepath{stroke,fill}%
\end{pgfscope}%
\begin{pgfscope}%
\pgfpathrectangle{\pgfqpoint{0.563297in}{0.529443in}}{\pgfqpoint{1.636659in}{1.745990in}}%
\pgfusepath{clip}%
\pgfsetbuttcap%
\pgfsetroundjoin%
\definecolor{currentfill}{rgb}{0.031373,0.627451,0.913725}%
\pgfsetfillcolor{currentfill}%
\pgfsetfillopacity{0.900000}%
\pgfsetlinewidth{0.507862pt}%
\definecolor{currentstroke}{rgb}{1.000000,1.000000,1.000000}%
\pgfsetstrokecolor{currentstroke}%
\pgfsetstrokeopacity{0.900000}%
\pgfsetdash{}{0pt}%
\pgfpathmoveto{\pgfqpoint{1.752512in}{2.152149in}}%
\pgfpathcurveto{\pgfqpoint{1.764160in}{2.152149in}}{\pgfqpoint{1.775332in}{2.156777in}}{\pgfqpoint{1.783568in}{2.165013in}}%
\pgfpathcurveto{\pgfqpoint{1.791804in}{2.173249in}}{\pgfqpoint{1.796432in}{2.184422in}}{\pgfqpoint{1.796432in}{2.196069in}}%
\pgfpathcurveto{\pgfqpoint{1.796432in}{2.207717in}}{\pgfqpoint{1.791804in}{2.218890in}}{\pgfqpoint{1.783568in}{2.227126in}}%
\pgfpathcurveto{\pgfqpoint{1.775332in}{2.235362in}}{\pgfqpoint{1.764160in}{2.239990in}}{\pgfqpoint{1.752512in}{2.239990in}}%
\pgfpathcurveto{\pgfqpoint{1.740864in}{2.239990in}}{\pgfqpoint{1.729691in}{2.235362in}}{\pgfqpoint{1.721455in}{2.227126in}}%
\pgfpathcurveto{\pgfqpoint{1.713219in}{2.218890in}}{\pgfqpoint{1.708591in}{2.207717in}}{\pgfqpoint{1.708591in}{2.196069in}}%
\pgfpathcurveto{\pgfqpoint{1.708591in}{2.184422in}}{\pgfqpoint{1.713219in}{2.173249in}}{\pgfqpoint{1.721455in}{2.165013in}}%
\pgfpathcurveto{\pgfqpoint{1.729691in}{2.156777in}}{\pgfqpoint{1.740864in}{2.152149in}}{\pgfqpoint{1.752512in}{2.152149in}}%
\pgfpathlineto{\pgfqpoint{1.752512in}{2.152149in}}%
\pgfpathclose%
\pgfusepath{stroke,fill}%
\end{pgfscope}%
\begin{pgfscope}%
\pgfpathrectangle{\pgfqpoint{0.563297in}{0.529443in}}{\pgfqpoint{1.636659in}{1.745990in}}%
\pgfusepath{clip}%
\pgfsetbuttcap%
\pgfsetroundjoin%
\definecolor{currentfill}{rgb}{0.031373,0.627451,0.913725}%
\pgfsetfillcolor{currentfill}%
\pgfsetfillopacity{0.900000}%
\pgfsetlinewidth{0.507862pt}%
\definecolor{currentstroke}{rgb}{1.000000,1.000000,1.000000}%
\pgfsetstrokecolor{currentstroke}%
\pgfsetstrokeopacity{0.900000}%
\pgfsetdash{}{0pt}%
\pgfpathmoveto{\pgfqpoint{1.772861in}{1.957439in}}%
\pgfpathcurveto{\pgfqpoint{1.784508in}{1.957439in}}{\pgfqpoint{1.795681in}{1.962067in}}{\pgfqpoint{1.803917in}{1.970303in}}%
\pgfpathcurveto{\pgfqpoint{1.812153in}{1.978539in}}{\pgfqpoint{1.816781in}{1.989712in}}{\pgfqpoint{1.816781in}{2.001360in}}%
\pgfpathcurveto{\pgfqpoint{1.816781in}{2.013008in}}{\pgfqpoint{1.812153in}{2.024180in}}{\pgfqpoint{1.803917in}{2.032416in}}%
\pgfpathcurveto{\pgfqpoint{1.795681in}{2.040652in}}{\pgfqpoint{1.784508in}{2.045280in}}{\pgfqpoint{1.772861in}{2.045280in}}%
\pgfpathcurveto{\pgfqpoint{1.761213in}{2.045280in}}{\pgfqpoint{1.750040in}{2.040652in}}{\pgfqpoint{1.741804in}{2.032416in}}%
\pgfpathcurveto{\pgfqpoint{1.733568in}{2.024180in}}{\pgfqpoint{1.728940in}{2.013008in}}{\pgfqpoint{1.728940in}{2.001360in}}%
\pgfpathcurveto{\pgfqpoint{1.728940in}{1.989712in}}{\pgfqpoint{1.733568in}{1.978539in}}{\pgfqpoint{1.741804in}{1.970303in}}%
\pgfpathcurveto{\pgfqpoint{1.750040in}{1.962067in}}{\pgfqpoint{1.761213in}{1.957439in}}{\pgfqpoint{1.772861in}{1.957439in}}%
\pgfpathlineto{\pgfqpoint{1.772861in}{1.957439in}}%
\pgfpathclose%
\pgfusepath{stroke,fill}%
\end{pgfscope}%
\begin{pgfscope}%
\pgfsetrectcap%
\pgfsetmiterjoin%
\pgfsetlinewidth{1.254687pt}%
\definecolor{currentstroke}{rgb}{0.800000,0.800000,0.800000}%
\pgfsetstrokecolor{currentstroke}%
\pgfsetdash{}{0pt}%
\pgfpathmoveto{\pgfqpoint{0.563297in}{0.529443in}}%
\pgfpathlineto{\pgfqpoint{0.563297in}{2.275433in}}%
\pgfusepath{stroke}%
\end{pgfscope}%
\begin{pgfscope}%
\pgfsetrectcap%
\pgfsetmiterjoin%
\pgfsetlinewidth{1.254687pt}%
\definecolor{currentstroke}{rgb}{0.800000,0.800000,0.800000}%
\pgfsetstrokecolor{currentstroke}%
\pgfsetdash{}{0pt}%
\pgfpathmoveto{\pgfqpoint{2.199956in}{0.529443in}}%
\pgfpathlineto{\pgfqpoint{2.199956in}{2.275433in}}%
\pgfusepath{stroke}%
\end{pgfscope}%
\begin{pgfscope}%
\pgfsetrectcap%
\pgfsetmiterjoin%
\pgfsetlinewidth{1.254687pt}%
\definecolor{currentstroke}{rgb}{0.800000,0.800000,0.800000}%
\pgfsetstrokecolor{currentstroke}%
\pgfsetdash{}{0pt}%
\pgfpathmoveto{\pgfqpoint{0.563297in}{0.529443in}}%
\pgfpathlineto{\pgfqpoint{2.199956in}{0.529443in}}%
\pgfusepath{stroke}%
\end{pgfscope}%
\begin{pgfscope}%
\pgfsetrectcap%
\pgfsetmiterjoin%
\pgfsetlinewidth{1.254687pt}%
\definecolor{currentstroke}{rgb}{0.800000,0.800000,0.800000}%
\pgfsetstrokecolor{currentstroke}%
\pgfsetdash{}{0pt}%
\pgfpathmoveto{\pgfqpoint{0.563297in}{2.275433in}}%
\pgfpathlineto{\pgfqpoint{2.199956in}{2.275433in}}%
\pgfusepath{stroke}%
\end{pgfscope}%
\begin{pgfscope}%
\definecolor{textcolor}{rgb}{0.150000,0.150000,0.150000}%
\pgfsetstrokecolor{textcolor}%
\pgfsetfillcolor{textcolor}%
\pgftext[x=1.381627in,y=2.358766in,,base]{\color{textcolor}{\rmfamily\fontsize{11.000000}{13.200000}\selectfont\catcode`\^=\active\def^{\ifmmode\sp\else\^{}\fi}\catcode`\%=\active\def%{\%}Graph2Vec}}%
\end{pgfscope}%
\begin{pgfscope}%
\pgfsetbuttcap%
\pgfsetmiterjoin%
\definecolor{currentfill}{rgb}{1.000000,1.000000,1.000000}%
\pgfsetfillcolor{currentfill}%
\pgfsetlinewidth{0.000000pt}%
\definecolor{currentstroke}{rgb}{0.000000,0.000000,0.000000}%
\pgfsetstrokecolor{currentstroke}%
\pgfsetstrokeopacity{0.000000}%
\pgfsetdash{}{0pt}%
\pgfpathmoveto{\pgfqpoint{2.717318in}{0.529443in}}%
\pgfpathlineto{\pgfqpoint{4.353977in}{0.529443in}}%
\pgfpathlineto{\pgfqpoint{4.353977in}{2.275433in}}%
\pgfpathlineto{\pgfqpoint{2.717318in}{2.275433in}}%
\pgfpathlineto{\pgfqpoint{2.717318in}{0.529443in}}%
\pgfpathclose%
\pgfusepath{fill}%
\end{pgfscope}%
\begin{pgfscope}%
\pgfpathrectangle{\pgfqpoint{2.717318in}{0.529443in}}{\pgfqpoint{1.636659in}{1.745990in}}%
\pgfusepath{clip}%
\pgfsetroundcap%
\pgfsetroundjoin%
\pgfsetlinewidth{1.003750pt}%
\definecolor{currentstroke}{rgb}{0.800000,0.800000,0.800000}%
\pgfsetstrokecolor{currentstroke}%
\pgfsetdash{}{0pt}%
\pgfpathmoveto{\pgfqpoint{2.761785in}{0.529443in}}%
\pgfpathlineto{\pgfqpoint{2.761785in}{2.275433in}}%
\pgfusepath{stroke}%
\end{pgfscope}%
\begin{pgfscope}%
\definecolor{textcolor}{rgb}{0.150000,0.150000,0.150000}%
\pgfsetstrokecolor{textcolor}%
\pgfsetfillcolor{textcolor}%
\pgftext[x=2.761785in,y=0.397499in,,top]{\color{textcolor}{\rmfamily\fontsize{8.000000}{9.600000}\selectfont\catcode`\^=\active\def^{\ifmmode\sp\else\^{}\fi}\catcode`\%=\active\def%{\%}0}}%
\end{pgfscope}%
\begin{pgfscope}%
\pgfpathrectangle{\pgfqpoint{2.717318in}{0.529443in}}{\pgfqpoint{1.636659in}{1.745990in}}%
\pgfusepath{clip}%
\pgfsetroundcap%
\pgfsetroundjoin%
\pgfsetlinewidth{1.003750pt}%
\definecolor{currentstroke}{rgb}{0.800000,0.800000,0.800000}%
\pgfsetstrokecolor{currentstroke}%
\pgfsetdash{}{0pt}%
\pgfpathmoveto{\pgfqpoint{3.172727in}{0.529443in}}%
\pgfpathlineto{\pgfqpoint{3.172727in}{2.275433in}}%
\pgfusepath{stroke}%
\end{pgfscope}%
\begin{pgfscope}%
\definecolor{textcolor}{rgb}{0.150000,0.150000,0.150000}%
\pgfsetstrokecolor{textcolor}%
\pgfsetfillcolor{textcolor}%
\pgftext[x=3.172727in,y=0.397499in,,top]{\color{textcolor}{\rmfamily\fontsize{8.000000}{9.600000}\selectfont\catcode`\^=\active\def^{\ifmmode\sp\else\^{}\fi}\catcode`\%=\active\def%{\%}10}}%
\end{pgfscope}%
\begin{pgfscope}%
\pgfpathrectangle{\pgfqpoint{2.717318in}{0.529443in}}{\pgfqpoint{1.636659in}{1.745990in}}%
\pgfusepath{clip}%
\pgfsetroundcap%
\pgfsetroundjoin%
\pgfsetlinewidth{1.003750pt}%
\definecolor{currentstroke}{rgb}{0.800000,0.800000,0.800000}%
\pgfsetstrokecolor{currentstroke}%
\pgfsetdash{}{0pt}%
\pgfpathmoveto{\pgfqpoint{3.583668in}{0.529443in}}%
\pgfpathlineto{\pgfqpoint{3.583668in}{2.275433in}}%
\pgfusepath{stroke}%
\end{pgfscope}%
\begin{pgfscope}%
\definecolor{textcolor}{rgb}{0.150000,0.150000,0.150000}%
\pgfsetstrokecolor{textcolor}%
\pgfsetfillcolor{textcolor}%
\pgftext[x=3.583668in,y=0.397499in,,top]{\color{textcolor}{\rmfamily\fontsize{8.000000}{9.600000}\selectfont\catcode`\^=\active\def^{\ifmmode\sp\else\^{}\fi}\catcode`\%=\active\def%{\%}20}}%
\end{pgfscope}%
\begin{pgfscope}%
\pgfpathrectangle{\pgfqpoint{2.717318in}{0.529443in}}{\pgfqpoint{1.636659in}{1.745990in}}%
\pgfusepath{clip}%
\pgfsetroundcap%
\pgfsetroundjoin%
\pgfsetlinewidth{1.003750pt}%
\definecolor{currentstroke}{rgb}{0.800000,0.800000,0.800000}%
\pgfsetstrokecolor{currentstroke}%
\pgfsetdash{}{0pt}%
\pgfpathmoveto{\pgfqpoint{3.994609in}{0.529443in}}%
\pgfpathlineto{\pgfqpoint{3.994609in}{2.275433in}}%
\pgfusepath{stroke}%
\end{pgfscope}%
\begin{pgfscope}%
\definecolor{textcolor}{rgb}{0.150000,0.150000,0.150000}%
\pgfsetstrokecolor{textcolor}%
\pgfsetfillcolor{textcolor}%
\pgftext[x=3.994609in,y=0.397499in,,top]{\color{textcolor}{\rmfamily\fontsize{8.000000}{9.600000}\selectfont\catcode`\^=\active\def^{\ifmmode\sp\else\^{}\fi}\catcode`\%=\active\def%{\%}30}}%
\end{pgfscope}%
\begin{pgfscope}%
\definecolor{textcolor}{rgb}{0.150000,0.150000,0.150000}%
\pgfsetstrokecolor{textcolor}%
\pgfsetfillcolor{textcolor}%
\pgftext[x=3.535647in,y=0.234413in,,top]{\color{textcolor}{\rmfamily\fontsize{10.000000}{12.000000}\selectfont\catcode`\^=\active\def^{\ifmmode\sp\else\^{}\fi}\catcode`\%=\active\def%{\%}UMAP 1}}%
\end{pgfscope}%
\begin{pgfscope}%
\pgfpathrectangle{\pgfqpoint{2.717318in}{0.529443in}}{\pgfqpoint{1.636659in}{1.745990in}}%
\pgfusepath{clip}%
\pgfsetroundcap%
\pgfsetroundjoin%
\pgfsetlinewidth{1.003750pt}%
\definecolor{currentstroke}{rgb}{0.800000,0.800000,0.800000}%
\pgfsetstrokecolor{currentstroke}%
\pgfsetdash{}{0pt}%
\pgfpathmoveto{\pgfqpoint{2.717318in}{0.700819in}}%
\pgfpathlineto{\pgfqpoint{4.353977in}{0.700819in}}%
\pgfusepath{stroke}%
\end{pgfscope}%
\begin{pgfscope}%
\definecolor{textcolor}{rgb}{0.150000,0.150000,0.150000}%
\pgfsetstrokecolor{textcolor}%
\pgfsetfillcolor{textcolor}%
\pgftext[x=2.408669in, y=0.658610in, left, base]{\color{textcolor}{\rmfamily\fontsize{8.000000}{9.600000}\selectfont\catcode`\^=\active\def^{\ifmmode\sp\else\^{}\fi}\catcode`\%=\active\def%{\%}7.5}}%
\end{pgfscope}%
\begin{pgfscope}%
\pgfpathrectangle{\pgfqpoint{2.717318in}{0.529443in}}{\pgfqpoint{1.636659in}{1.745990in}}%
\pgfusepath{clip}%
\pgfsetroundcap%
\pgfsetroundjoin%
\pgfsetlinewidth{1.003750pt}%
\definecolor{currentstroke}{rgb}{0.800000,0.800000,0.800000}%
\pgfsetstrokecolor{currentstroke}%
\pgfsetdash{}{0pt}%
\pgfpathmoveto{\pgfqpoint{2.717318in}{0.968182in}}%
\pgfpathlineto{\pgfqpoint{4.353977in}{0.968182in}}%
\pgfusepath{stroke}%
\end{pgfscope}%
\begin{pgfscope}%
\definecolor{textcolor}{rgb}{0.150000,0.150000,0.150000}%
\pgfsetstrokecolor{textcolor}%
\pgfsetfillcolor{textcolor}%
\pgftext[x=2.337977in, y=0.925973in, left, base]{\color{textcolor}{\rmfamily\fontsize{8.000000}{9.600000}\selectfont\catcode`\^=\active\def^{\ifmmode\sp\else\^{}\fi}\catcode`\%=\active\def%{\%}10.0}}%
\end{pgfscope}%
\begin{pgfscope}%
\pgfpathrectangle{\pgfqpoint{2.717318in}{0.529443in}}{\pgfqpoint{1.636659in}{1.745990in}}%
\pgfusepath{clip}%
\pgfsetroundcap%
\pgfsetroundjoin%
\pgfsetlinewidth{1.003750pt}%
\definecolor{currentstroke}{rgb}{0.800000,0.800000,0.800000}%
\pgfsetstrokecolor{currentstroke}%
\pgfsetdash{}{0pt}%
\pgfpathmoveto{\pgfqpoint{2.717318in}{1.235545in}}%
\pgfpathlineto{\pgfqpoint{4.353977in}{1.235545in}}%
\pgfusepath{stroke}%
\end{pgfscope}%
\begin{pgfscope}%
\definecolor{textcolor}{rgb}{0.150000,0.150000,0.150000}%
\pgfsetstrokecolor{textcolor}%
\pgfsetfillcolor{textcolor}%
\pgftext[x=2.337977in, y=1.193336in, left, base]{\color{textcolor}{\rmfamily\fontsize{8.000000}{9.600000}\selectfont\catcode`\^=\active\def^{\ifmmode\sp\else\^{}\fi}\catcode`\%=\active\def%{\%}12.5}}%
\end{pgfscope}%
\begin{pgfscope}%
\pgfpathrectangle{\pgfqpoint{2.717318in}{0.529443in}}{\pgfqpoint{1.636659in}{1.745990in}}%
\pgfusepath{clip}%
\pgfsetroundcap%
\pgfsetroundjoin%
\pgfsetlinewidth{1.003750pt}%
\definecolor{currentstroke}{rgb}{0.800000,0.800000,0.800000}%
\pgfsetstrokecolor{currentstroke}%
\pgfsetdash{}{0pt}%
\pgfpathmoveto{\pgfqpoint{2.717318in}{1.502908in}}%
\pgfpathlineto{\pgfqpoint{4.353977in}{1.502908in}}%
\pgfusepath{stroke}%
\end{pgfscope}%
\begin{pgfscope}%
\definecolor{textcolor}{rgb}{0.150000,0.150000,0.150000}%
\pgfsetstrokecolor{textcolor}%
\pgfsetfillcolor{textcolor}%
\pgftext[x=2.337977in, y=1.460699in, left, base]{\color{textcolor}{\rmfamily\fontsize{8.000000}{9.600000}\selectfont\catcode`\^=\active\def^{\ifmmode\sp\else\^{}\fi}\catcode`\%=\active\def%{\%}15.0}}%
\end{pgfscope}%
\begin{pgfscope}%
\pgfpathrectangle{\pgfqpoint{2.717318in}{0.529443in}}{\pgfqpoint{1.636659in}{1.745990in}}%
\pgfusepath{clip}%
\pgfsetroundcap%
\pgfsetroundjoin%
\pgfsetlinewidth{1.003750pt}%
\definecolor{currentstroke}{rgb}{0.800000,0.800000,0.800000}%
\pgfsetstrokecolor{currentstroke}%
\pgfsetdash{}{0pt}%
\pgfpathmoveto{\pgfqpoint{2.717318in}{1.770271in}}%
\pgfpathlineto{\pgfqpoint{4.353977in}{1.770271in}}%
\pgfusepath{stroke}%
\end{pgfscope}%
\begin{pgfscope}%
\definecolor{textcolor}{rgb}{0.150000,0.150000,0.150000}%
\pgfsetstrokecolor{textcolor}%
\pgfsetfillcolor{textcolor}%
\pgftext[x=2.337977in, y=1.728062in, left, base]{\color{textcolor}{\rmfamily\fontsize{8.000000}{9.600000}\selectfont\catcode`\^=\active\def^{\ifmmode\sp\else\^{}\fi}\catcode`\%=\active\def%{\%}17.5}}%
\end{pgfscope}%
\begin{pgfscope}%
\pgfpathrectangle{\pgfqpoint{2.717318in}{0.529443in}}{\pgfqpoint{1.636659in}{1.745990in}}%
\pgfusepath{clip}%
\pgfsetroundcap%
\pgfsetroundjoin%
\pgfsetlinewidth{1.003750pt}%
\definecolor{currentstroke}{rgb}{0.800000,0.800000,0.800000}%
\pgfsetstrokecolor{currentstroke}%
\pgfsetdash{}{0pt}%
\pgfpathmoveto{\pgfqpoint{2.717318in}{2.037634in}}%
\pgfpathlineto{\pgfqpoint{4.353977in}{2.037634in}}%
\pgfusepath{stroke}%
\end{pgfscope}%
\begin{pgfscope}%
\definecolor{textcolor}{rgb}{0.150000,0.150000,0.150000}%
\pgfsetstrokecolor{textcolor}%
\pgfsetfillcolor{textcolor}%
\pgftext[x=2.337977in, y=1.995425in, left, base]{\color{textcolor}{\rmfamily\fontsize{8.000000}{9.600000}\selectfont\catcode`\^=\active\def^{\ifmmode\sp\else\^{}\fi}\catcode`\%=\active\def%{\%}20.0}}%
\end{pgfscope}%
\begin{pgfscope}%
\pgfpathrectangle{\pgfqpoint{2.717318in}{0.529443in}}{\pgfqpoint{1.636659in}{1.745990in}}%
\pgfusepath{clip}%
\pgfsetbuttcap%
\pgfsetroundjoin%
\definecolor{currentfill}{rgb}{0.003922,0.003922,0.003922}%
\pgfsetfillcolor{currentfill}%
\pgfsetfillopacity{0.900000}%
\pgfsetlinewidth{0.507862pt}%
\definecolor{currentstroke}{rgb}{1.000000,1.000000,1.000000}%
\pgfsetstrokecolor{currentstroke}%
\pgfsetstrokeopacity{0.900000}%
\pgfsetdash{}{0pt}%
\pgfpathmoveto{\pgfqpoint{2.858592in}{0.572006in}}%
\pgfpathcurveto{\pgfqpoint{2.870240in}{0.572006in}}{\pgfqpoint{2.881412in}{0.576633in}}{\pgfqpoint{2.889648in}{0.584870in}}%
\pgfpathcurveto{\pgfqpoint{2.897885in}{0.593106in}}{\pgfqpoint{2.902512in}{0.604278in}}{\pgfqpoint{2.902512in}{0.615926in}}%
\pgfpathcurveto{\pgfqpoint{2.902512in}{0.627574in}}{\pgfqpoint{2.897885in}{0.638746in}}{\pgfqpoint{2.889648in}{0.646983in}}%
\pgfpathcurveto{\pgfqpoint{2.881412in}{0.655219in}}{\pgfqpoint{2.870240in}{0.659847in}}{\pgfqpoint{2.858592in}{0.659847in}}%
\pgfpathcurveto{\pgfqpoint{2.846944in}{0.659847in}}{\pgfqpoint{2.835772in}{0.655219in}}{\pgfqpoint{2.827535in}{0.646983in}}%
\pgfpathcurveto{\pgfqpoint{2.819299in}{0.638746in}}{\pgfqpoint{2.814671in}{0.627574in}}{\pgfqpoint{2.814671in}{0.615926in}}%
\pgfpathcurveto{\pgfqpoint{2.814671in}{0.604278in}}{\pgfqpoint{2.819299in}{0.593106in}}{\pgfqpoint{2.827535in}{0.584870in}}%
\pgfpathcurveto{\pgfqpoint{2.835772in}{0.576633in}}{\pgfqpoint{2.846944in}{0.572006in}}{\pgfqpoint{2.858592in}{0.572006in}}%
\pgfpathlineto{\pgfqpoint{2.858592in}{0.572006in}}%
\pgfpathclose%
\pgfusepath{stroke,fill}%
\end{pgfscope}%
\begin{pgfscope}%
\pgfpathrectangle{\pgfqpoint{2.717318in}{0.529443in}}{\pgfqpoint{1.636659in}{1.745990in}}%
\pgfusepath{clip}%
\pgfsetbuttcap%
\pgfsetroundjoin%
\definecolor{currentfill}{rgb}{0.003922,0.003922,0.003922}%
\pgfsetfillcolor{currentfill}%
\pgfsetfillopacity{0.900000}%
\pgfsetlinewidth{0.507862pt}%
\definecolor{currentstroke}{rgb}{1.000000,1.000000,1.000000}%
\pgfsetstrokecolor{currentstroke}%
\pgfsetstrokeopacity{0.900000}%
\pgfsetdash{}{0pt}%
\pgfpathmoveto{\pgfqpoint{4.207605in}{2.090327in}}%
\pgfpathcurveto{\pgfqpoint{4.219252in}{2.090327in}}{\pgfqpoint{4.230425in}{2.094955in}}{\pgfqpoint{4.238661in}{2.103191in}}%
\pgfpathcurveto{\pgfqpoint{4.246897in}{2.111428in}}{\pgfqpoint{4.251525in}{2.122600in}}{\pgfqpoint{4.251525in}{2.134248in}}%
\pgfpathcurveto{\pgfqpoint{4.251525in}{2.145896in}}{\pgfqpoint{4.246897in}{2.157068in}}{\pgfqpoint{4.238661in}{2.165304in}}%
\pgfpathcurveto{\pgfqpoint{4.230425in}{2.173541in}}{\pgfqpoint{4.219252in}{2.178168in}}{\pgfqpoint{4.207605in}{2.178168in}}%
\pgfpathcurveto{\pgfqpoint{4.195957in}{2.178168in}}{\pgfqpoint{4.184784in}{2.173541in}}{\pgfqpoint{4.176548in}{2.165304in}}%
\pgfpathcurveto{\pgfqpoint{4.168312in}{2.157068in}}{\pgfqpoint{4.163684in}{2.145896in}}{\pgfqpoint{4.163684in}{2.134248in}}%
\pgfpathcurveto{\pgfqpoint{4.163684in}{2.122600in}}{\pgfqpoint{4.168312in}{2.111428in}}{\pgfqpoint{4.176548in}{2.103191in}}%
\pgfpathcurveto{\pgfqpoint{4.184784in}{2.094955in}}{\pgfqpoint{4.195957in}{2.090327in}}{\pgfqpoint{4.207605in}{2.090327in}}%
\pgfpathlineto{\pgfqpoint{4.207605in}{2.090327in}}%
\pgfpathclose%
\pgfusepath{stroke,fill}%
\end{pgfscope}%
\begin{pgfscope}%
\pgfpathrectangle{\pgfqpoint{2.717318in}{0.529443in}}{\pgfqpoint{1.636659in}{1.745990in}}%
\pgfusepath{clip}%
\pgfsetbuttcap%
\pgfsetroundjoin%
\definecolor{currentfill}{rgb}{0.003922,0.003922,0.003922}%
\pgfsetfillcolor{currentfill}%
\pgfsetfillopacity{0.900000}%
\pgfsetlinewidth{0.507862pt}%
\definecolor{currentstroke}{rgb}{1.000000,1.000000,1.000000}%
\pgfsetstrokecolor{currentstroke}%
\pgfsetstrokeopacity{0.900000}%
\pgfsetdash{}{0pt}%
\pgfpathmoveto{\pgfqpoint{2.843639in}{0.628014in}}%
\pgfpathcurveto{\pgfqpoint{2.855287in}{0.628014in}}{\pgfqpoint{2.866460in}{0.632642in}}{\pgfqpoint{2.874696in}{0.640878in}}%
\pgfpathcurveto{\pgfqpoint{2.882932in}{0.649114in}}{\pgfqpoint{2.887560in}{0.660287in}}{\pgfqpoint{2.887560in}{0.671935in}}%
\pgfpathcurveto{\pgfqpoint{2.887560in}{0.683582in}}{\pgfqpoint{2.882932in}{0.694755in}}{\pgfqpoint{2.874696in}{0.702991in}}%
\pgfpathcurveto{\pgfqpoint{2.866460in}{0.711227in}}{\pgfqpoint{2.855287in}{0.715855in}}{\pgfqpoint{2.843639in}{0.715855in}}%
\pgfpathcurveto{\pgfqpoint{2.831992in}{0.715855in}}{\pgfqpoint{2.820819in}{0.711227in}}{\pgfqpoint{2.812583in}{0.702991in}}%
\pgfpathcurveto{\pgfqpoint{2.804347in}{0.694755in}}{\pgfqpoint{2.799719in}{0.683582in}}{\pgfqpoint{2.799719in}{0.671935in}}%
\pgfpathcurveto{\pgfqpoint{2.799719in}{0.660287in}}{\pgfqpoint{2.804347in}{0.649114in}}{\pgfqpoint{2.812583in}{0.640878in}}%
\pgfpathcurveto{\pgfqpoint{2.820819in}{0.632642in}}{\pgfqpoint{2.831992in}{0.628014in}}{\pgfqpoint{2.843639in}{0.628014in}}%
\pgfpathlineto{\pgfqpoint{2.843639in}{0.628014in}}%
\pgfpathclose%
\pgfusepath{stroke,fill}%
\end{pgfscope}%
\begin{pgfscope}%
\pgfpathrectangle{\pgfqpoint{2.717318in}{0.529443in}}{\pgfqpoint{1.636659in}{1.745990in}}%
\pgfusepath{clip}%
\pgfsetbuttcap%
\pgfsetroundjoin%
\definecolor{currentfill}{rgb}{0.003922,0.003922,0.003922}%
\pgfsetfillcolor{currentfill}%
\pgfsetfillopacity{0.900000}%
\pgfsetlinewidth{0.507862pt}%
\definecolor{currentstroke}{rgb}{1.000000,1.000000,1.000000}%
\pgfsetstrokecolor{currentstroke}%
\pgfsetstrokeopacity{0.900000}%
\pgfsetdash{}{0pt}%
\pgfpathmoveto{\pgfqpoint{4.260098in}{2.085929in}}%
\pgfpathcurveto{\pgfqpoint{4.271745in}{2.085929in}}{\pgfqpoint{4.282918in}{2.090557in}}{\pgfqpoint{4.291154in}{2.098793in}}%
\pgfpathcurveto{\pgfqpoint{4.299390in}{2.107029in}}{\pgfqpoint{4.304018in}{2.118202in}}{\pgfqpoint{4.304018in}{2.129849in}}%
\pgfpathcurveto{\pgfqpoint{4.304018in}{2.141497in}}{\pgfqpoint{4.299390in}{2.152670in}}{\pgfqpoint{4.291154in}{2.160906in}}%
\pgfpathcurveto{\pgfqpoint{4.282918in}{2.169142in}}{\pgfqpoint{4.271745in}{2.173770in}}{\pgfqpoint{4.260098in}{2.173770in}}%
\pgfpathcurveto{\pgfqpoint{4.248450in}{2.173770in}}{\pgfqpoint{4.237277in}{2.169142in}}{\pgfqpoint{4.229041in}{2.160906in}}%
\pgfpathcurveto{\pgfqpoint{4.220805in}{2.152670in}}{\pgfqpoint{4.216177in}{2.141497in}}{\pgfqpoint{4.216177in}{2.129849in}}%
\pgfpathcurveto{\pgfqpoint{4.216177in}{2.118202in}}{\pgfqpoint{4.220805in}{2.107029in}}{\pgfqpoint{4.229041in}{2.098793in}}%
\pgfpathcurveto{\pgfqpoint{4.237277in}{2.090557in}}{\pgfqpoint{4.248450in}{2.085929in}}{\pgfqpoint{4.260098in}{2.085929in}}%
\pgfpathlineto{\pgfqpoint{4.260098in}{2.085929in}}%
\pgfpathclose%
\pgfusepath{stroke,fill}%
\end{pgfscope}%
\begin{pgfscope}%
\pgfpathrectangle{\pgfqpoint{2.717318in}{0.529443in}}{\pgfqpoint{1.636659in}{1.745990in}}%
\pgfusepath{clip}%
\pgfsetbuttcap%
\pgfsetroundjoin%
\definecolor{currentfill}{rgb}{0.003922,0.003922,0.003922}%
\pgfsetfillcolor{currentfill}%
\pgfsetfillopacity{0.900000}%
\pgfsetlinewidth{0.507862pt}%
\definecolor{currentstroke}{rgb}{1.000000,1.000000,1.000000}%
\pgfsetstrokecolor{currentstroke}%
\pgfsetstrokeopacity{0.900000}%
\pgfsetdash{}{0pt}%
\pgfpathmoveto{\pgfqpoint{4.225225in}{1.948791in}}%
\pgfpathcurveto{\pgfqpoint{4.236873in}{1.948791in}}{\pgfqpoint{4.248045in}{1.953419in}}{\pgfqpoint{4.256282in}{1.961655in}}%
\pgfpathcurveto{\pgfqpoint{4.264518in}{1.969891in}}{\pgfqpoint{4.269146in}{1.981063in}}{\pgfqpoint{4.269146in}{1.992711in}}%
\pgfpathcurveto{\pgfqpoint{4.269146in}{2.004359in}}{\pgfqpoint{4.264518in}{2.015532in}}{\pgfqpoint{4.256282in}{2.023768in}}%
\pgfpathcurveto{\pgfqpoint{4.248045in}{2.032004in}}{\pgfqpoint{4.236873in}{2.036632in}}{\pgfqpoint{4.225225in}{2.036632in}}%
\pgfpathcurveto{\pgfqpoint{4.213577in}{2.036632in}}{\pgfqpoint{4.202405in}{2.032004in}}{\pgfqpoint{4.194169in}{2.023768in}}%
\pgfpathcurveto{\pgfqpoint{4.185932in}{2.015532in}}{\pgfqpoint{4.181305in}{2.004359in}}{\pgfqpoint{4.181305in}{1.992711in}}%
\pgfpathcurveto{\pgfqpoint{4.181305in}{1.981063in}}{\pgfqpoint{4.185932in}{1.969891in}}{\pgfqpoint{4.194169in}{1.961655in}}%
\pgfpathcurveto{\pgfqpoint{4.202405in}{1.953419in}}{\pgfqpoint{4.213577in}{1.948791in}}{\pgfqpoint{4.225225in}{1.948791in}}%
\pgfpathlineto{\pgfqpoint{4.225225in}{1.948791in}}%
\pgfpathclose%
\pgfusepath{stroke,fill}%
\end{pgfscope}%
\begin{pgfscope}%
\pgfpathrectangle{\pgfqpoint{2.717318in}{0.529443in}}{\pgfqpoint{1.636659in}{1.745990in}}%
\pgfusepath{clip}%
\pgfsetbuttcap%
\pgfsetroundjoin%
\definecolor{currentfill}{rgb}{0.003922,0.003922,0.003922}%
\pgfsetfillcolor{currentfill}%
\pgfsetfillopacity{0.900000}%
\pgfsetlinewidth{0.507862pt}%
\definecolor{currentstroke}{rgb}{1.000000,1.000000,1.000000}%
\pgfsetstrokecolor{currentstroke}%
\pgfsetstrokeopacity{0.900000}%
\pgfsetdash{}{0pt}%
\pgfpathmoveto{\pgfqpoint{4.239096in}{1.972634in}}%
\pgfpathcurveto{\pgfqpoint{4.250744in}{1.972634in}}{\pgfqpoint{4.261916in}{1.977262in}}{\pgfqpoint{4.270153in}{1.985498in}}%
\pgfpathcurveto{\pgfqpoint{4.278389in}{1.993735in}}{\pgfqpoint{4.283017in}{2.004907in}}{\pgfqpoint{4.283017in}{2.016555in}}%
\pgfpathcurveto{\pgfqpoint{4.283017in}{2.028203in}}{\pgfqpoint{4.278389in}{2.039375in}}{\pgfqpoint{4.270153in}{2.047611in}}%
\pgfpathcurveto{\pgfqpoint{4.261916in}{2.055848in}}{\pgfqpoint{4.250744in}{2.060475in}}{\pgfqpoint{4.239096in}{2.060475in}}%
\pgfpathcurveto{\pgfqpoint{4.227448in}{2.060475in}}{\pgfqpoint{4.216276in}{2.055848in}}{\pgfqpoint{4.208040in}{2.047611in}}%
\pgfpathcurveto{\pgfqpoint{4.199803in}{2.039375in}}{\pgfqpoint{4.195176in}{2.028203in}}{\pgfqpoint{4.195176in}{2.016555in}}%
\pgfpathcurveto{\pgfqpoint{4.195176in}{2.004907in}}{\pgfqpoint{4.199803in}{1.993735in}}{\pgfqpoint{4.208040in}{1.985498in}}%
\pgfpathcurveto{\pgfqpoint{4.216276in}{1.977262in}}{\pgfqpoint{4.227448in}{1.972634in}}{\pgfqpoint{4.239096in}{1.972634in}}%
\pgfpathlineto{\pgfqpoint{4.239096in}{1.972634in}}%
\pgfpathclose%
\pgfusepath{stroke,fill}%
\end{pgfscope}%
\begin{pgfscope}%
\pgfpathrectangle{\pgfqpoint{2.717318in}{0.529443in}}{\pgfqpoint{1.636659in}{1.745990in}}%
\pgfusepath{clip}%
\pgfsetbuttcap%
\pgfsetroundjoin%
\definecolor{currentfill}{rgb}{0.003922,0.003922,0.003922}%
\pgfsetfillcolor{currentfill}%
\pgfsetfillopacity{0.900000}%
\pgfsetlinewidth{0.507862pt}%
\definecolor{currentstroke}{rgb}{1.000000,1.000000,1.000000}%
\pgfsetstrokecolor{currentstroke}%
\pgfsetstrokeopacity{0.900000}%
\pgfsetdash{}{0pt}%
\pgfpathmoveto{\pgfqpoint{4.262524in}{2.036725in}}%
\pgfpathcurveto{\pgfqpoint{4.274172in}{2.036725in}}{\pgfqpoint{4.285345in}{2.041353in}}{\pgfqpoint{4.293581in}{2.049589in}}%
\pgfpathcurveto{\pgfqpoint{4.301817in}{2.057825in}}{\pgfqpoint{4.306445in}{2.068998in}}{\pgfqpoint{4.306445in}{2.080645in}}%
\pgfpathcurveto{\pgfqpoint{4.306445in}{2.092293in}}{\pgfqpoint{4.301817in}{2.103466in}}{\pgfqpoint{4.293581in}{2.111702in}}%
\pgfpathcurveto{\pgfqpoint{4.285345in}{2.119938in}}{\pgfqpoint{4.274172in}{2.124566in}}{\pgfqpoint{4.262524in}{2.124566in}}%
\pgfpathcurveto{\pgfqpoint{4.250877in}{2.124566in}}{\pgfqpoint{4.239704in}{2.119938in}}{\pgfqpoint{4.231468in}{2.111702in}}%
\pgfpathcurveto{\pgfqpoint{4.223232in}{2.103466in}}{\pgfqpoint{4.218604in}{2.092293in}}{\pgfqpoint{4.218604in}{2.080645in}}%
\pgfpathcurveto{\pgfqpoint{4.218604in}{2.068998in}}{\pgfqpoint{4.223232in}{2.057825in}}{\pgfqpoint{4.231468in}{2.049589in}}%
\pgfpathcurveto{\pgfqpoint{4.239704in}{2.041353in}}{\pgfqpoint{4.250877in}{2.036725in}}{\pgfqpoint{4.262524in}{2.036725in}}%
\pgfpathlineto{\pgfqpoint{4.262524in}{2.036725in}}%
\pgfpathclose%
\pgfusepath{stroke,fill}%
\end{pgfscope}%
\begin{pgfscope}%
\pgfpathrectangle{\pgfqpoint{2.717318in}{0.529443in}}{\pgfqpoint{1.636659in}{1.745990in}}%
\pgfusepath{clip}%
\pgfsetbuttcap%
\pgfsetroundjoin%
\definecolor{currentfill}{rgb}{0.003922,0.003922,0.003922}%
\pgfsetfillcolor{currentfill}%
\pgfsetfillopacity{0.900000}%
\pgfsetlinewidth{0.507862pt}%
\definecolor{currentstroke}{rgb}{1.000000,1.000000,1.000000}%
\pgfsetstrokecolor{currentstroke}%
\pgfsetstrokeopacity{0.900000}%
\pgfsetdash{}{0pt}%
\pgfpathmoveto{\pgfqpoint{4.177256in}{2.085526in}}%
\pgfpathcurveto{\pgfqpoint{4.188904in}{2.085526in}}{\pgfqpoint{4.200076in}{2.090154in}}{\pgfqpoint{4.208313in}{2.098390in}}%
\pgfpathcurveto{\pgfqpoint{4.216549in}{2.106626in}}{\pgfqpoint{4.221177in}{2.117799in}}{\pgfqpoint{4.221177in}{2.129446in}}%
\pgfpathcurveto{\pgfqpoint{4.221177in}{2.141094in}}{\pgfqpoint{4.216549in}{2.152267in}}{\pgfqpoint{4.208313in}{2.160503in}}%
\pgfpathcurveto{\pgfqpoint{4.200076in}{2.168739in}}{\pgfqpoint{4.188904in}{2.173367in}}{\pgfqpoint{4.177256in}{2.173367in}}%
\pgfpathcurveto{\pgfqpoint{4.165608in}{2.173367in}}{\pgfqpoint{4.154436in}{2.168739in}}{\pgfqpoint{4.146200in}{2.160503in}}%
\pgfpathcurveto{\pgfqpoint{4.137963in}{2.152267in}}{\pgfqpoint{4.133336in}{2.141094in}}{\pgfqpoint{4.133336in}{2.129446in}}%
\pgfpathcurveto{\pgfqpoint{4.133336in}{2.117799in}}{\pgfqpoint{4.137963in}{2.106626in}}{\pgfqpoint{4.146200in}{2.098390in}}%
\pgfpathcurveto{\pgfqpoint{4.154436in}{2.090154in}}{\pgfqpoint{4.165608in}{2.085526in}}{\pgfqpoint{4.177256in}{2.085526in}}%
\pgfpathlineto{\pgfqpoint{4.177256in}{2.085526in}}%
\pgfpathclose%
\pgfusepath{stroke,fill}%
\end{pgfscope}%
\begin{pgfscope}%
\pgfpathrectangle{\pgfqpoint{2.717318in}{0.529443in}}{\pgfqpoint{1.636659in}{1.745990in}}%
\pgfusepath{clip}%
\pgfsetbuttcap%
\pgfsetroundjoin%
\definecolor{currentfill}{rgb}{0.003922,0.003922,0.003922}%
\pgfsetfillcolor{currentfill}%
\pgfsetfillopacity{0.900000}%
\pgfsetlinewidth{0.507862pt}%
\definecolor{currentstroke}{rgb}{1.000000,1.000000,1.000000}%
\pgfsetstrokecolor{currentstroke}%
\pgfsetstrokeopacity{0.900000}%
\pgfsetdash{}{0pt}%
\pgfpathmoveto{\pgfqpoint{2.862941in}{0.622670in}}%
\pgfpathcurveto{\pgfqpoint{2.874589in}{0.622670in}}{\pgfqpoint{2.885761in}{0.627298in}}{\pgfqpoint{2.893997in}{0.635534in}}%
\pgfpathcurveto{\pgfqpoint{2.902234in}{0.643771in}}{\pgfqpoint{2.906861in}{0.654943in}}{\pgfqpoint{2.906861in}{0.666591in}}%
\pgfpathcurveto{\pgfqpoint{2.906861in}{0.678239in}}{\pgfqpoint{2.902234in}{0.689411in}}{\pgfqpoint{2.893997in}{0.697647in}}%
\pgfpathcurveto{\pgfqpoint{2.885761in}{0.705884in}}{\pgfqpoint{2.874589in}{0.710511in}}{\pgfqpoint{2.862941in}{0.710511in}}%
\pgfpathcurveto{\pgfqpoint{2.851293in}{0.710511in}}{\pgfqpoint{2.840121in}{0.705884in}}{\pgfqpoint{2.831884in}{0.697647in}}%
\pgfpathcurveto{\pgfqpoint{2.823648in}{0.689411in}}{\pgfqpoint{2.819020in}{0.678239in}}{\pgfqpoint{2.819020in}{0.666591in}}%
\pgfpathcurveto{\pgfqpoint{2.819020in}{0.654943in}}{\pgfqpoint{2.823648in}{0.643771in}}{\pgfqpoint{2.831884in}{0.635534in}}%
\pgfpathcurveto{\pgfqpoint{2.840121in}{0.627298in}}{\pgfqpoint{2.851293in}{0.622670in}}{\pgfqpoint{2.862941in}{0.622670in}}%
\pgfpathlineto{\pgfqpoint{2.862941in}{0.622670in}}%
\pgfpathclose%
\pgfusepath{stroke,fill}%
\end{pgfscope}%
\begin{pgfscope}%
\pgfpathrectangle{\pgfqpoint{2.717318in}{0.529443in}}{\pgfqpoint{1.636659in}{1.745990in}}%
\pgfusepath{clip}%
\pgfsetbuttcap%
\pgfsetroundjoin%
\definecolor{currentfill}{rgb}{0.003922,0.003922,0.003922}%
\pgfsetfillcolor{currentfill}%
\pgfsetfillopacity{0.900000}%
\pgfsetlinewidth{0.507862pt}%
\definecolor{currentstroke}{rgb}{1.000000,1.000000,1.000000}%
\pgfsetstrokecolor{currentstroke}%
\pgfsetstrokeopacity{0.900000}%
\pgfsetdash{}{0pt}%
\pgfpathmoveto{\pgfqpoint{4.214391in}{2.043613in}}%
\pgfpathcurveto{\pgfqpoint{4.226038in}{2.043613in}}{\pgfqpoint{4.237211in}{2.048240in}}{\pgfqpoint{4.245447in}{2.056477in}}%
\pgfpathcurveto{\pgfqpoint{4.253683in}{2.064713in}}{\pgfqpoint{4.258311in}{2.075885in}}{\pgfqpoint{4.258311in}{2.087533in}}%
\pgfpathcurveto{\pgfqpoint{4.258311in}{2.099181in}}{\pgfqpoint{4.253683in}{2.110353in}}{\pgfqpoint{4.245447in}{2.118590in}}%
\pgfpathcurveto{\pgfqpoint{4.237211in}{2.126826in}}{\pgfqpoint{4.226038in}{2.131454in}}{\pgfqpoint{4.214391in}{2.131454in}}%
\pgfpathcurveto{\pgfqpoint{4.202743in}{2.131454in}}{\pgfqpoint{4.191570in}{2.126826in}}{\pgfqpoint{4.183334in}{2.118590in}}%
\pgfpathcurveto{\pgfqpoint{4.175098in}{2.110353in}}{\pgfqpoint{4.170470in}{2.099181in}}{\pgfqpoint{4.170470in}{2.087533in}}%
\pgfpathcurveto{\pgfqpoint{4.170470in}{2.075885in}}{\pgfqpoint{4.175098in}{2.064713in}}{\pgfqpoint{4.183334in}{2.056477in}}%
\pgfpathcurveto{\pgfqpoint{4.191570in}{2.048240in}}{\pgfqpoint{4.202743in}{2.043613in}}{\pgfqpoint{4.214391in}{2.043613in}}%
\pgfpathlineto{\pgfqpoint{4.214391in}{2.043613in}}%
\pgfpathclose%
\pgfusepath{stroke,fill}%
\end{pgfscope}%
\begin{pgfscope}%
\pgfpathrectangle{\pgfqpoint{2.717318in}{0.529443in}}{\pgfqpoint{1.636659in}{1.745990in}}%
\pgfusepath{clip}%
\pgfsetbuttcap%
\pgfsetroundjoin%
\definecolor{currentfill}{rgb}{0.003922,0.003922,0.003922}%
\pgfsetfillcolor{currentfill}%
\pgfsetfillopacity{0.900000}%
\pgfsetlinewidth{0.507862pt}%
\definecolor{currentstroke}{rgb}{1.000000,1.000000,1.000000}%
\pgfsetstrokecolor{currentstroke}%
\pgfsetstrokeopacity{0.900000}%
\pgfsetdash{}{0pt}%
\pgfpathmoveto{\pgfqpoint{2.805427in}{0.626312in}}%
\pgfpathcurveto{\pgfqpoint{2.817075in}{0.626312in}}{\pgfqpoint{2.828247in}{0.630940in}}{\pgfqpoint{2.836483in}{0.639177in}}%
\pgfpathcurveto{\pgfqpoint{2.844720in}{0.647413in}}{\pgfqpoint{2.849347in}{0.658585in}}{\pgfqpoint{2.849347in}{0.670233in}}%
\pgfpathcurveto{\pgfqpoint{2.849347in}{0.681881in}}{\pgfqpoint{2.844720in}{0.693053in}}{\pgfqpoint{2.836483in}{0.701290in}}%
\pgfpathcurveto{\pgfqpoint{2.828247in}{0.709526in}}{\pgfqpoint{2.817075in}{0.714154in}}{\pgfqpoint{2.805427in}{0.714154in}}%
\pgfpathcurveto{\pgfqpoint{2.793779in}{0.714154in}}{\pgfqpoint{2.782607in}{0.709526in}}{\pgfqpoint{2.774370in}{0.701290in}}%
\pgfpathcurveto{\pgfqpoint{2.766134in}{0.693053in}}{\pgfqpoint{2.761506in}{0.681881in}}{\pgfqpoint{2.761506in}{0.670233in}}%
\pgfpathcurveto{\pgfqpoint{2.761506in}{0.658585in}}{\pgfqpoint{2.766134in}{0.647413in}}{\pgfqpoint{2.774370in}{0.639177in}}%
\pgfpathcurveto{\pgfqpoint{2.782607in}{0.630940in}}{\pgfqpoint{2.793779in}{0.626312in}}{\pgfqpoint{2.805427in}{0.626312in}}%
\pgfpathlineto{\pgfqpoint{2.805427in}{0.626312in}}%
\pgfpathclose%
\pgfusepath{stroke,fill}%
\end{pgfscope}%
\begin{pgfscope}%
\pgfpathrectangle{\pgfqpoint{2.717318in}{0.529443in}}{\pgfqpoint{1.636659in}{1.745990in}}%
\pgfusepath{clip}%
\pgfsetbuttcap%
\pgfsetroundjoin%
\definecolor{currentfill}{rgb}{0.003922,0.003922,0.003922}%
\pgfsetfillcolor{currentfill}%
\pgfsetfillopacity{0.900000}%
\pgfsetlinewidth{0.507862pt}%
\definecolor{currentstroke}{rgb}{1.000000,1.000000,1.000000}%
\pgfsetstrokecolor{currentstroke}%
\pgfsetstrokeopacity{0.900000}%
\pgfsetdash{}{0pt}%
\pgfpathmoveto{\pgfqpoint{2.810747in}{0.658232in}}%
\pgfpathcurveto{\pgfqpoint{2.822395in}{0.658232in}}{\pgfqpoint{2.833567in}{0.662860in}}{\pgfqpoint{2.841803in}{0.671096in}}%
\pgfpathcurveto{\pgfqpoint{2.850040in}{0.679333in}}{\pgfqpoint{2.854667in}{0.690505in}}{\pgfqpoint{2.854667in}{0.702153in}}%
\pgfpathcurveto{\pgfqpoint{2.854667in}{0.713801in}}{\pgfqpoint{2.850040in}{0.724973in}}{\pgfqpoint{2.841803in}{0.733209in}}%
\pgfpathcurveto{\pgfqpoint{2.833567in}{0.741446in}}{\pgfqpoint{2.822395in}{0.746073in}}{\pgfqpoint{2.810747in}{0.746073in}}%
\pgfpathcurveto{\pgfqpoint{2.799099in}{0.746073in}}{\pgfqpoint{2.787927in}{0.741446in}}{\pgfqpoint{2.779690in}{0.733209in}}%
\pgfpathcurveto{\pgfqpoint{2.771454in}{0.724973in}}{\pgfqpoint{2.766826in}{0.713801in}}{\pgfqpoint{2.766826in}{0.702153in}}%
\pgfpathcurveto{\pgfqpoint{2.766826in}{0.690505in}}{\pgfqpoint{2.771454in}{0.679333in}}{\pgfqpoint{2.779690in}{0.671096in}}%
\pgfpathcurveto{\pgfqpoint{2.787927in}{0.662860in}}{\pgfqpoint{2.799099in}{0.658232in}}{\pgfqpoint{2.810747in}{0.658232in}}%
\pgfpathlineto{\pgfqpoint{2.810747in}{0.658232in}}%
\pgfpathclose%
\pgfusepath{stroke,fill}%
\end{pgfscope}%
\begin{pgfscope}%
\pgfpathrectangle{\pgfqpoint{2.717318in}{0.529443in}}{\pgfqpoint{1.636659in}{1.745990in}}%
\pgfusepath{clip}%
\pgfsetbuttcap%
\pgfsetroundjoin%
\definecolor{currentfill}{rgb}{0.003922,0.003922,0.003922}%
\pgfsetfillcolor{currentfill}%
\pgfsetfillopacity{0.900000}%
\pgfsetlinewidth{0.507862pt}%
\definecolor{currentstroke}{rgb}{1.000000,1.000000,1.000000}%
\pgfsetstrokecolor{currentstroke}%
\pgfsetstrokeopacity{0.900000}%
\pgfsetdash{}{0pt}%
\pgfpathmoveto{\pgfqpoint{2.831280in}{0.771187in}}%
\pgfpathcurveto{\pgfqpoint{2.842928in}{0.771187in}}{\pgfqpoint{2.854100in}{0.775815in}}{\pgfqpoint{2.862336in}{0.784051in}}%
\pgfpathcurveto{\pgfqpoint{2.870573in}{0.792288in}}{\pgfqpoint{2.875201in}{0.803460in}}{\pgfqpoint{2.875201in}{0.815108in}}%
\pgfpathcurveto{\pgfqpoint{2.875201in}{0.826756in}}{\pgfqpoint{2.870573in}{0.837928in}}{\pgfqpoint{2.862336in}{0.846164in}}%
\pgfpathcurveto{\pgfqpoint{2.854100in}{0.854401in}}{\pgfqpoint{2.842928in}{0.859028in}}{\pgfqpoint{2.831280in}{0.859028in}}%
\pgfpathcurveto{\pgfqpoint{2.819632in}{0.859028in}}{\pgfqpoint{2.808460in}{0.854401in}}{\pgfqpoint{2.800223in}{0.846164in}}%
\pgfpathcurveto{\pgfqpoint{2.791987in}{0.837928in}}{\pgfqpoint{2.787359in}{0.826756in}}{\pgfqpoint{2.787359in}{0.815108in}}%
\pgfpathcurveto{\pgfqpoint{2.787359in}{0.803460in}}{\pgfqpoint{2.791987in}{0.792288in}}{\pgfqpoint{2.800223in}{0.784051in}}%
\pgfpathcurveto{\pgfqpoint{2.808460in}{0.775815in}}{\pgfqpoint{2.819632in}{0.771187in}}{\pgfqpoint{2.831280in}{0.771187in}}%
\pgfpathlineto{\pgfqpoint{2.831280in}{0.771187in}}%
\pgfpathclose%
\pgfusepath{stroke,fill}%
\end{pgfscope}%
\begin{pgfscope}%
\pgfpathrectangle{\pgfqpoint{2.717318in}{0.529443in}}{\pgfqpoint{1.636659in}{1.745990in}}%
\pgfusepath{clip}%
\pgfsetbuttcap%
\pgfsetroundjoin%
\definecolor{currentfill}{rgb}{0.003922,0.003922,0.003922}%
\pgfsetfillcolor{currentfill}%
\pgfsetfillopacity{0.900000}%
\pgfsetlinewidth{0.507862pt}%
\definecolor{currentstroke}{rgb}{1.000000,1.000000,1.000000}%
\pgfsetstrokecolor{currentstroke}%
\pgfsetstrokeopacity{0.900000}%
\pgfsetdash{}{0pt}%
\pgfpathmoveto{\pgfqpoint{4.223011in}{2.081970in}}%
\pgfpathcurveto{\pgfqpoint{4.234658in}{2.081970in}}{\pgfqpoint{4.245831in}{2.086598in}}{\pgfqpoint{4.254067in}{2.094834in}}%
\pgfpathcurveto{\pgfqpoint{4.262303in}{2.103071in}}{\pgfqpoint{4.266931in}{2.114243in}}{\pgfqpoint{4.266931in}{2.125891in}}%
\pgfpathcurveto{\pgfqpoint{4.266931in}{2.137539in}}{\pgfqpoint{4.262303in}{2.148711in}}{\pgfqpoint{4.254067in}{2.156947in}}%
\pgfpathcurveto{\pgfqpoint{4.245831in}{2.165184in}}{\pgfqpoint{4.234658in}{2.169812in}}{\pgfqpoint{4.223011in}{2.169812in}}%
\pgfpathcurveto{\pgfqpoint{4.211363in}{2.169812in}}{\pgfqpoint{4.200190in}{2.165184in}}{\pgfqpoint{4.191954in}{2.156947in}}%
\pgfpathcurveto{\pgfqpoint{4.183718in}{2.148711in}}{\pgfqpoint{4.179090in}{2.137539in}}{\pgfqpoint{4.179090in}{2.125891in}}%
\pgfpathcurveto{\pgfqpoint{4.179090in}{2.114243in}}{\pgfqpoint{4.183718in}{2.103071in}}{\pgfqpoint{4.191954in}{2.094834in}}%
\pgfpathcurveto{\pgfqpoint{4.200190in}{2.086598in}}{\pgfqpoint{4.211363in}{2.081970in}}{\pgfqpoint{4.223011in}{2.081970in}}%
\pgfpathlineto{\pgfqpoint{4.223011in}{2.081970in}}%
\pgfpathclose%
\pgfusepath{stroke,fill}%
\end{pgfscope}%
\begin{pgfscope}%
\pgfpathrectangle{\pgfqpoint{2.717318in}{0.529443in}}{\pgfqpoint{1.636659in}{1.745990in}}%
\pgfusepath{clip}%
\pgfsetbuttcap%
\pgfsetroundjoin%
\definecolor{currentfill}{rgb}{0.003922,0.003922,0.003922}%
\pgfsetfillcolor{currentfill}%
\pgfsetfillopacity{0.900000}%
\pgfsetlinewidth{0.507862pt}%
\definecolor{currentstroke}{rgb}{1.000000,1.000000,1.000000}%
\pgfsetstrokecolor{currentstroke}%
\pgfsetstrokeopacity{0.900000}%
\pgfsetdash{}{0pt}%
\pgfpathmoveto{\pgfqpoint{4.233094in}{2.007844in}}%
\pgfpathcurveto{\pgfqpoint{4.244742in}{2.007844in}}{\pgfqpoint{4.255914in}{2.012472in}}{\pgfqpoint{4.264150in}{2.020708in}}%
\pgfpathcurveto{\pgfqpoint{4.272387in}{2.028944in}}{\pgfqpoint{4.277014in}{2.040117in}}{\pgfqpoint{4.277014in}{2.051764in}}%
\pgfpathcurveto{\pgfqpoint{4.277014in}{2.063412in}}{\pgfqpoint{4.272387in}{2.074585in}}{\pgfqpoint{4.264150in}{2.082821in}}%
\pgfpathcurveto{\pgfqpoint{4.255914in}{2.091057in}}{\pgfqpoint{4.244742in}{2.095685in}}{\pgfqpoint{4.233094in}{2.095685in}}%
\pgfpathcurveto{\pgfqpoint{4.221446in}{2.095685in}}{\pgfqpoint{4.210274in}{2.091057in}}{\pgfqpoint{4.202037in}{2.082821in}}%
\pgfpathcurveto{\pgfqpoint{4.193801in}{2.074585in}}{\pgfqpoint{4.189173in}{2.063412in}}{\pgfqpoint{4.189173in}{2.051764in}}%
\pgfpathcurveto{\pgfqpoint{4.189173in}{2.040117in}}{\pgfqpoint{4.193801in}{2.028944in}}{\pgfqpoint{4.202037in}{2.020708in}}%
\pgfpathcurveto{\pgfqpoint{4.210274in}{2.012472in}}{\pgfqpoint{4.221446in}{2.007844in}}{\pgfqpoint{4.233094in}{2.007844in}}%
\pgfpathlineto{\pgfqpoint{4.233094in}{2.007844in}}%
\pgfpathclose%
\pgfusepath{stroke,fill}%
\end{pgfscope}%
\begin{pgfscope}%
\pgfpathrectangle{\pgfqpoint{2.717318in}{0.529443in}}{\pgfqpoint{1.636659in}{1.745990in}}%
\pgfusepath{clip}%
\pgfsetbuttcap%
\pgfsetroundjoin%
\definecolor{currentfill}{rgb}{0.003922,0.003922,0.003922}%
\pgfsetfillcolor{currentfill}%
\pgfsetfillopacity{0.900000}%
\pgfsetlinewidth{0.507862pt}%
\definecolor{currentstroke}{rgb}{1.000000,1.000000,1.000000}%
\pgfsetstrokecolor{currentstroke}%
\pgfsetstrokeopacity{0.900000}%
\pgfsetdash{}{0pt}%
\pgfpathmoveto{\pgfqpoint{4.233366in}{2.055544in}}%
\pgfpathcurveto{\pgfqpoint{4.245013in}{2.055544in}}{\pgfqpoint{4.256186in}{2.060171in}}{\pgfqpoint{4.264422in}{2.068408in}}%
\pgfpathcurveto{\pgfqpoint{4.272658in}{2.076644in}}{\pgfqpoint{4.277286in}{2.087816in}}{\pgfqpoint{4.277286in}{2.099464in}}%
\pgfpathcurveto{\pgfqpoint{4.277286in}{2.111112in}}{\pgfqpoint{4.272658in}{2.122284in}}{\pgfqpoint{4.264422in}{2.130521in}}%
\pgfpathcurveto{\pgfqpoint{4.256186in}{2.138757in}}{\pgfqpoint{4.245013in}{2.143385in}}{\pgfqpoint{4.233366in}{2.143385in}}%
\pgfpathcurveto{\pgfqpoint{4.221718in}{2.143385in}}{\pgfqpoint{4.210545in}{2.138757in}}{\pgfqpoint{4.202309in}{2.130521in}}%
\pgfpathcurveto{\pgfqpoint{4.194073in}{2.122284in}}{\pgfqpoint{4.189445in}{2.111112in}}{\pgfqpoint{4.189445in}{2.099464in}}%
\pgfpathcurveto{\pgfqpoint{4.189445in}{2.087816in}}{\pgfqpoint{4.194073in}{2.076644in}}{\pgfqpoint{4.202309in}{2.068408in}}%
\pgfpathcurveto{\pgfqpoint{4.210545in}{2.060171in}}{\pgfqpoint{4.221718in}{2.055544in}}{\pgfqpoint{4.233366in}{2.055544in}}%
\pgfpathlineto{\pgfqpoint{4.233366in}{2.055544in}}%
\pgfpathclose%
\pgfusepath{stroke,fill}%
\end{pgfscope}%
\begin{pgfscope}%
\pgfpathrectangle{\pgfqpoint{2.717318in}{0.529443in}}{\pgfqpoint{1.636659in}{1.745990in}}%
\pgfusepath{clip}%
\pgfsetbuttcap%
\pgfsetroundjoin%
\definecolor{currentfill}{rgb}{0.003922,0.003922,0.003922}%
\pgfsetfillcolor{currentfill}%
\pgfsetfillopacity{0.900000}%
\pgfsetlinewidth{0.507862pt}%
\definecolor{currentstroke}{rgb}{1.000000,1.000000,1.000000}%
\pgfsetstrokecolor{currentstroke}%
\pgfsetstrokeopacity{0.900000}%
\pgfsetdash{}{0pt}%
\pgfpathmoveto{\pgfqpoint{2.868131in}{0.635589in}}%
\pgfpathcurveto{\pgfqpoint{2.879779in}{0.635589in}}{\pgfqpoint{2.890951in}{0.640217in}}{\pgfqpoint{2.899187in}{0.648453in}}%
\pgfpathcurveto{\pgfqpoint{2.907424in}{0.656690in}}{\pgfqpoint{2.912051in}{0.667862in}}{\pgfqpoint{2.912051in}{0.679510in}}%
\pgfpathcurveto{\pgfqpoint{2.912051in}{0.691158in}}{\pgfqpoint{2.907424in}{0.702330in}}{\pgfqpoint{2.899187in}{0.710566in}}%
\pgfpathcurveto{\pgfqpoint{2.890951in}{0.718803in}}{\pgfqpoint{2.879779in}{0.723430in}}{\pgfqpoint{2.868131in}{0.723430in}}%
\pgfpathcurveto{\pgfqpoint{2.856483in}{0.723430in}}{\pgfqpoint{2.845311in}{0.718803in}}{\pgfqpoint{2.837074in}{0.710566in}}%
\pgfpathcurveto{\pgfqpoint{2.828838in}{0.702330in}}{\pgfqpoint{2.824210in}{0.691158in}}{\pgfqpoint{2.824210in}{0.679510in}}%
\pgfpathcurveto{\pgfqpoint{2.824210in}{0.667862in}}{\pgfqpoint{2.828838in}{0.656690in}}{\pgfqpoint{2.837074in}{0.648453in}}%
\pgfpathcurveto{\pgfqpoint{2.845311in}{0.640217in}}{\pgfqpoint{2.856483in}{0.635589in}}{\pgfqpoint{2.868131in}{0.635589in}}%
\pgfpathlineto{\pgfqpoint{2.868131in}{0.635589in}}%
\pgfpathclose%
\pgfusepath{stroke,fill}%
\end{pgfscope}%
\begin{pgfscope}%
\pgfpathrectangle{\pgfqpoint{2.717318in}{0.529443in}}{\pgfqpoint{1.636659in}{1.745990in}}%
\pgfusepath{clip}%
\pgfsetbuttcap%
\pgfsetroundjoin%
\definecolor{currentfill}{rgb}{0.003922,0.003922,0.003922}%
\pgfsetfillcolor{currentfill}%
\pgfsetfillopacity{0.900000}%
\pgfsetlinewidth{0.507862pt}%
\definecolor{currentstroke}{rgb}{1.000000,1.000000,1.000000}%
\pgfsetstrokecolor{currentstroke}%
\pgfsetstrokeopacity{0.900000}%
\pgfsetdash{}{0pt}%
\pgfpathmoveto{\pgfqpoint{4.254866in}{1.985479in}}%
\pgfpathcurveto{\pgfqpoint{4.266514in}{1.985479in}}{\pgfqpoint{4.277686in}{1.990106in}}{\pgfqpoint{4.285923in}{1.998343in}}%
\pgfpathcurveto{\pgfqpoint{4.294159in}{2.006579in}}{\pgfqpoint{4.298787in}{2.017751in}}{\pgfqpoint{4.298787in}{2.029399in}}%
\pgfpathcurveto{\pgfqpoint{4.298787in}{2.041047in}}{\pgfqpoint{4.294159in}{2.052219in}}{\pgfqpoint{4.285923in}{2.060456in}}%
\pgfpathcurveto{\pgfqpoint{4.277686in}{2.068692in}}{\pgfqpoint{4.266514in}{2.073320in}}{\pgfqpoint{4.254866in}{2.073320in}}%
\pgfpathcurveto{\pgfqpoint{4.243218in}{2.073320in}}{\pgfqpoint{4.232046in}{2.068692in}}{\pgfqpoint{4.223810in}{2.060456in}}%
\pgfpathcurveto{\pgfqpoint{4.215573in}{2.052219in}}{\pgfqpoint{4.210946in}{2.041047in}}{\pgfqpoint{4.210946in}{2.029399in}}%
\pgfpathcurveto{\pgfqpoint{4.210946in}{2.017751in}}{\pgfqpoint{4.215573in}{2.006579in}}{\pgfqpoint{4.223810in}{1.998343in}}%
\pgfpathcurveto{\pgfqpoint{4.232046in}{1.990106in}}{\pgfqpoint{4.243218in}{1.985479in}}{\pgfqpoint{4.254866in}{1.985479in}}%
\pgfpathlineto{\pgfqpoint{4.254866in}{1.985479in}}%
\pgfpathclose%
\pgfusepath{stroke,fill}%
\end{pgfscope}%
\begin{pgfscope}%
\pgfpathrectangle{\pgfqpoint{2.717318in}{0.529443in}}{\pgfqpoint{1.636659in}{1.745990in}}%
\pgfusepath{clip}%
\pgfsetbuttcap%
\pgfsetroundjoin%
\definecolor{currentfill}{rgb}{0.003922,0.003922,0.003922}%
\pgfsetfillcolor{currentfill}%
\pgfsetfillopacity{0.900000}%
\pgfsetlinewidth{0.507862pt}%
\definecolor{currentstroke}{rgb}{1.000000,1.000000,1.000000}%
\pgfsetstrokecolor{currentstroke}%
\pgfsetstrokeopacity{0.900000}%
\pgfsetdash{}{0pt}%
\pgfpathmoveto{\pgfqpoint{4.198453in}{2.028837in}}%
\pgfpathcurveto{\pgfqpoint{4.210101in}{2.028837in}}{\pgfqpoint{4.221273in}{2.033464in}}{\pgfqpoint{4.229509in}{2.041701in}}%
\pgfpathcurveto{\pgfqpoint{4.237746in}{2.049937in}}{\pgfqpoint{4.242373in}{2.061109in}}{\pgfqpoint{4.242373in}{2.072757in}}%
\pgfpathcurveto{\pgfqpoint{4.242373in}{2.084405in}}{\pgfqpoint{4.237746in}{2.095577in}}{\pgfqpoint{4.229509in}{2.103814in}}%
\pgfpathcurveto{\pgfqpoint{4.221273in}{2.112050in}}{\pgfqpoint{4.210101in}{2.116678in}}{\pgfqpoint{4.198453in}{2.116678in}}%
\pgfpathcurveto{\pgfqpoint{4.186805in}{2.116678in}}{\pgfqpoint{4.175633in}{2.112050in}}{\pgfqpoint{4.167396in}{2.103814in}}%
\pgfpathcurveto{\pgfqpoint{4.159160in}{2.095577in}}{\pgfqpoint{4.154532in}{2.084405in}}{\pgfqpoint{4.154532in}{2.072757in}}%
\pgfpathcurveto{\pgfqpoint{4.154532in}{2.061109in}}{\pgfqpoint{4.159160in}{2.049937in}}{\pgfqpoint{4.167396in}{2.041701in}}%
\pgfpathcurveto{\pgfqpoint{4.175633in}{2.033464in}}{\pgfqpoint{4.186805in}{2.028837in}}{\pgfqpoint{4.198453in}{2.028837in}}%
\pgfpathlineto{\pgfqpoint{4.198453in}{2.028837in}}%
\pgfpathclose%
\pgfusepath{stroke,fill}%
\end{pgfscope}%
\begin{pgfscope}%
\pgfpathrectangle{\pgfqpoint{2.717318in}{0.529443in}}{\pgfqpoint{1.636659in}{1.745990in}}%
\pgfusepath{clip}%
\pgfsetbuttcap%
\pgfsetroundjoin%
\definecolor{currentfill}{rgb}{0.003922,0.003922,0.003922}%
\pgfsetfillcolor{currentfill}%
\pgfsetfillopacity{0.900000}%
\pgfsetlinewidth{0.507862pt}%
\definecolor{currentstroke}{rgb}{1.000000,1.000000,1.000000}%
\pgfsetstrokecolor{currentstroke}%
\pgfsetstrokeopacity{0.900000}%
\pgfsetdash{}{0pt}%
\pgfpathmoveto{\pgfqpoint{2.855918in}{0.650464in}}%
\pgfpathcurveto{\pgfqpoint{2.867566in}{0.650464in}}{\pgfqpoint{2.878738in}{0.655091in}}{\pgfqpoint{2.886974in}{0.663328in}}%
\pgfpathcurveto{\pgfqpoint{2.895211in}{0.671564in}}{\pgfqpoint{2.899838in}{0.682736in}}{\pgfqpoint{2.899838in}{0.694384in}}%
\pgfpathcurveto{\pgfqpoint{2.899838in}{0.706032in}}{\pgfqpoint{2.895211in}{0.717204in}}{\pgfqpoint{2.886974in}{0.725441in}}%
\pgfpathcurveto{\pgfqpoint{2.878738in}{0.733677in}}{\pgfqpoint{2.867566in}{0.738305in}}{\pgfqpoint{2.855918in}{0.738305in}}%
\pgfpathcurveto{\pgfqpoint{2.844270in}{0.738305in}}{\pgfqpoint{2.833098in}{0.733677in}}{\pgfqpoint{2.824861in}{0.725441in}}%
\pgfpathcurveto{\pgfqpoint{2.816625in}{0.717204in}}{\pgfqpoint{2.811997in}{0.706032in}}{\pgfqpoint{2.811997in}{0.694384in}}%
\pgfpathcurveto{\pgfqpoint{2.811997in}{0.682736in}}{\pgfqpoint{2.816625in}{0.671564in}}{\pgfqpoint{2.824861in}{0.663328in}}%
\pgfpathcurveto{\pgfqpoint{2.833098in}{0.655091in}}{\pgfqpoint{2.844270in}{0.650464in}}{\pgfqpoint{2.855918in}{0.650464in}}%
\pgfpathlineto{\pgfqpoint{2.855918in}{0.650464in}}%
\pgfpathclose%
\pgfusepath{stroke,fill}%
\end{pgfscope}%
\begin{pgfscope}%
\pgfpathrectangle{\pgfqpoint{2.717318in}{0.529443in}}{\pgfqpoint{1.636659in}{1.745990in}}%
\pgfusepath{clip}%
\pgfsetbuttcap%
\pgfsetroundjoin%
\definecolor{currentfill}{rgb}{0.003922,0.003922,0.003922}%
\pgfsetfillcolor{currentfill}%
\pgfsetfillopacity{0.900000}%
\pgfsetlinewidth{0.507862pt}%
\definecolor{currentstroke}{rgb}{1.000000,1.000000,1.000000}%
\pgfsetstrokecolor{currentstroke}%
\pgfsetstrokeopacity{0.900000}%
\pgfsetdash{}{0pt}%
\pgfpathmoveto{\pgfqpoint{4.219632in}{2.002631in}}%
\pgfpathcurveto{\pgfqpoint{4.231280in}{2.002631in}}{\pgfqpoint{4.242452in}{2.007259in}}{\pgfqpoint{4.250689in}{2.015495in}}%
\pgfpathcurveto{\pgfqpoint{4.258925in}{2.023731in}}{\pgfqpoint{4.263553in}{2.034904in}}{\pgfqpoint{4.263553in}{2.046552in}}%
\pgfpathcurveto{\pgfqpoint{4.263553in}{2.058200in}}{\pgfqpoint{4.258925in}{2.069372in}}{\pgfqpoint{4.250689in}{2.077608in}}%
\pgfpathcurveto{\pgfqpoint{4.242452in}{2.085844in}}{\pgfqpoint{4.231280in}{2.090472in}}{\pgfqpoint{4.219632in}{2.090472in}}%
\pgfpathcurveto{\pgfqpoint{4.207984in}{2.090472in}}{\pgfqpoint{4.196812in}{2.085844in}}{\pgfqpoint{4.188576in}{2.077608in}}%
\pgfpathcurveto{\pgfqpoint{4.180339in}{2.069372in}}{\pgfqpoint{4.175711in}{2.058200in}}{\pgfqpoint{4.175711in}{2.046552in}}%
\pgfpathcurveto{\pgfqpoint{4.175711in}{2.034904in}}{\pgfqpoint{4.180339in}{2.023731in}}{\pgfqpoint{4.188576in}{2.015495in}}%
\pgfpathcurveto{\pgfqpoint{4.196812in}{2.007259in}}{\pgfqpoint{4.207984in}{2.002631in}}{\pgfqpoint{4.219632in}{2.002631in}}%
\pgfpathlineto{\pgfqpoint{4.219632in}{2.002631in}}%
\pgfpathclose%
\pgfusepath{stroke,fill}%
\end{pgfscope}%
\begin{pgfscope}%
\pgfpathrectangle{\pgfqpoint{2.717318in}{0.529443in}}{\pgfqpoint{1.636659in}{1.745990in}}%
\pgfusepath{clip}%
\pgfsetbuttcap%
\pgfsetroundjoin%
\definecolor{currentfill}{rgb}{0.003922,0.003922,0.003922}%
\pgfsetfillcolor{currentfill}%
\pgfsetfillopacity{0.900000}%
\pgfsetlinewidth{0.507862pt}%
\definecolor{currentstroke}{rgb}{1.000000,1.000000,1.000000}%
\pgfsetstrokecolor{currentstroke}%
\pgfsetstrokeopacity{0.900000}%
\pgfsetdash{}{0pt}%
\pgfpathmoveto{\pgfqpoint{4.226048in}{2.152149in}}%
\pgfpathcurveto{\pgfqpoint{4.237696in}{2.152149in}}{\pgfqpoint{4.248869in}{2.156777in}}{\pgfqpoint{4.257105in}{2.165013in}}%
\pgfpathcurveto{\pgfqpoint{4.265341in}{2.173249in}}{\pgfqpoint{4.269969in}{2.184422in}}{\pgfqpoint{4.269969in}{2.196069in}}%
\pgfpathcurveto{\pgfqpoint{4.269969in}{2.207717in}}{\pgfqpoint{4.265341in}{2.218890in}}{\pgfqpoint{4.257105in}{2.227126in}}%
\pgfpathcurveto{\pgfqpoint{4.248869in}{2.235362in}}{\pgfqpoint{4.237696in}{2.239990in}}{\pgfqpoint{4.226048in}{2.239990in}}%
\pgfpathcurveto{\pgfqpoint{4.214401in}{2.239990in}}{\pgfqpoint{4.203228in}{2.235362in}}{\pgfqpoint{4.194992in}{2.227126in}}%
\pgfpathcurveto{\pgfqpoint{4.186756in}{2.218890in}}{\pgfqpoint{4.182128in}{2.207717in}}{\pgfqpoint{4.182128in}{2.196069in}}%
\pgfpathcurveto{\pgfqpoint{4.182128in}{2.184422in}}{\pgfqpoint{4.186756in}{2.173249in}}{\pgfqpoint{4.194992in}{2.165013in}}%
\pgfpathcurveto{\pgfqpoint{4.203228in}{2.156777in}}{\pgfqpoint{4.214401in}{2.152149in}}{\pgfqpoint{4.226048in}{2.152149in}}%
\pgfpathlineto{\pgfqpoint{4.226048in}{2.152149in}}%
\pgfpathclose%
\pgfusepath{stroke,fill}%
\end{pgfscope}%
\begin{pgfscope}%
\pgfpathrectangle{\pgfqpoint{2.717318in}{0.529443in}}{\pgfqpoint{1.636659in}{1.745990in}}%
\pgfusepath{clip}%
\pgfsetbuttcap%
\pgfsetroundjoin%
\definecolor{currentfill}{rgb}{0.003922,0.003922,0.003922}%
\pgfsetfillcolor{currentfill}%
\pgfsetfillopacity{0.900000}%
\pgfsetlinewidth{0.507862pt}%
\definecolor{currentstroke}{rgb}{1.000000,1.000000,1.000000}%
\pgfsetstrokecolor{currentstroke}%
\pgfsetstrokeopacity{0.900000}%
\pgfsetdash{}{0pt}%
\pgfpathmoveto{\pgfqpoint{4.245421in}{2.035035in}}%
\pgfpathcurveto{\pgfqpoint{4.257069in}{2.035035in}}{\pgfqpoint{4.268242in}{2.039663in}}{\pgfqpoint{4.276478in}{2.047899in}}%
\pgfpathcurveto{\pgfqpoint{4.284714in}{2.056136in}}{\pgfqpoint{4.289342in}{2.067308in}}{\pgfqpoint{4.289342in}{2.078956in}}%
\pgfpathcurveto{\pgfqpoint{4.289342in}{2.090604in}}{\pgfqpoint{4.284714in}{2.101776in}}{\pgfqpoint{4.276478in}{2.110012in}}%
\pgfpathcurveto{\pgfqpoint{4.268242in}{2.118249in}}{\pgfqpoint{4.257069in}{2.122876in}}{\pgfqpoint{4.245421in}{2.122876in}}%
\pgfpathcurveto{\pgfqpoint{4.233773in}{2.122876in}}{\pgfqpoint{4.222601in}{2.118249in}}{\pgfqpoint{4.214365in}{2.110012in}}%
\pgfpathcurveto{\pgfqpoint{4.206129in}{2.101776in}}{\pgfqpoint{4.201501in}{2.090604in}}{\pgfqpoint{4.201501in}{2.078956in}}%
\pgfpathcurveto{\pgfqpoint{4.201501in}{2.067308in}}{\pgfqpoint{4.206129in}{2.056136in}}{\pgfqpoint{4.214365in}{2.047899in}}%
\pgfpathcurveto{\pgfqpoint{4.222601in}{2.039663in}}{\pgfqpoint{4.233773in}{2.035035in}}{\pgfqpoint{4.245421in}{2.035035in}}%
\pgfpathlineto{\pgfqpoint{4.245421in}{2.035035in}}%
\pgfpathclose%
\pgfusepath{stroke,fill}%
\end{pgfscope}%
\begin{pgfscope}%
\pgfpathrectangle{\pgfqpoint{2.717318in}{0.529443in}}{\pgfqpoint{1.636659in}{1.745990in}}%
\pgfusepath{clip}%
\pgfsetbuttcap%
\pgfsetroundjoin%
\definecolor{currentfill}{rgb}{0.003922,0.003922,0.003922}%
\pgfsetfillcolor{currentfill}%
\pgfsetfillopacity{0.900000}%
\pgfsetlinewidth{0.507862pt}%
\definecolor{currentstroke}{rgb}{1.000000,1.000000,1.000000}%
\pgfsetstrokecolor{currentstroke}%
\pgfsetstrokeopacity{0.900000}%
\pgfsetdash{}{0pt}%
\pgfpathmoveto{\pgfqpoint{4.212553in}{2.131189in}}%
\pgfpathcurveto{\pgfqpoint{4.224201in}{2.131189in}}{\pgfqpoint{4.235374in}{2.135817in}}{\pgfqpoint{4.243610in}{2.144053in}}%
\pgfpathcurveto{\pgfqpoint{4.251846in}{2.152289in}}{\pgfqpoint{4.256474in}{2.163461in}}{\pgfqpoint{4.256474in}{2.175109in}}%
\pgfpathcurveto{\pgfqpoint{4.256474in}{2.186757in}}{\pgfqpoint{4.251846in}{2.197930in}}{\pgfqpoint{4.243610in}{2.206166in}}%
\pgfpathcurveto{\pgfqpoint{4.235374in}{2.214402in}}{\pgfqpoint{4.224201in}{2.219030in}}{\pgfqpoint{4.212553in}{2.219030in}}%
\pgfpathcurveto{\pgfqpoint{4.200906in}{2.219030in}}{\pgfqpoint{4.189733in}{2.214402in}}{\pgfqpoint{4.181497in}{2.206166in}}%
\pgfpathcurveto{\pgfqpoint{4.173261in}{2.197930in}}{\pgfqpoint{4.168633in}{2.186757in}}{\pgfqpoint{4.168633in}{2.175109in}}%
\pgfpathcurveto{\pgfqpoint{4.168633in}{2.163461in}}{\pgfqpoint{4.173261in}{2.152289in}}{\pgfqpoint{4.181497in}{2.144053in}}%
\pgfpathcurveto{\pgfqpoint{4.189733in}{2.135817in}}{\pgfqpoint{4.200906in}{2.131189in}}{\pgfqpoint{4.212553in}{2.131189in}}%
\pgfpathlineto{\pgfqpoint{4.212553in}{2.131189in}}%
\pgfpathclose%
\pgfusepath{stroke,fill}%
\end{pgfscope}%
\begin{pgfscope}%
\pgfpathrectangle{\pgfqpoint{2.717318in}{0.529443in}}{\pgfqpoint{1.636659in}{1.745990in}}%
\pgfusepath{clip}%
\pgfsetbuttcap%
\pgfsetroundjoin%
\definecolor{currentfill}{rgb}{0.003922,0.003922,0.003922}%
\pgfsetfillcolor{currentfill}%
\pgfsetfillopacity{0.900000}%
\pgfsetlinewidth{0.507862pt}%
\definecolor{currentstroke}{rgb}{1.000000,1.000000,1.000000}%
\pgfsetstrokecolor{currentstroke}%
\pgfsetstrokeopacity{0.900000}%
\pgfsetdash{}{0pt}%
\pgfpathmoveto{\pgfqpoint{2.835068in}{0.564886in}}%
\pgfpathcurveto{\pgfqpoint{2.846716in}{0.564886in}}{\pgfqpoint{2.857888in}{0.569513in}}{\pgfqpoint{2.866125in}{0.577750in}}%
\pgfpathcurveto{\pgfqpoint{2.874361in}{0.585986in}}{\pgfqpoint{2.878989in}{0.597158in}}{\pgfqpoint{2.878989in}{0.608806in}}%
\pgfpathcurveto{\pgfqpoint{2.878989in}{0.620454in}}{\pgfqpoint{2.874361in}{0.631626in}}{\pgfqpoint{2.866125in}{0.639863in}}%
\pgfpathcurveto{\pgfqpoint{2.857888in}{0.648099in}}{\pgfqpoint{2.846716in}{0.652727in}}{\pgfqpoint{2.835068in}{0.652727in}}%
\pgfpathcurveto{\pgfqpoint{2.823420in}{0.652727in}}{\pgfqpoint{2.812248in}{0.648099in}}{\pgfqpoint{2.804012in}{0.639863in}}%
\pgfpathcurveto{\pgfqpoint{2.795775in}{0.631626in}}{\pgfqpoint{2.791148in}{0.620454in}}{\pgfqpoint{2.791148in}{0.608806in}}%
\pgfpathcurveto{\pgfqpoint{2.791148in}{0.597158in}}{\pgfqpoint{2.795775in}{0.585986in}}{\pgfqpoint{2.804012in}{0.577750in}}%
\pgfpathcurveto{\pgfqpoint{2.812248in}{0.569513in}}{\pgfqpoint{2.823420in}{0.564886in}}{\pgfqpoint{2.835068in}{0.564886in}}%
\pgfpathlineto{\pgfqpoint{2.835068in}{0.564886in}}%
\pgfpathclose%
\pgfusepath{stroke,fill}%
\end{pgfscope}%
\begin{pgfscope}%
\pgfpathrectangle{\pgfqpoint{2.717318in}{0.529443in}}{\pgfqpoint{1.636659in}{1.745990in}}%
\pgfusepath{clip}%
\pgfsetbuttcap%
\pgfsetroundjoin%
\definecolor{currentfill}{rgb}{0.003922,0.003922,0.003922}%
\pgfsetfillcolor{currentfill}%
\pgfsetfillopacity{0.900000}%
\pgfsetlinewidth{0.507862pt}%
\definecolor{currentstroke}{rgb}{1.000000,1.000000,1.000000}%
\pgfsetstrokecolor{currentstroke}%
\pgfsetstrokeopacity{0.900000}%
\pgfsetdash{}{0pt}%
\pgfpathmoveto{\pgfqpoint{2.839483in}{0.570893in}}%
\pgfpathcurveto{\pgfqpoint{2.851131in}{0.570893in}}{\pgfqpoint{2.862303in}{0.575521in}}{\pgfqpoint{2.870539in}{0.583757in}}%
\pgfpathcurveto{\pgfqpoint{2.878775in}{0.591993in}}{\pgfqpoint{2.883403in}{0.603166in}}{\pgfqpoint{2.883403in}{0.614814in}}%
\pgfpathcurveto{\pgfqpoint{2.883403in}{0.626462in}}{\pgfqpoint{2.878775in}{0.637634in}}{\pgfqpoint{2.870539in}{0.645870in}}%
\pgfpathcurveto{\pgfqpoint{2.862303in}{0.654106in}}{\pgfqpoint{2.851131in}{0.658734in}}{\pgfqpoint{2.839483in}{0.658734in}}%
\pgfpathcurveto{\pgfqpoint{2.827835in}{0.658734in}}{\pgfqpoint{2.816662in}{0.654106in}}{\pgfqpoint{2.808426in}{0.645870in}}%
\pgfpathcurveto{\pgfqpoint{2.800190in}{0.637634in}}{\pgfqpoint{2.795562in}{0.626462in}}{\pgfqpoint{2.795562in}{0.614814in}}%
\pgfpathcurveto{\pgfqpoint{2.795562in}{0.603166in}}{\pgfqpoint{2.800190in}{0.591993in}}{\pgfqpoint{2.808426in}{0.583757in}}%
\pgfpathcurveto{\pgfqpoint{2.816662in}{0.575521in}}{\pgfqpoint{2.827835in}{0.570893in}}{\pgfqpoint{2.839483in}{0.570893in}}%
\pgfpathlineto{\pgfqpoint{2.839483in}{0.570893in}}%
\pgfpathclose%
\pgfusepath{stroke,fill}%
\end{pgfscope}%
\begin{pgfscope}%
\pgfpathrectangle{\pgfqpoint{2.717318in}{0.529443in}}{\pgfqpoint{1.636659in}{1.745990in}}%
\pgfusepath{clip}%
\pgfsetbuttcap%
\pgfsetroundjoin%
\definecolor{currentfill}{rgb}{0.003922,0.003922,0.003922}%
\pgfsetfillcolor{currentfill}%
\pgfsetfillopacity{0.900000}%
\pgfsetlinewidth{0.507862pt}%
\definecolor{currentstroke}{rgb}{1.000000,1.000000,1.000000}%
\pgfsetstrokecolor{currentstroke}%
\pgfsetstrokeopacity{0.900000}%
\pgfsetdash{}{0pt}%
\pgfpathmoveto{\pgfqpoint{4.246346in}{2.132714in}}%
\pgfpathcurveto{\pgfqpoint{4.257994in}{2.132714in}}{\pgfqpoint{4.269166in}{2.137342in}}{\pgfqpoint{4.277403in}{2.145578in}}%
\pgfpathcurveto{\pgfqpoint{4.285639in}{2.153814in}}{\pgfqpoint{4.290267in}{2.164987in}}{\pgfqpoint{4.290267in}{2.176635in}}%
\pgfpathcurveto{\pgfqpoint{4.290267in}{2.188282in}}{\pgfqpoint{4.285639in}{2.199455in}}{\pgfqpoint{4.277403in}{2.207691in}}%
\pgfpathcurveto{\pgfqpoint{4.269166in}{2.215927in}}{\pgfqpoint{4.257994in}{2.220555in}}{\pgfqpoint{4.246346in}{2.220555in}}%
\pgfpathcurveto{\pgfqpoint{4.234698in}{2.220555in}}{\pgfqpoint{4.223526in}{2.215927in}}{\pgfqpoint{4.215290in}{2.207691in}}%
\pgfpathcurveto{\pgfqpoint{4.207053in}{2.199455in}}{\pgfqpoint{4.202426in}{2.188282in}}{\pgfqpoint{4.202426in}{2.176635in}}%
\pgfpathcurveto{\pgfqpoint{4.202426in}{2.164987in}}{\pgfqpoint{4.207053in}{2.153814in}}{\pgfqpoint{4.215290in}{2.145578in}}%
\pgfpathcurveto{\pgfqpoint{4.223526in}{2.137342in}}{\pgfqpoint{4.234698in}{2.132714in}}{\pgfqpoint{4.246346in}{2.132714in}}%
\pgfpathlineto{\pgfqpoint{4.246346in}{2.132714in}}%
\pgfpathclose%
\pgfusepath{stroke,fill}%
\end{pgfscope}%
\begin{pgfscope}%
\pgfpathrectangle{\pgfqpoint{2.717318in}{0.529443in}}{\pgfqpoint{1.636659in}{1.745990in}}%
\pgfusepath{clip}%
\pgfsetbuttcap%
\pgfsetroundjoin%
\definecolor{currentfill}{rgb}{0.003922,0.003922,0.003922}%
\pgfsetfillcolor{currentfill}%
\pgfsetfillopacity{0.900000}%
\pgfsetlinewidth{0.507862pt}%
\definecolor{currentstroke}{rgb}{1.000000,1.000000,1.000000}%
\pgfsetstrokecolor{currentstroke}%
\pgfsetstrokeopacity{0.900000}%
\pgfsetdash{}{0pt}%
\pgfpathmoveto{\pgfqpoint{2.796601in}{0.675423in}}%
\pgfpathcurveto{\pgfqpoint{2.808249in}{0.675423in}}{\pgfqpoint{2.819421in}{0.680051in}}{\pgfqpoint{2.827658in}{0.688287in}}%
\pgfpathcurveto{\pgfqpoint{2.835894in}{0.696524in}}{\pgfqpoint{2.840522in}{0.707696in}}{\pgfqpoint{2.840522in}{0.719344in}}%
\pgfpathcurveto{\pgfqpoint{2.840522in}{0.730992in}}{\pgfqpoint{2.835894in}{0.742164in}}{\pgfqpoint{2.827658in}{0.750400in}}%
\pgfpathcurveto{\pgfqpoint{2.819421in}{0.758637in}}{\pgfqpoint{2.808249in}{0.763264in}}{\pgfqpoint{2.796601in}{0.763264in}}%
\pgfpathcurveto{\pgfqpoint{2.784953in}{0.763264in}}{\pgfqpoint{2.773781in}{0.758637in}}{\pgfqpoint{2.765545in}{0.750400in}}%
\pgfpathcurveto{\pgfqpoint{2.757308in}{0.742164in}}{\pgfqpoint{2.752680in}{0.730992in}}{\pgfqpoint{2.752680in}{0.719344in}}%
\pgfpathcurveto{\pgfqpoint{2.752680in}{0.707696in}}{\pgfqpoint{2.757308in}{0.696524in}}{\pgfqpoint{2.765545in}{0.688287in}}%
\pgfpathcurveto{\pgfqpoint{2.773781in}{0.680051in}}{\pgfqpoint{2.784953in}{0.675423in}}{\pgfqpoint{2.796601in}{0.675423in}}%
\pgfpathlineto{\pgfqpoint{2.796601in}{0.675423in}}%
\pgfpathclose%
\pgfusepath{stroke,fill}%
\end{pgfscope}%
\begin{pgfscope}%
\pgfpathrectangle{\pgfqpoint{2.717318in}{0.529443in}}{\pgfqpoint{1.636659in}{1.745990in}}%
\pgfusepath{clip}%
\pgfsetbuttcap%
\pgfsetroundjoin%
\definecolor{currentfill}{rgb}{0.003922,0.003922,0.003922}%
\pgfsetfillcolor{currentfill}%
\pgfsetfillopacity{0.900000}%
\pgfsetlinewidth{0.507862pt}%
\definecolor{currentstroke}{rgb}{1.000000,1.000000,1.000000}%
\pgfsetstrokecolor{currentstroke}%
\pgfsetstrokeopacity{0.900000}%
\pgfsetdash{}{0pt}%
\pgfpathmoveto{\pgfqpoint{4.239554in}{2.098632in}}%
\pgfpathcurveto{\pgfqpoint{4.251202in}{2.098632in}}{\pgfqpoint{4.262374in}{2.103260in}}{\pgfqpoint{4.270611in}{2.111496in}}%
\pgfpathcurveto{\pgfqpoint{4.278847in}{2.119732in}}{\pgfqpoint{4.283475in}{2.130905in}}{\pgfqpoint{4.283475in}{2.142552in}}%
\pgfpathcurveto{\pgfqpoint{4.283475in}{2.154200in}}{\pgfqpoint{4.278847in}{2.165373in}}{\pgfqpoint{4.270611in}{2.173609in}}%
\pgfpathcurveto{\pgfqpoint{4.262374in}{2.181845in}}{\pgfqpoint{4.251202in}{2.186473in}}{\pgfqpoint{4.239554in}{2.186473in}}%
\pgfpathcurveto{\pgfqpoint{4.227906in}{2.186473in}}{\pgfqpoint{4.216734in}{2.181845in}}{\pgfqpoint{4.208498in}{2.173609in}}%
\pgfpathcurveto{\pgfqpoint{4.200261in}{2.165373in}}{\pgfqpoint{4.195634in}{2.154200in}}{\pgfqpoint{4.195634in}{2.142552in}}%
\pgfpathcurveto{\pgfqpoint{4.195634in}{2.130905in}}{\pgfqpoint{4.200261in}{2.119732in}}{\pgfqpoint{4.208498in}{2.111496in}}%
\pgfpathcurveto{\pgfqpoint{4.216734in}{2.103260in}}{\pgfqpoint{4.227906in}{2.098632in}}{\pgfqpoint{4.239554in}{2.098632in}}%
\pgfpathlineto{\pgfqpoint{4.239554in}{2.098632in}}%
\pgfpathclose%
\pgfusepath{stroke,fill}%
\end{pgfscope}%
\begin{pgfscope}%
\pgfpathrectangle{\pgfqpoint{2.717318in}{0.529443in}}{\pgfqpoint{1.636659in}{1.745990in}}%
\pgfusepath{clip}%
\pgfsetbuttcap%
\pgfsetroundjoin%
\definecolor{currentfill}{rgb}{0.003922,0.003922,0.003922}%
\pgfsetfillcolor{currentfill}%
\pgfsetfillopacity{0.900000}%
\pgfsetlinewidth{0.507862pt}%
\definecolor{currentstroke}{rgb}{1.000000,1.000000,1.000000}%
\pgfsetstrokecolor{currentstroke}%
\pgfsetstrokeopacity{0.900000}%
\pgfsetdash{}{0pt}%
\pgfpathmoveto{\pgfqpoint{2.843908in}{0.701058in}}%
\pgfpathcurveto{\pgfqpoint{2.855556in}{0.701058in}}{\pgfqpoint{2.866728in}{0.705685in}}{\pgfqpoint{2.874965in}{0.713922in}}%
\pgfpathcurveto{\pgfqpoint{2.883201in}{0.722158in}}{\pgfqpoint{2.887829in}{0.733330in}}{\pgfqpoint{2.887829in}{0.744978in}}%
\pgfpathcurveto{\pgfqpoint{2.887829in}{0.756626in}}{\pgfqpoint{2.883201in}{0.767798in}}{\pgfqpoint{2.874965in}{0.776035in}}%
\pgfpathcurveto{\pgfqpoint{2.866728in}{0.784271in}}{\pgfqpoint{2.855556in}{0.788899in}}{\pgfqpoint{2.843908in}{0.788899in}}%
\pgfpathcurveto{\pgfqpoint{2.832260in}{0.788899in}}{\pgfqpoint{2.821088in}{0.784271in}}{\pgfqpoint{2.812852in}{0.776035in}}%
\pgfpathcurveto{\pgfqpoint{2.804615in}{0.767798in}}{\pgfqpoint{2.799987in}{0.756626in}}{\pgfqpoint{2.799987in}{0.744978in}}%
\pgfpathcurveto{\pgfqpoint{2.799987in}{0.733330in}}{\pgfqpoint{2.804615in}{0.722158in}}{\pgfqpoint{2.812852in}{0.713922in}}%
\pgfpathcurveto{\pgfqpoint{2.821088in}{0.705685in}}{\pgfqpoint{2.832260in}{0.701058in}}{\pgfqpoint{2.843908in}{0.701058in}}%
\pgfpathlineto{\pgfqpoint{2.843908in}{0.701058in}}%
\pgfpathclose%
\pgfusepath{stroke,fill}%
\end{pgfscope}%
\begin{pgfscope}%
\pgfpathrectangle{\pgfqpoint{2.717318in}{0.529443in}}{\pgfqpoint{1.636659in}{1.745990in}}%
\pgfusepath{clip}%
\pgfsetbuttcap%
\pgfsetroundjoin%
\definecolor{currentfill}{rgb}{0.003922,0.003922,0.003922}%
\pgfsetfillcolor{currentfill}%
\pgfsetfillopacity{0.900000}%
\pgfsetlinewidth{0.507862pt}%
\definecolor{currentstroke}{rgb}{1.000000,1.000000,1.000000}%
\pgfsetstrokecolor{currentstroke}%
\pgfsetstrokeopacity{0.900000}%
\pgfsetdash{}{0pt}%
\pgfpathmoveto{\pgfqpoint{2.791711in}{0.652968in}}%
\pgfpathcurveto{\pgfqpoint{2.803359in}{0.652968in}}{\pgfqpoint{2.814531in}{0.657596in}}{\pgfqpoint{2.822768in}{0.665832in}}%
\pgfpathcurveto{\pgfqpoint{2.831004in}{0.674068in}}{\pgfqpoint{2.835632in}{0.685241in}}{\pgfqpoint{2.835632in}{0.696889in}}%
\pgfpathcurveto{\pgfqpoint{2.835632in}{0.708536in}}{\pgfqpoint{2.831004in}{0.719709in}}{\pgfqpoint{2.822768in}{0.727945in}}%
\pgfpathcurveto{\pgfqpoint{2.814531in}{0.736181in}}{\pgfqpoint{2.803359in}{0.740809in}}{\pgfqpoint{2.791711in}{0.740809in}}%
\pgfpathcurveto{\pgfqpoint{2.780063in}{0.740809in}}{\pgfqpoint{2.768891in}{0.736181in}}{\pgfqpoint{2.760655in}{0.727945in}}%
\pgfpathcurveto{\pgfqpoint{2.752418in}{0.719709in}}{\pgfqpoint{2.747791in}{0.708536in}}{\pgfqpoint{2.747791in}{0.696889in}}%
\pgfpathcurveto{\pgfqpoint{2.747791in}{0.685241in}}{\pgfqpoint{2.752418in}{0.674068in}}{\pgfqpoint{2.760655in}{0.665832in}}%
\pgfpathcurveto{\pgfqpoint{2.768891in}{0.657596in}}{\pgfqpoint{2.780063in}{0.652968in}}{\pgfqpoint{2.791711in}{0.652968in}}%
\pgfpathlineto{\pgfqpoint{2.791711in}{0.652968in}}%
\pgfpathclose%
\pgfusepath{stroke,fill}%
\end{pgfscope}%
\begin{pgfscope}%
\pgfpathrectangle{\pgfqpoint{2.717318in}{0.529443in}}{\pgfqpoint{1.636659in}{1.745990in}}%
\pgfusepath{clip}%
\pgfsetbuttcap%
\pgfsetroundjoin%
\definecolor{currentfill}{rgb}{0.003922,0.003922,0.003922}%
\pgfsetfillcolor{currentfill}%
\pgfsetfillopacity{0.900000}%
\pgfsetlinewidth{0.507862pt}%
\definecolor{currentstroke}{rgb}{1.000000,1.000000,1.000000}%
\pgfsetstrokecolor{currentstroke}%
\pgfsetstrokeopacity{0.900000}%
\pgfsetdash{}{0pt}%
\pgfpathmoveto{\pgfqpoint{2.812514in}{0.739555in}}%
\pgfpathcurveto{\pgfqpoint{2.824162in}{0.739555in}}{\pgfqpoint{2.835335in}{0.744183in}}{\pgfqpoint{2.843571in}{0.752419in}}%
\pgfpathcurveto{\pgfqpoint{2.851807in}{0.760655in}}{\pgfqpoint{2.856435in}{0.771828in}}{\pgfqpoint{2.856435in}{0.783475in}}%
\pgfpathcurveto{\pgfqpoint{2.856435in}{0.795123in}}{\pgfqpoint{2.851807in}{0.806296in}}{\pgfqpoint{2.843571in}{0.814532in}}%
\pgfpathcurveto{\pgfqpoint{2.835335in}{0.822768in}}{\pgfqpoint{2.824162in}{0.827396in}}{\pgfqpoint{2.812514in}{0.827396in}}%
\pgfpathcurveto{\pgfqpoint{2.800867in}{0.827396in}}{\pgfqpoint{2.789694in}{0.822768in}}{\pgfqpoint{2.781458in}{0.814532in}}%
\pgfpathcurveto{\pgfqpoint{2.773222in}{0.806296in}}{\pgfqpoint{2.768594in}{0.795123in}}{\pgfqpoint{2.768594in}{0.783475in}}%
\pgfpathcurveto{\pgfqpoint{2.768594in}{0.771828in}}{\pgfqpoint{2.773222in}{0.760655in}}{\pgfqpoint{2.781458in}{0.752419in}}%
\pgfpathcurveto{\pgfqpoint{2.789694in}{0.744183in}}{\pgfqpoint{2.800867in}{0.739555in}}{\pgfqpoint{2.812514in}{0.739555in}}%
\pgfpathlineto{\pgfqpoint{2.812514in}{0.739555in}}%
\pgfpathclose%
\pgfusepath{stroke,fill}%
\end{pgfscope}%
\begin{pgfscope}%
\pgfpathrectangle{\pgfqpoint{2.717318in}{0.529443in}}{\pgfqpoint{1.636659in}{1.745990in}}%
\pgfusepath{clip}%
\pgfsetbuttcap%
\pgfsetroundjoin%
\definecolor{currentfill}{rgb}{0.003922,0.003922,0.003922}%
\pgfsetfillcolor{currentfill}%
\pgfsetfillopacity{0.900000}%
\pgfsetlinewidth{0.507862pt}%
\definecolor{currentstroke}{rgb}{1.000000,1.000000,1.000000}%
\pgfsetstrokecolor{currentstroke}%
\pgfsetstrokeopacity{0.900000}%
\pgfsetdash{}{0pt}%
\pgfpathmoveto{\pgfqpoint{4.206833in}{1.942492in}}%
\pgfpathcurveto{\pgfqpoint{4.218480in}{1.942492in}}{\pgfqpoint{4.229653in}{1.947120in}}{\pgfqpoint{4.237889in}{1.955356in}}%
\pgfpathcurveto{\pgfqpoint{4.246125in}{1.963592in}}{\pgfqpoint{4.250753in}{1.974764in}}{\pgfqpoint{4.250753in}{1.986412in}}%
\pgfpathcurveto{\pgfqpoint{4.250753in}{1.998060in}}{\pgfqpoint{4.246125in}{2.009233in}}{\pgfqpoint{4.237889in}{2.017469in}}%
\pgfpathcurveto{\pgfqpoint{4.229653in}{2.025705in}}{\pgfqpoint{4.218480in}{2.030333in}}{\pgfqpoint{4.206833in}{2.030333in}}%
\pgfpathcurveto{\pgfqpoint{4.195185in}{2.030333in}}{\pgfqpoint{4.184012in}{2.025705in}}{\pgfqpoint{4.175776in}{2.017469in}}%
\pgfpathcurveto{\pgfqpoint{4.167540in}{2.009233in}}{\pgfqpoint{4.162912in}{1.998060in}}{\pgfqpoint{4.162912in}{1.986412in}}%
\pgfpathcurveto{\pgfqpoint{4.162912in}{1.974764in}}{\pgfqpoint{4.167540in}{1.963592in}}{\pgfqpoint{4.175776in}{1.955356in}}%
\pgfpathcurveto{\pgfqpoint{4.184012in}{1.947120in}}{\pgfqpoint{4.195185in}{1.942492in}}{\pgfqpoint{4.206833in}{1.942492in}}%
\pgfpathlineto{\pgfqpoint{4.206833in}{1.942492in}}%
\pgfpathclose%
\pgfusepath{stroke,fill}%
\end{pgfscope}%
\begin{pgfscope}%
\pgfpathrectangle{\pgfqpoint{2.717318in}{0.529443in}}{\pgfqpoint{1.636659in}{1.745990in}}%
\pgfusepath{clip}%
\pgfsetbuttcap%
\pgfsetroundjoin%
\definecolor{currentfill}{rgb}{0.003922,0.003922,0.003922}%
\pgfsetfillcolor{currentfill}%
\pgfsetfillopacity{0.900000}%
\pgfsetlinewidth{0.507862pt}%
\definecolor{currentstroke}{rgb}{1.000000,1.000000,1.000000}%
\pgfsetstrokecolor{currentstroke}%
\pgfsetstrokeopacity{0.900000}%
\pgfsetdash{}{0pt}%
\pgfpathmoveto{\pgfqpoint{2.834599in}{0.591548in}}%
\pgfpathcurveto{\pgfqpoint{2.846247in}{0.591548in}}{\pgfqpoint{2.857419in}{0.596176in}}{\pgfqpoint{2.865655in}{0.604412in}}%
\pgfpathcurveto{\pgfqpoint{2.873892in}{0.612649in}}{\pgfqpoint{2.878519in}{0.623821in}}{\pgfqpoint{2.878519in}{0.635469in}}%
\pgfpathcurveto{\pgfqpoint{2.878519in}{0.647117in}}{\pgfqpoint{2.873892in}{0.658289in}}{\pgfqpoint{2.865655in}{0.666525in}}%
\pgfpathcurveto{\pgfqpoint{2.857419in}{0.674762in}}{\pgfqpoint{2.846247in}{0.679389in}}{\pgfqpoint{2.834599in}{0.679389in}}%
\pgfpathcurveto{\pgfqpoint{2.822951in}{0.679389in}}{\pgfqpoint{2.811779in}{0.674762in}}{\pgfqpoint{2.803542in}{0.666525in}}%
\pgfpathcurveto{\pgfqpoint{2.795306in}{0.658289in}}{\pgfqpoint{2.790678in}{0.647117in}}{\pgfqpoint{2.790678in}{0.635469in}}%
\pgfpathcurveto{\pgfqpoint{2.790678in}{0.623821in}}{\pgfqpoint{2.795306in}{0.612649in}}{\pgfqpoint{2.803542in}{0.604412in}}%
\pgfpathcurveto{\pgfqpoint{2.811779in}{0.596176in}}{\pgfqpoint{2.822951in}{0.591548in}}{\pgfqpoint{2.834599in}{0.591548in}}%
\pgfpathlineto{\pgfqpoint{2.834599in}{0.591548in}}%
\pgfpathclose%
\pgfusepath{stroke,fill}%
\end{pgfscope}%
\begin{pgfscope}%
\pgfpathrectangle{\pgfqpoint{2.717318in}{0.529443in}}{\pgfqpoint{1.636659in}{1.745990in}}%
\pgfusepath{clip}%
\pgfsetbuttcap%
\pgfsetroundjoin%
\definecolor{currentfill}{rgb}{0.003922,0.003922,0.003922}%
\pgfsetfillcolor{currentfill}%
\pgfsetfillopacity{0.900000}%
\pgfsetlinewidth{0.507862pt}%
\definecolor{currentstroke}{rgb}{1.000000,1.000000,1.000000}%
\pgfsetstrokecolor{currentstroke}%
\pgfsetstrokeopacity{0.900000}%
\pgfsetdash{}{0pt}%
\pgfpathmoveto{\pgfqpoint{4.279583in}{2.036607in}}%
\pgfpathcurveto{\pgfqpoint{4.291231in}{2.036607in}}{\pgfqpoint{4.302403in}{2.041235in}}{\pgfqpoint{4.310640in}{2.049471in}}%
\pgfpathcurveto{\pgfqpoint{4.318876in}{2.057708in}}{\pgfqpoint{4.323504in}{2.068880in}}{\pgfqpoint{4.323504in}{2.080528in}}%
\pgfpathcurveto{\pgfqpoint{4.323504in}{2.092176in}}{\pgfqpoint{4.318876in}{2.103348in}}{\pgfqpoint{4.310640in}{2.111584in}}%
\pgfpathcurveto{\pgfqpoint{4.302403in}{2.119821in}}{\pgfqpoint{4.291231in}{2.124448in}}{\pgfqpoint{4.279583in}{2.124448in}}%
\pgfpathcurveto{\pgfqpoint{4.267935in}{2.124448in}}{\pgfqpoint{4.256763in}{2.119821in}}{\pgfqpoint{4.248527in}{2.111584in}}%
\pgfpathcurveto{\pgfqpoint{4.240290in}{2.103348in}}{\pgfqpoint{4.235662in}{2.092176in}}{\pgfqpoint{4.235662in}{2.080528in}}%
\pgfpathcurveto{\pgfqpoint{4.235662in}{2.068880in}}{\pgfqpoint{4.240290in}{2.057708in}}{\pgfqpoint{4.248527in}{2.049471in}}%
\pgfpathcurveto{\pgfqpoint{4.256763in}{2.041235in}}{\pgfqpoint{4.267935in}{2.036607in}}{\pgfqpoint{4.279583in}{2.036607in}}%
\pgfpathlineto{\pgfqpoint{4.279583in}{2.036607in}}%
\pgfpathclose%
\pgfusepath{stroke,fill}%
\end{pgfscope}%
\begin{pgfscope}%
\pgfpathrectangle{\pgfqpoint{2.717318in}{0.529443in}}{\pgfqpoint{1.636659in}{1.745990in}}%
\pgfusepath{clip}%
\pgfsetbuttcap%
\pgfsetroundjoin%
\definecolor{currentfill}{rgb}{0.003922,0.003922,0.003922}%
\pgfsetfillcolor{currentfill}%
\pgfsetfillopacity{0.900000}%
\pgfsetlinewidth{0.507862pt}%
\definecolor{currentstroke}{rgb}{1.000000,1.000000,1.000000}%
\pgfsetstrokecolor{currentstroke}%
\pgfsetstrokeopacity{0.900000}%
\pgfsetdash{}{0pt}%
\pgfpathmoveto{\pgfqpoint{4.166491in}{2.120208in}}%
\pgfpathcurveto{\pgfqpoint{4.178139in}{2.120208in}}{\pgfqpoint{4.189311in}{2.124835in}}{\pgfqpoint{4.197547in}{2.133072in}}%
\pgfpathcurveto{\pgfqpoint{4.205784in}{2.141308in}}{\pgfqpoint{4.210411in}{2.152480in}}{\pgfqpoint{4.210411in}{2.164128in}}%
\pgfpathcurveto{\pgfqpoint{4.210411in}{2.175776in}}{\pgfqpoint{4.205784in}{2.186948in}}{\pgfqpoint{4.197547in}{2.195185in}}%
\pgfpathcurveto{\pgfqpoint{4.189311in}{2.203421in}}{\pgfqpoint{4.178139in}{2.208049in}}{\pgfqpoint{4.166491in}{2.208049in}}%
\pgfpathcurveto{\pgfqpoint{4.154843in}{2.208049in}}{\pgfqpoint{4.143671in}{2.203421in}}{\pgfqpoint{4.135434in}{2.195185in}}%
\pgfpathcurveto{\pgfqpoint{4.127198in}{2.186948in}}{\pgfqpoint{4.122570in}{2.175776in}}{\pgfqpoint{4.122570in}{2.164128in}}%
\pgfpathcurveto{\pgfqpoint{4.122570in}{2.152480in}}{\pgfqpoint{4.127198in}{2.141308in}}{\pgfqpoint{4.135434in}{2.133072in}}%
\pgfpathcurveto{\pgfqpoint{4.143671in}{2.124835in}}{\pgfqpoint{4.154843in}{2.120208in}}{\pgfqpoint{4.166491in}{2.120208in}}%
\pgfpathlineto{\pgfqpoint{4.166491in}{2.120208in}}%
\pgfpathclose%
\pgfusepath{stroke,fill}%
\end{pgfscope}%
\begin{pgfscope}%
\pgfpathrectangle{\pgfqpoint{2.717318in}{0.529443in}}{\pgfqpoint{1.636659in}{1.745990in}}%
\pgfusepath{clip}%
\pgfsetbuttcap%
\pgfsetroundjoin%
\definecolor{currentfill}{rgb}{0.031373,0.627451,0.913725}%
\pgfsetfillcolor{currentfill}%
\pgfsetfillopacity{0.900000}%
\pgfsetlinewidth{0.507862pt}%
\definecolor{currentstroke}{rgb}{1.000000,1.000000,1.000000}%
\pgfsetstrokecolor{currentstroke}%
\pgfsetstrokeopacity{0.900000}%
\pgfsetdash{}{0pt}%
\pgfpathmoveto{\pgfqpoint{2.830217in}{0.694409in}}%
\pgfpathcurveto{\pgfqpoint{2.841864in}{0.694409in}}{\pgfqpoint{2.853037in}{0.699036in}}{\pgfqpoint{2.861273in}{0.707273in}}%
\pgfpathcurveto{\pgfqpoint{2.869509in}{0.715509in}}{\pgfqpoint{2.874137in}{0.726681in}}{\pgfqpoint{2.874137in}{0.738329in}}%
\pgfpathcurveto{\pgfqpoint{2.874137in}{0.749977in}}{\pgfqpoint{2.869509in}{0.761149in}}{\pgfqpoint{2.861273in}{0.769386in}}%
\pgfpathcurveto{\pgfqpoint{2.853037in}{0.777622in}}{\pgfqpoint{2.841864in}{0.782250in}}{\pgfqpoint{2.830217in}{0.782250in}}%
\pgfpathcurveto{\pgfqpoint{2.818569in}{0.782250in}}{\pgfqpoint{2.807396in}{0.777622in}}{\pgfqpoint{2.799160in}{0.769386in}}%
\pgfpathcurveto{\pgfqpoint{2.790924in}{0.761149in}}{\pgfqpoint{2.786296in}{0.749977in}}{\pgfqpoint{2.786296in}{0.738329in}}%
\pgfpathcurveto{\pgfqpoint{2.786296in}{0.726681in}}{\pgfqpoint{2.790924in}{0.715509in}}{\pgfqpoint{2.799160in}{0.707273in}}%
\pgfpathcurveto{\pgfqpoint{2.807396in}{0.699036in}}{\pgfqpoint{2.818569in}{0.694409in}}{\pgfqpoint{2.830217in}{0.694409in}}%
\pgfpathlineto{\pgfqpoint{2.830217in}{0.694409in}}%
\pgfpathclose%
\pgfusepath{stroke,fill}%
\end{pgfscope}%
\begin{pgfscope}%
\pgfpathrectangle{\pgfqpoint{2.717318in}{0.529443in}}{\pgfqpoint{1.636659in}{1.745990in}}%
\pgfusepath{clip}%
\pgfsetbuttcap%
\pgfsetroundjoin%
\definecolor{currentfill}{rgb}{0.031373,0.627451,0.913725}%
\pgfsetfillcolor{currentfill}%
\pgfsetfillopacity{0.900000}%
\pgfsetlinewidth{0.507862pt}%
\definecolor{currentstroke}{rgb}{1.000000,1.000000,1.000000}%
\pgfsetstrokecolor{currentstroke}%
\pgfsetstrokeopacity{0.900000}%
\pgfsetdash{}{0pt}%
\pgfpathmoveto{\pgfqpoint{2.816257in}{0.572122in}}%
\pgfpathcurveto{\pgfqpoint{2.827905in}{0.572122in}}{\pgfqpoint{2.839077in}{0.576750in}}{\pgfqpoint{2.847313in}{0.584986in}}%
\pgfpathcurveto{\pgfqpoint{2.855550in}{0.593223in}}{\pgfqpoint{2.860177in}{0.604395in}}{\pgfqpoint{2.860177in}{0.616043in}}%
\pgfpathcurveto{\pgfqpoint{2.860177in}{0.627691in}}{\pgfqpoint{2.855550in}{0.638863in}}{\pgfqpoint{2.847313in}{0.647099in}}%
\pgfpathcurveto{\pgfqpoint{2.839077in}{0.655336in}}{\pgfqpoint{2.827905in}{0.659963in}}{\pgfqpoint{2.816257in}{0.659963in}}%
\pgfpathcurveto{\pgfqpoint{2.804609in}{0.659963in}}{\pgfqpoint{2.793437in}{0.655336in}}{\pgfqpoint{2.785200in}{0.647099in}}%
\pgfpathcurveto{\pgfqpoint{2.776964in}{0.638863in}}{\pgfqpoint{2.772336in}{0.627691in}}{\pgfqpoint{2.772336in}{0.616043in}}%
\pgfpathcurveto{\pgfqpoint{2.772336in}{0.604395in}}{\pgfqpoint{2.776964in}{0.593223in}}{\pgfqpoint{2.785200in}{0.584986in}}%
\pgfpathcurveto{\pgfqpoint{2.793437in}{0.576750in}}{\pgfqpoint{2.804609in}{0.572122in}}{\pgfqpoint{2.816257in}{0.572122in}}%
\pgfpathlineto{\pgfqpoint{2.816257in}{0.572122in}}%
\pgfpathclose%
\pgfusepath{stroke,fill}%
\end{pgfscope}%
\begin{pgfscope}%
\pgfpathrectangle{\pgfqpoint{2.717318in}{0.529443in}}{\pgfqpoint{1.636659in}{1.745990in}}%
\pgfusepath{clip}%
\pgfsetbuttcap%
\pgfsetroundjoin%
\definecolor{currentfill}{rgb}{0.031373,0.627451,0.913725}%
\pgfsetfillcolor{currentfill}%
\pgfsetfillopacity{0.900000}%
\pgfsetlinewidth{0.507862pt}%
\definecolor{currentstroke}{rgb}{1.000000,1.000000,1.000000}%
\pgfsetstrokecolor{currentstroke}%
\pgfsetstrokeopacity{0.900000}%
\pgfsetdash{}{0pt}%
\pgfpathmoveto{\pgfqpoint{4.211753in}{1.979995in}}%
\pgfpathcurveto{\pgfqpoint{4.223401in}{1.979995in}}{\pgfqpoint{4.234573in}{1.984623in}}{\pgfqpoint{4.242809in}{1.992859in}}%
\pgfpathcurveto{\pgfqpoint{4.251045in}{2.001095in}}{\pgfqpoint{4.255673in}{2.012268in}}{\pgfqpoint{4.255673in}{2.023915in}}%
\pgfpathcurveto{\pgfqpoint{4.255673in}{2.035563in}}{\pgfqpoint{4.251045in}{2.046736in}}{\pgfqpoint{4.242809in}{2.054972in}}%
\pgfpathcurveto{\pgfqpoint{4.234573in}{2.063208in}}{\pgfqpoint{4.223401in}{2.067836in}}{\pgfqpoint{4.211753in}{2.067836in}}%
\pgfpathcurveto{\pgfqpoint{4.200105in}{2.067836in}}{\pgfqpoint{4.188932in}{2.063208in}}{\pgfqpoint{4.180696in}{2.054972in}}%
\pgfpathcurveto{\pgfqpoint{4.172460in}{2.046736in}}{\pgfqpoint{4.167832in}{2.035563in}}{\pgfqpoint{4.167832in}{2.023915in}}%
\pgfpathcurveto{\pgfqpoint{4.167832in}{2.012268in}}{\pgfqpoint{4.172460in}{2.001095in}}{\pgfqpoint{4.180696in}{1.992859in}}%
\pgfpathcurveto{\pgfqpoint{4.188932in}{1.984623in}}{\pgfqpoint{4.200105in}{1.979995in}}{\pgfqpoint{4.211753in}{1.979995in}}%
\pgfpathlineto{\pgfqpoint{4.211753in}{1.979995in}}%
\pgfpathclose%
\pgfusepath{stroke,fill}%
\end{pgfscope}%
\begin{pgfscope}%
\pgfpathrectangle{\pgfqpoint{2.717318in}{0.529443in}}{\pgfqpoint{1.636659in}{1.745990in}}%
\pgfusepath{clip}%
\pgfsetbuttcap%
\pgfsetroundjoin%
\definecolor{currentfill}{rgb}{0.031373,0.627451,0.913725}%
\pgfsetfillcolor{currentfill}%
\pgfsetfillopacity{0.900000}%
\pgfsetlinewidth{0.507862pt}%
\definecolor{currentstroke}{rgb}{1.000000,1.000000,1.000000}%
\pgfsetstrokecolor{currentstroke}%
\pgfsetstrokeopacity{0.900000}%
\pgfsetdash{}{0pt}%
\pgfpathmoveto{\pgfqpoint{4.185473in}{1.993901in}}%
\pgfpathcurveto{\pgfqpoint{4.197121in}{1.993901in}}{\pgfqpoint{4.208294in}{1.998529in}}{\pgfqpoint{4.216530in}{2.006765in}}%
\pgfpathcurveto{\pgfqpoint{4.224766in}{2.015001in}}{\pgfqpoint{4.229394in}{2.026174in}}{\pgfqpoint{4.229394in}{2.037821in}}%
\pgfpathcurveto{\pgfqpoint{4.229394in}{2.049469in}}{\pgfqpoint{4.224766in}{2.060642in}}{\pgfqpoint{4.216530in}{2.068878in}}%
\pgfpathcurveto{\pgfqpoint{4.208294in}{2.077114in}}{\pgfqpoint{4.197121in}{2.081742in}}{\pgfqpoint{4.185473in}{2.081742in}}%
\pgfpathcurveto{\pgfqpoint{4.173825in}{2.081742in}}{\pgfqpoint{4.162653in}{2.077114in}}{\pgfqpoint{4.154417in}{2.068878in}}%
\pgfpathcurveto{\pgfqpoint{4.146181in}{2.060642in}}{\pgfqpoint{4.141553in}{2.049469in}}{\pgfqpoint{4.141553in}{2.037821in}}%
\pgfpathcurveto{\pgfqpoint{4.141553in}{2.026174in}}{\pgfqpoint{4.146181in}{2.015001in}}{\pgfqpoint{4.154417in}{2.006765in}}%
\pgfpathcurveto{\pgfqpoint{4.162653in}{1.998529in}}{\pgfqpoint{4.173825in}{1.993901in}}{\pgfqpoint{4.185473in}{1.993901in}}%
\pgfpathlineto{\pgfqpoint{4.185473in}{1.993901in}}%
\pgfpathclose%
\pgfusepath{stroke,fill}%
\end{pgfscope}%
\begin{pgfscope}%
\pgfpathrectangle{\pgfqpoint{2.717318in}{0.529443in}}{\pgfqpoint{1.636659in}{1.745990in}}%
\pgfusepath{clip}%
\pgfsetbuttcap%
\pgfsetroundjoin%
\definecolor{currentfill}{rgb}{0.031373,0.627451,0.913725}%
\pgfsetfillcolor{currentfill}%
\pgfsetfillopacity{0.900000}%
\pgfsetlinewidth{0.507862pt}%
\definecolor{currentstroke}{rgb}{1.000000,1.000000,1.000000}%
\pgfsetstrokecolor{currentstroke}%
\pgfsetstrokeopacity{0.900000}%
\pgfsetdash{}{0pt}%
\pgfpathmoveto{\pgfqpoint{4.265811in}{2.119058in}}%
\pgfpathcurveto{\pgfqpoint{4.277459in}{2.119058in}}{\pgfqpoint{4.288632in}{2.123686in}}{\pgfqpoint{4.296868in}{2.131922in}}%
\pgfpathcurveto{\pgfqpoint{4.305104in}{2.140158in}}{\pgfqpoint{4.309732in}{2.151330in}}{\pgfqpoint{4.309732in}{2.162978in}}%
\pgfpathcurveto{\pgfqpoint{4.309732in}{2.174626in}}{\pgfqpoint{4.305104in}{2.185799in}}{\pgfqpoint{4.296868in}{2.194035in}}%
\pgfpathcurveto{\pgfqpoint{4.288632in}{2.202271in}}{\pgfqpoint{4.277459in}{2.206899in}}{\pgfqpoint{4.265811in}{2.206899in}}%
\pgfpathcurveto{\pgfqpoint{4.254164in}{2.206899in}}{\pgfqpoint{4.242991in}{2.202271in}}{\pgfqpoint{4.234755in}{2.194035in}}%
\pgfpathcurveto{\pgfqpoint{4.226519in}{2.185799in}}{\pgfqpoint{4.221891in}{2.174626in}}{\pgfqpoint{4.221891in}{2.162978in}}%
\pgfpathcurveto{\pgfqpoint{4.221891in}{2.151330in}}{\pgfqpoint{4.226519in}{2.140158in}}{\pgfqpoint{4.234755in}{2.131922in}}%
\pgfpathcurveto{\pgfqpoint{4.242991in}{2.123686in}}{\pgfqpoint{4.254164in}{2.119058in}}{\pgfqpoint{4.265811in}{2.119058in}}%
\pgfpathlineto{\pgfqpoint{4.265811in}{2.119058in}}%
\pgfpathclose%
\pgfusepath{stroke,fill}%
\end{pgfscope}%
\begin{pgfscope}%
\pgfpathrectangle{\pgfqpoint{2.717318in}{0.529443in}}{\pgfqpoint{1.636659in}{1.745990in}}%
\pgfusepath{clip}%
\pgfsetbuttcap%
\pgfsetroundjoin%
\definecolor{currentfill}{rgb}{0.031373,0.627451,0.913725}%
\pgfsetfillcolor{currentfill}%
\pgfsetfillopacity{0.900000}%
\pgfsetlinewidth{0.507862pt}%
\definecolor{currentstroke}{rgb}{1.000000,1.000000,1.000000}%
\pgfsetstrokecolor{currentstroke}%
\pgfsetstrokeopacity{0.900000}%
\pgfsetdash{}{0pt}%
\pgfpathmoveto{\pgfqpoint{4.244577in}{1.927055in}}%
\pgfpathcurveto{\pgfqpoint{4.256224in}{1.927055in}}{\pgfqpoint{4.267397in}{1.931683in}}{\pgfqpoint{4.275633in}{1.939919in}}%
\pgfpathcurveto{\pgfqpoint{4.283869in}{1.948155in}}{\pgfqpoint{4.288497in}{1.959328in}}{\pgfqpoint{4.288497in}{1.970976in}}%
\pgfpathcurveto{\pgfqpoint{4.288497in}{1.982623in}}{\pgfqpoint{4.283869in}{1.993796in}}{\pgfqpoint{4.275633in}{2.002032in}}%
\pgfpathcurveto{\pgfqpoint{4.267397in}{2.010268in}}{\pgfqpoint{4.256224in}{2.014896in}}{\pgfqpoint{4.244577in}{2.014896in}}%
\pgfpathcurveto{\pgfqpoint{4.232929in}{2.014896in}}{\pgfqpoint{4.221756in}{2.010268in}}{\pgfqpoint{4.213520in}{2.002032in}}%
\pgfpathcurveto{\pgfqpoint{4.205284in}{1.993796in}}{\pgfqpoint{4.200656in}{1.982623in}}{\pgfqpoint{4.200656in}{1.970976in}}%
\pgfpathcurveto{\pgfqpoint{4.200656in}{1.959328in}}{\pgfqpoint{4.205284in}{1.948155in}}{\pgfqpoint{4.213520in}{1.939919in}}%
\pgfpathcurveto{\pgfqpoint{4.221756in}{1.931683in}}{\pgfqpoint{4.232929in}{1.927055in}}{\pgfqpoint{4.244577in}{1.927055in}}%
\pgfpathlineto{\pgfqpoint{4.244577in}{1.927055in}}%
\pgfpathclose%
\pgfusepath{stroke,fill}%
\end{pgfscope}%
\begin{pgfscope}%
\pgfsetrectcap%
\pgfsetmiterjoin%
\pgfsetlinewidth{1.254687pt}%
\definecolor{currentstroke}{rgb}{0.800000,0.800000,0.800000}%
\pgfsetstrokecolor{currentstroke}%
\pgfsetdash{}{0pt}%
\pgfpathmoveto{\pgfqpoint{2.717318in}{0.529443in}}%
\pgfpathlineto{\pgfqpoint{2.717318in}{2.275433in}}%
\pgfusepath{stroke}%
\end{pgfscope}%
\begin{pgfscope}%
\pgfsetrectcap%
\pgfsetmiterjoin%
\pgfsetlinewidth{1.254687pt}%
\definecolor{currentstroke}{rgb}{0.800000,0.800000,0.800000}%
\pgfsetstrokecolor{currentstroke}%
\pgfsetdash{}{0pt}%
\pgfpathmoveto{\pgfqpoint{4.353977in}{0.529443in}}%
\pgfpathlineto{\pgfqpoint{4.353977in}{2.275433in}}%
\pgfusepath{stroke}%
\end{pgfscope}%
\begin{pgfscope}%
\pgfsetrectcap%
\pgfsetmiterjoin%
\pgfsetlinewidth{1.254687pt}%
\definecolor{currentstroke}{rgb}{0.800000,0.800000,0.800000}%
\pgfsetstrokecolor{currentstroke}%
\pgfsetdash{}{0pt}%
\pgfpathmoveto{\pgfqpoint{2.717318in}{0.529443in}}%
\pgfpathlineto{\pgfqpoint{4.353977in}{0.529443in}}%
\pgfusepath{stroke}%
\end{pgfscope}%
\begin{pgfscope}%
\pgfsetrectcap%
\pgfsetmiterjoin%
\pgfsetlinewidth{1.254687pt}%
\definecolor{currentstroke}{rgb}{0.800000,0.800000,0.800000}%
\pgfsetstrokecolor{currentstroke}%
\pgfsetdash{}{0pt}%
\pgfpathmoveto{\pgfqpoint{2.717318in}{2.275433in}}%
\pgfpathlineto{\pgfqpoint{4.353977in}{2.275433in}}%
\pgfusepath{stroke}%
\end{pgfscope}%
\begin{pgfscope}%
\definecolor{textcolor}{rgb}{0.150000,0.150000,0.150000}%
\pgfsetstrokecolor{textcolor}%
\pgfsetfillcolor{textcolor}%
\pgftext[x=3.535647in,y=2.358766in,,base]{\color{textcolor}{\rmfamily\fontsize{11.000000}{13.200000}\selectfont\catcode`\^=\active\def^{\ifmmode\sp\else\^{}\fi}\catcode`\%=\active\def%{\%}Bag-Of-Subgraphs}}%
\end{pgfscope}%
\begin{pgfscope}%
\pgfsetbuttcap%
\pgfsetmiterjoin%
\definecolor{currentfill}{rgb}{1.000000,1.000000,1.000000}%
\pgfsetfillcolor{currentfill}%
\pgfsetlinewidth{0.000000pt}%
\definecolor{currentstroke}{rgb}{0.000000,0.000000,0.000000}%
\pgfsetstrokecolor{currentstroke}%
\pgfsetstrokeopacity{0.000000}%
\pgfsetdash{}{0pt}%
\pgfpathmoveto{\pgfqpoint{4.871338in}{0.529443in}}%
\pgfpathlineto{\pgfqpoint{6.507997in}{0.529443in}}%
\pgfpathlineto{\pgfqpoint{6.507997in}{2.275433in}}%
\pgfpathlineto{\pgfqpoint{4.871338in}{2.275433in}}%
\pgfpathlineto{\pgfqpoint{4.871338in}{0.529443in}}%
\pgfpathclose%
\pgfusepath{fill}%
\end{pgfscope}%
\begin{pgfscope}%
\pgfpathrectangle{\pgfqpoint{4.871338in}{0.529443in}}{\pgfqpoint{1.636659in}{1.745990in}}%
\pgfusepath{clip}%
\pgfsetroundcap%
\pgfsetroundjoin%
\pgfsetlinewidth{1.003750pt}%
\definecolor{currentstroke}{rgb}{0.800000,0.800000,0.800000}%
\pgfsetstrokecolor{currentstroke}%
\pgfsetdash{}{0pt}%
\pgfpathmoveto{\pgfqpoint{5.191212in}{0.529443in}}%
\pgfpathlineto{\pgfqpoint{5.191212in}{2.275433in}}%
\pgfusepath{stroke}%
\end{pgfscope}%
\begin{pgfscope}%
\definecolor{textcolor}{rgb}{0.150000,0.150000,0.150000}%
\pgfsetstrokecolor{textcolor}%
\pgfsetfillcolor{textcolor}%
\pgftext[x=5.191212in,y=0.397499in,,top]{\color{textcolor}{\rmfamily\fontsize{8.000000}{9.600000}\selectfont\catcode`\^=\active\def^{\ifmmode\sp\else\^{}\fi}\catcode`\%=\active\def%{\%}0}}%
\end{pgfscope}%
\begin{pgfscope}%
\pgfpathrectangle{\pgfqpoint{4.871338in}{0.529443in}}{\pgfqpoint{1.636659in}{1.745990in}}%
\pgfusepath{clip}%
\pgfsetroundcap%
\pgfsetroundjoin%
\pgfsetlinewidth{1.003750pt}%
\definecolor{currentstroke}{rgb}{0.800000,0.800000,0.800000}%
\pgfsetstrokecolor{currentstroke}%
\pgfsetdash{}{0pt}%
\pgfpathmoveto{\pgfqpoint{5.830817in}{0.529443in}}%
\pgfpathlineto{\pgfqpoint{5.830817in}{2.275433in}}%
\pgfusepath{stroke}%
\end{pgfscope}%
\begin{pgfscope}%
\definecolor{textcolor}{rgb}{0.150000,0.150000,0.150000}%
\pgfsetstrokecolor{textcolor}%
\pgfsetfillcolor{textcolor}%
\pgftext[x=5.830817in,y=0.397499in,,top]{\color{textcolor}{\rmfamily\fontsize{8.000000}{9.600000}\selectfont\catcode`\^=\active\def^{\ifmmode\sp\else\^{}\fi}\catcode`\%=\active\def%{\%}5}}%
\end{pgfscope}%
\begin{pgfscope}%
\pgfpathrectangle{\pgfqpoint{4.871338in}{0.529443in}}{\pgfqpoint{1.636659in}{1.745990in}}%
\pgfusepath{clip}%
\pgfsetroundcap%
\pgfsetroundjoin%
\pgfsetlinewidth{1.003750pt}%
\definecolor{currentstroke}{rgb}{0.800000,0.800000,0.800000}%
\pgfsetstrokecolor{currentstroke}%
\pgfsetdash{}{0pt}%
\pgfpathmoveto{\pgfqpoint{6.470421in}{0.529443in}}%
\pgfpathlineto{\pgfqpoint{6.470421in}{2.275433in}}%
\pgfusepath{stroke}%
\end{pgfscope}%
\begin{pgfscope}%
\definecolor{textcolor}{rgb}{0.150000,0.150000,0.150000}%
\pgfsetstrokecolor{textcolor}%
\pgfsetfillcolor{textcolor}%
\pgftext[x=6.470421in,y=0.397499in,,top]{\color{textcolor}{\rmfamily\fontsize{8.000000}{9.600000}\selectfont\catcode`\^=\active\def^{\ifmmode\sp\else\^{}\fi}\catcode`\%=\active\def%{\%}10}}%
\end{pgfscope}%
\begin{pgfscope}%
\definecolor{textcolor}{rgb}{0.150000,0.150000,0.150000}%
\pgfsetstrokecolor{textcolor}%
\pgfsetfillcolor{textcolor}%
\pgftext[x=5.689667in,y=0.234413in,,top]{\color{textcolor}{\rmfamily\fontsize{10.000000}{12.000000}\selectfont\catcode`\^=\active\def^{\ifmmode\sp\else\^{}\fi}\catcode`\%=\active\def%{\%}UMAP 1}}%
\end{pgfscope}%
\begin{pgfscope}%
\pgfpathrectangle{\pgfqpoint{4.871338in}{0.529443in}}{\pgfqpoint{1.636659in}{1.745990in}}%
\pgfusepath{clip}%
\pgfsetroundcap%
\pgfsetroundjoin%
\pgfsetlinewidth{1.003750pt}%
\definecolor{currentstroke}{rgb}{0.800000,0.800000,0.800000}%
\pgfsetstrokecolor{currentstroke}%
\pgfsetdash{}{0pt}%
\pgfpathmoveto{\pgfqpoint{4.871338in}{0.816783in}}%
\pgfpathlineto{\pgfqpoint{6.507997in}{0.816783in}}%
\pgfusepath{stroke}%
\end{pgfscope}%
\begin{pgfscope}%
\definecolor{textcolor}{rgb}{0.150000,0.150000,0.150000}%
\pgfsetstrokecolor{textcolor}%
\pgfsetfillcolor{textcolor}%
\pgftext[x=4.506186in, y=0.774574in, left, base]{\color{textcolor}{\rmfamily\fontsize{8.000000}{9.600000}\selectfont\catcode`\^=\active\def^{\ifmmode\sp\else\^{}\fi}\catcode`\%=\active\def%{\%}\ensuremath{-}10}}%
\end{pgfscope}%
\begin{pgfscope}%
\pgfpathrectangle{\pgfqpoint{4.871338in}{0.529443in}}{\pgfqpoint{1.636659in}{1.745990in}}%
\pgfusepath{clip}%
\pgfsetroundcap%
\pgfsetroundjoin%
\pgfsetlinewidth{1.003750pt}%
\definecolor{currentstroke}{rgb}{0.800000,0.800000,0.800000}%
\pgfsetstrokecolor{currentstroke}%
\pgfsetdash{}{0pt}%
\pgfpathmoveto{\pgfqpoint{4.871338in}{1.294017in}}%
\pgfpathlineto{\pgfqpoint{6.507997in}{1.294017in}}%
\pgfusepath{stroke}%
\end{pgfscope}%
\begin{pgfscope}%
\definecolor{textcolor}{rgb}{0.150000,0.150000,0.150000}%
\pgfsetstrokecolor{textcolor}%
\pgfsetfillcolor{textcolor}%
\pgftext[x=4.668701in, y=1.251808in, left, base]{\color{textcolor}{\rmfamily\fontsize{8.000000}{9.600000}\selectfont\catcode`\^=\active\def^{\ifmmode\sp\else\^{}\fi}\catcode`\%=\active\def%{\%}0}}%
\end{pgfscope}%
\begin{pgfscope}%
\pgfpathrectangle{\pgfqpoint{4.871338in}{0.529443in}}{\pgfqpoint{1.636659in}{1.745990in}}%
\pgfusepath{clip}%
\pgfsetroundcap%
\pgfsetroundjoin%
\pgfsetlinewidth{1.003750pt}%
\definecolor{currentstroke}{rgb}{0.800000,0.800000,0.800000}%
\pgfsetstrokecolor{currentstroke}%
\pgfsetdash{}{0pt}%
\pgfpathmoveto{\pgfqpoint{4.871338in}{1.771251in}}%
\pgfpathlineto{\pgfqpoint{6.507997in}{1.771251in}}%
\pgfusepath{stroke}%
\end{pgfscope}%
\begin{pgfscope}%
\definecolor{textcolor}{rgb}{0.150000,0.150000,0.150000}%
\pgfsetstrokecolor{textcolor}%
\pgfsetfillcolor{textcolor}%
\pgftext[x=4.598009in, y=1.729041in, left, base]{\color{textcolor}{\rmfamily\fontsize{8.000000}{9.600000}\selectfont\catcode`\^=\active\def^{\ifmmode\sp\else\^{}\fi}\catcode`\%=\active\def%{\%}10}}%
\end{pgfscope}%
\begin{pgfscope}%
\pgfpathrectangle{\pgfqpoint{4.871338in}{0.529443in}}{\pgfqpoint{1.636659in}{1.745990in}}%
\pgfusepath{clip}%
\pgfsetroundcap%
\pgfsetroundjoin%
\pgfsetlinewidth{1.003750pt}%
\definecolor{currentstroke}{rgb}{0.800000,0.800000,0.800000}%
\pgfsetstrokecolor{currentstroke}%
\pgfsetdash{}{0pt}%
\pgfpathmoveto{\pgfqpoint{4.871338in}{2.248485in}}%
\pgfpathlineto{\pgfqpoint{6.507997in}{2.248485in}}%
\pgfusepath{stroke}%
\end{pgfscope}%
\begin{pgfscope}%
\definecolor{textcolor}{rgb}{0.150000,0.150000,0.150000}%
\pgfsetstrokecolor{textcolor}%
\pgfsetfillcolor{textcolor}%
\pgftext[x=4.598009in, y=2.206275in, left, base]{\color{textcolor}{\rmfamily\fontsize{8.000000}{9.600000}\selectfont\catcode`\^=\active\def^{\ifmmode\sp\else\^{}\fi}\catcode`\%=\active\def%{\%}20}}%
\end{pgfscope}%
\begin{pgfscope}%
\pgfpathrectangle{\pgfqpoint{4.871338in}{0.529443in}}{\pgfqpoint{1.636659in}{1.745990in}}%
\pgfusepath{clip}%
\pgfsetbuttcap%
\pgfsetroundjoin%
\definecolor{currentfill}{rgb}{0.003922,0.003922,0.003922}%
\pgfsetfillcolor{currentfill}%
\pgfsetfillopacity{0.900000}%
\pgfsetlinewidth{0.507862pt}%
\definecolor{currentstroke}{rgb}{1.000000,1.000000,1.000000}%
\pgfsetstrokecolor{currentstroke}%
\pgfsetstrokeopacity{0.900000}%
\pgfsetdash{}{0pt}%
\pgfpathmoveto{\pgfqpoint{5.050560in}{2.056272in}}%
\pgfpathcurveto{\pgfqpoint{5.062208in}{2.056272in}}{\pgfqpoint{5.073381in}{2.060900in}}{\pgfqpoint{5.081617in}{2.069136in}}%
\pgfpathcurveto{\pgfqpoint{5.089853in}{2.077373in}}{\pgfqpoint{5.094481in}{2.088545in}}{\pgfqpoint{5.094481in}{2.100193in}}%
\pgfpathcurveto{\pgfqpoint{5.094481in}{2.111841in}}{\pgfqpoint{5.089853in}{2.123013in}}{\pgfqpoint{5.081617in}{2.131249in}}%
\pgfpathcurveto{\pgfqpoint{5.073381in}{2.139486in}}{\pgfqpoint{5.062208in}{2.144113in}}{\pgfqpoint{5.050560in}{2.144113in}}%
\pgfpathcurveto{\pgfqpoint{5.038913in}{2.144113in}}{\pgfqpoint{5.027740in}{2.139486in}}{\pgfqpoint{5.019504in}{2.131249in}}%
\pgfpathcurveto{\pgfqpoint{5.011268in}{2.123013in}}{\pgfqpoint{5.006640in}{2.111841in}}{\pgfqpoint{5.006640in}{2.100193in}}%
\pgfpathcurveto{\pgfqpoint{5.006640in}{2.088545in}}{\pgfqpoint{5.011268in}{2.077373in}}{\pgfqpoint{5.019504in}{2.069136in}}%
\pgfpathcurveto{\pgfqpoint{5.027740in}{2.060900in}}{\pgfqpoint{5.038913in}{2.056272in}}{\pgfqpoint{5.050560in}{2.056272in}}%
\pgfpathlineto{\pgfqpoint{5.050560in}{2.056272in}}%
\pgfpathclose%
\pgfusepath{stroke,fill}%
\end{pgfscope}%
\begin{pgfscope}%
\pgfpathrectangle{\pgfqpoint{4.871338in}{0.529443in}}{\pgfqpoint{1.636659in}{1.745990in}}%
\pgfusepath{clip}%
\pgfsetbuttcap%
\pgfsetroundjoin%
\definecolor{currentfill}{rgb}{0.003922,0.003922,0.003922}%
\pgfsetfillcolor{currentfill}%
\pgfsetfillopacity{0.900000}%
\pgfsetlinewidth{0.507862pt}%
\definecolor{currentstroke}{rgb}{1.000000,1.000000,1.000000}%
\pgfsetstrokecolor{currentstroke}%
\pgfsetstrokeopacity{0.900000}%
\pgfsetdash{}{0pt}%
\pgfpathmoveto{\pgfqpoint{6.373516in}{0.604069in}}%
\pgfpathcurveto{\pgfqpoint{6.385164in}{0.604069in}}{\pgfqpoint{6.396336in}{0.608697in}}{\pgfqpoint{6.404572in}{0.616933in}}%
\pgfpathcurveto{\pgfqpoint{6.412808in}{0.625169in}}{\pgfqpoint{6.417436in}{0.636342in}}{\pgfqpoint{6.417436in}{0.647990in}}%
\pgfpathcurveto{\pgfqpoint{6.417436in}{0.659638in}}{\pgfqpoint{6.412808in}{0.670810in}}{\pgfqpoint{6.404572in}{0.679046in}}%
\pgfpathcurveto{\pgfqpoint{6.396336in}{0.687282in}}{\pgfqpoint{6.385164in}{0.691910in}}{\pgfqpoint{6.373516in}{0.691910in}}%
\pgfpathcurveto{\pgfqpoint{6.361868in}{0.691910in}}{\pgfqpoint{6.350696in}{0.687282in}}{\pgfqpoint{6.342459in}{0.679046in}}%
\pgfpathcurveto{\pgfqpoint{6.334223in}{0.670810in}}{\pgfqpoint{6.329595in}{0.659638in}}{\pgfqpoint{6.329595in}{0.647990in}}%
\pgfpathcurveto{\pgfqpoint{6.329595in}{0.636342in}}{\pgfqpoint{6.334223in}{0.625169in}}{\pgfqpoint{6.342459in}{0.616933in}}%
\pgfpathcurveto{\pgfqpoint{6.350696in}{0.608697in}}{\pgfqpoint{6.361868in}{0.604069in}}{\pgfqpoint{6.373516in}{0.604069in}}%
\pgfpathlineto{\pgfqpoint{6.373516in}{0.604069in}}%
\pgfpathclose%
\pgfusepath{stroke,fill}%
\end{pgfscope}%
\begin{pgfscope}%
\pgfpathrectangle{\pgfqpoint{4.871338in}{0.529443in}}{\pgfqpoint{1.636659in}{1.745990in}}%
\pgfusepath{clip}%
\pgfsetbuttcap%
\pgfsetroundjoin%
\definecolor{currentfill}{rgb}{0.003922,0.003922,0.003922}%
\pgfsetfillcolor{currentfill}%
\pgfsetfillopacity{0.900000}%
\pgfsetlinewidth{0.507862pt}%
\definecolor{currentstroke}{rgb}{1.000000,1.000000,1.000000}%
\pgfsetstrokecolor{currentstroke}%
\pgfsetstrokeopacity{0.900000}%
\pgfsetdash{}{0pt}%
\pgfpathmoveto{\pgfqpoint{5.025744in}{2.047539in}}%
\pgfpathcurveto{\pgfqpoint{5.037391in}{2.047539in}}{\pgfqpoint{5.048564in}{2.052167in}}{\pgfqpoint{5.056800in}{2.060403in}}%
\pgfpathcurveto{\pgfqpoint{5.065036in}{2.068640in}}{\pgfqpoint{5.069664in}{2.079812in}}{\pgfqpoint{5.069664in}{2.091460in}}%
\pgfpathcurveto{\pgfqpoint{5.069664in}{2.103108in}}{\pgfqpoint{5.065036in}{2.114280in}}{\pgfqpoint{5.056800in}{2.122516in}}%
\pgfpathcurveto{\pgfqpoint{5.048564in}{2.130753in}}{\pgfqpoint{5.037391in}{2.135380in}}{\pgfqpoint{5.025744in}{2.135380in}}%
\pgfpathcurveto{\pgfqpoint{5.014096in}{2.135380in}}{\pgfqpoint{5.002923in}{2.130753in}}{\pgfqpoint{4.994687in}{2.122516in}}%
\pgfpathcurveto{\pgfqpoint{4.986451in}{2.114280in}}{\pgfqpoint{4.981823in}{2.103108in}}{\pgfqpoint{4.981823in}{2.091460in}}%
\pgfpathcurveto{\pgfqpoint{4.981823in}{2.079812in}}{\pgfqpoint{4.986451in}{2.068640in}}{\pgfqpoint{4.994687in}{2.060403in}}%
\pgfpathcurveto{\pgfqpoint{5.002923in}{2.052167in}}{\pgfqpoint{5.014096in}{2.047539in}}{\pgfqpoint{5.025744in}{2.047539in}}%
\pgfpathlineto{\pgfqpoint{5.025744in}{2.047539in}}%
\pgfpathclose%
\pgfusepath{stroke,fill}%
\end{pgfscope}%
\begin{pgfscope}%
\pgfpathrectangle{\pgfqpoint{4.871338in}{0.529443in}}{\pgfqpoint{1.636659in}{1.745990in}}%
\pgfusepath{clip}%
\pgfsetbuttcap%
\pgfsetroundjoin%
\definecolor{currentfill}{rgb}{0.003922,0.003922,0.003922}%
\pgfsetfillcolor{currentfill}%
\pgfsetfillopacity{0.900000}%
\pgfsetlinewidth{0.507862pt}%
\definecolor{currentstroke}{rgb}{1.000000,1.000000,1.000000}%
\pgfsetstrokecolor{currentstroke}%
\pgfsetstrokeopacity{0.900000}%
\pgfsetdash{}{0pt}%
\pgfpathmoveto{\pgfqpoint{6.276038in}{0.637585in}}%
\pgfpathcurveto{\pgfqpoint{6.287686in}{0.637585in}}{\pgfqpoint{6.298858in}{0.642212in}}{\pgfqpoint{6.307094in}{0.650449in}}%
\pgfpathcurveto{\pgfqpoint{6.315331in}{0.658685in}}{\pgfqpoint{6.319958in}{0.669857in}}{\pgfqpoint{6.319958in}{0.681505in}}%
\pgfpathcurveto{\pgfqpoint{6.319958in}{0.693153in}}{\pgfqpoint{6.315331in}{0.704325in}}{\pgfqpoint{6.307094in}{0.712562in}}%
\pgfpathcurveto{\pgfqpoint{6.298858in}{0.720798in}}{\pgfqpoint{6.287686in}{0.725426in}}{\pgfqpoint{6.276038in}{0.725426in}}%
\pgfpathcurveto{\pgfqpoint{6.264390in}{0.725426in}}{\pgfqpoint{6.253218in}{0.720798in}}{\pgfqpoint{6.244981in}{0.712562in}}%
\pgfpathcurveto{\pgfqpoint{6.236745in}{0.704325in}}{\pgfqpoint{6.232117in}{0.693153in}}{\pgfqpoint{6.232117in}{0.681505in}}%
\pgfpathcurveto{\pgfqpoint{6.232117in}{0.669857in}}{\pgfqpoint{6.236745in}{0.658685in}}{\pgfqpoint{6.244981in}{0.650449in}}%
\pgfpathcurveto{\pgfqpoint{6.253218in}{0.642212in}}{\pgfqpoint{6.264390in}{0.637585in}}{\pgfqpoint{6.276038in}{0.637585in}}%
\pgfpathlineto{\pgfqpoint{6.276038in}{0.637585in}}%
\pgfpathclose%
\pgfusepath{stroke,fill}%
\end{pgfscope}%
\begin{pgfscope}%
\pgfpathrectangle{\pgfqpoint{4.871338in}{0.529443in}}{\pgfqpoint{1.636659in}{1.745990in}}%
\pgfusepath{clip}%
\pgfsetbuttcap%
\pgfsetroundjoin%
\definecolor{currentfill}{rgb}{0.003922,0.003922,0.003922}%
\pgfsetfillcolor{currentfill}%
\pgfsetfillopacity{0.900000}%
\pgfsetlinewidth{0.507862pt}%
\definecolor{currentstroke}{rgb}{1.000000,1.000000,1.000000}%
\pgfsetstrokecolor{currentstroke}%
\pgfsetstrokeopacity{0.900000}%
\pgfsetdash{}{0pt}%
\pgfpathmoveto{\pgfqpoint{6.348799in}{0.619691in}}%
\pgfpathcurveto{\pgfqpoint{6.360447in}{0.619691in}}{\pgfqpoint{6.371620in}{0.624318in}}{\pgfqpoint{6.379856in}{0.632555in}}%
\pgfpathcurveto{\pgfqpoint{6.388092in}{0.640791in}}{\pgfqpoint{6.392720in}{0.651963in}}{\pgfqpoint{6.392720in}{0.663611in}}%
\pgfpathcurveto{\pgfqpoint{6.392720in}{0.675259in}}{\pgfqpoint{6.388092in}{0.686431in}}{\pgfqpoint{6.379856in}{0.694668in}}%
\pgfpathcurveto{\pgfqpoint{6.371620in}{0.702904in}}{\pgfqpoint{6.360447in}{0.707532in}}{\pgfqpoint{6.348799in}{0.707532in}}%
\pgfpathcurveto{\pgfqpoint{6.337151in}{0.707532in}}{\pgfqpoint{6.325979in}{0.702904in}}{\pgfqpoint{6.317743in}{0.694668in}}%
\pgfpathcurveto{\pgfqpoint{6.309507in}{0.686431in}}{\pgfqpoint{6.304879in}{0.675259in}}{\pgfqpoint{6.304879in}{0.663611in}}%
\pgfpathcurveto{\pgfqpoint{6.304879in}{0.651963in}}{\pgfqpoint{6.309507in}{0.640791in}}{\pgfqpoint{6.317743in}{0.632555in}}%
\pgfpathcurveto{\pgfqpoint{6.325979in}{0.624318in}}{\pgfqpoint{6.337151in}{0.619691in}}{\pgfqpoint{6.348799in}{0.619691in}}%
\pgfpathlineto{\pgfqpoint{6.348799in}{0.619691in}}%
\pgfpathclose%
\pgfusepath{stroke,fill}%
\end{pgfscope}%
\begin{pgfscope}%
\pgfpathrectangle{\pgfqpoint{4.871338in}{0.529443in}}{\pgfqpoint{1.636659in}{1.745990in}}%
\pgfusepath{clip}%
\pgfsetbuttcap%
\pgfsetroundjoin%
\definecolor{currentfill}{rgb}{0.003922,0.003922,0.003922}%
\pgfsetfillcolor{currentfill}%
\pgfsetfillopacity{0.900000}%
\pgfsetlinewidth{0.507862pt}%
\definecolor{currentstroke}{rgb}{1.000000,1.000000,1.000000}%
\pgfsetstrokecolor{currentstroke}%
\pgfsetstrokeopacity{0.900000}%
\pgfsetdash{}{0pt}%
\pgfpathmoveto{\pgfqpoint{6.358154in}{0.588387in}}%
\pgfpathcurveto{\pgfqpoint{6.369802in}{0.588387in}}{\pgfqpoint{6.380974in}{0.593014in}}{\pgfqpoint{6.389211in}{0.601251in}}%
\pgfpathcurveto{\pgfqpoint{6.397447in}{0.609487in}}{\pgfqpoint{6.402075in}{0.620659in}}{\pgfqpoint{6.402075in}{0.632307in}}%
\pgfpathcurveto{\pgfqpoint{6.402075in}{0.643955in}}{\pgfqpoint{6.397447in}{0.655127in}}{\pgfqpoint{6.389211in}{0.663364in}}%
\pgfpathcurveto{\pgfqpoint{6.380974in}{0.671600in}}{\pgfqpoint{6.369802in}{0.676228in}}{\pgfqpoint{6.358154in}{0.676228in}}%
\pgfpathcurveto{\pgfqpoint{6.346506in}{0.676228in}}{\pgfqpoint{6.335334in}{0.671600in}}{\pgfqpoint{6.327098in}{0.663364in}}%
\pgfpathcurveto{\pgfqpoint{6.318861in}{0.655127in}}{\pgfqpoint{6.314234in}{0.643955in}}{\pgfqpoint{6.314234in}{0.632307in}}%
\pgfpathcurveto{\pgfqpoint{6.314234in}{0.620659in}}{\pgfqpoint{6.318861in}{0.609487in}}{\pgfqpoint{6.327098in}{0.601251in}}%
\pgfpathcurveto{\pgfqpoint{6.335334in}{0.593014in}}{\pgfqpoint{6.346506in}{0.588387in}}{\pgfqpoint{6.358154in}{0.588387in}}%
\pgfpathlineto{\pgfqpoint{6.358154in}{0.588387in}}%
\pgfpathclose%
\pgfusepath{stroke,fill}%
\end{pgfscope}%
\begin{pgfscope}%
\pgfpathrectangle{\pgfqpoint{4.871338in}{0.529443in}}{\pgfqpoint{1.636659in}{1.745990in}}%
\pgfusepath{clip}%
\pgfsetbuttcap%
\pgfsetroundjoin%
\definecolor{currentfill}{rgb}{0.003922,0.003922,0.003922}%
\pgfsetfillcolor{currentfill}%
\pgfsetfillopacity{0.900000}%
\pgfsetlinewidth{0.507862pt}%
\definecolor{currentstroke}{rgb}{1.000000,1.000000,1.000000}%
\pgfsetstrokecolor{currentstroke}%
\pgfsetstrokeopacity{0.900000}%
\pgfsetdash{}{0pt}%
\pgfpathmoveto{\pgfqpoint{6.426164in}{0.590383in}}%
\pgfpathcurveto{\pgfqpoint{6.437812in}{0.590383in}}{\pgfqpoint{6.448984in}{0.595011in}}{\pgfqpoint{6.457220in}{0.603247in}}%
\pgfpathcurveto{\pgfqpoint{6.465457in}{0.611483in}}{\pgfqpoint{6.470084in}{0.622656in}}{\pgfqpoint{6.470084in}{0.634304in}}%
\pgfpathcurveto{\pgfqpoint{6.470084in}{0.645952in}}{\pgfqpoint{6.465457in}{0.657124in}}{\pgfqpoint{6.457220in}{0.665360in}}%
\pgfpathcurveto{\pgfqpoint{6.448984in}{0.673596in}}{\pgfqpoint{6.437812in}{0.678224in}}{\pgfqpoint{6.426164in}{0.678224in}}%
\pgfpathcurveto{\pgfqpoint{6.414516in}{0.678224in}}{\pgfqpoint{6.403344in}{0.673596in}}{\pgfqpoint{6.395107in}{0.665360in}}%
\pgfpathcurveto{\pgfqpoint{6.386871in}{0.657124in}}{\pgfqpoint{6.382243in}{0.645952in}}{\pgfqpoint{6.382243in}{0.634304in}}%
\pgfpathcurveto{\pgfqpoint{6.382243in}{0.622656in}}{\pgfqpoint{6.386871in}{0.611483in}}{\pgfqpoint{6.395107in}{0.603247in}}%
\pgfpathcurveto{\pgfqpoint{6.403344in}{0.595011in}}{\pgfqpoint{6.414516in}{0.590383in}}{\pgfqpoint{6.426164in}{0.590383in}}%
\pgfpathlineto{\pgfqpoint{6.426164in}{0.590383in}}%
\pgfpathclose%
\pgfusepath{stroke,fill}%
\end{pgfscope}%
\begin{pgfscope}%
\pgfpathrectangle{\pgfqpoint{4.871338in}{0.529443in}}{\pgfqpoint{1.636659in}{1.745990in}}%
\pgfusepath{clip}%
\pgfsetbuttcap%
\pgfsetroundjoin%
\definecolor{currentfill}{rgb}{0.003922,0.003922,0.003922}%
\pgfsetfillcolor{currentfill}%
\pgfsetfillopacity{0.900000}%
\pgfsetlinewidth{0.507862pt}%
\definecolor{currentstroke}{rgb}{1.000000,1.000000,1.000000}%
\pgfsetstrokecolor{currentstroke}%
\pgfsetstrokeopacity{0.900000}%
\pgfsetdash{}{0pt}%
\pgfpathmoveto{\pgfqpoint{5.054407in}{2.133440in}}%
\pgfpathcurveto{\pgfqpoint{5.066055in}{2.133440in}}{\pgfqpoint{5.077228in}{2.138068in}}{\pgfqpoint{5.085464in}{2.146304in}}%
\pgfpathcurveto{\pgfqpoint{5.093700in}{2.154540in}}{\pgfqpoint{5.098328in}{2.165713in}}{\pgfqpoint{5.098328in}{2.177361in}}%
\pgfpathcurveto{\pgfqpoint{5.098328in}{2.189008in}}{\pgfqpoint{5.093700in}{2.200181in}}{\pgfqpoint{5.085464in}{2.208417in}}%
\pgfpathcurveto{\pgfqpoint{5.077228in}{2.216653in}}{\pgfqpoint{5.066055in}{2.221281in}}{\pgfqpoint{5.054407in}{2.221281in}}%
\pgfpathcurveto{\pgfqpoint{5.042760in}{2.221281in}}{\pgfqpoint{5.031587in}{2.216653in}}{\pgfqpoint{5.023351in}{2.208417in}}%
\pgfpathcurveto{\pgfqpoint{5.015115in}{2.200181in}}{\pgfqpoint{5.010487in}{2.189008in}}{\pgfqpoint{5.010487in}{2.177361in}}%
\pgfpathcurveto{\pgfqpoint{5.010487in}{2.165713in}}{\pgfqpoint{5.015115in}{2.154540in}}{\pgfqpoint{5.023351in}{2.146304in}}%
\pgfpathcurveto{\pgfqpoint{5.031587in}{2.138068in}}{\pgfqpoint{5.042760in}{2.133440in}}{\pgfqpoint{5.054407in}{2.133440in}}%
\pgfpathlineto{\pgfqpoint{5.054407in}{2.133440in}}%
\pgfpathclose%
\pgfusepath{stroke,fill}%
\end{pgfscope}%
\begin{pgfscope}%
\pgfpathrectangle{\pgfqpoint{4.871338in}{0.529443in}}{\pgfqpoint{1.636659in}{1.745990in}}%
\pgfusepath{clip}%
\pgfsetbuttcap%
\pgfsetroundjoin%
\definecolor{currentfill}{rgb}{0.003922,0.003922,0.003922}%
\pgfsetfillcolor{currentfill}%
\pgfsetfillopacity{0.900000}%
\pgfsetlinewidth{0.507862pt}%
\definecolor{currentstroke}{rgb}{1.000000,1.000000,1.000000}%
\pgfsetstrokecolor{currentstroke}%
\pgfsetstrokeopacity{0.900000}%
\pgfsetdash{}{0pt}%
\pgfpathmoveto{\pgfqpoint{5.069993in}{2.106054in}}%
\pgfpathcurveto{\pgfqpoint{5.081640in}{2.106054in}}{\pgfqpoint{5.092813in}{2.110682in}}{\pgfqpoint{5.101049in}{2.118918in}}%
\pgfpathcurveto{\pgfqpoint{5.109285in}{2.127154in}}{\pgfqpoint{5.113913in}{2.138327in}}{\pgfqpoint{5.113913in}{2.149975in}}%
\pgfpathcurveto{\pgfqpoint{5.113913in}{2.161623in}}{\pgfqpoint{5.109285in}{2.172795in}}{\pgfqpoint{5.101049in}{2.181031in}}%
\pgfpathcurveto{\pgfqpoint{5.092813in}{2.189267in}}{\pgfqpoint{5.081640in}{2.193895in}}{\pgfqpoint{5.069993in}{2.193895in}}%
\pgfpathcurveto{\pgfqpoint{5.058345in}{2.193895in}}{\pgfqpoint{5.047172in}{2.189267in}}{\pgfqpoint{5.038936in}{2.181031in}}%
\pgfpathcurveto{\pgfqpoint{5.030700in}{2.172795in}}{\pgfqpoint{5.026072in}{2.161623in}}{\pgfqpoint{5.026072in}{2.149975in}}%
\pgfpathcurveto{\pgfqpoint{5.026072in}{2.138327in}}{\pgfqpoint{5.030700in}{2.127154in}}{\pgfqpoint{5.038936in}{2.118918in}}%
\pgfpathcurveto{\pgfqpoint{5.047172in}{2.110682in}}{\pgfqpoint{5.058345in}{2.106054in}}{\pgfqpoint{5.069993in}{2.106054in}}%
\pgfpathlineto{\pgfqpoint{5.069993in}{2.106054in}}%
\pgfpathclose%
\pgfusepath{stroke,fill}%
\end{pgfscope}%
\begin{pgfscope}%
\pgfpathrectangle{\pgfqpoint{4.871338in}{0.529443in}}{\pgfqpoint{1.636659in}{1.745990in}}%
\pgfusepath{clip}%
\pgfsetbuttcap%
\pgfsetroundjoin%
\definecolor{currentfill}{rgb}{0.003922,0.003922,0.003922}%
\pgfsetfillcolor{currentfill}%
\pgfsetfillopacity{0.900000}%
\pgfsetlinewidth{0.507862pt}%
\definecolor{currentstroke}{rgb}{1.000000,1.000000,1.000000}%
\pgfsetstrokecolor{currentstroke}%
\pgfsetstrokeopacity{0.900000}%
\pgfsetdash{}{0pt}%
\pgfpathmoveto{\pgfqpoint{6.322935in}{0.652701in}}%
\pgfpathcurveto{\pgfqpoint{6.334583in}{0.652701in}}{\pgfqpoint{6.345755in}{0.657328in}}{\pgfqpoint{6.353991in}{0.665565in}}%
\pgfpathcurveto{\pgfqpoint{6.362228in}{0.673801in}}{\pgfqpoint{6.366855in}{0.684973in}}{\pgfqpoint{6.366855in}{0.696621in}}%
\pgfpathcurveto{\pgfqpoint{6.366855in}{0.708269in}}{\pgfqpoint{6.362228in}{0.719441in}}{\pgfqpoint{6.353991in}{0.727678in}}%
\pgfpathcurveto{\pgfqpoint{6.345755in}{0.735914in}}{\pgfqpoint{6.334583in}{0.740542in}}{\pgfqpoint{6.322935in}{0.740542in}}%
\pgfpathcurveto{\pgfqpoint{6.311287in}{0.740542in}}{\pgfqpoint{6.300115in}{0.735914in}}{\pgfqpoint{6.291878in}{0.727678in}}%
\pgfpathcurveto{\pgfqpoint{6.283642in}{0.719441in}}{\pgfqpoint{6.279014in}{0.708269in}}{\pgfqpoint{6.279014in}{0.696621in}}%
\pgfpathcurveto{\pgfqpoint{6.279014in}{0.684973in}}{\pgfqpoint{6.283642in}{0.673801in}}{\pgfqpoint{6.291878in}{0.665565in}}%
\pgfpathcurveto{\pgfqpoint{6.300115in}{0.657328in}}{\pgfqpoint{6.311287in}{0.652701in}}{\pgfqpoint{6.322935in}{0.652701in}}%
\pgfpathlineto{\pgfqpoint{6.322935in}{0.652701in}}%
\pgfpathclose%
\pgfusepath{stroke,fill}%
\end{pgfscope}%
\begin{pgfscope}%
\pgfpathrectangle{\pgfqpoint{4.871338in}{0.529443in}}{\pgfqpoint{1.636659in}{1.745990in}}%
\pgfusepath{clip}%
\pgfsetbuttcap%
\pgfsetroundjoin%
\definecolor{currentfill}{rgb}{0.003922,0.003922,0.003922}%
\pgfsetfillcolor{currentfill}%
\pgfsetfillopacity{0.900000}%
\pgfsetlinewidth{0.507862pt}%
\definecolor{currentstroke}{rgb}{1.000000,1.000000,1.000000}%
\pgfsetstrokecolor{currentstroke}%
\pgfsetstrokeopacity{0.900000}%
\pgfsetdash{}{0pt}%
\pgfpathmoveto{\pgfqpoint{5.001970in}{2.022016in}}%
\pgfpathcurveto{\pgfqpoint{5.013618in}{2.022016in}}{\pgfqpoint{5.024791in}{2.026644in}}{\pgfqpoint{5.033027in}{2.034880in}}%
\pgfpathcurveto{\pgfqpoint{5.041263in}{2.043116in}}{\pgfqpoint{5.045891in}{2.054289in}}{\pgfqpoint{5.045891in}{2.065937in}}%
\pgfpathcurveto{\pgfqpoint{5.045891in}{2.077584in}}{\pgfqpoint{5.041263in}{2.088757in}}{\pgfqpoint{5.033027in}{2.096993in}}%
\pgfpathcurveto{\pgfqpoint{5.024791in}{2.105229in}}{\pgfqpoint{5.013618in}{2.109857in}}{\pgfqpoint{5.001970in}{2.109857in}}%
\pgfpathcurveto{\pgfqpoint{4.990322in}{2.109857in}}{\pgfqpoint{4.979150in}{2.105229in}}{\pgfqpoint{4.970914in}{2.096993in}}%
\pgfpathcurveto{\pgfqpoint{4.962678in}{2.088757in}}{\pgfqpoint{4.958050in}{2.077584in}}{\pgfqpoint{4.958050in}{2.065937in}}%
\pgfpathcurveto{\pgfqpoint{4.958050in}{2.054289in}}{\pgfqpoint{4.962678in}{2.043116in}}{\pgfqpoint{4.970914in}{2.034880in}}%
\pgfpathcurveto{\pgfqpoint{4.979150in}{2.026644in}}{\pgfqpoint{4.990322in}{2.022016in}}{\pgfqpoint{5.001970in}{2.022016in}}%
\pgfpathlineto{\pgfqpoint{5.001970in}{2.022016in}}%
\pgfpathclose%
\pgfusepath{stroke,fill}%
\end{pgfscope}%
\begin{pgfscope}%
\pgfpathrectangle{\pgfqpoint{4.871338in}{0.529443in}}{\pgfqpoint{1.636659in}{1.745990in}}%
\pgfusepath{clip}%
\pgfsetbuttcap%
\pgfsetroundjoin%
\definecolor{currentfill}{rgb}{0.003922,0.003922,0.003922}%
\pgfsetfillcolor{currentfill}%
\pgfsetfillopacity{0.900000}%
\pgfsetlinewidth{0.507862pt}%
\definecolor{currentstroke}{rgb}{1.000000,1.000000,1.000000}%
\pgfsetstrokecolor{currentstroke}%
\pgfsetstrokeopacity{0.900000}%
\pgfsetdash{}{0pt}%
\pgfpathmoveto{\pgfqpoint{6.215972in}{0.596162in}}%
\pgfpathcurveto{\pgfqpoint{6.227620in}{0.596162in}}{\pgfqpoint{6.238792in}{0.600790in}}{\pgfqpoint{6.247028in}{0.609026in}}%
\pgfpathcurveto{\pgfqpoint{6.255265in}{0.617262in}}{\pgfqpoint{6.259892in}{0.628435in}}{\pgfqpoint{6.259892in}{0.640082in}}%
\pgfpathcurveto{\pgfqpoint{6.259892in}{0.651730in}}{\pgfqpoint{6.255265in}{0.662903in}}{\pgfqpoint{6.247028in}{0.671139in}}%
\pgfpathcurveto{\pgfqpoint{6.238792in}{0.679375in}}{\pgfqpoint{6.227620in}{0.684003in}}{\pgfqpoint{6.215972in}{0.684003in}}%
\pgfpathcurveto{\pgfqpoint{6.204324in}{0.684003in}}{\pgfqpoint{6.193152in}{0.679375in}}{\pgfqpoint{6.184915in}{0.671139in}}%
\pgfpathcurveto{\pgfqpoint{6.176679in}{0.662903in}}{\pgfqpoint{6.172051in}{0.651730in}}{\pgfqpoint{6.172051in}{0.640082in}}%
\pgfpathcurveto{\pgfqpoint{6.172051in}{0.628435in}}{\pgfqpoint{6.176679in}{0.617262in}}{\pgfqpoint{6.184915in}{0.609026in}}%
\pgfpathcurveto{\pgfqpoint{6.193152in}{0.600790in}}{\pgfqpoint{6.204324in}{0.596162in}}{\pgfqpoint{6.215972in}{0.596162in}}%
\pgfpathlineto{\pgfqpoint{6.215972in}{0.596162in}}%
\pgfpathclose%
\pgfusepath{stroke,fill}%
\end{pgfscope}%
\begin{pgfscope}%
\pgfpathrectangle{\pgfqpoint{4.871338in}{0.529443in}}{\pgfqpoint{1.636659in}{1.745990in}}%
\pgfusepath{clip}%
\pgfsetbuttcap%
\pgfsetroundjoin%
\definecolor{currentfill}{rgb}{0.003922,0.003922,0.003922}%
\pgfsetfillcolor{currentfill}%
\pgfsetfillopacity{0.900000}%
\pgfsetlinewidth{0.507862pt}%
\definecolor{currentstroke}{rgb}{1.000000,1.000000,1.000000}%
\pgfsetstrokecolor{currentstroke}%
\pgfsetstrokeopacity{0.900000}%
\pgfsetdash{}{0pt}%
\pgfpathmoveto{\pgfqpoint{5.097626in}{2.026222in}}%
\pgfpathcurveto{\pgfqpoint{5.109273in}{2.026222in}}{\pgfqpoint{5.120446in}{2.030850in}}{\pgfqpoint{5.128682in}{2.039086in}}%
\pgfpathcurveto{\pgfqpoint{5.136918in}{2.047322in}}{\pgfqpoint{5.141546in}{2.058495in}}{\pgfqpoint{5.141546in}{2.070143in}}%
\pgfpathcurveto{\pgfqpoint{5.141546in}{2.081790in}}{\pgfqpoint{5.136918in}{2.092963in}}{\pgfqpoint{5.128682in}{2.101199in}}%
\pgfpathcurveto{\pgfqpoint{5.120446in}{2.109435in}}{\pgfqpoint{5.109273in}{2.114063in}}{\pgfqpoint{5.097626in}{2.114063in}}%
\pgfpathcurveto{\pgfqpoint{5.085978in}{2.114063in}}{\pgfqpoint{5.074805in}{2.109435in}}{\pgfqpoint{5.066569in}{2.101199in}}%
\pgfpathcurveto{\pgfqpoint{5.058333in}{2.092963in}}{\pgfqpoint{5.053705in}{2.081790in}}{\pgfqpoint{5.053705in}{2.070143in}}%
\pgfpathcurveto{\pgfqpoint{5.053705in}{2.058495in}}{\pgfqpoint{5.058333in}{2.047322in}}{\pgfqpoint{5.066569in}{2.039086in}}%
\pgfpathcurveto{\pgfqpoint{5.074805in}{2.030850in}}{\pgfqpoint{5.085978in}{2.026222in}}{\pgfqpoint{5.097626in}{2.026222in}}%
\pgfpathlineto{\pgfqpoint{5.097626in}{2.026222in}}%
\pgfpathclose%
\pgfusepath{stroke,fill}%
\end{pgfscope}%
\begin{pgfscope}%
\pgfpathrectangle{\pgfqpoint{4.871338in}{0.529443in}}{\pgfqpoint{1.636659in}{1.745990in}}%
\pgfusepath{clip}%
\pgfsetbuttcap%
\pgfsetroundjoin%
\definecolor{currentfill}{rgb}{0.003922,0.003922,0.003922}%
\pgfsetfillcolor{currentfill}%
\pgfsetfillopacity{0.900000}%
\pgfsetlinewidth{0.507862pt}%
\definecolor{currentstroke}{rgb}{1.000000,1.000000,1.000000}%
\pgfsetstrokecolor{currentstroke}%
\pgfsetstrokeopacity{0.900000}%
\pgfsetdash{}{0pt}%
\pgfpathmoveto{\pgfqpoint{6.312351in}{0.603727in}}%
\pgfpathcurveto{\pgfqpoint{6.323999in}{0.603727in}}{\pgfqpoint{6.335171in}{0.608354in}}{\pgfqpoint{6.343407in}{0.616591in}}%
\pgfpathcurveto{\pgfqpoint{6.351643in}{0.624827in}}{\pgfqpoint{6.356271in}{0.635999in}}{\pgfqpoint{6.356271in}{0.647647in}}%
\pgfpathcurveto{\pgfqpoint{6.356271in}{0.659295in}}{\pgfqpoint{6.351643in}{0.670467in}}{\pgfqpoint{6.343407in}{0.678704in}}%
\pgfpathcurveto{\pgfqpoint{6.335171in}{0.686940in}}{\pgfqpoint{6.323999in}{0.691568in}}{\pgfqpoint{6.312351in}{0.691568in}}%
\pgfpathcurveto{\pgfqpoint{6.300703in}{0.691568in}}{\pgfqpoint{6.289530in}{0.686940in}}{\pgfqpoint{6.281294in}{0.678704in}}%
\pgfpathcurveto{\pgfqpoint{6.273058in}{0.670467in}}{\pgfqpoint{6.268430in}{0.659295in}}{\pgfqpoint{6.268430in}{0.647647in}}%
\pgfpathcurveto{\pgfqpoint{6.268430in}{0.635999in}}{\pgfqpoint{6.273058in}{0.624827in}}{\pgfqpoint{6.281294in}{0.616591in}}%
\pgfpathcurveto{\pgfqpoint{6.289530in}{0.608354in}}{\pgfqpoint{6.300703in}{0.603727in}}{\pgfqpoint{6.312351in}{0.603727in}}%
\pgfpathlineto{\pgfqpoint{6.312351in}{0.603727in}}%
\pgfpathclose%
\pgfusepath{stroke,fill}%
\end{pgfscope}%
\begin{pgfscope}%
\pgfpathrectangle{\pgfqpoint{4.871338in}{0.529443in}}{\pgfqpoint{1.636659in}{1.745990in}}%
\pgfusepath{clip}%
\pgfsetbuttcap%
\pgfsetroundjoin%
\definecolor{currentfill}{rgb}{0.003922,0.003922,0.003922}%
\pgfsetfillcolor{currentfill}%
\pgfsetfillopacity{0.900000}%
\pgfsetlinewidth{0.507862pt}%
\definecolor{currentstroke}{rgb}{1.000000,1.000000,1.000000}%
\pgfsetstrokecolor{currentstroke}%
\pgfsetstrokeopacity{0.900000}%
\pgfsetdash{}{0pt}%
\pgfpathmoveto{\pgfqpoint{6.277923in}{0.590650in}}%
\pgfpathcurveto{\pgfqpoint{6.289571in}{0.590650in}}{\pgfqpoint{6.300743in}{0.595278in}}{\pgfqpoint{6.308980in}{0.603514in}}%
\pgfpathcurveto{\pgfqpoint{6.317216in}{0.611750in}}{\pgfqpoint{6.321844in}{0.622923in}}{\pgfqpoint{6.321844in}{0.634571in}}%
\pgfpathcurveto{\pgfqpoint{6.321844in}{0.646219in}}{\pgfqpoint{6.317216in}{0.657391in}}{\pgfqpoint{6.308980in}{0.665627in}}%
\pgfpathcurveto{\pgfqpoint{6.300743in}{0.673863in}}{\pgfqpoint{6.289571in}{0.678491in}}{\pgfqpoint{6.277923in}{0.678491in}}%
\pgfpathcurveto{\pgfqpoint{6.266275in}{0.678491in}}{\pgfqpoint{6.255103in}{0.673863in}}{\pgfqpoint{6.246867in}{0.665627in}}%
\pgfpathcurveto{\pgfqpoint{6.238630in}{0.657391in}}{\pgfqpoint{6.234003in}{0.646219in}}{\pgfqpoint{6.234003in}{0.634571in}}%
\pgfpathcurveto{\pgfqpoint{6.234003in}{0.622923in}}{\pgfqpoint{6.238630in}{0.611750in}}{\pgfqpoint{6.246867in}{0.603514in}}%
\pgfpathcurveto{\pgfqpoint{6.255103in}{0.595278in}}{\pgfqpoint{6.266275in}{0.590650in}}{\pgfqpoint{6.277923in}{0.590650in}}%
\pgfpathlineto{\pgfqpoint{6.277923in}{0.590650in}}%
\pgfpathclose%
\pgfusepath{stroke,fill}%
\end{pgfscope}%
\begin{pgfscope}%
\pgfpathrectangle{\pgfqpoint{4.871338in}{0.529443in}}{\pgfqpoint{1.636659in}{1.745990in}}%
\pgfusepath{clip}%
\pgfsetbuttcap%
\pgfsetroundjoin%
\definecolor{currentfill}{rgb}{0.003922,0.003922,0.003922}%
\pgfsetfillcolor{currentfill}%
\pgfsetfillopacity{0.900000}%
\pgfsetlinewidth{0.507862pt}%
\definecolor{currentstroke}{rgb}{1.000000,1.000000,1.000000}%
\pgfsetstrokecolor{currentstroke}%
\pgfsetstrokeopacity{0.900000}%
\pgfsetdash{}{0pt}%
\pgfpathmoveto{\pgfqpoint{6.257166in}{0.613958in}}%
\pgfpathcurveto{\pgfqpoint{6.268813in}{0.613958in}}{\pgfqpoint{6.279986in}{0.618586in}}{\pgfqpoint{6.288222in}{0.626822in}}%
\pgfpathcurveto{\pgfqpoint{6.296458in}{0.635059in}}{\pgfqpoint{6.301086in}{0.646231in}}{\pgfqpoint{6.301086in}{0.657879in}}%
\pgfpathcurveto{\pgfqpoint{6.301086in}{0.669527in}}{\pgfqpoint{6.296458in}{0.680699in}}{\pgfqpoint{6.288222in}{0.688935in}}%
\pgfpathcurveto{\pgfqpoint{6.279986in}{0.697172in}}{\pgfqpoint{6.268813in}{0.701799in}}{\pgfqpoint{6.257166in}{0.701799in}}%
\pgfpathcurveto{\pgfqpoint{6.245518in}{0.701799in}}{\pgfqpoint{6.234345in}{0.697172in}}{\pgfqpoint{6.226109in}{0.688935in}}%
\pgfpathcurveto{\pgfqpoint{6.217873in}{0.680699in}}{\pgfqpoint{6.213245in}{0.669527in}}{\pgfqpoint{6.213245in}{0.657879in}}%
\pgfpathcurveto{\pgfqpoint{6.213245in}{0.646231in}}{\pgfqpoint{6.217873in}{0.635059in}}{\pgfqpoint{6.226109in}{0.626822in}}%
\pgfpathcurveto{\pgfqpoint{6.234345in}{0.618586in}}{\pgfqpoint{6.245518in}{0.613958in}}{\pgfqpoint{6.257166in}{0.613958in}}%
\pgfpathlineto{\pgfqpoint{6.257166in}{0.613958in}}%
\pgfpathclose%
\pgfusepath{stroke,fill}%
\end{pgfscope}%
\begin{pgfscope}%
\pgfpathrectangle{\pgfqpoint{4.871338in}{0.529443in}}{\pgfqpoint{1.636659in}{1.745990in}}%
\pgfusepath{clip}%
\pgfsetbuttcap%
\pgfsetroundjoin%
\definecolor{currentfill}{rgb}{0.003922,0.003922,0.003922}%
\pgfsetfillcolor{currentfill}%
\pgfsetfillopacity{0.900000}%
\pgfsetlinewidth{0.507862pt}%
\definecolor{currentstroke}{rgb}{1.000000,1.000000,1.000000}%
\pgfsetstrokecolor{currentstroke}%
\pgfsetstrokeopacity{0.900000}%
\pgfsetdash{}{0pt}%
\pgfpathmoveto{\pgfqpoint{5.038218in}{2.123022in}}%
\pgfpathcurveto{\pgfqpoint{5.049866in}{2.123022in}}{\pgfqpoint{5.061039in}{2.127650in}}{\pgfqpoint{5.069275in}{2.135886in}}%
\pgfpathcurveto{\pgfqpoint{5.077511in}{2.144122in}}{\pgfqpoint{5.082139in}{2.155294in}}{\pgfqpoint{5.082139in}{2.166942in}}%
\pgfpathcurveto{\pgfqpoint{5.082139in}{2.178590in}}{\pgfqpoint{5.077511in}{2.189763in}}{\pgfqpoint{5.069275in}{2.197999in}}%
\pgfpathcurveto{\pgfqpoint{5.061039in}{2.206235in}}{\pgfqpoint{5.049866in}{2.210863in}}{\pgfqpoint{5.038218in}{2.210863in}}%
\pgfpathcurveto{\pgfqpoint{5.026570in}{2.210863in}}{\pgfqpoint{5.015398in}{2.206235in}}{\pgfqpoint{5.007162in}{2.197999in}}%
\pgfpathcurveto{\pgfqpoint{4.998926in}{2.189763in}}{\pgfqpoint{4.994298in}{2.178590in}}{\pgfqpoint{4.994298in}{2.166942in}}%
\pgfpathcurveto{\pgfqpoint{4.994298in}{2.155294in}}{\pgfqpoint{4.998926in}{2.144122in}}{\pgfqpoint{5.007162in}{2.135886in}}%
\pgfpathcurveto{\pgfqpoint{5.015398in}{2.127650in}}{\pgfqpoint{5.026570in}{2.123022in}}{\pgfqpoint{5.038218in}{2.123022in}}%
\pgfpathlineto{\pgfqpoint{5.038218in}{2.123022in}}%
\pgfpathclose%
\pgfusepath{stroke,fill}%
\end{pgfscope}%
\begin{pgfscope}%
\pgfpathrectangle{\pgfqpoint{4.871338in}{0.529443in}}{\pgfqpoint{1.636659in}{1.745990in}}%
\pgfusepath{clip}%
\pgfsetbuttcap%
\pgfsetroundjoin%
\definecolor{currentfill}{rgb}{0.003922,0.003922,0.003922}%
\pgfsetfillcolor{currentfill}%
\pgfsetfillopacity{0.900000}%
\pgfsetlinewidth{0.507862pt}%
\definecolor{currentstroke}{rgb}{1.000000,1.000000,1.000000}%
\pgfsetstrokecolor{currentstroke}%
\pgfsetstrokeopacity{0.900000}%
\pgfsetdash{}{0pt}%
\pgfpathmoveto{\pgfqpoint{6.380256in}{0.568071in}}%
\pgfpathcurveto{\pgfqpoint{6.391904in}{0.568071in}}{\pgfqpoint{6.403076in}{0.572698in}}{\pgfqpoint{6.411312in}{0.580935in}}%
\pgfpathcurveto{\pgfqpoint{6.419549in}{0.589171in}}{\pgfqpoint{6.424176in}{0.600343in}}{\pgfqpoint{6.424176in}{0.611991in}}%
\pgfpathcurveto{\pgfqpoint{6.424176in}{0.623639in}}{\pgfqpoint{6.419549in}{0.634811in}}{\pgfqpoint{6.411312in}{0.643048in}}%
\pgfpathcurveto{\pgfqpoint{6.403076in}{0.651284in}}{\pgfqpoint{6.391904in}{0.655912in}}{\pgfqpoint{6.380256in}{0.655912in}}%
\pgfpathcurveto{\pgfqpoint{6.368608in}{0.655912in}}{\pgfqpoint{6.357436in}{0.651284in}}{\pgfqpoint{6.349199in}{0.643048in}}%
\pgfpathcurveto{\pgfqpoint{6.340963in}{0.634811in}}{\pgfqpoint{6.336335in}{0.623639in}}{\pgfqpoint{6.336335in}{0.611991in}}%
\pgfpathcurveto{\pgfqpoint{6.336335in}{0.600343in}}{\pgfqpoint{6.340963in}{0.589171in}}{\pgfqpoint{6.349199in}{0.580935in}}%
\pgfpathcurveto{\pgfqpoint{6.357436in}{0.572698in}}{\pgfqpoint{6.368608in}{0.568071in}}{\pgfqpoint{6.380256in}{0.568071in}}%
\pgfpathlineto{\pgfqpoint{6.380256in}{0.568071in}}%
\pgfpathclose%
\pgfusepath{stroke,fill}%
\end{pgfscope}%
\begin{pgfscope}%
\pgfpathrectangle{\pgfqpoint{4.871338in}{0.529443in}}{\pgfqpoint{1.636659in}{1.745990in}}%
\pgfusepath{clip}%
\pgfsetbuttcap%
\pgfsetroundjoin%
\definecolor{currentfill}{rgb}{0.003922,0.003922,0.003922}%
\pgfsetfillcolor{currentfill}%
\pgfsetfillopacity{0.900000}%
\pgfsetlinewidth{0.507862pt}%
\definecolor{currentstroke}{rgb}{1.000000,1.000000,1.000000}%
\pgfsetstrokecolor{currentstroke}%
\pgfsetstrokeopacity{0.900000}%
\pgfsetdash{}{0pt}%
\pgfpathmoveto{\pgfqpoint{6.324839in}{0.569288in}}%
\pgfpathcurveto{\pgfqpoint{6.336487in}{0.569288in}}{\pgfqpoint{6.347659in}{0.573916in}}{\pgfqpoint{6.355895in}{0.582152in}}%
\pgfpathcurveto{\pgfqpoint{6.364131in}{0.590389in}}{\pgfqpoint{6.368759in}{0.601561in}}{\pgfqpoint{6.368759in}{0.613209in}}%
\pgfpathcurveto{\pgfqpoint{6.368759in}{0.624857in}}{\pgfqpoint{6.364131in}{0.636029in}}{\pgfqpoint{6.355895in}{0.644265in}}%
\pgfpathcurveto{\pgfqpoint{6.347659in}{0.652502in}}{\pgfqpoint{6.336487in}{0.657129in}}{\pgfqpoint{6.324839in}{0.657129in}}%
\pgfpathcurveto{\pgfqpoint{6.313191in}{0.657129in}}{\pgfqpoint{6.302018in}{0.652502in}}{\pgfqpoint{6.293782in}{0.644265in}}%
\pgfpathcurveto{\pgfqpoint{6.285546in}{0.636029in}}{\pgfqpoint{6.280918in}{0.624857in}}{\pgfqpoint{6.280918in}{0.613209in}}%
\pgfpathcurveto{\pgfqpoint{6.280918in}{0.601561in}}{\pgfqpoint{6.285546in}{0.590389in}}{\pgfqpoint{6.293782in}{0.582152in}}%
\pgfpathcurveto{\pgfqpoint{6.302018in}{0.573916in}}{\pgfqpoint{6.313191in}{0.569288in}}{\pgfqpoint{6.324839in}{0.569288in}}%
\pgfpathlineto{\pgfqpoint{6.324839in}{0.569288in}}%
\pgfpathclose%
\pgfusepath{stroke,fill}%
\end{pgfscope}%
\begin{pgfscope}%
\pgfpathrectangle{\pgfqpoint{4.871338in}{0.529443in}}{\pgfqpoint{1.636659in}{1.745990in}}%
\pgfusepath{clip}%
\pgfsetbuttcap%
\pgfsetroundjoin%
\definecolor{currentfill}{rgb}{0.003922,0.003922,0.003922}%
\pgfsetfillcolor{currentfill}%
\pgfsetfillopacity{0.900000}%
\pgfsetlinewidth{0.507862pt}%
\definecolor{currentstroke}{rgb}{1.000000,1.000000,1.000000}%
\pgfsetstrokecolor{currentstroke}%
\pgfsetstrokeopacity{0.900000}%
\pgfsetdash{}{0pt}%
\pgfpathmoveto{\pgfqpoint{4.981767in}{2.145229in}}%
\pgfpathcurveto{\pgfqpoint{4.993414in}{2.145229in}}{\pgfqpoint{5.004587in}{2.149857in}}{\pgfqpoint{5.012823in}{2.158093in}}%
\pgfpathcurveto{\pgfqpoint{5.021059in}{2.166329in}}{\pgfqpoint{5.025687in}{2.177501in}}{\pgfqpoint{5.025687in}{2.189149in}}%
\pgfpathcurveto{\pgfqpoint{5.025687in}{2.200797in}}{\pgfqpoint{5.021059in}{2.211970in}}{\pgfqpoint{5.012823in}{2.220206in}}%
\pgfpathcurveto{\pgfqpoint{5.004587in}{2.228442in}}{\pgfqpoint{4.993414in}{2.233070in}}{\pgfqpoint{4.981767in}{2.233070in}}%
\pgfpathcurveto{\pgfqpoint{4.970119in}{2.233070in}}{\pgfqpoint{4.958946in}{2.228442in}}{\pgfqpoint{4.950710in}{2.220206in}}%
\pgfpathcurveto{\pgfqpoint{4.942474in}{2.211970in}}{\pgfqpoint{4.937846in}{2.200797in}}{\pgfqpoint{4.937846in}{2.189149in}}%
\pgfpathcurveto{\pgfqpoint{4.937846in}{2.177501in}}{\pgfqpoint{4.942474in}{2.166329in}}{\pgfqpoint{4.950710in}{2.158093in}}%
\pgfpathcurveto{\pgfqpoint{4.958946in}{2.149857in}}{\pgfqpoint{4.970119in}{2.145229in}}{\pgfqpoint{4.981767in}{2.145229in}}%
\pgfpathlineto{\pgfqpoint{4.981767in}{2.145229in}}%
\pgfpathclose%
\pgfusepath{stroke,fill}%
\end{pgfscope}%
\begin{pgfscope}%
\pgfpathrectangle{\pgfqpoint{4.871338in}{0.529443in}}{\pgfqpoint{1.636659in}{1.745990in}}%
\pgfusepath{clip}%
\pgfsetbuttcap%
\pgfsetroundjoin%
\definecolor{currentfill}{rgb}{0.003922,0.003922,0.003922}%
\pgfsetfillcolor{currentfill}%
\pgfsetfillopacity{0.900000}%
\pgfsetlinewidth{0.507862pt}%
\definecolor{currentstroke}{rgb}{1.000000,1.000000,1.000000}%
\pgfsetstrokecolor{currentstroke}%
\pgfsetstrokeopacity{0.900000}%
\pgfsetdash{}{0pt}%
\pgfpathmoveto{\pgfqpoint{6.187061in}{0.615455in}}%
\pgfpathcurveto{\pgfqpoint{6.198708in}{0.615455in}}{\pgfqpoint{6.209881in}{0.620083in}}{\pgfqpoint{6.218117in}{0.628319in}}%
\pgfpathcurveto{\pgfqpoint{6.226353in}{0.636555in}}{\pgfqpoint{6.230981in}{0.647728in}}{\pgfqpoint{6.230981in}{0.659376in}}%
\pgfpathcurveto{\pgfqpoint{6.230981in}{0.671024in}}{\pgfqpoint{6.226353in}{0.682196in}}{\pgfqpoint{6.218117in}{0.690432in}}%
\pgfpathcurveto{\pgfqpoint{6.209881in}{0.698668in}}{\pgfqpoint{6.198708in}{0.703296in}}{\pgfqpoint{6.187061in}{0.703296in}}%
\pgfpathcurveto{\pgfqpoint{6.175413in}{0.703296in}}{\pgfqpoint{6.164240in}{0.698668in}}{\pgfqpoint{6.156004in}{0.690432in}}%
\pgfpathcurveto{\pgfqpoint{6.147768in}{0.682196in}}{\pgfqpoint{6.143140in}{0.671024in}}{\pgfqpoint{6.143140in}{0.659376in}}%
\pgfpathcurveto{\pgfqpoint{6.143140in}{0.647728in}}{\pgfqpoint{6.147768in}{0.636555in}}{\pgfqpoint{6.156004in}{0.628319in}}%
\pgfpathcurveto{\pgfqpoint{6.164240in}{0.620083in}}{\pgfqpoint{6.175413in}{0.615455in}}{\pgfqpoint{6.187061in}{0.615455in}}%
\pgfpathlineto{\pgfqpoint{6.187061in}{0.615455in}}%
\pgfpathclose%
\pgfusepath{stroke,fill}%
\end{pgfscope}%
\begin{pgfscope}%
\pgfpathrectangle{\pgfqpoint{4.871338in}{0.529443in}}{\pgfqpoint{1.636659in}{1.745990in}}%
\pgfusepath{clip}%
\pgfsetbuttcap%
\pgfsetroundjoin%
\definecolor{currentfill}{rgb}{0.003922,0.003922,0.003922}%
\pgfsetfillcolor{currentfill}%
\pgfsetfillopacity{0.900000}%
\pgfsetlinewidth{0.507862pt}%
\definecolor{currentstroke}{rgb}{1.000000,1.000000,1.000000}%
\pgfsetstrokecolor{currentstroke}%
\pgfsetstrokeopacity{0.900000}%
\pgfsetdash{}{0pt}%
\pgfpathmoveto{\pgfqpoint{5.064644in}{2.152149in}}%
\pgfpathcurveto{\pgfqpoint{5.076292in}{2.152149in}}{\pgfqpoint{5.087465in}{2.156777in}}{\pgfqpoint{5.095701in}{2.165013in}}%
\pgfpathcurveto{\pgfqpoint{5.103937in}{2.173249in}}{\pgfqpoint{5.108565in}{2.184422in}}{\pgfqpoint{5.108565in}{2.196069in}}%
\pgfpathcurveto{\pgfqpoint{5.108565in}{2.207717in}}{\pgfqpoint{5.103937in}{2.218890in}}{\pgfqpoint{5.095701in}{2.227126in}}%
\pgfpathcurveto{\pgfqpoint{5.087465in}{2.235362in}}{\pgfqpoint{5.076292in}{2.239990in}}{\pgfqpoint{5.064644in}{2.239990in}}%
\pgfpathcurveto{\pgfqpoint{5.052996in}{2.239990in}}{\pgfqpoint{5.041824in}{2.235362in}}{\pgfqpoint{5.033588in}{2.227126in}}%
\pgfpathcurveto{\pgfqpoint{5.025352in}{2.218890in}}{\pgfqpoint{5.020724in}{2.207717in}}{\pgfqpoint{5.020724in}{2.196069in}}%
\pgfpathcurveto{\pgfqpoint{5.020724in}{2.184422in}}{\pgfqpoint{5.025352in}{2.173249in}}{\pgfqpoint{5.033588in}{2.165013in}}%
\pgfpathcurveto{\pgfqpoint{5.041824in}{2.156777in}}{\pgfqpoint{5.052996in}{2.152149in}}{\pgfqpoint{5.064644in}{2.152149in}}%
\pgfpathlineto{\pgfqpoint{5.064644in}{2.152149in}}%
\pgfpathclose%
\pgfusepath{stroke,fill}%
\end{pgfscope}%
\begin{pgfscope}%
\pgfpathrectangle{\pgfqpoint{4.871338in}{0.529443in}}{\pgfqpoint{1.636659in}{1.745990in}}%
\pgfusepath{clip}%
\pgfsetbuttcap%
\pgfsetroundjoin%
\definecolor{currentfill}{rgb}{0.003922,0.003922,0.003922}%
\pgfsetfillcolor{currentfill}%
\pgfsetfillopacity{0.900000}%
\pgfsetlinewidth{0.507862pt}%
\definecolor{currentstroke}{rgb}{1.000000,1.000000,1.000000}%
\pgfsetstrokecolor{currentstroke}%
\pgfsetstrokeopacity{0.900000}%
\pgfsetdash{}{0pt}%
\pgfpathmoveto{\pgfqpoint{6.286352in}{0.619886in}}%
\pgfpathcurveto{\pgfqpoint{6.298000in}{0.619886in}}{\pgfqpoint{6.309172in}{0.624514in}}{\pgfqpoint{6.317409in}{0.632750in}}%
\pgfpathcurveto{\pgfqpoint{6.325645in}{0.640987in}}{\pgfqpoint{6.330273in}{0.652159in}}{\pgfqpoint{6.330273in}{0.663807in}}%
\pgfpathcurveto{\pgfqpoint{6.330273in}{0.675455in}}{\pgfqpoint{6.325645in}{0.686627in}}{\pgfqpoint{6.317409in}{0.694863in}}%
\pgfpathcurveto{\pgfqpoint{6.309172in}{0.703100in}}{\pgfqpoint{6.298000in}{0.707727in}}{\pgfqpoint{6.286352in}{0.707727in}}%
\pgfpathcurveto{\pgfqpoint{6.274704in}{0.707727in}}{\pgfqpoint{6.263532in}{0.703100in}}{\pgfqpoint{6.255296in}{0.694863in}}%
\pgfpathcurveto{\pgfqpoint{6.247059in}{0.686627in}}{\pgfqpoint{6.242432in}{0.675455in}}{\pgfqpoint{6.242432in}{0.663807in}}%
\pgfpathcurveto{\pgfqpoint{6.242432in}{0.652159in}}{\pgfqpoint{6.247059in}{0.640987in}}{\pgfqpoint{6.255296in}{0.632750in}}%
\pgfpathcurveto{\pgfqpoint{6.263532in}{0.624514in}}{\pgfqpoint{6.274704in}{0.619886in}}{\pgfqpoint{6.286352in}{0.619886in}}%
\pgfpathlineto{\pgfqpoint{6.286352in}{0.619886in}}%
\pgfpathclose%
\pgfusepath{stroke,fill}%
\end{pgfscope}%
\begin{pgfscope}%
\pgfpathrectangle{\pgfqpoint{4.871338in}{0.529443in}}{\pgfqpoint{1.636659in}{1.745990in}}%
\pgfusepath{clip}%
\pgfsetbuttcap%
\pgfsetroundjoin%
\definecolor{currentfill}{rgb}{0.003922,0.003922,0.003922}%
\pgfsetfillcolor{currentfill}%
\pgfsetfillopacity{0.900000}%
\pgfsetlinewidth{0.507862pt}%
\definecolor{currentstroke}{rgb}{1.000000,1.000000,1.000000}%
\pgfsetstrokecolor{currentstroke}%
\pgfsetstrokeopacity{0.900000}%
\pgfsetdash{}{0pt}%
\pgfpathmoveto{\pgfqpoint{6.318765in}{0.631502in}}%
\pgfpathcurveto{\pgfqpoint{6.330413in}{0.631502in}}{\pgfqpoint{6.341585in}{0.636130in}}{\pgfqpoint{6.349821in}{0.644366in}}%
\pgfpathcurveto{\pgfqpoint{6.358057in}{0.652603in}}{\pgfqpoint{6.362685in}{0.663775in}}{\pgfqpoint{6.362685in}{0.675423in}}%
\pgfpathcurveto{\pgfqpoint{6.362685in}{0.687071in}}{\pgfqpoint{6.358057in}{0.698243in}}{\pgfqpoint{6.349821in}{0.706479in}}%
\pgfpathcurveto{\pgfqpoint{6.341585in}{0.714716in}}{\pgfqpoint{6.330413in}{0.719343in}}{\pgfqpoint{6.318765in}{0.719343in}}%
\pgfpathcurveto{\pgfqpoint{6.307117in}{0.719343in}}{\pgfqpoint{6.295944in}{0.714716in}}{\pgfqpoint{6.287708in}{0.706479in}}%
\pgfpathcurveto{\pgfqpoint{6.279472in}{0.698243in}}{\pgfqpoint{6.274844in}{0.687071in}}{\pgfqpoint{6.274844in}{0.675423in}}%
\pgfpathcurveto{\pgfqpoint{6.274844in}{0.663775in}}{\pgfqpoint{6.279472in}{0.652603in}}{\pgfqpoint{6.287708in}{0.644366in}}%
\pgfpathcurveto{\pgfqpoint{6.295944in}{0.636130in}}{\pgfqpoint{6.307117in}{0.631502in}}{\pgfqpoint{6.318765in}{0.631502in}}%
\pgfpathlineto{\pgfqpoint{6.318765in}{0.631502in}}%
\pgfpathclose%
\pgfusepath{stroke,fill}%
\end{pgfscope}%
\begin{pgfscope}%
\pgfpathrectangle{\pgfqpoint{4.871338in}{0.529443in}}{\pgfqpoint{1.636659in}{1.745990in}}%
\pgfusepath{clip}%
\pgfsetbuttcap%
\pgfsetroundjoin%
\definecolor{currentfill}{rgb}{0.003922,0.003922,0.003922}%
\pgfsetfillcolor{currentfill}%
\pgfsetfillopacity{0.900000}%
\pgfsetlinewidth{0.507862pt}%
\definecolor{currentstroke}{rgb}{1.000000,1.000000,1.000000}%
\pgfsetstrokecolor{currentstroke}%
\pgfsetstrokeopacity{0.900000}%
\pgfsetdash{}{0pt}%
\pgfpathmoveto{\pgfqpoint{6.433603in}{0.614392in}}%
\pgfpathcurveto{\pgfqpoint{6.445251in}{0.614392in}}{\pgfqpoint{6.456423in}{0.619020in}}{\pgfqpoint{6.464660in}{0.627256in}}%
\pgfpathcurveto{\pgfqpoint{6.472896in}{0.635492in}}{\pgfqpoint{6.477524in}{0.646665in}}{\pgfqpoint{6.477524in}{0.658312in}}%
\pgfpathcurveto{\pgfqpoint{6.477524in}{0.669960in}}{\pgfqpoint{6.472896in}{0.681133in}}{\pgfqpoint{6.464660in}{0.689369in}}%
\pgfpathcurveto{\pgfqpoint{6.456423in}{0.697605in}}{\pgfqpoint{6.445251in}{0.702233in}}{\pgfqpoint{6.433603in}{0.702233in}}%
\pgfpathcurveto{\pgfqpoint{6.421955in}{0.702233in}}{\pgfqpoint{6.410783in}{0.697605in}}{\pgfqpoint{6.402547in}{0.689369in}}%
\pgfpathcurveto{\pgfqpoint{6.394310in}{0.681133in}}{\pgfqpoint{6.389683in}{0.669960in}}{\pgfqpoint{6.389683in}{0.658312in}}%
\pgfpathcurveto{\pgfqpoint{6.389683in}{0.646665in}}{\pgfqpoint{6.394310in}{0.635492in}}{\pgfqpoint{6.402547in}{0.627256in}}%
\pgfpathcurveto{\pgfqpoint{6.410783in}{0.619020in}}{\pgfqpoint{6.421955in}{0.614392in}}{\pgfqpoint{6.433603in}{0.614392in}}%
\pgfpathlineto{\pgfqpoint{6.433603in}{0.614392in}}%
\pgfpathclose%
\pgfusepath{stroke,fill}%
\end{pgfscope}%
\begin{pgfscope}%
\pgfpathrectangle{\pgfqpoint{4.871338in}{0.529443in}}{\pgfqpoint{1.636659in}{1.745990in}}%
\pgfusepath{clip}%
\pgfsetbuttcap%
\pgfsetroundjoin%
\definecolor{currentfill}{rgb}{0.003922,0.003922,0.003922}%
\pgfsetfillcolor{currentfill}%
\pgfsetfillopacity{0.900000}%
\pgfsetlinewidth{0.507862pt}%
\definecolor{currentstroke}{rgb}{1.000000,1.000000,1.000000}%
\pgfsetstrokecolor{currentstroke}%
\pgfsetstrokeopacity{0.900000}%
\pgfsetdash{}{0pt}%
\pgfpathmoveto{\pgfqpoint{4.978664in}{2.068396in}}%
\pgfpathcurveto{\pgfqpoint{4.990312in}{2.068396in}}{\pgfqpoint{5.001484in}{2.073024in}}{\pgfqpoint{5.009720in}{2.081260in}}%
\pgfpathcurveto{\pgfqpoint{5.017957in}{2.089496in}}{\pgfqpoint{5.022584in}{2.100668in}}{\pgfqpoint{5.022584in}{2.112316in}}%
\pgfpathcurveto{\pgfqpoint{5.022584in}{2.123964in}}{\pgfqpoint{5.017957in}{2.135137in}}{\pgfqpoint{5.009720in}{2.143373in}}%
\pgfpathcurveto{\pgfqpoint{5.001484in}{2.151609in}}{\pgfqpoint{4.990312in}{2.156237in}}{\pgfqpoint{4.978664in}{2.156237in}}%
\pgfpathcurveto{\pgfqpoint{4.967016in}{2.156237in}}{\pgfqpoint{4.955844in}{2.151609in}}{\pgfqpoint{4.947607in}{2.143373in}}%
\pgfpathcurveto{\pgfqpoint{4.939371in}{2.135137in}}{\pgfqpoint{4.934743in}{2.123964in}}{\pgfqpoint{4.934743in}{2.112316in}}%
\pgfpathcurveto{\pgfqpoint{4.934743in}{2.100668in}}{\pgfqpoint{4.939371in}{2.089496in}}{\pgfqpoint{4.947607in}{2.081260in}}%
\pgfpathcurveto{\pgfqpoint{4.955844in}{2.073024in}}{\pgfqpoint{4.967016in}{2.068396in}}{\pgfqpoint{4.978664in}{2.068396in}}%
\pgfpathlineto{\pgfqpoint{4.978664in}{2.068396in}}%
\pgfpathclose%
\pgfusepath{stroke,fill}%
\end{pgfscope}%
\begin{pgfscope}%
\pgfpathrectangle{\pgfqpoint{4.871338in}{0.529443in}}{\pgfqpoint{1.636659in}{1.745990in}}%
\pgfusepath{clip}%
\pgfsetbuttcap%
\pgfsetroundjoin%
\definecolor{currentfill}{rgb}{0.003922,0.003922,0.003922}%
\pgfsetfillcolor{currentfill}%
\pgfsetfillopacity{0.900000}%
\pgfsetlinewidth{0.507862pt}%
\definecolor{currentstroke}{rgb}{1.000000,1.000000,1.000000}%
\pgfsetstrokecolor{currentstroke}%
\pgfsetstrokeopacity{0.900000}%
\pgfsetdash{}{0pt}%
\pgfpathmoveto{\pgfqpoint{6.126701in}{0.622383in}}%
\pgfpathcurveto{\pgfqpoint{6.138349in}{0.622383in}}{\pgfqpoint{6.149522in}{0.627011in}}{\pgfqpoint{6.157758in}{0.635247in}}%
\pgfpathcurveto{\pgfqpoint{6.165994in}{0.643483in}}{\pgfqpoint{6.170622in}{0.654656in}}{\pgfqpoint{6.170622in}{0.666303in}}%
\pgfpathcurveto{\pgfqpoint{6.170622in}{0.677951in}}{\pgfqpoint{6.165994in}{0.689124in}}{\pgfqpoint{6.157758in}{0.697360in}}%
\pgfpathcurveto{\pgfqpoint{6.149522in}{0.705596in}}{\pgfqpoint{6.138349in}{0.710224in}}{\pgfqpoint{6.126701in}{0.710224in}}%
\pgfpathcurveto{\pgfqpoint{6.115054in}{0.710224in}}{\pgfqpoint{6.103881in}{0.705596in}}{\pgfqpoint{6.095645in}{0.697360in}}%
\pgfpathcurveto{\pgfqpoint{6.087409in}{0.689124in}}{\pgfqpoint{6.082781in}{0.677951in}}{\pgfqpoint{6.082781in}{0.666303in}}%
\pgfpathcurveto{\pgfqpoint{6.082781in}{0.654656in}}{\pgfqpoint{6.087409in}{0.643483in}}{\pgfqpoint{6.095645in}{0.635247in}}%
\pgfpathcurveto{\pgfqpoint{6.103881in}{0.627011in}}{\pgfqpoint{6.115054in}{0.622383in}}{\pgfqpoint{6.126701in}{0.622383in}}%
\pgfpathlineto{\pgfqpoint{6.126701in}{0.622383in}}%
\pgfpathclose%
\pgfusepath{stroke,fill}%
\end{pgfscope}%
\begin{pgfscope}%
\pgfpathrectangle{\pgfqpoint{4.871338in}{0.529443in}}{\pgfqpoint{1.636659in}{1.745990in}}%
\pgfusepath{clip}%
\pgfsetbuttcap%
\pgfsetroundjoin%
\definecolor{currentfill}{rgb}{0.003922,0.003922,0.003922}%
\pgfsetfillcolor{currentfill}%
\pgfsetfillopacity{0.900000}%
\pgfsetlinewidth{0.507862pt}%
\definecolor{currentstroke}{rgb}{1.000000,1.000000,1.000000}%
\pgfsetstrokecolor{currentstroke}%
\pgfsetstrokeopacity{0.900000}%
\pgfsetdash{}{0pt}%
\pgfpathmoveto{\pgfqpoint{5.039352in}{2.151340in}}%
\pgfpathcurveto{\pgfqpoint{5.051000in}{2.151340in}}{\pgfqpoint{5.062173in}{2.155968in}}{\pgfqpoint{5.070409in}{2.164204in}}%
\pgfpathcurveto{\pgfqpoint{5.078645in}{2.172440in}}{\pgfqpoint{5.083273in}{2.183612in}}{\pgfqpoint{5.083273in}{2.195260in}}%
\pgfpathcurveto{\pgfqpoint{5.083273in}{2.206908in}}{\pgfqpoint{5.078645in}{2.218081in}}{\pgfqpoint{5.070409in}{2.226317in}}%
\pgfpathcurveto{\pgfqpoint{5.062173in}{2.234553in}}{\pgfqpoint{5.051000in}{2.239181in}}{\pgfqpoint{5.039352in}{2.239181in}}%
\pgfpathcurveto{\pgfqpoint{5.027704in}{2.239181in}}{\pgfqpoint{5.016532in}{2.234553in}}{\pgfqpoint{5.008296in}{2.226317in}}%
\pgfpathcurveto{\pgfqpoint{5.000060in}{2.218081in}}{\pgfqpoint{4.995432in}{2.206908in}}{\pgfqpoint{4.995432in}{2.195260in}}%
\pgfpathcurveto{\pgfqpoint{4.995432in}{2.183612in}}{\pgfqpoint{5.000060in}{2.172440in}}{\pgfqpoint{5.008296in}{2.164204in}}%
\pgfpathcurveto{\pgfqpoint{5.016532in}{2.155968in}}{\pgfqpoint{5.027704in}{2.151340in}}{\pgfqpoint{5.039352in}{2.151340in}}%
\pgfpathlineto{\pgfqpoint{5.039352in}{2.151340in}}%
\pgfpathclose%
\pgfusepath{stroke,fill}%
\end{pgfscope}%
\begin{pgfscope}%
\pgfpathrectangle{\pgfqpoint{4.871338in}{0.529443in}}{\pgfqpoint{1.636659in}{1.745990in}}%
\pgfusepath{clip}%
\pgfsetbuttcap%
\pgfsetroundjoin%
\definecolor{currentfill}{rgb}{0.003922,0.003922,0.003922}%
\pgfsetfillcolor{currentfill}%
\pgfsetfillopacity{0.900000}%
\pgfsetlinewidth{0.507862pt}%
\definecolor{currentstroke}{rgb}{1.000000,1.000000,1.000000}%
\pgfsetstrokecolor{currentstroke}%
\pgfsetstrokeopacity{0.900000}%
\pgfsetdash{}{0pt}%
\pgfpathmoveto{\pgfqpoint{6.267189in}{0.662948in}}%
\pgfpathcurveto{\pgfqpoint{6.278837in}{0.662948in}}{\pgfqpoint{6.290009in}{0.667576in}}{\pgfqpoint{6.298246in}{0.675812in}}%
\pgfpathcurveto{\pgfqpoint{6.306482in}{0.684049in}}{\pgfqpoint{6.311110in}{0.695221in}}{\pgfqpoint{6.311110in}{0.706869in}}%
\pgfpathcurveto{\pgfqpoint{6.311110in}{0.718517in}}{\pgfqpoint{6.306482in}{0.729689in}}{\pgfqpoint{6.298246in}{0.737925in}}%
\pgfpathcurveto{\pgfqpoint{6.290009in}{0.746162in}}{\pgfqpoint{6.278837in}{0.750789in}}{\pgfqpoint{6.267189in}{0.750789in}}%
\pgfpathcurveto{\pgfqpoint{6.255541in}{0.750789in}}{\pgfqpoint{6.244369in}{0.746162in}}{\pgfqpoint{6.236133in}{0.737925in}}%
\pgfpathcurveto{\pgfqpoint{6.227896in}{0.729689in}}{\pgfqpoint{6.223269in}{0.718517in}}{\pgfqpoint{6.223269in}{0.706869in}}%
\pgfpathcurveto{\pgfqpoint{6.223269in}{0.695221in}}{\pgfqpoint{6.227896in}{0.684049in}}{\pgfqpoint{6.236133in}{0.675812in}}%
\pgfpathcurveto{\pgfqpoint{6.244369in}{0.667576in}}{\pgfqpoint{6.255541in}{0.662948in}}{\pgfqpoint{6.267189in}{0.662948in}}%
\pgfpathlineto{\pgfqpoint{6.267189in}{0.662948in}}%
\pgfpathclose%
\pgfusepath{stroke,fill}%
\end{pgfscope}%
\begin{pgfscope}%
\pgfpathrectangle{\pgfqpoint{4.871338in}{0.529443in}}{\pgfqpoint{1.636659in}{1.745990in}}%
\pgfusepath{clip}%
\pgfsetbuttcap%
\pgfsetroundjoin%
\definecolor{currentfill}{rgb}{0.003922,0.003922,0.003922}%
\pgfsetfillcolor{currentfill}%
\pgfsetfillopacity{0.900000}%
\pgfsetlinewidth{0.507862pt}%
\definecolor{currentstroke}{rgb}{1.000000,1.000000,1.000000}%
\pgfsetstrokecolor{currentstroke}%
\pgfsetstrokeopacity{0.900000}%
\pgfsetdash{}{0pt}%
\pgfpathmoveto{\pgfqpoint{5.014251in}{2.093022in}}%
\pgfpathcurveto{\pgfqpoint{5.025899in}{2.093022in}}{\pgfqpoint{5.037072in}{2.097650in}}{\pgfqpoint{5.045308in}{2.105886in}}%
\pgfpathcurveto{\pgfqpoint{5.053544in}{2.114122in}}{\pgfqpoint{5.058172in}{2.125295in}}{\pgfqpoint{5.058172in}{2.136942in}}%
\pgfpathcurveto{\pgfqpoint{5.058172in}{2.148590in}}{\pgfqpoint{5.053544in}{2.159763in}}{\pgfqpoint{5.045308in}{2.167999in}}%
\pgfpathcurveto{\pgfqpoint{5.037072in}{2.176235in}}{\pgfqpoint{5.025899in}{2.180863in}}{\pgfqpoint{5.014251in}{2.180863in}}%
\pgfpathcurveto{\pgfqpoint{5.002604in}{2.180863in}}{\pgfqpoint{4.991431in}{2.176235in}}{\pgfqpoint{4.983195in}{2.167999in}}%
\pgfpathcurveto{\pgfqpoint{4.974959in}{2.159763in}}{\pgfqpoint{4.970331in}{2.148590in}}{\pgfqpoint{4.970331in}{2.136942in}}%
\pgfpathcurveto{\pgfqpoint{4.970331in}{2.125295in}}{\pgfqpoint{4.974959in}{2.114122in}}{\pgfqpoint{4.983195in}{2.105886in}}%
\pgfpathcurveto{\pgfqpoint{4.991431in}{2.097650in}}{\pgfqpoint{5.002604in}{2.093022in}}{\pgfqpoint{5.014251in}{2.093022in}}%
\pgfpathlineto{\pgfqpoint{5.014251in}{2.093022in}}%
\pgfpathclose%
\pgfusepath{stroke,fill}%
\end{pgfscope}%
\begin{pgfscope}%
\pgfpathrectangle{\pgfqpoint{4.871338in}{0.529443in}}{\pgfqpoint{1.636659in}{1.745990in}}%
\pgfusepath{clip}%
\pgfsetbuttcap%
\pgfsetroundjoin%
\definecolor{currentfill}{rgb}{0.003922,0.003922,0.003922}%
\pgfsetfillcolor{currentfill}%
\pgfsetfillopacity{0.900000}%
\pgfsetlinewidth{0.507862pt}%
\definecolor{currentstroke}{rgb}{1.000000,1.000000,1.000000}%
\pgfsetstrokecolor{currentstroke}%
\pgfsetstrokeopacity{0.900000}%
\pgfsetdash{}{0pt}%
\pgfpathmoveto{\pgfqpoint{6.225126in}{0.564886in}}%
\pgfpathcurveto{\pgfqpoint{6.236774in}{0.564886in}}{\pgfqpoint{6.247947in}{0.569513in}}{\pgfqpoint{6.256183in}{0.577750in}}%
\pgfpathcurveto{\pgfqpoint{6.264419in}{0.585986in}}{\pgfqpoint{6.269047in}{0.597158in}}{\pgfqpoint{6.269047in}{0.608806in}}%
\pgfpathcurveto{\pgfqpoint{6.269047in}{0.620454in}}{\pgfqpoint{6.264419in}{0.631626in}}{\pgfqpoint{6.256183in}{0.639863in}}%
\pgfpathcurveto{\pgfqpoint{6.247947in}{0.648099in}}{\pgfqpoint{6.236774in}{0.652727in}}{\pgfqpoint{6.225126in}{0.652727in}}%
\pgfpathcurveto{\pgfqpoint{6.213478in}{0.652727in}}{\pgfqpoint{6.202306in}{0.648099in}}{\pgfqpoint{6.194070in}{0.639863in}}%
\pgfpathcurveto{\pgfqpoint{6.185834in}{0.631626in}}{\pgfqpoint{6.181206in}{0.620454in}}{\pgfqpoint{6.181206in}{0.608806in}}%
\pgfpathcurveto{\pgfqpoint{6.181206in}{0.597158in}}{\pgfqpoint{6.185834in}{0.585986in}}{\pgfqpoint{6.194070in}{0.577750in}}%
\pgfpathcurveto{\pgfqpoint{6.202306in}{0.569513in}}{\pgfqpoint{6.213478in}{0.564886in}}{\pgfqpoint{6.225126in}{0.564886in}}%
\pgfpathlineto{\pgfqpoint{6.225126in}{0.564886in}}%
\pgfpathclose%
\pgfusepath{stroke,fill}%
\end{pgfscope}%
\begin{pgfscope}%
\pgfpathrectangle{\pgfqpoint{4.871338in}{0.529443in}}{\pgfqpoint{1.636659in}{1.745990in}}%
\pgfusepath{clip}%
\pgfsetbuttcap%
\pgfsetroundjoin%
\definecolor{currentfill}{rgb}{0.003922,0.003922,0.003922}%
\pgfsetfillcolor{currentfill}%
\pgfsetfillopacity{0.900000}%
\pgfsetlinewidth{0.507862pt}%
\definecolor{currentstroke}{rgb}{1.000000,1.000000,1.000000}%
\pgfsetstrokecolor{currentstroke}%
\pgfsetstrokeopacity{0.900000}%
\pgfsetdash{}{0pt}%
\pgfpathmoveto{\pgfqpoint{4.985351in}{2.045217in}}%
\pgfpathcurveto{\pgfqpoint{4.996999in}{2.045217in}}{\pgfqpoint{5.008171in}{2.049844in}}{\pgfqpoint{5.016407in}{2.058081in}}%
\pgfpathcurveto{\pgfqpoint{5.024644in}{2.066317in}}{\pgfqpoint{5.029271in}{2.077489in}}{\pgfqpoint{5.029271in}{2.089137in}}%
\pgfpathcurveto{\pgfqpoint{5.029271in}{2.100785in}}{\pgfqpoint{5.024644in}{2.111957in}}{\pgfqpoint{5.016407in}{2.120194in}}%
\pgfpathcurveto{\pgfqpoint{5.008171in}{2.128430in}}{\pgfqpoint{4.996999in}{2.133058in}}{\pgfqpoint{4.985351in}{2.133058in}}%
\pgfpathcurveto{\pgfqpoint{4.973703in}{2.133058in}}{\pgfqpoint{4.962531in}{2.128430in}}{\pgfqpoint{4.954294in}{2.120194in}}%
\pgfpathcurveto{\pgfqpoint{4.946058in}{2.111957in}}{\pgfqpoint{4.941430in}{2.100785in}}{\pgfqpoint{4.941430in}{2.089137in}}%
\pgfpathcurveto{\pgfqpoint{4.941430in}{2.077489in}}{\pgfqpoint{4.946058in}{2.066317in}}{\pgfqpoint{4.954294in}{2.058081in}}%
\pgfpathcurveto{\pgfqpoint{4.962531in}{2.049844in}}{\pgfqpoint{4.973703in}{2.045217in}}{\pgfqpoint{4.985351in}{2.045217in}}%
\pgfpathlineto{\pgfqpoint{4.985351in}{2.045217in}}%
\pgfpathclose%
\pgfusepath{stroke,fill}%
\end{pgfscope}%
\begin{pgfscope}%
\pgfpathrectangle{\pgfqpoint{4.871338in}{0.529443in}}{\pgfqpoint{1.636659in}{1.745990in}}%
\pgfusepath{clip}%
\pgfsetbuttcap%
\pgfsetroundjoin%
\definecolor{currentfill}{rgb}{0.003922,0.003922,0.003922}%
\pgfsetfillcolor{currentfill}%
\pgfsetfillopacity{0.900000}%
\pgfsetlinewidth{0.507862pt}%
\definecolor{currentstroke}{rgb}{1.000000,1.000000,1.000000}%
\pgfsetstrokecolor{currentstroke}%
\pgfsetstrokeopacity{0.900000}%
\pgfsetdash{}{0pt}%
\pgfpathmoveto{\pgfqpoint{6.223465in}{0.654130in}}%
\pgfpathcurveto{\pgfqpoint{6.235113in}{0.654130in}}{\pgfqpoint{6.246285in}{0.658757in}}{\pgfqpoint{6.254521in}{0.666994in}}%
\pgfpathcurveto{\pgfqpoint{6.262758in}{0.675230in}}{\pgfqpoint{6.267385in}{0.686402in}}{\pgfqpoint{6.267385in}{0.698050in}}%
\pgfpathcurveto{\pgfqpoint{6.267385in}{0.709698in}}{\pgfqpoint{6.262758in}{0.720870in}}{\pgfqpoint{6.254521in}{0.729107in}}%
\pgfpathcurveto{\pgfqpoint{6.246285in}{0.737343in}}{\pgfqpoint{6.235113in}{0.741971in}}{\pgfqpoint{6.223465in}{0.741971in}}%
\pgfpathcurveto{\pgfqpoint{6.211817in}{0.741971in}}{\pgfqpoint{6.200645in}{0.737343in}}{\pgfqpoint{6.192408in}{0.729107in}}%
\pgfpathcurveto{\pgfqpoint{6.184172in}{0.720870in}}{\pgfqpoint{6.179544in}{0.709698in}}{\pgfqpoint{6.179544in}{0.698050in}}%
\pgfpathcurveto{\pgfqpoint{6.179544in}{0.686402in}}{\pgfqpoint{6.184172in}{0.675230in}}{\pgfqpoint{6.192408in}{0.666994in}}%
\pgfpathcurveto{\pgfqpoint{6.200645in}{0.658757in}}{\pgfqpoint{6.211817in}{0.654130in}}{\pgfqpoint{6.223465in}{0.654130in}}%
\pgfpathlineto{\pgfqpoint{6.223465in}{0.654130in}}%
\pgfpathclose%
\pgfusepath{stroke,fill}%
\end{pgfscope}%
\begin{pgfscope}%
\pgfpathrectangle{\pgfqpoint{4.871338in}{0.529443in}}{\pgfqpoint{1.636659in}{1.745990in}}%
\pgfusepath{clip}%
\pgfsetbuttcap%
\pgfsetroundjoin%
\definecolor{currentfill}{rgb}{0.003922,0.003922,0.003922}%
\pgfsetfillcolor{currentfill}%
\pgfsetfillopacity{0.900000}%
\pgfsetlinewidth{0.507862pt}%
\definecolor{currentstroke}{rgb}{1.000000,1.000000,1.000000}%
\pgfsetstrokecolor{currentstroke}%
\pgfsetstrokeopacity{0.900000}%
\pgfsetdash{}{0pt}%
\pgfpathmoveto{\pgfqpoint{5.000770in}{2.141247in}}%
\pgfpathcurveto{\pgfqpoint{5.012418in}{2.141247in}}{\pgfqpoint{5.023590in}{2.145875in}}{\pgfqpoint{5.031826in}{2.154111in}}%
\pgfpathcurveto{\pgfqpoint{5.040063in}{2.162347in}}{\pgfqpoint{5.044690in}{2.173520in}}{\pgfqpoint{5.044690in}{2.185168in}}%
\pgfpathcurveto{\pgfqpoint{5.044690in}{2.196816in}}{\pgfqpoint{5.040063in}{2.207988in}}{\pgfqpoint{5.031826in}{2.216224in}}%
\pgfpathcurveto{\pgfqpoint{5.023590in}{2.224460in}}{\pgfqpoint{5.012418in}{2.229088in}}{\pgfqpoint{5.000770in}{2.229088in}}%
\pgfpathcurveto{\pgfqpoint{4.989122in}{2.229088in}}{\pgfqpoint{4.977950in}{2.224460in}}{\pgfqpoint{4.969713in}{2.216224in}}%
\pgfpathcurveto{\pgfqpoint{4.961477in}{2.207988in}}{\pgfqpoint{4.956849in}{2.196816in}}{\pgfqpoint{4.956849in}{2.185168in}}%
\pgfpathcurveto{\pgfqpoint{4.956849in}{2.173520in}}{\pgfqpoint{4.961477in}{2.162347in}}{\pgfqpoint{4.969713in}{2.154111in}}%
\pgfpathcurveto{\pgfqpoint{4.977950in}{2.145875in}}{\pgfqpoint{4.989122in}{2.141247in}}{\pgfqpoint{5.000770in}{2.141247in}}%
\pgfpathlineto{\pgfqpoint{5.000770in}{2.141247in}}%
\pgfpathclose%
\pgfusepath{stroke,fill}%
\end{pgfscope}%
\begin{pgfscope}%
\pgfpathrectangle{\pgfqpoint{4.871338in}{0.529443in}}{\pgfqpoint{1.636659in}{1.745990in}}%
\pgfusepath{clip}%
\pgfsetbuttcap%
\pgfsetroundjoin%
\definecolor{currentfill}{rgb}{0.003922,0.003922,0.003922}%
\pgfsetfillcolor{currentfill}%
\pgfsetfillopacity{0.900000}%
\pgfsetlinewidth{0.507862pt}%
\definecolor{currentstroke}{rgb}{1.000000,1.000000,1.000000}%
\pgfsetstrokecolor{currentstroke}%
\pgfsetstrokeopacity{0.900000}%
\pgfsetdash{}{0pt}%
\pgfpathmoveto{\pgfqpoint{6.424696in}{0.644938in}}%
\pgfpathcurveto{\pgfqpoint{6.436344in}{0.644938in}}{\pgfqpoint{6.447516in}{0.649566in}}{\pgfqpoint{6.455753in}{0.657802in}}%
\pgfpathcurveto{\pgfqpoint{6.463989in}{0.666039in}}{\pgfqpoint{6.468617in}{0.677211in}}{\pgfqpoint{6.468617in}{0.688859in}}%
\pgfpathcurveto{\pgfqpoint{6.468617in}{0.700507in}}{\pgfqpoint{6.463989in}{0.711679in}}{\pgfqpoint{6.455753in}{0.719915in}}%
\pgfpathcurveto{\pgfqpoint{6.447516in}{0.728152in}}{\pgfqpoint{6.436344in}{0.732779in}}{\pgfqpoint{6.424696in}{0.732779in}}%
\pgfpathcurveto{\pgfqpoint{6.413048in}{0.732779in}}{\pgfqpoint{6.401876in}{0.728152in}}{\pgfqpoint{6.393640in}{0.719915in}}%
\pgfpathcurveto{\pgfqpoint{6.385403in}{0.711679in}}{\pgfqpoint{6.380776in}{0.700507in}}{\pgfqpoint{6.380776in}{0.688859in}}%
\pgfpathcurveto{\pgfqpoint{6.380776in}{0.677211in}}{\pgfqpoint{6.385403in}{0.666039in}}{\pgfqpoint{6.393640in}{0.657802in}}%
\pgfpathcurveto{\pgfqpoint{6.401876in}{0.649566in}}{\pgfqpoint{6.413048in}{0.644938in}}{\pgfqpoint{6.424696in}{0.644938in}}%
\pgfpathlineto{\pgfqpoint{6.424696in}{0.644938in}}%
\pgfpathclose%
\pgfusepath{stroke,fill}%
\end{pgfscope}%
\begin{pgfscope}%
\pgfpathrectangle{\pgfqpoint{4.871338in}{0.529443in}}{\pgfqpoint{1.636659in}{1.745990in}}%
\pgfusepath{clip}%
\pgfsetbuttcap%
\pgfsetroundjoin%
\definecolor{currentfill}{rgb}{0.003922,0.003922,0.003922}%
\pgfsetfillcolor{currentfill}%
\pgfsetfillopacity{0.900000}%
\pgfsetlinewidth{0.507862pt}%
\definecolor{currentstroke}{rgb}{1.000000,1.000000,1.000000}%
\pgfsetstrokecolor{currentstroke}%
\pgfsetstrokeopacity{0.900000}%
\pgfsetdash{}{0pt}%
\pgfpathmoveto{\pgfqpoint{4.969327in}{2.113057in}}%
\pgfpathcurveto{\pgfqpoint{4.980975in}{2.113057in}}{\pgfqpoint{4.992148in}{2.117685in}}{\pgfqpoint{5.000384in}{2.125921in}}%
\pgfpathcurveto{\pgfqpoint{5.008620in}{2.134157in}}{\pgfqpoint{5.013248in}{2.145330in}}{\pgfqpoint{5.013248in}{2.156978in}}%
\pgfpathcurveto{\pgfqpoint{5.013248in}{2.168625in}}{\pgfqpoint{5.008620in}{2.179798in}}{\pgfqpoint{5.000384in}{2.188034in}}%
\pgfpathcurveto{\pgfqpoint{4.992148in}{2.196270in}}{\pgfqpoint{4.980975in}{2.200898in}}{\pgfqpoint{4.969327in}{2.200898in}}%
\pgfpathcurveto{\pgfqpoint{4.957680in}{2.200898in}}{\pgfqpoint{4.946507in}{2.196270in}}{\pgfqpoint{4.938271in}{2.188034in}}%
\pgfpathcurveto{\pgfqpoint{4.930035in}{2.179798in}}{\pgfqpoint{4.925407in}{2.168625in}}{\pgfqpoint{4.925407in}{2.156978in}}%
\pgfpathcurveto{\pgfqpoint{4.925407in}{2.145330in}}{\pgfqpoint{4.930035in}{2.134157in}}{\pgfqpoint{4.938271in}{2.125921in}}%
\pgfpathcurveto{\pgfqpoint{4.946507in}{2.117685in}}{\pgfqpoint{4.957680in}{2.113057in}}{\pgfqpoint{4.969327in}{2.113057in}}%
\pgfpathlineto{\pgfqpoint{4.969327in}{2.113057in}}%
\pgfpathclose%
\pgfusepath{stroke,fill}%
\end{pgfscope}%
\begin{pgfscope}%
\pgfpathrectangle{\pgfqpoint{4.871338in}{0.529443in}}{\pgfqpoint{1.636659in}{1.745990in}}%
\pgfusepath{clip}%
\pgfsetbuttcap%
\pgfsetroundjoin%
\definecolor{currentfill}{rgb}{0.031373,0.627451,0.913725}%
\pgfsetfillcolor{currentfill}%
\pgfsetfillopacity{0.900000}%
\pgfsetlinewidth{0.507862pt}%
\definecolor{currentstroke}{rgb}{1.000000,1.000000,1.000000}%
\pgfsetstrokecolor{currentstroke}%
\pgfsetstrokeopacity{0.900000}%
\pgfsetdash{}{0pt}%
\pgfpathmoveto{\pgfqpoint{5.051679in}{2.080024in}}%
\pgfpathcurveto{\pgfqpoint{5.063326in}{2.080024in}}{\pgfqpoint{5.074499in}{2.084652in}}{\pgfqpoint{5.082735in}{2.092888in}}%
\pgfpathcurveto{\pgfqpoint{5.090971in}{2.101124in}}{\pgfqpoint{5.095599in}{2.112297in}}{\pgfqpoint{5.095599in}{2.123945in}}%
\pgfpathcurveto{\pgfqpoint{5.095599in}{2.135592in}}{\pgfqpoint{5.090971in}{2.146765in}}{\pgfqpoint{5.082735in}{2.155001in}}%
\pgfpathcurveto{\pgfqpoint{5.074499in}{2.163237in}}{\pgfqpoint{5.063326in}{2.167865in}}{\pgfqpoint{5.051679in}{2.167865in}}%
\pgfpathcurveto{\pgfqpoint{5.040031in}{2.167865in}}{\pgfqpoint{5.028858in}{2.163237in}}{\pgfqpoint{5.020622in}{2.155001in}}%
\pgfpathcurveto{\pgfqpoint{5.012386in}{2.146765in}}{\pgfqpoint{5.007758in}{2.135592in}}{\pgfqpoint{5.007758in}{2.123945in}}%
\pgfpathcurveto{\pgfqpoint{5.007758in}{2.112297in}}{\pgfqpoint{5.012386in}{2.101124in}}{\pgfqpoint{5.020622in}{2.092888in}}%
\pgfpathcurveto{\pgfqpoint{5.028858in}{2.084652in}}{\pgfqpoint{5.040031in}{2.080024in}}{\pgfqpoint{5.051679in}{2.080024in}}%
\pgfpathlineto{\pgfqpoint{5.051679in}{2.080024in}}%
\pgfpathclose%
\pgfusepath{stroke,fill}%
\end{pgfscope}%
\begin{pgfscope}%
\pgfpathrectangle{\pgfqpoint{4.871338in}{0.529443in}}{\pgfqpoint{1.636659in}{1.745990in}}%
\pgfusepath{clip}%
\pgfsetbuttcap%
\pgfsetroundjoin%
\definecolor{currentfill}{rgb}{0.031373,0.627451,0.913725}%
\pgfsetfillcolor{currentfill}%
\pgfsetfillopacity{0.900000}%
\pgfsetlinewidth{0.507862pt}%
\definecolor{currentstroke}{rgb}{1.000000,1.000000,1.000000}%
\pgfsetstrokecolor{currentstroke}%
\pgfsetstrokeopacity{0.900000}%
\pgfsetdash{}{0pt}%
\pgfpathmoveto{\pgfqpoint{4.945731in}{2.135868in}}%
\pgfpathcurveto{\pgfqpoint{4.957379in}{2.135868in}}{\pgfqpoint{4.968552in}{2.140496in}}{\pgfqpoint{4.976788in}{2.148732in}}%
\pgfpathcurveto{\pgfqpoint{4.985024in}{2.156969in}}{\pgfqpoint{4.989652in}{2.168141in}}{\pgfqpoint{4.989652in}{2.179789in}}%
\pgfpathcurveto{\pgfqpoint{4.989652in}{2.191437in}}{\pgfqpoint{4.985024in}{2.202609in}}{\pgfqpoint{4.976788in}{2.210845in}}%
\pgfpathcurveto{\pgfqpoint{4.968552in}{2.219082in}}{\pgfqpoint{4.957379in}{2.223710in}}{\pgfqpoint{4.945731in}{2.223710in}}%
\pgfpathcurveto{\pgfqpoint{4.934083in}{2.223710in}}{\pgfqpoint{4.922911in}{2.219082in}}{\pgfqpoint{4.914675in}{2.210845in}}%
\pgfpathcurveto{\pgfqpoint{4.906439in}{2.202609in}}{\pgfqpoint{4.901811in}{2.191437in}}{\pgfqpoint{4.901811in}{2.179789in}}%
\pgfpathcurveto{\pgfqpoint{4.901811in}{2.168141in}}{\pgfqpoint{4.906439in}{2.156969in}}{\pgfqpoint{4.914675in}{2.148732in}}%
\pgfpathcurveto{\pgfqpoint{4.922911in}{2.140496in}}{\pgfqpoint{4.934083in}{2.135868in}}{\pgfqpoint{4.945731in}{2.135868in}}%
\pgfpathlineto{\pgfqpoint{4.945731in}{2.135868in}}%
\pgfpathclose%
\pgfusepath{stroke,fill}%
\end{pgfscope}%
\begin{pgfscope}%
\pgfpathrectangle{\pgfqpoint{4.871338in}{0.529443in}}{\pgfqpoint{1.636659in}{1.745990in}}%
\pgfusepath{clip}%
\pgfsetbuttcap%
\pgfsetroundjoin%
\definecolor{currentfill}{rgb}{0.031373,0.627451,0.913725}%
\pgfsetfillcolor{currentfill}%
\pgfsetfillopacity{0.900000}%
\pgfsetlinewidth{0.507862pt}%
\definecolor{currentstroke}{rgb}{1.000000,1.000000,1.000000}%
\pgfsetstrokecolor{currentstroke}%
\pgfsetstrokeopacity{0.900000}%
\pgfsetdash{}{0pt}%
\pgfpathmoveto{\pgfqpoint{6.253324in}{0.572653in}}%
\pgfpathcurveto{\pgfqpoint{6.264972in}{0.572653in}}{\pgfqpoint{6.276144in}{0.577280in}}{\pgfqpoint{6.284380in}{0.585517in}}%
\pgfpathcurveto{\pgfqpoint{6.292617in}{0.593753in}}{\pgfqpoint{6.297244in}{0.604925in}}{\pgfqpoint{6.297244in}{0.616573in}}%
\pgfpathcurveto{\pgfqpoint{6.297244in}{0.628221in}}{\pgfqpoint{6.292617in}{0.639393in}}{\pgfqpoint{6.284380in}{0.647630in}}%
\pgfpathcurveto{\pgfqpoint{6.276144in}{0.655866in}}{\pgfqpoint{6.264972in}{0.660494in}}{\pgfqpoint{6.253324in}{0.660494in}}%
\pgfpathcurveto{\pgfqpoint{6.241676in}{0.660494in}}{\pgfqpoint{6.230504in}{0.655866in}}{\pgfqpoint{6.222267in}{0.647630in}}%
\pgfpathcurveto{\pgfqpoint{6.214031in}{0.639393in}}{\pgfqpoint{6.209403in}{0.628221in}}{\pgfqpoint{6.209403in}{0.616573in}}%
\pgfpathcurveto{\pgfqpoint{6.209403in}{0.604925in}}{\pgfqpoint{6.214031in}{0.593753in}}{\pgfqpoint{6.222267in}{0.585517in}}%
\pgfpathcurveto{\pgfqpoint{6.230504in}{0.577280in}}{\pgfqpoint{6.241676in}{0.572653in}}{\pgfqpoint{6.253324in}{0.572653in}}%
\pgfpathlineto{\pgfqpoint{6.253324in}{0.572653in}}%
\pgfpathclose%
\pgfusepath{stroke,fill}%
\end{pgfscope}%
\begin{pgfscope}%
\pgfpathrectangle{\pgfqpoint{4.871338in}{0.529443in}}{\pgfqpoint{1.636659in}{1.745990in}}%
\pgfusepath{clip}%
\pgfsetbuttcap%
\pgfsetroundjoin%
\definecolor{currentfill}{rgb}{0.031373,0.627451,0.913725}%
\pgfsetfillcolor{currentfill}%
\pgfsetfillopacity{0.900000}%
\pgfsetlinewidth{0.507862pt}%
\definecolor{currentstroke}{rgb}{1.000000,1.000000,1.000000}%
\pgfsetstrokecolor{currentstroke}%
\pgfsetstrokeopacity{0.900000}%
\pgfsetdash{}{0pt}%
\pgfpathmoveto{\pgfqpoint{6.168540in}{0.583233in}}%
\pgfpathcurveto{\pgfqpoint{6.180188in}{0.583233in}}{\pgfqpoint{6.191361in}{0.587861in}}{\pgfqpoint{6.199597in}{0.596097in}}%
\pgfpathcurveto{\pgfqpoint{6.207833in}{0.604333in}}{\pgfqpoint{6.212461in}{0.615506in}}{\pgfqpoint{6.212461in}{0.627154in}}%
\pgfpathcurveto{\pgfqpoint{6.212461in}{0.638801in}}{\pgfqpoint{6.207833in}{0.649974in}}{\pgfqpoint{6.199597in}{0.658210in}}%
\pgfpathcurveto{\pgfqpoint{6.191361in}{0.666446in}}{\pgfqpoint{6.180188in}{0.671074in}}{\pgfqpoint{6.168540in}{0.671074in}}%
\pgfpathcurveto{\pgfqpoint{6.156893in}{0.671074in}}{\pgfqpoint{6.145720in}{0.666446in}}{\pgfqpoint{6.137484in}{0.658210in}}%
\pgfpathcurveto{\pgfqpoint{6.129248in}{0.649974in}}{\pgfqpoint{6.124620in}{0.638801in}}{\pgfqpoint{6.124620in}{0.627154in}}%
\pgfpathcurveto{\pgfqpoint{6.124620in}{0.615506in}}{\pgfqpoint{6.129248in}{0.604333in}}{\pgfqpoint{6.137484in}{0.596097in}}%
\pgfpathcurveto{\pgfqpoint{6.145720in}{0.587861in}}{\pgfqpoint{6.156893in}{0.583233in}}{\pgfqpoint{6.168540in}{0.583233in}}%
\pgfpathlineto{\pgfqpoint{6.168540in}{0.583233in}}%
\pgfpathclose%
\pgfusepath{stroke,fill}%
\end{pgfscope}%
\begin{pgfscope}%
\pgfpathrectangle{\pgfqpoint{4.871338in}{0.529443in}}{\pgfqpoint{1.636659in}{1.745990in}}%
\pgfusepath{clip}%
\pgfsetbuttcap%
\pgfsetroundjoin%
\definecolor{currentfill}{rgb}{0.031373,0.627451,0.913725}%
\pgfsetfillcolor{currentfill}%
\pgfsetfillopacity{0.900000}%
\pgfsetlinewidth{0.507862pt}%
\definecolor{currentstroke}{rgb}{1.000000,1.000000,1.000000}%
\pgfsetstrokecolor{currentstroke}%
\pgfsetstrokeopacity{0.900000}%
\pgfsetdash{}{0pt}%
\pgfpathmoveto{\pgfqpoint{6.375942in}{0.640283in}}%
\pgfpathcurveto{\pgfqpoint{6.387590in}{0.640283in}}{\pgfqpoint{6.398762in}{0.644911in}}{\pgfqpoint{6.406998in}{0.653147in}}%
\pgfpathcurveto{\pgfqpoint{6.415235in}{0.661383in}}{\pgfqpoint{6.419862in}{0.672556in}}{\pgfqpoint{6.419862in}{0.684204in}}%
\pgfpathcurveto{\pgfqpoint{6.419862in}{0.695851in}}{\pgfqpoint{6.415235in}{0.707024in}}{\pgfqpoint{6.406998in}{0.715260in}}%
\pgfpathcurveto{\pgfqpoint{6.398762in}{0.723496in}}{\pgfqpoint{6.387590in}{0.728124in}}{\pgfqpoint{6.375942in}{0.728124in}}%
\pgfpathcurveto{\pgfqpoint{6.364294in}{0.728124in}}{\pgfqpoint{6.353122in}{0.723496in}}{\pgfqpoint{6.344885in}{0.715260in}}%
\pgfpathcurveto{\pgfqpoint{6.336649in}{0.707024in}}{\pgfqpoint{6.332021in}{0.695851in}}{\pgfqpoint{6.332021in}{0.684204in}}%
\pgfpathcurveto{\pgfqpoint{6.332021in}{0.672556in}}{\pgfqpoint{6.336649in}{0.661383in}}{\pgfqpoint{6.344885in}{0.653147in}}%
\pgfpathcurveto{\pgfqpoint{6.353122in}{0.644911in}}{\pgfqpoint{6.364294in}{0.640283in}}{\pgfqpoint{6.375942in}{0.640283in}}%
\pgfpathlineto{\pgfqpoint{6.375942in}{0.640283in}}%
\pgfpathclose%
\pgfusepath{stroke,fill}%
\end{pgfscope}%
\begin{pgfscope}%
\pgfpathrectangle{\pgfqpoint{4.871338in}{0.529443in}}{\pgfqpoint{1.636659in}{1.745990in}}%
\pgfusepath{clip}%
\pgfsetbuttcap%
\pgfsetroundjoin%
\definecolor{currentfill}{rgb}{0.031373,0.627451,0.913725}%
\pgfsetfillcolor{currentfill}%
\pgfsetfillopacity{0.900000}%
\pgfsetlinewidth{0.507862pt}%
\definecolor{currentstroke}{rgb}{1.000000,1.000000,1.000000}%
\pgfsetstrokecolor{currentstroke}%
\pgfsetstrokeopacity{0.900000}%
\pgfsetdash{}{0pt}%
\pgfpathmoveto{\pgfqpoint{6.200835in}{0.633571in}}%
\pgfpathcurveto{\pgfqpoint{6.212483in}{0.633571in}}{\pgfqpoint{6.223656in}{0.638199in}}{\pgfqpoint{6.231892in}{0.646435in}}%
\pgfpathcurveto{\pgfqpoint{6.240128in}{0.654671in}}{\pgfqpoint{6.244756in}{0.665844in}}{\pgfqpoint{6.244756in}{0.677492in}}%
\pgfpathcurveto{\pgfqpoint{6.244756in}{0.689140in}}{\pgfqpoint{6.240128in}{0.700312in}}{\pgfqpoint{6.231892in}{0.708548in}}%
\pgfpathcurveto{\pgfqpoint{6.223656in}{0.716784in}}{\pgfqpoint{6.212483in}{0.721412in}}{\pgfqpoint{6.200835in}{0.721412in}}%
\pgfpathcurveto{\pgfqpoint{6.189187in}{0.721412in}}{\pgfqpoint{6.178015in}{0.716784in}}{\pgfqpoint{6.169779in}{0.708548in}}%
\pgfpathcurveto{\pgfqpoint{6.161543in}{0.700312in}}{\pgfqpoint{6.156915in}{0.689140in}}{\pgfqpoint{6.156915in}{0.677492in}}%
\pgfpathcurveto{\pgfqpoint{6.156915in}{0.665844in}}{\pgfqpoint{6.161543in}{0.654671in}}{\pgfqpoint{6.169779in}{0.646435in}}%
\pgfpathcurveto{\pgfqpoint{6.178015in}{0.638199in}}{\pgfqpoint{6.189187in}{0.633571in}}{\pgfqpoint{6.200835in}{0.633571in}}%
\pgfpathlineto{\pgfqpoint{6.200835in}{0.633571in}}%
\pgfpathclose%
\pgfusepath{stroke,fill}%
\end{pgfscope}%
\begin{pgfscope}%
\pgfsetrectcap%
\pgfsetmiterjoin%
\pgfsetlinewidth{1.254687pt}%
\definecolor{currentstroke}{rgb}{0.800000,0.800000,0.800000}%
\pgfsetstrokecolor{currentstroke}%
\pgfsetdash{}{0pt}%
\pgfpathmoveto{\pgfqpoint{4.871338in}{0.529443in}}%
\pgfpathlineto{\pgfqpoint{4.871338in}{2.275433in}}%
\pgfusepath{stroke}%
\end{pgfscope}%
\begin{pgfscope}%
\pgfsetrectcap%
\pgfsetmiterjoin%
\pgfsetlinewidth{1.254687pt}%
\definecolor{currentstroke}{rgb}{0.800000,0.800000,0.800000}%
\pgfsetstrokecolor{currentstroke}%
\pgfsetdash{}{0pt}%
\pgfpathmoveto{\pgfqpoint{6.507997in}{0.529443in}}%
\pgfpathlineto{\pgfqpoint{6.507997in}{2.275433in}}%
\pgfusepath{stroke}%
\end{pgfscope}%
\begin{pgfscope}%
\pgfsetrectcap%
\pgfsetmiterjoin%
\pgfsetlinewidth{1.254687pt}%
\definecolor{currentstroke}{rgb}{0.800000,0.800000,0.800000}%
\pgfsetstrokecolor{currentstroke}%
\pgfsetdash{}{0pt}%
\pgfpathmoveto{\pgfqpoint{4.871338in}{0.529443in}}%
\pgfpathlineto{\pgfqpoint{6.507997in}{0.529443in}}%
\pgfusepath{stroke}%
\end{pgfscope}%
\begin{pgfscope}%
\pgfsetrectcap%
\pgfsetmiterjoin%
\pgfsetlinewidth{1.254687pt}%
\definecolor{currentstroke}{rgb}{0.800000,0.800000,0.800000}%
\pgfsetstrokecolor{currentstroke}%
\pgfsetdash{}{0pt}%
\pgfpathmoveto{\pgfqpoint{4.871338in}{2.275433in}}%
\pgfpathlineto{\pgfqpoint{6.507997in}{2.275433in}}%
\pgfusepath{stroke}%
\end{pgfscope}%
\begin{pgfscope}%
\definecolor{textcolor}{rgb}{0.150000,0.150000,0.150000}%
\pgfsetstrokecolor{textcolor}%
\pgfsetfillcolor{textcolor}%
\pgftext[x=5.689667in,y=2.358766in,,base]{\color{textcolor}{\rmfamily\fontsize{11.000000}{13.200000}\selectfont\catcode`\^=\active\def^{\ifmmode\sp\else\^{}\fi}\catcode`\%=\active\def%{\%}Sentiment Bag-Of-Subgraphs}}%
\end{pgfscope}%
\begin{pgfscope}%
\pgfsetbuttcap%
\pgfsetmiterjoin%
\definecolor{currentfill}{rgb}{1.000000,1.000000,1.000000}%
\pgfsetfillcolor{currentfill}%
\pgfsetfillopacity{0.800000}%
\pgfsetlinewidth{1.003750pt}%
\definecolor{currentstroke}{rgb}{0.800000,0.800000,0.800000}%
\pgfsetstrokecolor{currentstroke}%
\pgfsetstrokeopacity{0.800000}%
\pgfsetdash{}{0pt}%
\pgfpathmoveto{\pgfqpoint{5.665701in}{1.860372in}}%
\pgfpathlineto{\pgfqpoint{6.430219in}{1.860372in}}%
\pgfpathquadraticcurveto{\pgfqpoint{6.452441in}{1.860372in}}{\pgfqpoint{6.452441in}{1.882594in}}%
\pgfpathlineto{\pgfqpoint{6.452441in}{2.197655in}}%
\pgfpathquadraticcurveto{\pgfqpoint{6.452441in}{2.219877in}}{\pgfqpoint{6.430219in}{2.219877in}}%
\pgfpathlineto{\pgfqpoint{5.665701in}{2.219877in}}%
\pgfpathquadraticcurveto{\pgfqpoint{5.643479in}{2.219877in}}{\pgfqpoint{5.643479in}{2.197655in}}%
\pgfpathlineto{\pgfqpoint{5.643479in}{1.882594in}}%
\pgfpathquadraticcurveto{\pgfqpoint{5.643479in}{1.860372in}}{\pgfqpoint{5.665701in}{1.860372in}}%
\pgfpathlineto{\pgfqpoint{5.665701in}{1.860372in}}%
\pgfpathclose%
\pgfusepath{stroke,fill}%
\end{pgfscope}%
\begin{pgfscope}%
\pgfsetbuttcap%
\pgfsetroundjoin%
\definecolor{currentfill}{rgb}{0.003922,0.003922,0.003922}%
\pgfsetfillcolor{currentfill}%
\pgfsetfillopacity{0.900000}%
\pgfsetlinewidth{0.507862pt}%
\definecolor{currentstroke}{rgb}{1.000000,1.000000,1.000000}%
\pgfsetstrokecolor{currentstroke}%
\pgfsetstrokeopacity{0.900000}%
\pgfsetdash{}{0pt}%
\pgfsys@defobject{currentmarker}{\pgfqpoint{-0.043921in}{-0.043921in}}{\pgfqpoint{0.043921in}{0.043921in}}{%
\pgfpathmoveto{\pgfqpoint{0.000000in}{-0.043921in}}%
\pgfpathcurveto{\pgfqpoint{0.011648in}{-0.043921in}}{\pgfqpoint{0.022820in}{-0.039293in}}{\pgfqpoint{0.031056in}{-0.031056in}}%
\pgfpathcurveto{\pgfqpoint{0.039293in}{-0.022820in}}{\pgfqpoint{0.043921in}{-0.011648in}}{\pgfqpoint{0.043921in}{0.000000in}}%
\pgfpathcurveto{\pgfqpoint{0.043921in}{0.011648in}}{\pgfqpoint{0.039293in}{0.022820in}}{\pgfqpoint{0.031056in}{0.031056in}}%
\pgfpathcurveto{\pgfqpoint{0.022820in}{0.039293in}}{\pgfqpoint{0.011648in}{0.043921in}}{\pgfqpoint{0.000000in}{0.043921in}}%
\pgfpathcurveto{\pgfqpoint{-0.011648in}{0.043921in}}{\pgfqpoint{-0.022820in}{0.039293in}}{\pgfqpoint{-0.031056in}{0.031056in}}%
\pgfpathcurveto{\pgfqpoint{-0.039293in}{0.022820in}}{\pgfqpoint{-0.043921in}{0.011648in}}{\pgfqpoint{-0.043921in}{0.000000in}}%
\pgfpathcurveto{\pgfqpoint{-0.043921in}{-0.011648in}}{\pgfqpoint{-0.039293in}{-0.022820in}}{\pgfqpoint{-0.031056in}{-0.031056in}}%
\pgfpathcurveto{\pgfqpoint{-0.022820in}{-0.039293in}}{\pgfqpoint{-0.011648in}{-0.043921in}}{\pgfqpoint{0.000000in}{-0.043921in}}%
\pgfpathlineto{\pgfqpoint{0.000000in}{-0.043921in}}%
\pgfpathclose%
\pgfusepath{stroke,fill}%
}%
\begin{pgfscope}%
\pgfsys@transformshift{5.799034in}{2.129903in}%
\pgfsys@useobject{currentmarker}{}%
\end{pgfscope}%
\end{pgfscope}%
\begin{pgfscope}%
\definecolor{textcolor}{rgb}{0.150000,0.150000,0.150000}%
\pgfsetstrokecolor{textcolor}%
\pgfsetfillcolor{textcolor}%
\pgftext[x=5.999034in,y=2.091014in,left,base]{\color{textcolor}{\rmfamily\fontsize{8.000000}{9.600000}\selectfont\catcode`\^=\active\def^{\ifmmode\sp\else\^{}\fi}\catcode`\%=\active\def%{\%}TikTok}}%
\end{pgfscope}%
\begin{pgfscope}%
\pgfsetbuttcap%
\pgfsetroundjoin%
\definecolor{currentfill}{rgb}{0.031373,0.627451,0.913725}%
\pgfsetfillcolor{currentfill}%
\pgfsetfillopacity{0.900000}%
\pgfsetlinewidth{0.507862pt}%
\definecolor{currentstroke}{rgb}{1.000000,1.000000,1.000000}%
\pgfsetstrokecolor{currentstroke}%
\pgfsetstrokeopacity{0.900000}%
\pgfsetdash{}{0pt}%
\pgfsys@defobject{currentmarker}{\pgfqpoint{-0.043921in}{-0.043921in}}{\pgfqpoint{0.043921in}{0.043921in}}{%
\pgfpathmoveto{\pgfqpoint{0.000000in}{-0.043921in}}%
\pgfpathcurveto{\pgfqpoint{0.011648in}{-0.043921in}}{\pgfqpoint{0.022820in}{-0.039293in}}{\pgfqpoint{0.031056in}{-0.031056in}}%
\pgfpathcurveto{\pgfqpoint{0.039293in}{-0.022820in}}{\pgfqpoint{0.043921in}{-0.011648in}}{\pgfqpoint{0.043921in}{0.000000in}}%
\pgfpathcurveto{\pgfqpoint{0.043921in}{0.011648in}}{\pgfqpoint{0.039293in}{0.022820in}}{\pgfqpoint{0.031056in}{0.031056in}}%
\pgfpathcurveto{\pgfqpoint{0.022820in}{0.039293in}}{\pgfqpoint{0.011648in}{0.043921in}}{\pgfqpoint{0.000000in}{0.043921in}}%
\pgfpathcurveto{\pgfqpoint{-0.011648in}{0.043921in}}{\pgfqpoint{-0.022820in}{0.039293in}}{\pgfqpoint{-0.031056in}{0.031056in}}%
\pgfpathcurveto{\pgfqpoint{-0.039293in}{0.022820in}}{\pgfqpoint{-0.043921in}{0.011648in}}{\pgfqpoint{-0.043921in}{0.000000in}}%
\pgfpathcurveto{\pgfqpoint{-0.043921in}{-0.011648in}}{\pgfqpoint{-0.039293in}{-0.022820in}}{\pgfqpoint{-0.031056in}{-0.031056in}}%
\pgfpathcurveto{\pgfqpoint{-0.022820in}{-0.039293in}}{\pgfqpoint{-0.011648in}{-0.043921in}}{\pgfqpoint{0.000000in}{-0.043921in}}%
\pgfpathlineto{\pgfqpoint{0.000000in}{-0.043921in}}%
\pgfpathclose%
\pgfusepath{stroke,fill}%
}%
\begin{pgfscope}%
\pgfsys@transformshift{5.799034in}{1.966817in}%
\pgfsys@useobject{currentmarker}{}%
\end{pgfscope}%
\end{pgfscope}%
\begin{pgfscope}%
\definecolor{textcolor}{rgb}{0.150000,0.150000,0.150000}%
\pgfsetstrokecolor{textcolor}%
\pgfsetfillcolor{textcolor}%
\pgftext[x=5.999034in,y=1.927928in,left,base]{\color{textcolor}{\rmfamily\fontsize{8.000000}{9.600000}\selectfont\catcode`\^=\active\def^{\ifmmode\sp\else\^{}\fi}\catcode`\%=\active\def%{\%}Twitter}}%
\end{pgfscope}%
\begin{pgfscope}%
\definecolor{textcolor}{rgb}{0.150000,0.150000,0.150000}%
\pgfsetstrokecolor{textcolor}%
\pgfsetfillcolor{textcolor}%
\pgftext[x=3.318714in,y=2.885276in,,top]{\color{textcolor}{\rmfamily\fontsize{12.000000}{14.400000}\selectfont\catcode`\^=\active\def^{\ifmmode\sp\else\^{}\fi}\catcode`\%=\active\def%{\%}User graph- and Twitter embeddings}}%
\end{pgfscope}%
\end{pgfpicture}%
\makeatother%
\endgroup%

    \end{adjustwidth}
    \caption{Comparison of user graph embeddings from TikTok and Twitter across three embedding spaces, dimensionality reduced with UMAP. While no distinct separation is observed for most graphs, larger Twitter graphs appear to occupy a slightly different region compared to TikTok graphs in the Graph2Vec embedding space. No such distinction is evident in either Bag-Of-Subgraphs.}
    \label{fig:twitter_scatter}
\end{figure}

Similar to the previous results, Figure \ref{fig:twitter_scatter} reveals no clear distinction between TikTok user graphs and Twitter reply networks across both variants of Bag-Of-Subgraph. However, in the Graph2Vec embedding space, larger Twitter graphs occupy a distinct region compared to TikTok graphs, suggesting potential differences in graph structures for larger graphs. Consistent with observations in the TikTok-only analysis, graph size remains the primary factor driving separation across all embedding spaces. Smaller Twitter graphs tend to cluster near smaller TikTok user graphs, while larger graphs from both platforms are similarly grouped. Overall, the embeddings do not show strong platform-specific grouping for the majority of graphs.

% g2v hdbscan nmi: 0.054 (12 outliers)
% motifs embedding nmi: 0.001 (1 outlier)
