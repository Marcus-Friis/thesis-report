\chapter{Acknowledgments}
In this project, we have used various generative AI tools. Specifically, GitHub Copilot was employed to assist with coding tasks, Writefull was used to enhance the quality of our writing, and ChatGPT for a combination of both. In particular, generative AI was not used to create original text, but was strictly limited to improving and clarifying the content we had already written. This means that generative AI was never used to introduce new content or new ideas and conclusions that we had not already written ourselves. A common example prompt used a lot for this paper is:
\begin{quote}
    Here is an excerpt from an academic paper: \newline
    [INSERT TEXT HERE] \newline
    Improve the wording of the last sentence.
\end{quote}

As for code assistants, GitHub Copilot was primarily used for code completion. ChatGPT was used primarily to debug existing code. For example, a common prompt was: 
\begin{quote}
    The below Python code throws an error, please help to identify why. \newline
    [INSERT CODE HERE] \newline
    We get this error: \newline
    [INSERT ERROR HERE]
\end{quote}
These are rough examples of prompts that exemplify the way we used generative AI.


%We have used OpenAI's ChatGPT (version 3.5 and 4o) as well as Google's Gemini (Gemini vanilla; the free version). They have primarily been used as 'consultants' to help with various code errors, LaTeX formatting, and summarizing relevant research papers. Moreover, ChatGPT has a custom GPT named "Consensus" which can find relevant research papers based on the prompt. It should be mentioned, that we never follow a generative AI tool blindly. All its claims have to be fact checked. However, what these tools really excel at, is to point us in the right direction when searching for various information. 
%OpenAI's Dalle-3 model generated the frontpage image.