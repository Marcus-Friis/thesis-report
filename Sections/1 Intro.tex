\chapter{Introduction}

In the 21st century, the landscape of public discourse has undergone a significant transformation with the rise of short-form video content platforms such as TikTok, Instagram Reels, and YouTube Shorts.  Among these, TikTok has seen a significant increase in popularity, becoming one of the most popular social media services, with more than $1.67$ billion active monthly users worldwide \citep{tiktok_popular}. In particular, teens have heavily adopted this new format type \citep{tiktok_teens}. With the recent introduction of the TikTok Research API \citep{tiktokresearchapi}, performing large scale studies of TikTok has become a possibility. 

The short-form video format has been adopted for all communicative purposes, with content ranging from comedic sketches to serious political debates. This multifaceted nature of TikTok’s content makes it a compelling platform to study. The shift from traditional text-dominated social media, such as Facebook and X (formerly known as Twitter), to short-form video content presents new challenges in computationally analyzing public discourse. While natural language processing methods have sufficed for older platforms, the multimodal character of short-form video content introduces analytical challenges that require a combination of visual, auditory, and textual analysis.

TikTok has a plethora of social features. Among these is the \textit{stitch}. The stitch allows users to create a reactionary video of another video, incorporating up to five seconds of the \textit{stitched} video. Similarly, the \textit{duet} allows users to add their own video alongside another user's video in a split screen, floating head, or a green screen format. Unlike stitches, which integrate parts of the original video into the new content, duets display the original and new videos simultaneously and are not bound by the five second rule. This content remixing functionality yields an explicit network structure, where content can have direct connections to other content. This is akin to X's \textit{reply} and \textit{quote} functionality, where posts can be explicit reactions to other posts. Although there exists research on TikTok, most of it is qualitative, and almost no research focuses on studying TikTok as a graph. 

This paper presents TikTok StitchGraph: possibly the first graph dataset detailing stitches on TikTok. A stitch network is constructed comprising $36$ different hashtags, containing all stitches created in May $2024$ using one of said hashtags. The dataset aims to explore and improve understanding of how people communicate using stitches on TikTok. The topological structure of TikTok stitch networks and the content of individual videos are analyzed to uncover patterns and insights into this emerging form of public discourse. Additionally, these findings are compared to Twitter to highlight structural differences and commonalities in how users interact across these two social media platforms. The approach combines basic network analysis, sentiment analysis, frequent subgraph mining, and network embeddings to gain an introductory understanding of the stitch networks collected from TikTok. Specifically, this paper aims to answer the following research questions:

\begin{itemize}
    \item How can stitch communication on TikTok be studied through the lens of network analysis?
    \item What stitch patterns are characteristic of TikTok?
    \item How do the stitch patterns vary between different content themes on TikTok?
    \item How do the discovered properties of TikTok stitch networks compare with Twitter reply networks?
\end{itemize}


\section{An introduction to TikTok}
To understand the methods and findings of this paper, one must be familiar with TikTok as a platform and the dynamics that shape its content creation, consumption, and user interactions. This includes understanding the platform's algorithmic structure, the role of trends and challenges, and the unique ways users engage with and remix other users' content with stitches and duets. Throughout this section, multiple TikTok terms and concepts are introduced. For a complete overview, see the Appendix Glossary \ref{glossary}.

TikTok originated as the Chinese app \textit{Douyin}, launched internationally in $2017$. In $2018$, it merged with \textit{Musical.ly}, an app introduced in $2014$ for sharing lip-sync videos under one minute in length. By $2017$, Musical.ly had accumulated $200$ million users before being acquired and integrated into TikTok. This historical context shapes TikTok’s present-day culture, characterized by its demographic of young users and the popularity of dance videos, as well as the platform's distinctive features for video creation \citep{IJoC14543}.

Users consume TikTok through a personalized video feed, curated by TikTok's proprietary algorithm, commonly referred to as \textit{The Algorithm}. The algorithm pre-sents videos to users, and based on user engagement, it adapts the recommended content. This personalized video feed is known as the \textit{For You Page (FYP)} and is the main attraction of TikTok. Anyone can upload a video and have the algorithm show it in the feeds of other users. The algorithm has strict guidelines towards what content is allowed and will penalize any content that is not compliant with TikTok's policies. This has led users to mask content behind obscure internal ways of speaking and writing, leading to a phenomenon called \textit{algospeak} \citep{doi:10.1177/20563051231194586}. 


Other content remixing functionalities include \textit{duets}, \textit{video replies to comments}, and \textit{sound sync}. \textit{Duets} enable users to create collaborative videos in a side-by-side or picture-in-picture format, facilitating interaction through real-time reactions or joint performances. Similarly, \textit{video replies to comments} transform text-based feedback into engaging multimedia interactions. Lastly, \textit{sound sync} lets users create different videos based on the same sound, linking creators through shared auditory elements. These features collectively contribute to an active and creative environment, where users can remix content and interact in different ways. However, by design of the 'For You Page', the exact content being interacted with is heavily influenced by what the algorithm recommends.

% For this study, we focus only on stitches to maintain a clear direction and avoid becoming overwhelmed by the vast range of interaction types on TikTok. Although duets and other forms of video replies offer valuable insights, the volume of data could make the analysis too complex. By narrowing our scope to stitches, we ensure that we can thoroughly explore the dynamics within this specific feature without losing focus. Additionally, there are more than enough data from stitches alone to provide a comprehensive view of communication patterns on TikTok. 
TikTok offers a variety of interactive features that shape the way users interact and create content. One of the key features is \textit{hashtags}, which serve as markers for different topics and communities, helping users find and engage with specific content. However, beyond hashtags, TikTok introduced the \textit{stitch} feature in $2020$, adding a new layer to communication \citep{introducingStitch}. This stitch feature is commonly used to create video reactions to other content. Importantly, users cannot stitch a video that is already a stitch of another video. TikTok also gives users control over this feature, allowing them to disable stitching for specific videos or for all of their content. When a video is stitched, the description will show “stitch with @username” and, most of the time\footnote{We have observed that some users remove the hyperlink from the video description, rendering us unable identify the stitched video.}, provide a direct link to the original video. Throughout this paper, we refer to the user creating the stitch as the "\textit{stitcher}" and the user whose video is being stitched as the "\textit{stitchee}".
