\chapter{Background}

This paper collects data for and quantitatively analyzes communication on TikTok. While qualitative methods and ethical considerations regarding TikTok have been fairly well documented, quantitative studies of communication on the platform remain scarce. For this reason, we also look towards similar social network studies conducted on Twitter, leveraging its similarities as a platform to further our understanding of TikTok.


\section{Research on TikTok}

The digital age has transformed how people engage with the media, moving from passive consumption to active participation. This shift has led to the growth of online communities, where users create and share their own content. This is known as participatory culture, where users are both consumers and creators of media \citep{jenkins2015participatory}. On platforms like TikTok, features such as stitching and duets enable users to remix and share videos, allowing trends and ideas to spread rapidly. The concept of "produsage" describes how users continuously build on and adapt existing content \citep{bruns2008blogs}.

TikTok’s stitching feature illustrates this by allowing users to incorporate parts of other videos into their own, blending the roles of creator and consumer \citep{tiktok_stitch_2020}. This creates an environment where content evolves through collaboration. As \cite{kaur2022customization} shows, TikTok also allows marginalized communities, such as migrant workers in Singapore, to document their poor experiences and challenge the mainstream narratives. Using TikTok's remix features, these workers engage in digital activism, bringing visibility to their unstable living conditions during the pandemic. This aligns with the concept of produsage, where users not only consume content but actively create and remix it, increasing internet exposure for underrepresented voices.

Although these participatory characteristics foster collaboration and community, they also raise legal and ethical questions. According to \cite{ripplexn2023stitch}, TikTok’s stitch function enables users to engage with existing content in ways that promote viral trends, but also introduces concerns regarding intellectual property. Users must navigate complex questions of content ownership and copyright, as the remixing culture enabled by stitches often blurs the line between original creation and derived work.

Similarly, the internet makes it easy for groups to form and collaborate without formal structures, allowing communities to emerge around shared interests \citep{shirky2008here}. On TikTok, these communities often form around hashtags, challenges, trends, and interactions. For this study, understanding these patterns is key as we explore how TikTok’s stitching feature influences communication. The way users engage through stitching aligns with the ideas of \cite{jenkins2015participatory}, \cite{bruns2008blogs}, and \cite{shirky2008here}, creating a space where users build on each other’s content and communities grow organically.

Although the role of features such as stitches in content creation has been acknowledged, there has been limited research specifically analyzing how these functions contribute to broader trends and connections within the TikTok ecosystem. In a systematic review of TikTok research, \cite{tiktok_review} found that most of the early TikTok studies used content analysis methods and focused on topics such as user behavior and platform governance. However, they noted significant gaps in research that looks into the unique capabilities of TikTok features, including tools like stitches and duets. This oversight underlines the need for studies such as ours, which aim to investigate how TikTok videos, particularly those using the stitching feature, form interconnected networks of content creation. To bridge this gap, researchers can now use TikTok’s research API to gather data at scale. The API provides a structured means of collecting information on videos, user engagement, and interactions, offering researchers a more systematic approach to analyzing TikTok’s various data. 

However, while the API opens up new opportunities for large-scale analyses, it also presents several notable limitations. As \cite{corso2024we} found, the API frequently fails to deliver the full quota of requested data, with researchers often receiving only $65\%$ of the expected content. This issue is made worse because it is unclear when and why some data is missing, especially with older videos that might have been deleted or set to private. 

%Furthermore, the API tends to over-represent certain geographic regions, particularly Asia, creating biases in the data that can skew results. These shortcomings highlight the challenges of relying solely on the API for comprehensive studies of TikTok’s global content network.

Despite these issues, the TikTok Research API remains a valuable tool for capturing large datasets that were previously inaccessible, and it allows for the examination of features like stitches in more detail than traditional methods. Using this API, this study aims to explore the connections between stitched videos and the broader content networks they form, addressing some of the gaps in understanding how TikTok’s unique affordances shape user interaction and content remixing.



\section{Previous Analyses of Social Networks}
% Communication networks are not exclusive to TikTok; they are a fundamental feature of many social media platforms, including Twitter\footnote{In this section, we refer to the platform as Twitter, as this was its name at the time of the studies cited. While the platform is now known as X, this distinction maintains historical consistency with prior research}. These platforms enable users to interact through explicit or implicit connections, forming complex networks that reveal patterns of communication, influence, and content propagation. For example, on Twitter, the retweet graph captures the relationships between users who share each other's posts, offering a structured way to study how information spreads and how users engage with content \citep{bild2015aggregate}. This concept closely parallels TikTok's stitch network, where users remix or respond to videos, creating a similar structure of communication.
Social networks have been widely studied, especially on platforms like Twitter\footnote{In this section, we use the name Twitter to match the platform’s name at the time of the studies mentioned. Although now called X, this helps maintain consistency with previous research.}, which has been a key focus for analyzing how people communicate online. These platforms allow users to interact through direct or indirect connections, forming networks that show patterns of communication, influence, and how content spreads. Twitter’s features, like retweets and replies, create clear links between users and their posts, making it easier to study how information flows. For example, a retweet graph maps the connections between users sharing each other’s posts, providing a way to analyze how content spreads \citep{bild2015aggregate}.

Research on Twitter has looked at many aspects of communication, such as political influence, user behavior, and network structures. \citet{torregrosa2020analyzing} demonstrate how relevance and centrality within a Twitter network correlate with the spread of extremist discourse, emphasizing that the role of influential users helps shape the flow of content. \citet{lassen2011twitter} examine how politicians use the unique features of Twitter to reach their audiences, finding that these features shape communication styles and visibility. \citet{doi:10.1126/sciadv.abq2044} explore polarization on the platform, showing how ideological divides and network clustering create echo chambers. Since networks play such a key role in understanding social media, studying similar patterns on other platforms is important. 

TikTok’s stitch feature provides a way to study communication through explicit connections, similar to how Twitter retweet networks have been analyzed. For instance, studies like \cite{garimella2018political} used Twitter data to model retweet graphs and examine interaction patterns around political events. 

Graph theory is an essential tool for studying these communication networks. Metrics such as centrality help identify key users or videos, while community detection reveals clusters of related activity. For example, analyses of Twitter retweet graphs by \cite{doi:10.1073/pnas.2023301118} have demonstrated how interactions often lead to clusters of like-minded users, sometimes forming echo chambers. Although our study does not focus on polarization or ideological alignment, we use similar techniques, such as centrality metrics to highlight influential nodes, and graph structure analysis to examine topological patterns. 

Datasets from Twitter, such as the "Coronavirus Tweet Ids" dataset \citep{kerchner2020coronavirus}, provide useful points of comparison. These datasets capture explicit user interactions, such as retweets and replies, which align closely with the stitched relationships in TikTok networks. Although the two platforms differ in format and features, their structural similarities enable direct comparisons through graph analysis. 

Prior social network research has focused heavily on Twitter. In this paper, we apply some of the presented methodologies to a newly composed TikTok stitch graph, while in parallel relating the findings to Twitter reply networks from \cite{doi:10.1126/sciadv.abq2044}. By conducting this analysis, we extend insights from Twitter studies to TikTok, addressing gaps in understanding its unique communication dynamics. 